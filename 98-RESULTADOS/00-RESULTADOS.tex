% \begin{itemize}
%     \item \add{Apresentação dos experimentos realizados: descrição dos cenários testados, procedimentos e configurações.}
%     \item \add{Comparação dos resultados do SLAM\_toolbox com os dados do sistema OptiTrack, apresentando gráficos, mapas, trajetórias e análises quantitativas.}
%     \item \add{Discussão dos resultados: análise dos pontos fortes e limitações da solução adotada, implicações para aplicações reais e comparação com os objetivos propostos.}
% \end{itemize}


\label{Cap04}
\section{Introdução}

% \add{
% \textbf{1. \txr{Objetivo do capitulo}}
% Neste capítulo serão apresentados os resultados obtidos.....
% \textbf{2. \txr{Temas a tratar}}
% De tal modo, o capítulo inicia descrevendo o banco de dados utilizado para treinamento e teste das abordagens propostas, 
% também são apresentados todos os resultados obtidos em diferentes etapas do processo, mostrando a evolução obtida a partir da implementação de algumas técnicas apresentadas anteriormente. 
% E por fim, é feita uma análise destes resultados e uma comparação com os resultados obtidos por outros trabalhos semelhantes com o intuito de validar as abordagens propostas.
% }


% ------------------ INICIO ----------------------
Utilizou-se a princípio a biblioteca SLAM Toolbox para criação de um mapa detalhado do laboratório, fazendo o robô percorrer diversos caminhos no interior do laboratório para que o mapa gerado fosse de alta qualidade e representativo do ambiente de testes. O mapa gerado por este método foi salvo e utilizado como mapa para os experimentos de localização do robô utilizando o modo de Localização da SLAM Toolbox (ver Figura \ref{fig:Exp_mapa_LABAIR}).

\begin{figure}[htb]
    \centering
    \caption{Mapa do LAB-AIR gerado com a SLAM Toolbox, mostrando os marcadores das posições e conexões do grafo de poses}
    \includegraphics[width=0.5\linewidth]{img/mapa_localizacao.png}
    \source
    \label{fig:Exp_mapa_LABAIR}
\end{figure}

Para verificar o desempenho da tarefa de localização usando a SLAM Toolbox, utilizou-se o controlador definido na Seção \ref{sec:Controle_Dif_LIMO} para guiar o robô em dois tipos diferentes de trajetórias, uma caracterizada por caminhos lineares entre posições específicas no referencial inercial do laboratório, e uma trajetória em formato de Lemniscata de Bernoulli. As duas trajetórias podem ser vistas na Figura \ref{fig:exp_trajetorias}. A trajetória que foi denominada como \ASPASDUPLAS{CASA} foi definida como retas ligando os pontos adjacentes do vetor de pontos: 

\begin{equation}
    \begin{bmatrix}
        -1.5 & \phantom{-}1.5\\ 
        -1.5 & -1.5 \\ 
        \phantom{-}1.5 & \phantom{-}1.5 \\ 
         \phantom{-}1.5 & -1.5 \\ 
        -1.5 & \phantom{-}1.5 \\
        \phantom{-}1.5 & \phantom{-}1.5 \\
        \phantom{-}2.0 & \phantom{-}0.0 \\
        \phantom{-}1.5 & -1.5 \\
        -1.5 & -1.5
    \end{bmatrix}
\end{equation}

\begin{figure}
    \centering
    \caption{Trajetórias Desejadas dos Experimentos}
    \begin{subfigure}[b]{0.4\textwidth}
        \includegraphics[width=\textwidth]{img/Trajetoria_Lemniscata.png}
        \caption{Trajetoria em Lemniscata de Bernoulli}
    \end{subfigure}
    \begin{subfigure}[b]{0.4\textwidth}
        \includegraphics[width=\textwidth]{img/Trajetoria_Casa.png}
        \caption{Trajetória Linear}
    \end{subfigure}
    \label{fig:exp_trajetorias}
    \sourceParbox[0.8\linewidth]
\end{figure}

Para avaliar o desempenho do sistema com o referencial do OptiTrack realizou-se 4 experimentos de controle de trajetória para cada uma das duas trajetórias definidas, usando primeiro informações provindas do sistema Optitrack para obter a pose do robô e posteriormente utilizando apenas as poses obtidas pela SLAM Toolbox. Os dois primeiros experimentos serviram para analisar o desempenho do controlador implementado utilizando o sistema de referência para as duas trajetórias, a fim de obter o desempenho do controlador para o sistema ideal.

Em todos os 4 experimentos armazenou-se tanto os dados de pose provindas do OptiTrack quanto da SLAM Toolbox, os sinais de controle calculados tanto com os dados de pose do OptiTrack quanto da SLAM Toolbox e e a trajetória a ser seguida a fim de gerar os gráficos apresentados neste capítulo. A pose inicial do robô foi resetada ao início de cada experimento, mantendo o ponto $(x,y)=(0.0 ,0.0)$ do referencial inercial do laboratório como posição inicial e orientação inicial sempre paralela ao eixo $X$ ($\psi = 0\degree$).

O primeiro experimento realizado foi o controle de trajetórias lineares que formam uma figura de casa. A trajetória realizada e a trajetória desejada podem ser vistas na Figura \ref{fig:Exp1_Trajetoria_VRPN_LINEAR}. Pode-se notar que a trajetória realizada pelo robô segue em grande parte a trajetória desejada, com alguns erros de seguimento da trajetória em pontos mais críticos de viradas mais bruscas e no início do experimento, quando o robô sai da origem do plano cartesiano e corre atrás do ponto da trajetória que está em movimento. A Figura \ref{fig:Exp1_XY_vs_tempo} mostra a evolução das posições obtidas do OptiTrack (\ASPASIMPLES{vrpn}) e da SLAM Toolbox(\ASPASIMPLES{slam} durante o experimento.

\begin{figure}[htb]
    \centering
    \caption{Experimento 1: Trajetoria Realizada}
    \resizebox{0.9\linewidth}{!}{
    \input{img/Resultados/Exp1_VRPN_Control_LINEAR/PGF_TIKZ/Trajetoria_vrpn.pgf}
    }
    \label{fig:Exp1_Trajetoria_VRPN_LINEAR}
    \source
\end{figure}

% \begin{figure}[htb]
%     \centering
%     \caption{Trajetória realizada pelo robô no primeiro experimento}
%     \begin{subfigure}[b]{0.49\textwidth}
%          \includegraphics[width=\textwidth]{img/Resultados/Exp1_VRPN_Control_LINEAR/Trajetoria_Goal.pdf}
%     \end{subfigure}
%     \begin{subfigure}[b]{0.49\textwidth}
%         \includegraphics[width=\textwidth]{img/Resultados/Exp1_VRPN_Control_LINEAR/Trajetoria_VRPN_Goal.pdf}
%     \end{subfigure}
%     \source
%     \label{fig:Exp1_Trajetoria_VRPN_LINEAR}
% \end{figure}

% \begin{figure}[htb]
%     \centering
%     \caption{Trajetória Realizada + Pose obtida pela SLAM Toolbox}
%     \includegraphics[width=0.8\linewidth]{img/Resultados/Exp1_VRPN_Control_LINEAR/Trajetoria_ALL.pdf}
%     \label{fig:Exp1_Trajetoria_All}
% \end{figure}

\begin{figure}[htb]
    \centering
    \caption{Evolução das coordenadas $x$ e $y$ no Experimento 1}
    \includegraphics[width=\linewidth]{img/Resultados/Exp1_VRPN_Control_LINEAR/Position_v_time.pdf}
    \sourceParbox[\linewidth]
    \label{fig:Exp1_XY_vs_tempo}
\end{figure}

% \begin{figure}[htb]
%     \centering
%     \includegraphics[width=1\linewidth]{img/Resultados/Exp1_VRPN_Control_LINEAR/Posicoes_v_tempo.pdf}
%     \caption{Enter Caption}
%     \label{fig:Exp1_Posicoes_Tempo}
% \end{figure}

% \begin{figure}[htb]
%     \centering
%     \caption{Experimento 1 - Erros de Posição}
%     \resizebox{\linewidth}{!}{
%     %% Creator: Matplotlib, PGF backend
%%
%% To include the figure in your LaTeX document, write
%%   \input{<filename>.pgf}
%%
%% Make sure the required packages are loaded in your preamble
%%   \usepackage{pgf}
%%
%% Also ensure that all the required font packages are loaded; for instance,
%% the lmodern package is sometimes necessary when using math font.
%%   \usepackage{lmodern}
%%
%% Figures using additional raster images can only be included by \input if
%% they are in the same directory as the main LaTeX file. For loading figures
%% from other directories you can use the `import` package
%%   \usepackage{import}
%%
%% and then include the figures with
%%   \import{<path to file>}{<filename>.pgf}
%%
%% Matplotlib used the following preamble
%%   
%%   \usepackage{fontspec}
%%   \setmainfont{DejaVuSerif.ttf}[Path=\detokenize{/home/maubp/.local/lib/python3.8/site-packages/matplotlib/mpl-data/fonts/ttf/}]
%%   \setsansfont{DejaVuSans.ttf}[Path=\detokenize{/home/maubp/.local/lib/python3.8/site-packages/matplotlib/mpl-data/fonts/ttf/}]
%%   \setmonofont{DejaVuSansMono.ttf}[Path=\detokenize{/home/maubp/.local/lib/python3.8/site-packages/matplotlib/mpl-data/fonts/ttf/}]
%%   \makeatletter\@ifpackageloaded{underscore}{}{\usepackage[strings]{underscore}}\makeatother
%%
\begingroup%
\makeatletter%
\begin{pgfpicture}%
\pgfpathrectangle{\pgfpointorigin}{\pgfqpoint{13.570000in}{9.770000in}}%
\pgfusepath{use as bounding box, clip}%
\begin{pgfscope}%
\pgfsetbuttcap%
\pgfsetmiterjoin%
\definecolor{currentfill}{rgb}{1.000000,1.000000,1.000000}%
\pgfsetfillcolor{currentfill}%
\pgfsetlinewidth{0.000000pt}%
\definecolor{currentstroke}{rgb}{1.000000,1.000000,1.000000}%
\pgfsetstrokecolor{currentstroke}%
\pgfsetdash{}{0pt}%
\pgfpathmoveto{\pgfqpoint{0.000000in}{0.000000in}}%
\pgfpathlineto{\pgfqpoint{13.570000in}{0.000000in}}%
\pgfpathlineto{\pgfqpoint{13.570000in}{9.770000in}}%
\pgfpathlineto{\pgfqpoint{0.000000in}{9.770000in}}%
\pgfpathlineto{\pgfqpoint{0.000000in}{0.000000in}}%
\pgfpathclose%
\pgfusepath{fill}%
\end{pgfscope}%
\begin{pgfscope}%
\pgfsetbuttcap%
\pgfsetmiterjoin%
\definecolor{currentfill}{rgb}{1.000000,1.000000,1.000000}%
\pgfsetfillcolor{currentfill}%
\pgfsetlinewidth{0.000000pt}%
\definecolor{currentstroke}{rgb}{0.000000,0.000000,0.000000}%
\pgfsetstrokecolor{currentstroke}%
\pgfsetstrokeopacity{0.000000}%
\pgfsetdash{}{0pt}%
\pgfpathmoveto{\pgfqpoint{1.021528in}{5.487778in}}%
\pgfpathlineto{\pgfqpoint{13.390000in}{5.487778in}}%
\pgfpathlineto{\pgfqpoint{13.390000in}{9.356667in}}%
\pgfpathlineto{\pgfqpoint{1.021528in}{9.356667in}}%
\pgfpathlineto{\pgfqpoint{1.021528in}{5.487778in}}%
\pgfpathclose%
\pgfusepath{fill}%
\end{pgfscope}%
\begin{pgfscope}%
\pgfpathrectangle{\pgfqpoint{1.021528in}{5.487778in}}{\pgfqpoint{12.368472in}{3.868889in}}%
\pgfusepath{clip}%
\pgfsetrectcap%
\pgfsetroundjoin%
\pgfsetlinewidth{0.803000pt}%
\definecolor{currentstroke}{rgb}{0.690196,0.690196,0.690196}%
\pgfsetstrokecolor{currentstroke}%
\pgfsetdash{}{0pt}%
\pgfpathmoveto{\pgfqpoint{1.021528in}{5.487778in}}%
\pgfpathlineto{\pgfqpoint{1.021528in}{9.356667in}}%
\pgfusepath{stroke}%
\end{pgfscope}%
\begin{pgfscope}%
\pgfsetbuttcap%
\pgfsetroundjoin%
\definecolor{currentfill}{rgb}{0.000000,0.000000,0.000000}%
\pgfsetfillcolor{currentfill}%
\pgfsetlinewidth{0.803000pt}%
\definecolor{currentstroke}{rgb}{0.000000,0.000000,0.000000}%
\pgfsetstrokecolor{currentstroke}%
\pgfsetdash{}{0pt}%
\pgfsys@defobject{currentmarker}{\pgfqpoint{0.000000in}{-0.048611in}}{\pgfqpoint{0.000000in}{0.000000in}}{%
\pgfpathmoveto{\pgfqpoint{0.000000in}{0.000000in}}%
\pgfpathlineto{\pgfqpoint{0.000000in}{-0.048611in}}%
\pgfusepath{stroke,fill}%
}%
\begin{pgfscope}%
\pgfsys@transformshift{1.021528in}{5.487778in}%
\pgfsys@useobject{currentmarker}{}%
\end{pgfscope}%
\end{pgfscope}%
\begin{pgfscope}%
\definecolor{textcolor}{rgb}{0.000000,0.000000,0.000000}%
\pgfsetstrokecolor{textcolor}%
\pgfsetfillcolor{textcolor}%
\pgftext[x=1.021528in,y=5.390556in,,top]{\color{textcolor}\sffamily\fontsize{12.000000}{14.400000}\selectfont 0}%
\end{pgfscope}%
\begin{pgfscope}%
\pgfpathrectangle{\pgfqpoint{1.021528in}{5.487778in}}{\pgfqpoint{12.368472in}{3.868889in}}%
\pgfusepath{clip}%
\pgfsetrectcap%
\pgfsetroundjoin%
\pgfsetlinewidth{0.803000pt}%
\definecolor{currentstroke}{rgb}{0.690196,0.690196,0.690196}%
\pgfsetstrokecolor{currentstroke}%
\pgfsetdash{}{0pt}%
\pgfpathmoveto{\pgfqpoint{3.209777in}{5.487778in}}%
\pgfpathlineto{\pgfqpoint{3.209777in}{9.356667in}}%
\pgfusepath{stroke}%
\end{pgfscope}%
\begin{pgfscope}%
\pgfsetbuttcap%
\pgfsetroundjoin%
\definecolor{currentfill}{rgb}{0.000000,0.000000,0.000000}%
\pgfsetfillcolor{currentfill}%
\pgfsetlinewidth{0.803000pt}%
\definecolor{currentstroke}{rgb}{0.000000,0.000000,0.000000}%
\pgfsetstrokecolor{currentstroke}%
\pgfsetdash{}{0pt}%
\pgfsys@defobject{currentmarker}{\pgfqpoint{0.000000in}{-0.048611in}}{\pgfqpoint{0.000000in}{0.000000in}}{%
\pgfpathmoveto{\pgfqpoint{0.000000in}{0.000000in}}%
\pgfpathlineto{\pgfqpoint{0.000000in}{-0.048611in}}%
\pgfusepath{stroke,fill}%
}%
\begin{pgfscope}%
\pgfsys@transformshift{3.209777in}{5.487778in}%
\pgfsys@useobject{currentmarker}{}%
\end{pgfscope}%
\end{pgfscope}%
\begin{pgfscope}%
\definecolor{textcolor}{rgb}{0.000000,0.000000,0.000000}%
\pgfsetstrokecolor{textcolor}%
\pgfsetfillcolor{textcolor}%
\pgftext[x=3.209777in,y=5.390556in,,top]{\color{textcolor}\sffamily\fontsize{12.000000}{14.400000}\selectfont 20}%
\end{pgfscope}%
\begin{pgfscope}%
\pgfpathrectangle{\pgfqpoint{1.021528in}{5.487778in}}{\pgfqpoint{12.368472in}{3.868889in}}%
\pgfusepath{clip}%
\pgfsetrectcap%
\pgfsetroundjoin%
\pgfsetlinewidth{0.803000pt}%
\definecolor{currentstroke}{rgb}{0.690196,0.690196,0.690196}%
\pgfsetstrokecolor{currentstroke}%
\pgfsetdash{}{0pt}%
\pgfpathmoveto{\pgfqpoint{5.398027in}{5.487778in}}%
\pgfpathlineto{\pgfqpoint{5.398027in}{9.356667in}}%
\pgfusepath{stroke}%
\end{pgfscope}%
\begin{pgfscope}%
\pgfsetbuttcap%
\pgfsetroundjoin%
\definecolor{currentfill}{rgb}{0.000000,0.000000,0.000000}%
\pgfsetfillcolor{currentfill}%
\pgfsetlinewidth{0.803000pt}%
\definecolor{currentstroke}{rgb}{0.000000,0.000000,0.000000}%
\pgfsetstrokecolor{currentstroke}%
\pgfsetdash{}{0pt}%
\pgfsys@defobject{currentmarker}{\pgfqpoint{0.000000in}{-0.048611in}}{\pgfqpoint{0.000000in}{0.000000in}}{%
\pgfpathmoveto{\pgfqpoint{0.000000in}{0.000000in}}%
\pgfpathlineto{\pgfqpoint{0.000000in}{-0.048611in}}%
\pgfusepath{stroke,fill}%
}%
\begin{pgfscope}%
\pgfsys@transformshift{5.398027in}{5.487778in}%
\pgfsys@useobject{currentmarker}{}%
\end{pgfscope}%
\end{pgfscope}%
\begin{pgfscope}%
\definecolor{textcolor}{rgb}{0.000000,0.000000,0.000000}%
\pgfsetstrokecolor{textcolor}%
\pgfsetfillcolor{textcolor}%
\pgftext[x=5.398027in,y=5.390556in,,top]{\color{textcolor}\sffamily\fontsize{12.000000}{14.400000}\selectfont 40}%
\end{pgfscope}%
\begin{pgfscope}%
\pgfpathrectangle{\pgfqpoint{1.021528in}{5.487778in}}{\pgfqpoint{12.368472in}{3.868889in}}%
\pgfusepath{clip}%
\pgfsetrectcap%
\pgfsetroundjoin%
\pgfsetlinewidth{0.803000pt}%
\definecolor{currentstroke}{rgb}{0.690196,0.690196,0.690196}%
\pgfsetstrokecolor{currentstroke}%
\pgfsetdash{}{0pt}%
\pgfpathmoveto{\pgfqpoint{7.586276in}{5.487778in}}%
\pgfpathlineto{\pgfqpoint{7.586276in}{9.356667in}}%
\pgfusepath{stroke}%
\end{pgfscope}%
\begin{pgfscope}%
\pgfsetbuttcap%
\pgfsetroundjoin%
\definecolor{currentfill}{rgb}{0.000000,0.000000,0.000000}%
\pgfsetfillcolor{currentfill}%
\pgfsetlinewidth{0.803000pt}%
\definecolor{currentstroke}{rgb}{0.000000,0.000000,0.000000}%
\pgfsetstrokecolor{currentstroke}%
\pgfsetdash{}{0pt}%
\pgfsys@defobject{currentmarker}{\pgfqpoint{0.000000in}{-0.048611in}}{\pgfqpoint{0.000000in}{0.000000in}}{%
\pgfpathmoveto{\pgfqpoint{0.000000in}{0.000000in}}%
\pgfpathlineto{\pgfqpoint{0.000000in}{-0.048611in}}%
\pgfusepath{stroke,fill}%
}%
\begin{pgfscope}%
\pgfsys@transformshift{7.586276in}{5.487778in}%
\pgfsys@useobject{currentmarker}{}%
\end{pgfscope}%
\end{pgfscope}%
\begin{pgfscope}%
\definecolor{textcolor}{rgb}{0.000000,0.000000,0.000000}%
\pgfsetstrokecolor{textcolor}%
\pgfsetfillcolor{textcolor}%
\pgftext[x=7.586276in,y=5.390556in,,top]{\color{textcolor}\sffamily\fontsize{12.000000}{14.400000}\selectfont 60}%
\end{pgfscope}%
\begin{pgfscope}%
\pgfpathrectangle{\pgfqpoint{1.021528in}{5.487778in}}{\pgfqpoint{12.368472in}{3.868889in}}%
\pgfusepath{clip}%
\pgfsetrectcap%
\pgfsetroundjoin%
\pgfsetlinewidth{0.803000pt}%
\definecolor{currentstroke}{rgb}{0.690196,0.690196,0.690196}%
\pgfsetstrokecolor{currentstroke}%
\pgfsetdash{}{0pt}%
\pgfpathmoveto{\pgfqpoint{9.774526in}{5.487778in}}%
\pgfpathlineto{\pgfqpoint{9.774526in}{9.356667in}}%
\pgfusepath{stroke}%
\end{pgfscope}%
\begin{pgfscope}%
\pgfsetbuttcap%
\pgfsetroundjoin%
\definecolor{currentfill}{rgb}{0.000000,0.000000,0.000000}%
\pgfsetfillcolor{currentfill}%
\pgfsetlinewidth{0.803000pt}%
\definecolor{currentstroke}{rgb}{0.000000,0.000000,0.000000}%
\pgfsetstrokecolor{currentstroke}%
\pgfsetdash{}{0pt}%
\pgfsys@defobject{currentmarker}{\pgfqpoint{0.000000in}{-0.048611in}}{\pgfqpoint{0.000000in}{0.000000in}}{%
\pgfpathmoveto{\pgfqpoint{0.000000in}{0.000000in}}%
\pgfpathlineto{\pgfqpoint{0.000000in}{-0.048611in}}%
\pgfusepath{stroke,fill}%
}%
\begin{pgfscope}%
\pgfsys@transformshift{9.774526in}{5.487778in}%
\pgfsys@useobject{currentmarker}{}%
\end{pgfscope}%
\end{pgfscope}%
\begin{pgfscope}%
\definecolor{textcolor}{rgb}{0.000000,0.000000,0.000000}%
\pgfsetstrokecolor{textcolor}%
\pgfsetfillcolor{textcolor}%
\pgftext[x=9.774526in,y=5.390556in,,top]{\color{textcolor}\sffamily\fontsize{12.000000}{14.400000}\selectfont 80}%
\end{pgfscope}%
\begin{pgfscope}%
\pgfpathrectangle{\pgfqpoint{1.021528in}{5.487778in}}{\pgfqpoint{12.368472in}{3.868889in}}%
\pgfusepath{clip}%
\pgfsetrectcap%
\pgfsetroundjoin%
\pgfsetlinewidth{0.803000pt}%
\definecolor{currentstroke}{rgb}{0.690196,0.690196,0.690196}%
\pgfsetstrokecolor{currentstroke}%
\pgfsetdash{}{0pt}%
\pgfpathmoveto{\pgfqpoint{11.962775in}{5.487778in}}%
\pgfpathlineto{\pgfqpoint{11.962775in}{9.356667in}}%
\pgfusepath{stroke}%
\end{pgfscope}%
\begin{pgfscope}%
\pgfsetbuttcap%
\pgfsetroundjoin%
\definecolor{currentfill}{rgb}{0.000000,0.000000,0.000000}%
\pgfsetfillcolor{currentfill}%
\pgfsetlinewidth{0.803000pt}%
\definecolor{currentstroke}{rgb}{0.000000,0.000000,0.000000}%
\pgfsetstrokecolor{currentstroke}%
\pgfsetdash{}{0pt}%
\pgfsys@defobject{currentmarker}{\pgfqpoint{0.000000in}{-0.048611in}}{\pgfqpoint{0.000000in}{0.000000in}}{%
\pgfpathmoveto{\pgfqpoint{0.000000in}{0.000000in}}%
\pgfpathlineto{\pgfqpoint{0.000000in}{-0.048611in}}%
\pgfusepath{stroke,fill}%
}%
\begin{pgfscope}%
\pgfsys@transformshift{11.962775in}{5.487778in}%
\pgfsys@useobject{currentmarker}{}%
\end{pgfscope}%
\end{pgfscope}%
\begin{pgfscope}%
\definecolor{textcolor}{rgb}{0.000000,0.000000,0.000000}%
\pgfsetstrokecolor{textcolor}%
\pgfsetfillcolor{textcolor}%
\pgftext[x=11.962775in,y=5.390556in,,top]{\color{textcolor}\sffamily\fontsize{12.000000}{14.400000}\selectfont 100}%
\end{pgfscope}%
\begin{pgfscope}%
\definecolor{textcolor}{rgb}{0.000000,0.000000,0.000000}%
\pgfsetstrokecolor{textcolor}%
\pgfsetfillcolor{textcolor}%
\pgftext[x=7.205764in,y=5.173705in,,top]{\color{textcolor}\sffamily\fontsize{12.000000}{14.400000}\selectfont Time (s)}%
\end{pgfscope}%
\begin{pgfscope}%
\pgfpathrectangle{\pgfqpoint{1.021528in}{5.487778in}}{\pgfqpoint{12.368472in}{3.868889in}}%
\pgfusepath{clip}%
\pgfsetrectcap%
\pgfsetroundjoin%
\pgfsetlinewidth{0.803000pt}%
\definecolor{currentstroke}{rgb}{0.690196,0.690196,0.690196}%
\pgfsetstrokecolor{currentstroke}%
\pgfsetdash{}{0pt}%
\pgfpathmoveto{\pgfqpoint{1.021528in}{5.793623in}}%
\pgfpathlineto{\pgfqpoint{13.390000in}{5.793623in}}%
\pgfusepath{stroke}%
\end{pgfscope}%
\begin{pgfscope}%
\pgfsetbuttcap%
\pgfsetroundjoin%
\definecolor{currentfill}{rgb}{0.000000,0.000000,0.000000}%
\pgfsetfillcolor{currentfill}%
\pgfsetlinewidth{0.803000pt}%
\definecolor{currentstroke}{rgb}{0.000000,0.000000,0.000000}%
\pgfsetstrokecolor{currentstroke}%
\pgfsetdash{}{0pt}%
\pgfsys@defobject{currentmarker}{\pgfqpoint{-0.048611in}{0.000000in}}{\pgfqpoint{-0.000000in}{0.000000in}}{%
\pgfpathmoveto{\pgfqpoint{-0.000000in}{0.000000in}}%
\pgfpathlineto{\pgfqpoint{-0.048611in}{0.000000in}}%
\pgfusepath{stroke,fill}%
}%
\begin{pgfscope}%
\pgfsys@transformshift{1.021528in}{5.793623in}%
\pgfsys@useobject{currentmarker}{}%
\end{pgfscope}%
\end{pgfscope}%
\begin{pgfscope}%
\definecolor{textcolor}{rgb}{0.000000,0.000000,0.000000}%
\pgfsetstrokecolor{textcolor}%
\pgfsetfillcolor{textcolor}%
\pgftext[x=0.423582in, y=5.730310in, left, base]{\color{textcolor}\sffamily\fontsize{12.000000}{14.400000}\selectfont \ensuremath{-}0.25}%
\end{pgfscope}%
\begin{pgfscope}%
\pgfpathrectangle{\pgfqpoint{1.021528in}{5.487778in}}{\pgfqpoint{12.368472in}{3.868889in}}%
\pgfusepath{clip}%
\pgfsetrectcap%
\pgfsetroundjoin%
\pgfsetlinewidth{0.803000pt}%
\definecolor{currentstroke}{rgb}{0.690196,0.690196,0.690196}%
\pgfsetstrokecolor{currentstroke}%
\pgfsetdash{}{0pt}%
\pgfpathmoveto{\pgfqpoint{1.021528in}{6.296516in}}%
\pgfpathlineto{\pgfqpoint{13.390000in}{6.296516in}}%
\pgfusepath{stroke}%
\end{pgfscope}%
\begin{pgfscope}%
\pgfsetbuttcap%
\pgfsetroundjoin%
\definecolor{currentfill}{rgb}{0.000000,0.000000,0.000000}%
\pgfsetfillcolor{currentfill}%
\pgfsetlinewidth{0.803000pt}%
\definecolor{currentstroke}{rgb}{0.000000,0.000000,0.000000}%
\pgfsetstrokecolor{currentstroke}%
\pgfsetdash{}{0pt}%
\pgfsys@defobject{currentmarker}{\pgfqpoint{-0.048611in}{0.000000in}}{\pgfqpoint{-0.000000in}{0.000000in}}{%
\pgfpathmoveto{\pgfqpoint{-0.000000in}{0.000000in}}%
\pgfpathlineto{\pgfqpoint{-0.048611in}{0.000000in}}%
\pgfusepath{stroke,fill}%
}%
\begin{pgfscope}%
\pgfsys@transformshift{1.021528in}{6.296516in}%
\pgfsys@useobject{currentmarker}{}%
\end{pgfscope}%
\end{pgfscope}%
\begin{pgfscope}%
\definecolor{textcolor}{rgb}{0.000000,0.000000,0.000000}%
\pgfsetstrokecolor{textcolor}%
\pgfsetfillcolor{textcolor}%
\pgftext[x=0.553212in, y=6.233202in, left, base]{\color{textcolor}\sffamily\fontsize{12.000000}{14.400000}\selectfont 0.00}%
\end{pgfscope}%
\begin{pgfscope}%
\pgfpathrectangle{\pgfqpoint{1.021528in}{5.487778in}}{\pgfqpoint{12.368472in}{3.868889in}}%
\pgfusepath{clip}%
\pgfsetrectcap%
\pgfsetroundjoin%
\pgfsetlinewidth{0.803000pt}%
\definecolor{currentstroke}{rgb}{0.690196,0.690196,0.690196}%
\pgfsetstrokecolor{currentstroke}%
\pgfsetdash{}{0pt}%
\pgfpathmoveto{\pgfqpoint{1.021528in}{6.799409in}}%
\pgfpathlineto{\pgfqpoint{13.390000in}{6.799409in}}%
\pgfusepath{stroke}%
\end{pgfscope}%
\begin{pgfscope}%
\pgfsetbuttcap%
\pgfsetroundjoin%
\definecolor{currentfill}{rgb}{0.000000,0.000000,0.000000}%
\pgfsetfillcolor{currentfill}%
\pgfsetlinewidth{0.803000pt}%
\definecolor{currentstroke}{rgb}{0.000000,0.000000,0.000000}%
\pgfsetstrokecolor{currentstroke}%
\pgfsetdash{}{0pt}%
\pgfsys@defobject{currentmarker}{\pgfqpoint{-0.048611in}{0.000000in}}{\pgfqpoint{-0.000000in}{0.000000in}}{%
\pgfpathmoveto{\pgfqpoint{-0.000000in}{0.000000in}}%
\pgfpathlineto{\pgfqpoint{-0.048611in}{0.000000in}}%
\pgfusepath{stroke,fill}%
}%
\begin{pgfscope}%
\pgfsys@transformshift{1.021528in}{6.799409in}%
\pgfsys@useobject{currentmarker}{}%
\end{pgfscope}%
\end{pgfscope}%
\begin{pgfscope}%
\definecolor{textcolor}{rgb}{0.000000,0.000000,0.000000}%
\pgfsetstrokecolor{textcolor}%
\pgfsetfillcolor{textcolor}%
\pgftext[x=0.553212in, y=6.736095in, left, base]{\color{textcolor}\sffamily\fontsize{12.000000}{14.400000}\selectfont 0.25}%
\end{pgfscope}%
\begin{pgfscope}%
\pgfpathrectangle{\pgfqpoint{1.021528in}{5.487778in}}{\pgfqpoint{12.368472in}{3.868889in}}%
\pgfusepath{clip}%
\pgfsetrectcap%
\pgfsetroundjoin%
\pgfsetlinewidth{0.803000pt}%
\definecolor{currentstroke}{rgb}{0.690196,0.690196,0.690196}%
\pgfsetstrokecolor{currentstroke}%
\pgfsetdash{}{0pt}%
\pgfpathmoveto{\pgfqpoint{1.021528in}{7.302301in}}%
\pgfpathlineto{\pgfqpoint{13.390000in}{7.302301in}}%
\pgfusepath{stroke}%
\end{pgfscope}%
\begin{pgfscope}%
\pgfsetbuttcap%
\pgfsetroundjoin%
\definecolor{currentfill}{rgb}{0.000000,0.000000,0.000000}%
\pgfsetfillcolor{currentfill}%
\pgfsetlinewidth{0.803000pt}%
\definecolor{currentstroke}{rgb}{0.000000,0.000000,0.000000}%
\pgfsetstrokecolor{currentstroke}%
\pgfsetdash{}{0pt}%
\pgfsys@defobject{currentmarker}{\pgfqpoint{-0.048611in}{0.000000in}}{\pgfqpoint{-0.000000in}{0.000000in}}{%
\pgfpathmoveto{\pgfqpoint{-0.000000in}{0.000000in}}%
\pgfpathlineto{\pgfqpoint{-0.048611in}{0.000000in}}%
\pgfusepath{stroke,fill}%
}%
\begin{pgfscope}%
\pgfsys@transformshift{1.021528in}{7.302301in}%
\pgfsys@useobject{currentmarker}{}%
\end{pgfscope}%
\end{pgfscope}%
\begin{pgfscope}%
\definecolor{textcolor}{rgb}{0.000000,0.000000,0.000000}%
\pgfsetstrokecolor{textcolor}%
\pgfsetfillcolor{textcolor}%
\pgftext[x=0.553212in, y=7.238988in, left, base]{\color{textcolor}\sffamily\fontsize{12.000000}{14.400000}\selectfont 0.50}%
\end{pgfscope}%
\begin{pgfscope}%
\pgfpathrectangle{\pgfqpoint{1.021528in}{5.487778in}}{\pgfqpoint{12.368472in}{3.868889in}}%
\pgfusepath{clip}%
\pgfsetrectcap%
\pgfsetroundjoin%
\pgfsetlinewidth{0.803000pt}%
\definecolor{currentstroke}{rgb}{0.690196,0.690196,0.690196}%
\pgfsetstrokecolor{currentstroke}%
\pgfsetdash{}{0pt}%
\pgfpathmoveto{\pgfqpoint{1.021528in}{7.805194in}}%
\pgfpathlineto{\pgfqpoint{13.390000in}{7.805194in}}%
\pgfusepath{stroke}%
\end{pgfscope}%
\begin{pgfscope}%
\pgfsetbuttcap%
\pgfsetroundjoin%
\definecolor{currentfill}{rgb}{0.000000,0.000000,0.000000}%
\pgfsetfillcolor{currentfill}%
\pgfsetlinewidth{0.803000pt}%
\definecolor{currentstroke}{rgb}{0.000000,0.000000,0.000000}%
\pgfsetstrokecolor{currentstroke}%
\pgfsetdash{}{0pt}%
\pgfsys@defobject{currentmarker}{\pgfqpoint{-0.048611in}{0.000000in}}{\pgfqpoint{-0.000000in}{0.000000in}}{%
\pgfpathmoveto{\pgfqpoint{-0.000000in}{0.000000in}}%
\pgfpathlineto{\pgfqpoint{-0.048611in}{0.000000in}}%
\pgfusepath{stroke,fill}%
}%
\begin{pgfscope}%
\pgfsys@transformshift{1.021528in}{7.805194in}%
\pgfsys@useobject{currentmarker}{}%
\end{pgfscope}%
\end{pgfscope}%
\begin{pgfscope}%
\definecolor{textcolor}{rgb}{0.000000,0.000000,0.000000}%
\pgfsetstrokecolor{textcolor}%
\pgfsetfillcolor{textcolor}%
\pgftext[x=0.553212in, y=7.741880in, left, base]{\color{textcolor}\sffamily\fontsize{12.000000}{14.400000}\selectfont 0.75}%
\end{pgfscope}%
\begin{pgfscope}%
\pgfpathrectangle{\pgfqpoint{1.021528in}{5.487778in}}{\pgfqpoint{12.368472in}{3.868889in}}%
\pgfusepath{clip}%
\pgfsetrectcap%
\pgfsetroundjoin%
\pgfsetlinewidth{0.803000pt}%
\definecolor{currentstroke}{rgb}{0.690196,0.690196,0.690196}%
\pgfsetstrokecolor{currentstroke}%
\pgfsetdash{}{0pt}%
\pgfpathmoveto{\pgfqpoint{1.021528in}{8.308087in}}%
\pgfpathlineto{\pgfqpoint{13.390000in}{8.308087in}}%
\pgfusepath{stroke}%
\end{pgfscope}%
\begin{pgfscope}%
\pgfsetbuttcap%
\pgfsetroundjoin%
\definecolor{currentfill}{rgb}{0.000000,0.000000,0.000000}%
\pgfsetfillcolor{currentfill}%
\pgfsetlinewidth{0.803000pt}%
\definecolor{currentstroke}{rgb}{0.000000,0.000000,0.000000}%
\pgfsetstrokecolor{currentstroke}%
\pgfsetdash{}{0pt}%
\pgfsys@defobject{currentmarker}{\pgfqpoint{-0.048611in}{0.000000in}}{\pgfqpoint{-0.000000in}{0.000000in}}{%
\pgfpathmoveto{\pgfqpoint{-0.000000in}{0.000000in}}%
\pgfpathlineto{\pgfqpoint{-0.048611in}{0.000000in}}%
\pgfusepath{stroke,fill}%
}%
\begin{pgfscope}%
\pgfsys@transformshift{1.021528in}{8.308087in}%
\pgfsys@useobject{currentmarker}{}%
\end{pgfscope}%
\end{pgfscope}%
\begin{pgfscope}%
\definecolor{textcolor}{rgb}{0.000000,0.000000,0.000000}%
\pgfsetstrokecolor{textcolor}%
\pgfsetfillcolor{textcolor}%
\pgftext[x=0.553212in, y=8.244773in, left, base]{\color{textcolor}\sffamily\fontsize{12.000000}{14.400000}\selectfont 1.00}%
\end{pgfscope}%
\begin{pgfscope}%
\pgfpathrectangle{\pgfqpoint{1.021528in}{5.487778in}}{\pgfqpoint{12.368472in}{3.868889in}}%
\pgfusepath{clip}%
\pgfsetrectcap%
\pgfsetroundjoin%
\pgfsetlinewidth{0.803000pt}%
\definecolor{currentstroke}{rgb}{0.690196,0.690196,0.690196}%
\pgfsetstrokecolor{currentstroke}%
\pgfsetdash{}{0pt}%
\pgfpathmoveto{\pgfqpoint{1.021528in}{8.810979in}}%
\pgfpathlineto{\pgfqpoint{13.390000in}{8.810979in}}%
\pgfusepath{stroke}%
\end{pgfscope}%
\begin{pgfscope}%
\pgfsetbuttcap%
\pgfsetroundjoin%
\definecolor{currentfill}{rgb}{0.000000,0.000000,0.000000}%
\pgfsetfillcolor{currentfill}%
\pgfsetlinewidth{0.803000pt}%
\definecolor{currentstroke}{rgb}{0.000000,0.000000,0.000000}%
\pgfsetstrokecolor{currentstroke}%
\pgfsetdash{}{0pt}%
\pgfsys@defobject{currentmarker}{\pgfqpoint{-0.048611in}{0.000000in}}{\pgfqpoint{-0.000000in}{0.000000in}}{%
\pgfpathmoveto{\pgfqpoint{-0.000000in}{0.000000in}}%
\pgfpathlineto{\pgfqpoint{-0.048611in}{0.000000in}}%
\pgfusepath{stroke,fill}%
}%
\begin{pgfscope}%
\pgfsys@transformshift{1.021528in}{8.810979in}%
\pgfsys@useobject{currentmarker}{}%
\end{pgfscope}%
\end{pgfscope}%
\begin{pgfscope}%
\definecolor{textcolor}{rgb}{0.000000,0.000000,0.000000}%
\pgfsetstrokecolor{textcolor}%
\pgfsetfillcolor{textcolor}%
\pgftext[x=0.553212in, y=8.747666in, left, base]{\color{textcolor}\sffamily\fontsize{12.000000}{14.400000}\selectfont 1.25}%
\end{pgfscope}%
\begin{pgfscope}%
\pgfpathrectangle{\pgfqpoint{1.021528in}{5.487778in}}{\pgfqpoint{12.368472in}{3.868889in}}%
\pgfusepath{clip}%
\pgfsetrectcap%
\pgfsetroundjoin%
\pgfsetlinewidth{0.803000pt}%
\definecolor{currentstroke}{rgb}{0.690196,0.690196,0.690196}%
\pgfsetstrokecolor{currentstroke}%
\pgfsetdash{}{0pt}%
\pgfpathmoveto{\pgfqpoint{1.021528in}{9.313872in}}%
\pgfpathlineto{\pgfqpoint{13.390000in}{9.313872in}}%
\pgfusepath{stroke}%
\end{pgfscope}%
\begin{pgfscope}%
\pgfsetbuttcap%
\pgfsetroundjoin%
\definecolor{currentfill}{rgb}{0.000000,0.000000,0.000000}%
\pgfsetfillcolor{currentfill}%
\pgfsetlinewidth{0.803000pt}%
\definecolor{currentstroke}{rgb}{0.000000,0.000000,0.000000}%
\pgfsetstrokecolor{currentstroke}%
\pgfsetdash{}{0pt}%
\pgfsys@defobject{currentmarker}{\pgfqpoint{-0.048611in}{0.000000in}}{\pgfqpoint{-0.000000in}{0.000000in}}{%
\pgfpathmoveto{\pgfqpoint{-0.000000in}{0.000000in}}%
\pgfpathlineto{\pgfqpoint{-0.048611in}{0.000000in}}%
\pgfusepath{stroke,fill}%
}%
\begin{pgfscope}%
\pgfsys@transformshift{1.021528in}{9.313872in}%
\pgfsys@useobject{currentmarker}{}%
\end{pgfscope}%
\end{pgfscope}%
\begin{pgfscope}%
\definecolor{textcolor}{rgb}{0.000000,0.000000,0.000000}%
\pgfsetstrokecolor{textcolor}%
\pgfsetfillcolor{textcolor}%
\pgftext[x=0.553212in, y=9.250558in, left, base]{\color{textcolor}\sffamily\fontsize{12.000000}{14.400000}\selectfont 1.50}%
\end{pgfscope}%
\begin{pgfscope}%
\definecolor{textcolor}{rgb}{0.000000,0.000000,0.000000}%
\pgfsetstrokecolor{textcolor}%
\pgfsetfillcolor{textcolor}%
\pgftext[x=0.368026in,y=7.422222in,,bottom,rotate=90.000000]{\color{textcolor}\sffamily\fontsize{12.000000}{14.400000}\selectfont X Error}%
\end{pgfscope}%
\begin{pgfscope}%
\pgfpathrectangle{\pgfqpoint{1.021528in}{5.487778in}}{\pgfqpoint{12.368472in}{3.868889in}}%
\pgfusepath{clip}%
\pgfsetrectcap%
\pgfsetroundjoin%
\pgfsetlinewidth{1.505625pt}%
\definecolor{currentstroke}{rgb}{0.121569,0.466667,0.705882}%
\pgfsetstrokecolor{currentstroke}%
\pgfsetdash{}{0pt}%
\pgfpathmoveto{\pgfqpoint{1.021528in}{9.179995in}}%
\pgfpathlineto{\pgfqpoint{1.030646in}{9.179976in}}%
\pgfpathlineto{\pgfqpoint{1.039763in}{9.179920in}}%
\pgfpathlineto{\pgfqpoint{1.059214in}{9.180027in}}%
\pgfpathlineto{\pgfqpoint{1.066508in}{9.179954in}}%
\pgfpathlineto{\pgfqpoint{1.084136in}{9.180035in}}%
\pgfpathlineto{\pgfqpoint{1.090215in}{9.179987in}}%
\pgfpathlineto{\pgfqpoint{1.126078in}{9.180007in}}%
\pgfpathlineto{\pgfqpoint{1.137019in}{9.180060in}}%
\pgfpathlineto{\pgfqpoint{1.144921in}{9.179974in}}%
\pgfpathlineto{\pgfqpoint{1.174705in}{9.180125in}}%
\pgfpathlineto{\pgfqpoint{1.259804in}{9.180153in}}%
\pgfpathlineto{\pgfqpoint{1.268314in}{9.180198in}}%
\pgfpathlineto{\pgfqpoint{1.273784in}{9.180069in}}%
\pgfpathlineto{\pgfqpoint{1.361922in}{9.179158in}}%
\pgfpathlineto{\pgfqpoint{1.364354in}{9.175367in}}%
\pgfpathlineto{\pgfqpoint{1.368001in}{9.163900in}}%
\pgfpathlineto{\pgfqpoint{1.371040in}{9.149070in}}%
\pgfpathlineto{\pgfqpoint{1.375903in}{9.112063in}}%
\pgfpathlineto{\pgfqpoint{1.382589in}{9.032536in}}%
\pgfpathlineto{\pgfqpoint{1.422099in}{8.528233in}}%
\pgfpathlineto{\pgfqpoint{1.439119in}{8.355815in}}%
\pgfpathlineto{\pgfqpoint{1.446413in}{8.309976in}}%
\pgfpathlineto{\pgfqpoint{1.453707in}{8.284525in}}%
\pgfpathlineto{\pgfqpoint{1.459786in}{8.275302in}}%
\pgfpathlineto{\pgfqpoint{1.463433in}{8.273093in}}%
\pgfpathlineto{\pgfqpoint{1.468295in}{8.273743in}}%
\pgfpathlineto{\pgfqpoint{1.471942in}{8.276117in}}%
\pgfpathlineto{\pgfqpoint{1.484707in}{8.295176in}}%
\pgfpathlineto{\pgfqpoint{1.497472in}{8.323103in}}%
\pgfpathlineto{\pgfqpoint{1.527864in}{8.420226in}}%
\pgfpathlineto{\pgfqpoint{1.530904in}{8.428005in}}%
\pgfpathlineto{\pgfqpoint{1.535766in}{8.440247in}}%
\pgfpathlineto{\pgfqpoint{1.541237in}{8.451136in}}%
\pgfpathlineto{\pgfqpoint{1.543061in}{8.453598in}}%
\pgfpathlineto{\pgfqpoint{1.552178in}{8.464559in}}%
\pgfpathlineto{\pgfqpoint{1.553394in}{8.465408in}}%
\pgfpathlineto{\pgfqpoint{1.560080in}{8.467846in}}%
\pgfpathlineto{\pgfqpoint{1.562512in}{8.467206in}}%
\pgfpathlineto{\pgfqpoint{1.565551in}{8.466816in}}%
\pgfpathlineto{\pgfqpoint{1.571629in}{8.462807in}}%
\pgfpathlineto{\pgfqpoint{1.575276in}{8.459280in}}%
\pgfpathlineto{\pgfqpoint{1.579531in}{8.453457in}}%
\pgfpathlineto{\pgfqpoint{1.586826in}{8.441285in}}%
\pgfpathlineto{\pgfqpoint{1.592904in}{8.426771in}}%
\pgfpathlineto{\pgfqpoint{1.597159in}{8.415704in}}%
\pgfpathlineto{\pgfqpoint{1.603845in}{8.393434in}}%
\pgfpathlineto{\pgfqpoint{1.619041in}{8.320685in}}%
\pgfpathlineto{\pgfqpoint{1.631806in}{8.245501in}}%
\pgfpathlineto{\pgfqpoint{1.643355in}{8.161819in}}%
\pgfpathlineto{\pgfqpoint{1.662199in}{8.011020in}}%
\pgfpathlineto{\pgfqpoint{1.675571in}{7.899114in}}%
\pgfpathlineto{\pgfqpoint{1.692591in}{7.742586in}}%
\pgfpathlineto{\pgfqpoint{1.711434in}{7.559650in}}%
\pgfpathlineto{\pgfqpoint{1.759454in}{7.107099in}}%
\pgfpathlineto{\pgfqpoint{1.789239in}{6.895964in}}%
\pgfpathlineto{\pgfqpoint{1.794709in}{6.863316in}}%
\pgfpathlineto{\pgfqpoint{1.814768in}{6.753002in}}%
\pgfpathlineto{\pgfqpoint{1.838474in}{6.642989in}}%
\pgfpathlineto{\pgfqpoint{1.876769in}{6.498779in}}%
\pgfpathlineto{\pgfqpoint{1.884671in}{6.473997in}}%
\pgfpathlineto{\pgfqpoint{1.895612in}{6.442838in}}%
\pgfpathlineto{\pgfqpoint{1.920534in}{6.383939in}}%
\pgfpathlineto{\pgfqpoint{1.933906in}{6.357426in}}%
\pgfpathlineto{\pgfqpoint{1.938769in}{6.348897in}}%
\pgfpathlineto{\pgfqpoint{1.944847in}{6.340227in}}%
\pgfpathlineto{\pgfqpoint{1.948495in}{6.333609in}}%
\pgfpathlineto{\pgfqpoint{1.952749in}{6.327454in}}%
\pgfpathlineto{\pgfqpoint{1.955789in}{6.323210in}}%
\pgfpathlineto{\pgfqpoint{1.961259in}{6.316169in}}%
\pgfpathlineto{\pgfqpoint{1.966122in}{6.309330in}}%
\pgfpathlineto{\pgfqpoint{1.969769in}{6.306252in}}%
\pgfpathlineto{\pgfqpoint{1.974632in}{6.299936in}}%
\pgfpathlineto{\pgfqpoint{1.977063in}{6.298246in}}%
\pgfpathlineto{\pgfqpoint{1.979495in}{6.294967in}}%
\pgfpathlineto{\pgfqpoint{1.982534in}{6.291688in}}%
\pgfpathlineto{\pgfqpoint{1.991044in}{6.284447in}}%
\pgfpathlineto{\pgfqpoint{1.994083in}{6.282294in}}%
\pgfpathlineto{\pgfqpoint{1.998946in}{6.279056in}}%
\pgfpathlineto{\pgfqpoint{2.000769in}{6.277909in}}%
\pgfpathlineto{\pgfqpoint{2.006848in}{6.274087in}}%
\pgfpathlineto{\pgfqpoint{2.009887in}{6.272176in}}%
\pgfpathlineto{\pgfqpoint{2.012319in}{6.270527in}}%
\pgfpathlineto{\pgfqpoint{2.014750in}{6.269440in}}%
\pgfpathlineto{\pgfqpoint{2.019005in}{6.267389in}}%
\pgfpathlineto{\pgfqpoint{2.045750in}{6.258558in}}%
\pgfpathlineto{\pgfqpoint{2.046966in}{6.258357in}}%
\pgfpathlineto{\pgfqpoint{2.050613in}{6.258055in}}%
\pgfpathlineto{\pgfqpoint{2.051221in}{6.258055in}}%
\pgfpathlineto{\pgfqpoint{2.052436in}{6.256848in}}%
\pgfpathlineto{\pgfqpoint{2.054260in}{6.256486in}}%
\pgfpathlineto{\pgfqpoint{2.059731in}{6.255500in}}%
\pgfpathlineto{\pgfqpoint{2.067025in}{6.254273in}}%
\pgfpathlineto{\pgfqpoint{2.070064in}{6.254313in}}%
\pgfpathlineto{\pgfqpoint{2.072495in}{6.253971in}}%
\pgfpathlineto{\pgfqpoint{2.075535in}{6.253851in}}%
\pgfpathlineto{\pgfqpoint{2.077358in}{6.253649in}}%
\pgfpathlineto{\pgfqpoint{2.081613in}{6.253569in}}%
\pgfpathlineto{\pgfqpoint{2.082829in}{6.254354in}}%
\pgfpathlineto{\pgfqpoint{2.083437in}{6.254394in}}%
\pgfpathlineto{\pgfqpoint{2.084044in}{6.252945in}}%
\pgfpathlineto{\pgfqpoint{2.084652in}{6.254112in}}%
\pgfpathlineto{\pgfqpoint{2.096809in}{6.254354in}}%
\pgfpathlineto{\pgfqpoint{2.098633in}{6.254595in}}%
\pgfpathlineto{\pgfqpoint{2.105319in}{6.254313in}}%
\pgfpathlineto{\pgfqpoint{2.129633in}{6.256707in}}%
\pgfpathlineto{\pgfqpoint{2.131457in}{6.256888in}}%
\pgfpathlineto{\pgfqpoint{2.133888in}{6.256486in}}%
\pgfpathlineto{\pgfqpoint{2.141790in}{6.257994in}}%
\pgfpathlineto{\pgfqpoint{2.144221in}{6.258196in}}%
\pgfpathlineto{\pgfqpoint{2.148476in}{6.258880in}}%
\pgfpathlineto{\pgfqpoint{2.152123in}{6.259262in}}%
\pgfpathlineto{\pgfqpoint{2.155770in}{6.260006in}}%
\pgfpathlineto{\pgfqpoint{2.158202in}{6.260449in}}%
\pgfpathlineto{\pgfqpoint{2.160025in}{6.260750in}}%
\pgfpathlineto{\pgfqpoint{2.162457in}{6.260770in}}%
\pgfpathlineto{\pgfqpoint{2.191026in}{6.265377in}}%
\pgfpathlineto{\pgfqpoint{2.193457in}{6.265960in}}%
\pgfpathlineto{\pgfqpoint{2.197712in}{6.266785in}}%
\pgfpathlineto{\pgfqpoint{2.203183in}{6.267228in}}%
\pgfpathlineto{\pgfqpoint{2.205614in}{6.267851in}}%
\pgfpathlineto{\pgfqpoint{2.211084in}{6.268756in}}%
\pgfpathlineto{\pgfqpoint{2.217771in}{6.270466in}}%
\pgfpathlineto{\pgfqpoint{2.219594in}{6.270285in}}%
\pgfpathlineto{\pgfqpoint{2.239653in}{6.273705in}}%
\pgfpathlineto{\pgfqpoint{2.240869in}{6.273946in}}%
\pgfpathlineto{\pgfqpoint{2.242693in}{6.274067in}}%
\pgfpathlineto{\pgfqpoint{2.253026in}{6.275757in}}%
\pgfpathlineto{\pgfqpoint{2.255457in}{6.276260in}}%
\pgfpathlineto{\pgfqpoint{2.257281in}{6.276199in}}%
\pgfpathlineto{\pgfqpoint{2.259104in}{6.276984in}}%
\pgfpathlineto{\pgfqpoint{2.260928in}{6.277145in}}%
\pgfpathlineto{\pgfqpoint{2.263359in}{6.277306in}}%
\pgfpathlineto{\pgfqpoint{2.273693in}{6.278392in}}%
\pgfpathlineto{\pgfqpoint{2.275516in}{6.278573in}}%
\pgfpathlineto{\pgfqpoint{2.277948in}{6.278653in}}%
\pgfpathlineto{\pgfqpoint{2.282203in}{6.279156in}}%
\pgfpathlineto{\pgfqpoint{2.285850in}{6.279538in}}%
\pgfpathlineto{\pgfqpoint{2.288889in}{6.279719in}}%
\pgfpathlineto{\pgfqpoint{2.290712in}{6.280082in}}%
\pgfpathlineto{\pgfqpoint{2.291928in}{6.279478in}}%
\pgfpathlineto{\pgfqpoint{2.293752in}{6.280142in}}%
\pgfpathlineto{\pgfqpoint{2.299830in}{6.280242in}}%
\pgfpathlineto{\pgfqpoint{2.301654in}{6.280403in}}%
\pgfpathlineto{\pgfqpoint{2.307124in}{6.279941in}}%
\pgfpathlineto{\pgfqpoint{2.313203in}{6.280826in}}%
\pgfpathlineto{\pgfqpoint{2.321105in}{6.280846in}}%
\pgfpathlineto{\pgfqpoint{2.322928in}{6.280826in}}%
\pgfpathlineto{\pgfqpoint{2.325360in}{6.280403in}}%
\pgfpathlineto{\pgfqpoint{2.330223in}{6.280605in}}%
\pgfpathlineto{\pgfqpoint{2.337517in}{6.280222in}}%
\pgfpathlineto{\pgfqpoint{2.339340in}{6.279921in}}%
\pgfpathlineto{\pgfqpoint{2.341164in}{6.280484in}}%
\pgfpathlineto{\pgfqpoint{2.349066in}{6.279538in}}%
\pgfpathlineto{\pgfqpoint{2.350889in}{6.279961in}}%
\pgfpathlineto{\pgfqpoint{2.362438in}{6.279317in}}%
\pgfpathlineto{\pgfqpoint{2.364262in}{6.279297in}}%
\pgfpathlineto{\pgfqpoint{2.367301in}{6.279377in}}%
\pgfpathlineto{\pgfqpoint{2.412890in}{6.275475in}}%
\pgfpathlineto{\pgfqpoint{2.415929in}{6.274791in}}%
\pgfpathlineto{\pgfqpoint{2.425047in}{6.274007in}}%
\pgfpathlineto{\pgfqpoint{2.432949in}{6.273142in}}%
\pgfpathlineto{\pgfqpoint{2.438419in}{6.272659in}}%
\pgfpathlineto{\pgfqpoint{2.441459in}{6.272377in}}%
\pgfpathlineto{\pgfqpoint{2.498596in}{6.267730in}}%
\pgfpathlineto{\pgfqpoint{2.502243in}{6.267972in}}%
\pgfpathlineto{\pgfqpoint{2.507106in}{6.267207in}}%
\pgfpathlineto{\pgfqpoint{2.508322in}{6.267650in}}%
\pgfpathlineto{\pgfqpoint{2.509537in}{6.266684in}}%
\pgfpathlineto{\pgfqpoint{2.510145in}{6.267409in}}%
\pgfpathlineto{\pgfqpoint{2.518047in}{6.266644in}}%
\pgfpathlineto{\pgfqpoint{2.524734in}{6.266906in}}%
\pgfpathlineto{\pgfqpoint{2.526557in}{6.266604in}}%
\pgfpathlineto{\pgfqpoint{2.539322in}{6.266363in}}%
\pgfpathlineto{\pgfqpoint{2.544793in}{6.265840in}}%
\pgfpathlineto{\pgfqpoint{2.546616in}{6.266805in}}%
\pgfpathlineto{\pgfqpoint{2.549047in}{6.265920in}}%
\pgfpathlineto{\pgfqpoint{2.552695in}{6.266564in}}%
\pgfpathlineto{\pgfqpoint{2.555126in}{6.265960in}}%
\pgfpathlineto{\pgfqpoint{2.560597in}{6.266443in}}%
\pgfpathlineto{\pgfqpoint{2.573969in}{6.266443in}}%
\pgfpathlineto{\pgfqpoint{2.577008in}{6.266765in}}%
\pgfpathlineto{\pgfqpoint{2.580656in}{6.266765in}}%
\pgfpathlineto{\pgfqpoint{2.586126in}{6.267529in}}%
\pgfpathlineto{\pgfqpoint{2.588557in}{6.267107in}}%
\pgfpathlineto{\pgfqpoint{2.761794in}{6.277688in}}%
\pgfpathlineto{\pgfqpoint{2.763617in}{6.277125in}}%
\pgfpathlineto{\pgfqpoint{2.767265in}{6.277990in}}%
\pgfpathlineto{\pgfqpoint{2.813461in}{6.277084in}}%
\pgfpathlineto{\pgfqpoint{2.815285in}{6.277245in}}%
\pgfpathlineto{\pgfqpoint{2.819539in}{6.277004in}}%
\pgfpathlineto{\pgfqpoint{2.831696in}{6.276863in}}%
\pgfpathlineto{\pgfqpoint{2.833520in}{6.276843in}}%
\pgfpathlineto{\pgfqpoint{2.835343in}{6.276461in}}%
\pgfpathlineto{\pgfqpoint{2.843245in}{6.276501in}}%
\pgfpathlineto{\pgfqpoint{2.847500in}{6.276199in}}%
\pgfpathlineto{\pgfqpoint{2.856010in}{6.275958in}}%
\pgfpathlineto{\pgfqpoint{2.859050in}{6.276159in}}%
\pgfpathlineto{\pgfqpoint{2.862089in}{6.275596in}}%
\pgfpathlineto{\pgfqpoint{2.863912in}{6.275777in}}%
\pgfpathlineto{\pgfqpoint{2.872422in}{6.275314in}}%
\pgfpathlineto{\pgfqpoint{2.875461in}{6.274952in}}%
\pgfpathlineto{\pgfqpoint{2.877893in}{6.275093in}}%
\pgfpathlineto{\pgfqpoint{2.882148in}{6.274711in}}%
\pgfpathlineto{\pgfqpoint{2.885187in}{6.274952in}}%
\pgfpathlineto{\pgfqpoint{2.887618in}{6.274449in}}%
\pgfpathlineto{\pgfqpoint{2.889442in}{6.274469in}}%
\pgfpathlineto{\pgfqpoint{2.892481in}{6.273624in}}%
\pgfpathlineto{\pgfqpoint{2.895520in}{6.274308in}}%
\pgfpathlineto{\pgfqpoint{2.902815in}{6.273725in}}%
\pgfpathlineto{\pgfqpoint{2.911932in}{6.273262in}}%
\pgfpathlineto{\pgfqpoint{2.938677in}{6.271693in}}%
\pgfpathlineto{\pgfqpoint{2.941109in}{6.271150in}}%
\pgfpathlineto{\pgfqpoint{2.941717in}{6.270486in}}%
\pgfpathlineto{\pgfqpoint{2.942325in}{6.271311in}}%
\pgfpathlineto{\pgfqpoint{2.983658in}{6.269059in}}%
\pgfpathlineto{\pgfqpoint{2.992168in}{6.237026in}}%
\pgfpathlineto{\pgfqpoint{2.997639in}{6.217003in}}%
\pgfpathlineto{\pgfqpoint{3.101580in}{5.852468in}}%
\pgfpathlineto{\pgfqpoint{3.111306in}{5.825135in}}%
\pgfpathlineto{\pgfqpoint{3.128326in}{5.783392in}}%
\pgfpathlineto{\pgfqpoint{3.140483in}{5.761442in}}%
\pgfpathlineto{\pgfqpoint{3.146561in}{5.753732in}}%
\pgfpathlineto{\pgfqpoint{3.156287in}{5.745998in}}%
\pgfpathlineto{\pgfqpoint{3.158110in}{5.744944in}}%
\pgfpathlineto{\pgfqpoint{3.163581in}{5.745555in}}%
\pgfpathlineto{\pgfqpoint{3.166012in}{5.746703in}}%
\pgfpathlineto{\pgfqpoint{3.167836in}{5.746550in}}%
\pgfpathlineto{\pgfqpoint{3.169659in}{5.745475in}}%
\pgfpathlineto{\pgfqpoint{3.176346in}{5.747630in}}%
\pgfpathlineto{\pgfqpoint{3.178777in}{5.749136in}}%
\pgfpathlineto{\pgfqpoint{3.186071in}{5.755201in}}%
\pgfpathlineto{\pgfqpoint{3.195189in}{5.766062in}}%
\pgfpathlineto{\pgfqpoint{3.198836in}{5.771228in}}%
\pgfpathlineto{\pgfqpoint{3.208562in}{5.788099in}}%
\pgfpathlineto{\pgfqpoint{3.221934in}{5.816137in}}%
\pgfpathlineto{\pgfqpoint{3.224973in}{5.821964in}}%
\pgfpathlineto{\pgfqpoint{3.244425in}{5.860424in}}%
\pgfpathlineto{\pgfqpoint{3.259013in}{5.882203in}}%
\pgfpathlineto{\pgfqpoint{3.269954in}{5.892188in}}%
\pgfpathlineto{\pgfqpoint{3.272993in}{5.893857in}}%
\pgfpathlineto{\pgfqpoint{3.279680in}{5.897617in}}%
\pgfpathlineto{\pgfqpoint{3.283327in}{5.898946in}}%
\pgfpathlineto{\pgfqpoint{3.285758in}{5.899567in}}%
\pgfpathlineto{\pgfqpoint{3.290013in}{5.900779in}}%
\pgfpathlineto{\pgfqpoint{3.291837in}{5.900937in}}%
\pgfpathlineto{\pgfqpoint{3.296092in}{5.902928in}}%
\pgfpathlineto{\pgfqpoint{3.299739in}{5.903078in}}%
\pgfpathlineto{\pgfqpoint{3.301562in}{5.902832in}}%
\pgfpathlineto{\pgfqpoint{3.303994in}{5.903362in}}%
\pgfpathlineto{\pgfqpoint{3.309464in}{5.903467in}}%
\pgfpathlineto{\pgfqpoint{3.318582in}{5.902380in}}%
\pgfpathlineto{\pgfqpoint{3.320405in}{5.903128in}}%
\pgfpathlineto{\pgfqpoint{3.324660in}{5.902325in}}%
\pgfpathlineto{\pgfqpoint{3.325876in}{5.902887in}}%
\pgfpathlineto{\pgfqpoint{3.327700in}{5.902073in}}%
\pgfpathlineto{\pgfqpoint{3.331955in}{5.901311in}}%
\pgfpathlineto{\pgfqpoint{3.335602in}{5.900484in}}%
\pgfpathlineto{\pgfqpoint{3.338641in}{5.900025in}}%
\pgfpathlineto{\pgfqpoint{3.344111in}{5.898944in}}%
\pgfpathlineto{\pgfqpoint{3.360523in}{5.894524in}}%
\pgfpathlineto{\pgfqpoint{3.364170in}{5.893625in}}%
\pgfpathlineto{\pgfqpoint{3.366602in}{5.893187in}}%
\pgfpathlineto{\pgfqpoint{3.373288in}{5.891440in}}%
\pgfpathlineto{\pgfqpoint{3.413406in}{5.883983in}}%
\pgfpathlineto{\pgfqpoint{3.426779in}{5.883882in}}%
\pgfpathlineto{\pgfqpoint{3.429210in}{5.884408in}}%
\pgfpathlineto{\pgfqpoint{3.433465in}{5.882961in}}%
\pgfpathlineto{\pgfqpoint{3.439543in}{5.883213in}}%
\pgfpathlineto{\pgfqpoint{3.441367in}{5.883214in}}%
\pgfpathlineto{\pgfqpoint{3.444406in}{5.883204in}}%
\pgfpathlineto{\pgfqpoint{3.448661in}{5.883982in}}%
\pgfpathlineto{\pgfqpoint{3.452308in}{5.883527in}}%
\pgfpathlineto{\pgfqpoint{3.455348in}{5.883728in}}%
\pgfpathlineto{\pgfqpoint{3.458995in}{5.883513in}}%
\pgfpathlineto{\pgfqpoint{3.462034in}{5.884190in}}%
\pgfpathlineto{\pgfqpoint{3.479661in}{5.886328in}}%
\pgfpathlineto{\pgfqpoint{3.482093in}{5.886258in}}%
\pgfpathlineto{\pgfqpoint{3.485132in}{5.886157in}}%
\pgfpathlineto{\pgfqpoint{3.491210in}{5.887008in}}%
\pgfpathlineto{\pgfqpoint{3.494250in}{5.886930in}}%
\pgfpathlineto{\pgfqpoint{3.496073in}{5.887178in}}%
\pgfpathlineto{\pgfqpoint{3.503975in}{5.887453in}}%
\pgfpathlineto{\pgfqpoint{3.506407in}{5.888268in}}%
\pgfpathlineto{\pgfqpoint{3.507622in}{5.888658in}}%
\pgfpathlineto{\pgfqpoint{3.514916in}{5.889988in}}%
\pgfpathlineto{\pgfqpoint{3.517956in}{5.890500in}}%
\pgfpathlineto{\pgfqpoint{3.520995in}{5.890401in}}%
\pgfpathlineto{\pgfqpoint{3.524642in}{5.890911in}}%
\pgfpathlineto{\pgfqpoint{3.546525in}{5.894036in}}%
\pgfpathlineto{\pgfqpoint{3.548956in}{5.894584in}}%
\pgfpathlineto{\pgfqpoint{3.557466in}{5.896490in}}%
\pgfpathlineto{\pgfqpoint{3.559289in}{5.896840in}}%
\pgfpathlineto{\pgfqpoint{3.641956in}{5.905810in}}%
\pgfpathlineto{\pgfqpoint{3.649251in}{5.906184in}}%
\pgfpathlineto{\pgfqpoint{3.656545in}{5.906159in}}%
\pgfpathlineto{\pgfqpoint{3.658976in}{5.906005in}}%
\pgfpathlineto{\pgfqpoint{3.661408in}{5.905822in}}%
\pgfpathlineto{\pgfqpoint{3.664447in}{5.906266in}}%
\pgfpathlineto{\pgfqpoint{3.668094in}{5.906314in}}%
\pgfpathlineto{\pgfqpoint{3.672349in}{5.906296in}}%
\pgfpathlineto{\pgfqpoint{3.675388in}{5.905981in}}%
\pgfpathlineto{\pgfqpoint{3.689369in}{5.905890in}}%
\pgfpathlineto{\pgfqpoint{3.691800in}{5.906482in}}%
\pgfpathlineto{\pgfqpoint{3.695447in}{5.907000in}}%
\pgfpathlineto{\pgfqpoint{3.697878in}{5.906406in}}%
\pgfpathlineto{\pgfqpoint{3.700310in}{5.906122in}}%
\pgfpathlineto{\pgfqpoint{3.706388in}{5.905719in}}%
\pgfpathlineto{\pgfqpoint{3.708212in}{5.905961in}}%
\pgfpathlineto{\pgfqpoint{3.713682in}{5.905409in}}%
\pgfpathlineto{\pgfqpoint{3.723408in}{5.904932in}}%
\pgfpathlineto{\pgfqpoint{3.725232in}{5.904624in}}%
\pgfpathlineto{\pgfqpoint{3.727663in}{5.903654in}}%
\pgfpathlineto{\pgfqpoint{3.730094in}{5.903476in}}%
\pgfpathlineto{\pgfqpoint{3.732526in}{5.903389in}}%
\pgfpathlineto{\pgfqpoint{3.733741in}{5.903358in}}%
\pgfpathlineto{\pgfqpoint{3.736173in}{5.902807in}}%
\pgfpathlineto{\pgfqpoint{3.739212in}{5.902859in}}%
\pgfpathlineto{\pgfqpoint{3.741036in}{5.903255in}}%
\pgfpathlineto{\pgfqpoint{3.743467in}{5.902770in}}%
\pgfpathlineto{\pgfqpoint{3.744683in}{5.902612in}}%
\pgfpathlineto{\pgfqpoint{3.746506in}{5.902522in}}%
\pgfpathlineto{\pgfqpoint{3.752585in}{5.902709in}}%
\pgfpathlineto{\pgfqpoint{3.755016in}{5.902795in}}%
\pgfpathlineto{\pgfqpoint{3.759271in}{5.903352in}}%
\pgfpathlineto{\pgfqpoint{3.761095in}{5.903402in}}%
\pgfpathlineto{\pgfqpoint{3.777506in}{5.901372in}}%
\pgfpathlineto{\pgfqpoint{3.787840in}{5.901236in}}%
\pgfpathlineto{\pgfqpoint{3.791487in}{5.901207in}}%
\pgfpathlineto{\pgfqpoint{3.809114in}{5.899063in}}%
\pgfpathlineto{\pgfqpoint{3.812154in}{5.898733in}}%
\pgfpathlineto{\pgfqpoint{3.813977in}{5.899242in}}%
\pgfpathlineto{\pgfqpoint{3.816409in}{5.899445in}}%
\pgfpathlineto{\pgfqpoint{3.818232in}{5.899307in}}%
\pgfpathlineto{\pgfqpoint{3.823095in}{5.897875in}}%
\pgfpathlineto{\pgfqpoint{3.825526in}{5.897343in}}%
\pgfpathlineto{\pgfqpoint{3.827958in}{5.897141in}}%
\pgfpathlineto{\pgfqpoint{3.829781in}{5.897540in}}%
\pgfpathlineto{\pgfqpoint{3.849840in}{5.898639in}}%
\pgfpathlineto{\pgfqpoint{3.852879in}{5.898515in}}%
\pgfpathlineto{\pgfqpoint{3.855311in}{5.899176in}}%
\pgfpathlineto{\pgfqpoint{3.889958in}{5.900981in}}%
\pgfpathlineto{\pgfqpoint{3.891782in}{5.901382in}}%
\pgfpathlineto{\pgfqpoint{3.898468in}{5.901741in}}%
\pgfpathlineto{\pgfqpoint{3.906370in}{5.902886in}}%
\pgfpathlineto{\pgfqpoint{3.908194in}{5.902931in}}%
\pgfpathlineto{\pgfqpoint{3.911233in}{5.902811in}}%
\pgfpathlineto{\pgfqpoint{3.913056in}{5.903435in}}%
\pgfpathlineto{\pgfqpoint{3.914880in}{5.903666in}}%
\pgfpathlineto{\pgfqpoint{3.920351in}{5.904201in}}%
\pgfpathlineto{\pgfqpoint{3.927037in}{5.904433in}}%
\pgfpathlineto{\pgfqpoint{3.938586in}{5.904976in}}%
\pgfpathlineto{\pgfqpoint{3.940409in}{5.904730in}}%
\pgfpathlineto{\pgfqpoint{3.942841in}{5.905192in}}%
\pgfpathlineto{\pgfqpoint{3.945272in}{5.905065in}}%
\pgfpathlineto{\pgfqpoint{3.950743in}{5.905560in}}%
\pgfpathlineto{\pgfqpoint{3.951959in}{5.905162in}}%
\pgfpathlineto{\pgfqpoint{3.954390in}{5.906111in}}%
\pgfpathlineto{\pgfqpoint{3.955606in}{5.906444in}}%
\pgfpathlineto{\pgfqpoint{3.958037in}{5.907582in}}%
\pgfpathlineto{\pgfqpoint{3.962900in}{5.905548in}}%
\pgfpathlineto{\pgfqpoint{3.963508in}{5.906503in}}%
\pgfpathlineto{\pgfqpoint{3.964116in}{5.905623in}}%
\pgfpathlineto{\pgfqpoint{3.966547in}{5.906545in}}%
\pgfpathlineto{\pgfqpoint{3.967155in}{5.906040in}}%
\pgfpathlineto{\pgfqpoint{3.967763in}{5.906937in}}%
\pgfpathlineto{\pgfqpoint{3.971410in}{5.906706in}}%
\pgfpathlineto{\pgfqpoint{3.972625in}{5.908142in}}%
\pgfpathlineto{\pgfqpoint{3.973233in}{5.907674in}}%
\pgfpathlineto{\pgfqpoint{3.975057in}{5.907188in}}%
\pgfpathlineto{\pgfqpoint{3.976272in}{5.907065in}}%
\pgfpathlineto{\pgfqpoint{3.976880in}{5.907960in}}%
\pgfpathlineto{\pgfqpoint{3.977488in}{5.906723in}}%
\pgfpathlineto{\pgfqpoint{3.978096in}{5.907051in}}%
\pgfpathlineto{\pgfqpoint{3.978704in}{5.908238in}}%
\pgfpathlineto{\pgfqpoint{3.979312in}{5.907474in}}%
\pgfpathlineto{\pgfqpoint{3.981743in}{5.907485in}}%
\pgfpathlineto{\pgfqpoint{3.984174in}{5.907706in}}%
\pgfpathlineto{\pgfqpoint{3.987214in}{5.908610in}}%
\pgfpathlineto{\pgfqpoint{3.988429in}{5.907799in}}%
\pgfpathlineto{\pgfqpoint{3.989645in}{5.907658in}}%
\pgfpathlineto{\pgfqpoint{3.991469in}{5.907931in}}%
\pgfpathlineto{\pgfqpoint{3.995724in}{5.908230in}}%
\pgfpathlineto{\pgfqpoint{3.996331in}{5.907876in}}%
\pgfpathlineto{\pgfqpoint{3.998155in}{5.908812in}}%
\pgfpathlineto{\pgfqpoint{3.999978in}{5.908424in}}%
\pgfpathlineto{\pgfqpoint{4.006057in}{5.908385in}}%
\pgfpathlineto{\pgfqpoint{4.008488in}{5.908622in}}%
\pgfpathlineto{\pgfqpoint{4.010920in}{5.909616in}}%
\pgfpathlineto{\pgfqpoint{4.012135in}{5.908922in}}%
\pgfpathlineto{\pgfqpoint{4.014567in}{5.909783in}}%
\pgfpathlineto{\pgfqpoint{4.019430in}{5.909618in}}%
\pgfpathlineto{\pgfqpoint{4.024900in}{5.908847in}}%
\pgfpathlineto{\pgfqpoint{4.027939in}{5.908384in}}%
\pgfpathlineto{\pgfqpoint{4.030979in}{5.907354in}}%
\pgfpathlineto{\pgfqpoint{4.034626in}{5.906270in}}%
\pgfpathlineto{\pgfqpoint{4.049822in}{5.906379in}}%
\pgfpathlineto{\pgfqpoint{4.051038in}{5.906301in}}%
\pgfpathlineto{\pgfqpoint{4.057724in}{5.906561in}}%
\pgfpathlineto{\pgfqpoint{4.060155in}{5.906938in}}%
\pgfpathlineto{\pgfqpoint{4.061979in}{5.907141in}}%
\pgfpathlineto{\pgfqpoint{4.068665in}{5.907260in}}%
\pgfpathlineto{\pgfqpoint{4.072312in}{5.907427in}}%
\pgfpathlineto{\pgfqpoint{4.082038in}{5.906626in}}%
\pgfpathlineto{\pgfqpoint{4.084469in}{5.906905in}}%
\pgfpathlineto{\pgfqpoint{4.085685in}{5.907033in}}%
\pgfpathlineto{\pgfqpoint{4.088116in}{5.907230in}}%
\pgfpathlineto{\pgfqpoint{4.127626in}{5.905355in}}%
\pgfpathlineto{\pgfqpoint{4.129450in}{5.905179in}}%
\pgfpathlineto{\pgfqpoint{4.133097in}{5.905164in}}%
\pgfpathlineto{\pgfqpoint{4.137352in}{5.904643in}}%
\pgfpathlineto{\pgfqpoint{4.141607in}{5.905252in}}%
\pgfpathlineto{\pgfqpoint{4.150117in}{5.904835in}}%
\pgfpathlineto{\pgfqpoint{4.152548in}{5.905312in}}%
\pgfpathlineto{\pgfqpoint{4.154979in}{5.904125in}}%
\pgfpathlineto{\pgfqpoint{4.161666in}{5.903327in}}%
\pgfpathlineto{\pgfqpoint{4.165313in}{5.903447in}}%
\pgfpathlineto{\pgfqpoint{4.167136in}{5.903480in}}%
\pgfpathlineto{\pgfqpoint{4.175038in}{5.903778in}}%
\pgfpathlineto{\pgfqpoint{4.182940in}{5.903212in}}%
\pgfpathlineto{\pgfqpoint{4.186587in}{5.903540in}}%
\pgfpathlineto{\pgfqpoint{4.189019in}{5.902928in}}%
\pgfpathlineto{\pgfqpoint{4.192666in}{5.903563in}}%
\pgfpathlineto{\pgfqpoint{4.195705in}{5.903177in}}%
\pgfpathlineto{\pgfqpoint{4.202999in}{5.902542in}}%
\pgfpathlineto{\pgfqpoint{4.204215in}{5.902593in}}%
\pgfpathlineto{\pgfqpoint{4.206039in}{5.902672in}}%
\pgfpathlineto{\pgfqpoint{4.216372in}{5.901841in}}%
\pgfpathlineto{\pgfqpoint{4.217588in}{5.901590in}}%
\pgfpathlineto{\pgfqpoint{4.219411in}{5.901810in}}%
\pgfpathlineto{\pgfqpoint{4.221843in}{5.901650in}}%
\pgfpathlineto{\pgfqpoint{4.223058in}{5.901884in}}%
\pgfpathlineto{\pgfqpoint{4.229745in}{5.901768in}}%
\pgfpathlineto{\pgfqpoint{4.230960in}{5.901519in}}%
\pgfpathlineto{\pgfqpoint{4.235215in}{5.901904in}}%
\pgfpathlineto{\pgfqpoint{4.236431in}{5.901933in}}%
\pgfpathlineto{\pgfqpoint{4.238254in}{5.901485in}}%
\pgfpathlineto{\pgfqpoint{4.244333in}{5.901957in}}%
\pgfpathlineto{\pgfqpoint{4.246156in}{5.902406in}}%
\pgfpathlineto{\pgfqpoint{4.247980in}{5.901689in}}%
\pgfpathlineto{\pgfqpoint{4.251627in}{5.901739in}}%
\pgfpathlineto{\pgfqpoint{4.253451in}{5.900521in}}%
\pgfpathlineto{\pgfqpoint{4.271686in}{5.900435in}}%
\pgfpathlineto{\pgfqpoint{4.274117in}{5.901119in}}%
\pgfpathlineto{\pgfqpoint{4.280804in}{5.900935in}}%
\pgfpathlineto{\pgfqpoint{4.283843in}{5.900473in}}%
\pgfpathlineto{\pgfqpoint{4.286882in}{5.900678in}}%
\pgfpathlineto{\pgfqpoint{4.289314in}{5.900527in}}%
\pgfpathlineto{\pgfqpoint{4.296000in}{5.901776in}}%
\pgfpathlineto{\pgfqpoint{4.306941in}{5.901608in}}%
\pgfpathlineto{\pgfqpoint{4.313628in}{5.901847in}}%
\pgfpathlineto{\pgfqpoint{4.318490in}{5.902096in}}%
\pgfpathlineto{\pgfqpoint{4.322745in}{5.902039in}}%
\pgfpathlineto{\pgfqpoint{4.324569in}{5.902154in}}%
\pgfpathlineto{\pgfqpoint{4.330647in}{5.902171in}}%
\pgfpathlineto{\pgfqpoint{4.344628in}{5.902073in}}%
\pgfpathlineto{\pgfqpoint{4.346451in}{5.902810in}}%
\pgfpathlineto{\pgfqpoint{4.354961in}{5.903026in}}%
\pgfpathlineto{\pgfqpoint{4.356177in}{5.902979in}}%
\pgfpathlineto{\pgfqpoint{4.358000in}{5.903071in}}%
\pgfpathlineto{\pgfqpoint{4.379883in}{5.904400in}}%
\pgfpathlineto{\pgfqpoint{4.381706in}{5.904784in}}%
\pgfpathlineto{\pgfqpoint{4.389608in}{5.904790in}}%
\pgfpathlineto{\pgfqpoint{4.395687in}{5.905628in}}%
\pgfpathlineto{\pgfqpoint{4.397510in}{5.905396in}}%
\pgfpathlineto{\pgfqpoint{4.399942in}{5.905331in}}%
\pgfpathlineto{\pgfqpoint{4.402981in}{5.905502in}}%
\pgfpathlineto{\pgfqpoint{4.421824in}{5.906418in}}%
\pgfpathlineto{\pgfqpoint{4.425471in}{5.906973in}}%
\pgfpathlineto{\pgfqpoint{4.427903in}{5.906306in}}%
\pgfpathlineto{\pgfqpoint{4.437021in}{5.906481in}}%
\pgfpathlineto{\pgfqpoint{4.439452in}{5.906831in}}%
\pgfpathlineto{\pgfqpoint{4.445530in}{5.906602in}}%
\pgfpathlineto{\pgfqpoint{4.450393in}{5.906627in}}%
\pgfpathlineto{\pgfqpoint{4.460119in}{5.906797in}}%
\pgfpathlineto{\pgfqpoint{4.462550in}{5.906242in}}%
\pgfpathlineto{\pgfqpoint{4.477746in}{5.906604in}}%
\pgfpathlineto{\pgfqpoint{4.481393in}{5.907277in}}%
\pgfpathlineto{\pgfqpoint{4.491727in}{5.907191in}}%
\pgfpathlineto{\pgfqpoint{4.496589in}{5.909252in}}%
\pgfpathlineto{\pgfqpoint{4.498413in}{5.909738in}}%
\pgfpathlineto{\pgfqpoint{4.503276in}{5.907719in}}%
\pgfpathlineto{\pgfqpoint{4.508746in}{5.908131in}}%
\pgfpathlineto{\pgfqpoint{4.511786in}{5.907924in}}%
\pgfpathlineto{\pgfqpoint{4.533668in}{5.908814in}}%
\pgfpathlineto{\pgfqpoint{4.535492in}{5.908759in}}%
\pgfpathlineto{\pgfqpoint{4.537315in}{5.909407in}}%
\pgfpathlineto{\pgfqpoint{4.542786in}{5.908468in}}%
\pgfpathlineto{\pgfqpoint{4.544609in}{5.909262in}}%
\pgfpathlineto{\pgfqpoint{4.551904in}{5.908930in}}%
\pgfpathlineto{\pgfqpoint{4.598100in}{5.906201in}}%
\pgfpathlineto{\pgfqpoint{4.604178in}{5.906944in}}%
\pgfpathlineto{\pgfqpoint{4.609041in}{5.906921in}}%
\pgfpathlineto{\pgfqpoint{4.612688in}{5.906780in}}%
\pgfpathlineto{\pgfqpoint{4.621198in}{5.906754in}}%
\pgfpathlineto{\pgfqpoint{4.633963in}{5.954227in}}%
\pgfpathlineto{\pgfqpoint{4.655238in}{6.014333in}}%
\pgfpathlineto{\pgfqpoint{4.666179in}{6.038572in}}%
\pgfpathlineto{\pgfqpoint{4.677120in}{6.056918in}}%
\pgfpathlineto{\pgfqpoint{4.685022in}{6.068867in}}%
\pgfpathlineto{\pgfqpoint{4.687454in}{6.075686in}}%
\pgfpathlineto{\pgfqpoint{4.693532in}{6.087272in}}%
\pgfpathlineto{\pgfqpoint{4.702042in}{6.099483in}}%
\pgfpathlineto{\pgfqpoint{4.705689in}{6.103808in}}%
\pgfpathlineto{\pgfqpoint{4.708120in}{6.107388in}}%
\pgfpathlineto{\pgfqpoint{4.733042in}{6.125211in}}%
\pgfpathlineto{\pgfqpoint{4.740944in}{6.128510in}}%
\pgfpathlineto{\pgfqpoint{4.742768in}{6.129757in}}%
\pgfpathlineto{\pgfqpoint{4.746415in}{6.130260in}}%
\pgfpathlineto{\pgfqpoint{4.754924in}{6.133056in}}%
\pgfpathlineto{\pgfqpoint{4.761611in}{6.134826in}}%
\pgfpathlineto{\pgfqpoint{4.768905in}{6.138024in}}%
\pgfpathlineto{\pgfqpoint{4.771336in}{6.139312in}}%
\pgfpathlineto{\pgfqpoint{4.778023in}{6.143114in}}%
\pgfpathlineto{\pgfqpoint{4.787748in}{6.150355in}}%
\pgfpathlineto{\pgfqpoint{4.790180in}{6.152990in}}%
\pgfpathlineto{\pgfqpoint{4.793827in}{6.155807in}}%
\pgfpathlineto{\pgfqpoint{4.795650in}{6.159267in}}%
\pgfpathlineto{\pgfqpoint{4.802944in}{6.167454in}}%
\pgfpathlineto{\pgfqpoint{4.805984in}{6.172281in}}%
\pgfpathlineto{\pgfqpoint{4.808415in}{6.174776in}}%
\pgfpathlineto{\pgfqpoint{4.812670in}{6.180891in}}%
\pgfpathlineto{\pgfqpoint{4.830298in}{6.212050in}}%
\pgfpathlineto{\pgfqpoint{4.888651in}{6.329486in}}%
\pgfpathlineto{\pgfqpoint{4.892906in}{6.336245in}}%
\pgfpathlineto{\pgfqpoint{4.898984in}{6.348394in}}%
\pgfpathlineto{\pgfqpoint{4.905671in}{6.364427in}}%
\pgfpathlineto{\pgfqpoint{4.909318in}{6.371206in}}%
\pgfpathlineto{\pgfqpoint{4.917220in}{6.384060in}}%
\pgfpathlineto{\pgfqpoint{4.918435in}{6.384985in}}%
\pgfpathlineto{\pgfqpoint{4.923298in}{6.391160in}}%
\pgfpathlineto{\pgfqpoint{4.929984in}{6.397155in}}%
\pgfpathlineto{\pgfqpoint{4.932416in}{6.399287in}}%
\pgfpathlineto{\pgfqpoint{4.937279in}{6.402244in}}%
\pgfpathlineto{\pgfqpoint{4.942749in}{6.405764in}}%
\pgfpathlineto{\pgfqpoint{4.947612in}{6.406368in}}%
\pgfpathlineto{\pgfqpoint{4.950043in}{6.406026in}}%
\pgfpathlineto{\pgfqpoint{4.972534in}{6.400474in}}%
\pgfpathlineto{\pgfqpoint{4.976789in}{6.397919in}}%
\pgfpathlineto{\pgfqpoint{4.980436in}{6.395666in}}%
\pgfpathlineto{\pgfqpoint{4.981652in}{6.395063in}}%
\pgfpathlineto{\pgfqpoint{4.989553in}{6.389330in}}%
\pgfpathlineto{\pgfqpoint{4.993808in}{6.386755in}}%
\pgfpathlineto{\pgfqpoint{5.004142in}{6.378789in}}%
\pgfpathlineto{\pgfqpoint{5.005965in}{6.377522in}}%
\pgfpathlineto{\pgfqpoint{5.011436in}{6.372875in}}%
\pgfpathlineto{\pgfqpoint{5.030887in}{6.356038in}}%
\pgfpathlineto{\pgfqpoint{5.032711in}{6.355073in}}%
\pgfpathlineto{\pgfqpoint{5.035142in}{6.352780in}}%
\pgfpathlineto{\pgfqpoint{5.084985in}{6.309330in}}%
\pgfpathlineto{\pgfqpoint{5.086809in}{6.308163in}}%
\pgfpathlineto{\pgfqpoint{5.097142in}{6.300398in}}%
\pgfpathlineto{\pgfqpoint{5.108084in}{6.292835in}}%
\pgfpathlineto{\pgfqpoint{5.110515in}{6.291910in}}%
\pgfpathlineto{\pgfqpoint{5.123888in}{6.282777in}}%
\pgfpathlineto{\pgfqpoint{5.128143in}{6.280444in}}%
\pgfpathlineto{\pgfqpoint{5.150025in}{6.269279in}}%
\pgfpathlineto{\pgfqpoint{5.155496in}{6.267489in}}%
\pgfpathlineto{\pgfqpoint{5.159751in}{6.264653in}}%
\pgfpathlineto{\pgfqpoint{5.162182in}{6.263868in}}%
\pgfpathlineto{\pgfqpoint{5.165221in}{6.262339in}}%
\pgfpathlineto{\pgfqpoint{5.170084in}{6.260368in}}%
\pgfpathlineto{\pgfqpoint{5.176770in}{6.258477in}}%
\pgfpathlineto{\pgfqpoint{5.182849in}{6.257693in}}%
\pgfpathlineto{\pgfqpoint{5.185280in}{6.257331in}}%
\pgfpathlineto{\pgfqpoint{5.188927in}{6.256868in}}%
\pgfpathlineto{\pgfqpoint{5.204731in}{6.254293in}}%
\pgfpathlineto{\pgfqpoint{5.213241in}{6.254012in}}%
\pgfpathlineto{\pgfqpoint{5.216888in}{6.254253in}}%
\pgfpathlineto{\pgfqpoint{5.224790in}{6.254997in}}%
\pgfpathlineto{\pgfqpoint{5.230261in}{6.255219in}}%
\pgfpathlineto{\pgfqpoint{5.233300in}{6.255178in}}%
\pgfpathlineto{\pgfqpoint{5.246673in}{6.256948in}}%
\pgfpathlineto{\pgfqpoint{5.249712in}{6.257532in}}%
\pgfpathlineto{\pgfqpoint{5.253967in}{6.257813in}}%
\pgfpathlineto{\pgfqpoint{5.263085in}{6.258940in}}%
\pgfpathlineto{\pgfqpoint{5.266732in}{6.258799in}}%
\pgfpathlineto{\pgfqpoint{5.268555in}{6.259805in}}%
\pgfpathlineto{\pgfqpoint{5.272202in}{6.260207in}}%
\pgfpathlineto{\pgfqpoint{5.274634in}{6.260730in}}%
\pgfpathlineto{\pgfqpoint{5.275242in}{6.259704in}}%
\pgfpathlineto{\pgfqpoint{5.275850in}{6.260610in}}%
\pgfpathlineto{\pgfqpoint{5.277673in}{6.261213in}}%
\pgfpathlineto{\pgfqpoint{5.278281in}{6.260509in}}%
\pgfpathlineto{\pgfqpoint{5.278889in}{6.261193in}}%
\pgfpathlineto{\pgfqpoint{5.283751in}{6.262219in}}%
\pgfpathlineto{\pgfqpoint{5.284359in}{6.261615in}}%
\pgfpathlineto{\pgfqpoint{5.284967in}{6.262299in}}%
\pgfpathlineto{\pgfqpoint{5.289222in}{6.263064in}}%
\pgfpathlineto{\pgfqpoint{5.291046in}{6.263305in}}%
\pgfpathlineto{\pgfqpoint{5.294085in}{6.263748in}}%
\pgfpathlineto{\pgfqpoint{5.297732in}{6.264351in}}%
\pgfpathlineto{\pgfqpoint{5.300163in}{6.264733in}}%
\pgfpathlineto{\pgfqpoint{5.302595in}{6.265317in}}%
\pgfpathlineto{\pgfqpoint{5.304418in}{6.265397in}}%
\pgfpathlineto{\pgfqpoint{5.305026in}{6.264391in}}%
\pgfpathlineto{\pgfqpoint{5.305634in}{6.265256in}}%
\pgfpathlineto{\pgfqpoint{5.308065in}{6.265900in}}%
\pgfpathlineto{\pgfqpoint{5.310497in}{6.266624in}}%
\pgfpathlineto{\pgfqpoint{5.311105in}{6.265478in}}%
\pgfpathlineto{\pgfqpoint{5.312928in}{6.266845in}}%
\pgfpathlineto{\pgfqpoint{5.334203in}{6.270748in}}%
\pgfpathlineto{\pgfqpoint{5.337850in}{6.271713in}}%
\pgfpathlineto{\pgfqpoint{5.343928in}{6.272498in}}%
\pgfpathlineto{\pgfqpoint{5.345144in}{6.271834in}}%
\pgfpathlineto{\pgfqpoint{5.346360in}{6.273182in}}%
\pgfpathlineto{\pgfqpoint{5.348791in}{6.272458in}}%
\pgfpathlineto{\pgfqpoint{5.349399in}{6.273886in}}%
\pgfpathlineto{\pgfqpoint{5.350007in}{6.273383in}}%
\pgfpathlineto{\pgfqpoint{5.352438in}{6.273524in}}%
\pgfpathlineto{\pgfqpoint{5.354262in}{6.274489in}}%
\pgfpathlineto{\pgfqpoint{5.368850in}{6.276863in}}%
\pgfpathlineto{\pgfqpoint{5.370674in}{6.276984in}}%
\pgfpathlineto{\pgfqpoint{5.372497in}{6.276964in}}%
\pgfpathlineto{\pgfqpoint{5.374321in}{6.277527in}}%
\pgfpathlineto{\pgfqpoint{5.376144in}{6.277406in}}%
\pgfpathlineto{\pgfqpoint{5.382831in}{6.278794in}}%
\pgfpathlineto{\pgfqpoint{5.383438in}{6.278492in}}%
\pgfpathlineto{\pgfqpoint{5.384046in}{6.279579in}}%
\pgfpathlineto{\pgfqpoint{5.384654in}{6.279036in}}%
\pgfpathlineto{\pgfqpoint{5.385870in}{6.278875in}}%
\pgfpathlineto{\pgfqpoint{5.387693in}{6.280061in}}%
\pgfpathlineto{\pgfqpoint{5.391340in}{6.280303in}}%
\pgfpathlineto{\pgfqpoint{5.393164in}{6.280826in}}%
\pgfpathlineto{\pgfqpoint{5.393772in}{6.280061in}}%
\pgfpathlineto{\pgfqpoint{5.394380in}{6.280645in}}%
\pgfpathlineto{\pgfqpoint{5.396203in}{6.281107in}}%
\pgfpathlineto{\pgfqpoint{5.425380in}{6.283280in}}%
\pgfpathlineto{\pgfqpoint{5.427203in}{6.283039in}}%
\pgfpathlineto{\pgfqpoint{5.431458in}{6.283280in}}%
\pgfpathlineto{\pgfqpoint{5.434498in}{6.283260in}}%
\pgfpathlineto{\pgfqpoint{5.436929in}{6.283421in}}%
\pgfpathlineto{\pgfqpoint{5.438145in}{6.283039in}}%
\pgfpathlineto{\pgfqpoint{5.441792in}{6.282978in}}%
\pgfpathlineto{\pgfqpoint{5.444831in}{6.282596in}}%
\pgfpathlineto{\pgfqpoint{5.446655in}{6.282958in}}%
\pgfpathlineto{\pgfqpoint{5.455772in}{6.282777in}}%
\pgfpathlineto{\pgfqpoint{5.467929in}{6.282536in}}%
\pgfpathlineto{\pgfqpoint{5.469145in}{6.282314in}}%
\pgfpathlineto{\pgfqpoint{5.472184in}{6.282234in}}%
\pgfpathlineto{\pgfqpoint{5.476439in}{6.281751in}}%
\pgfpathlineto{\pgfqpoint{5.524459in}{6.278171in}}%
\pgfpathlineto{\pgfqpoint{5.525067in}{6.277285in}}%
\pgfpathlineto{\pgfqpoint{5.525675in}{6.278130in}}%
\pgfpathlineto{\pgfqpoint{5.528106in}{6.276682in}}%
\pgfpathlineto{\pgfqpoint{5.529322in}{6.277889in}}%
\pgfpathlineto{\pgfqpoint{5.531145in}{6.276400in}}%
\pgfpathlineto{\pgfqpoint{5.532969in}{6.277225in}}%
\pgfpathlineto{\pgfqpoint{5.534792in}{6.276863in}}%
\pgfpathlineto{\pgfqpoint{5.536008in}{6.276803in}}%
\pgfpathlineto{\pgfqpoint{5.541479in}{6.276119in}}%
\pgfpathlineto{\pgfqpoint{5.543302in}{6.275596in}}%
\pgfpathlineto{\pgfqpoint{5.545734in}{6.275173in}}%
\pgfpathlineto{\pgfqpoint{5.550596in}{6.274489in}}%
\pgfpathlineto{\pgfqpoint{5.552420in}{6.274228in}}%
\pgfpathlineto{\pgfqpoint{5.581597in}{6.271050in}}%
\pgfpathlineto{\pgfqpoint{5.582204in}{6.271552in}}%
\pgfpathlineto{\pgfqpoint{5.582812in}{6.270708in}}%
\pgfpathlineto{\pgfqpoint{5.590106in}{6.269843in}}%
\pgfpathlineto{\pgfqpoint{5.610773in}{6.267610in}}%
\pgfpathlineto{\pgfqpoint{5.611989in}{6.267992in}}%
\pgfpathlineto{\pgfqpoint{5.641165in}{6.266141in}}%
\pgfpathlineto{\pgfqpoint{5.642989in}{6.265498in}}%
\pgfpathlineto{\pgfqpoint{5.645420in}{6.265679in}}%
\pgfpathlineto{\pgfqpoint{5.648460in}{6.265457in}}%
\pgfpathlineto{\pgfqpoint{5.652107in}{6.264914in}}%
\pgfpathlineto{\pgfqpoint{5.660617in}{6.264874in}}%
\pgfpathlineto{\pgfqpoint{5.661832in}{6.263044in}}%
\pgfpathlineto{\pgfqpoint{5.662440in}{6.265015in}}%
\pgfpathlineto{\pgfqpoint{5.663048in}{6.264653in}}%
\pgfpathlineto{\pgfqpoint{5.663656in}{6.264995in}}%
\pgfpathlineto{\pgfqpoint{5.665479in}{6.263406in}}%
\pgfpathlineto{\pgfqpoint{5.669734in}{6.264713in}}%
\pgfpathlineto{\pgfqpoint{5.670950in}{6.263164in}}%
\pgfpathlineto{\pgfqpoint{5.684930in}{6.263084in}}%
\pgfpathlineto{\pgfqpoint{5.685538in}{6.262661in}}%
\pgfpathlineto{\pgfqpoint{5.686146in}{6.264090in}}%
\pgfpathlineto{\pgfqpoint{5.686754in}{6.263808in}}%
\pgfpathlineto{\pgfqpoint{5.690401in}{6.263426in}}%
\pgfpathlineto{\pgfqpoint{5.695264in}{6.264351in}}%
\pgfpathlineto{\pgfqpoint{5.697695in}{6.263687in}}%
\pgfpathlineto{\pgfqpoint{5.700127in}{6.264210in}}%
\pgfpathlineto{\pgfqpoint{5.704382in}{6.263949in}}%
\pgfpathlineto{\pgfqpoint{5.706813in}{6.264472in}}%
\pgfpathlineto{\pgfqpoint{5.711068in}{6.264532in}}%
\pgfpathlineto{\pgfqpoint{5.715323in}{6.264713in}}%
\pgfpathlineto{\pgfqpoint{5.715931in}{6.264351in}}%
\pgfpathlineto{\pgfqpoint{5.717146in}{6.265659in}}%
\pgfpathlineto{\pgfqpoint{5.718970in}{6.265236in}}%
\pgfpathlineto{\pgfqpoint{5.728695in}{6.266001in}}%
\pgfpathlineto{\pgfqpoint{5.729911in}{6.266001in}}%
\pgfpathlineto{\pgfqpoint{5.731735in}{6.265779in}}%
\pgfpathlineto{\pgfqpoint{5.733558in}{6.265558in}}%
\pgfpathlineto{\pgfqpoint{5.734774in}{6.266222in}}%
\pgfpathlineto{\pgfqpoint{5.743284in}{6.267087in}}%
\pgfpathlineto{\pgfqpoint{5.745715in}{6.266765in}}%
\pgfpathlineto{\pgfqpoint{5.747539in}{6.267429in}}%
\pgfpathlineto{\pgfqpoint{5.764558in}{6.268837in}}%
\pgfpathlineto{\pgfqpoint{5.765774in}{6.268917in}}%
\pgfpathlineto{\pgfqpoint{5.776715in}{6.270486in}}%
\pgfpathlineto{\pgfqpoint{5.779147in}{6.271029in}}%
\pgfpathlineto{\pgfqpoint{5.782794in}{6.270728in}}%
\pgfpathlineto{\pgfqpoint{5.786441in}{6.271754in}}%
\pgfpathlineto{\pgfqpoint{5.788265in}{6.271673in}}%
\pgfpathlineto{\pgfqpoint{5.789480in}{6.270687in}}%
\pgfpathlineto{\pgfqpoint{5.791304in}{6.272538in}}%
\pgfpathlineto{\pgfqpoint{5.800421in}{6.272196in}}%
\pgfpathlineto{\pgfqpoint{5.802245in}{6.272639in}}%
\pgfpathlineto{\pgfqpoint{5.805284in}{6.273162in}}%
\pgfpathlineto{\pgfqpoint{5.807716in}{6.273202in}}%
\pgfpathlineto{\pgfqpoint{5.809539in}{6.273846in}}%
\pgfpathlineto{\pgfqpoint{5.811971in}{6.273886in}}%
\pgfpathlineto{\pgfqpoint{5.842363in}{6.276803in}}%
\pgfpathlineto{\pgfqpoint{5.850265in}{6.277587in}}%
\pgfpathlineto{\pgfqpoint{5.852088in}{6.277426in}}%
\pgfpathlineto{\pgfqpoint{5.853912in}{6.278090in}}%
\pgfpathlineto{\pgfqpoint{5.862422in}{6.278352in}}%
\pgfpathlineto{\pgfqpoint{5.865461in}{6.278673in}}%
\pgfpathlineto{\pgfqpoint{5.908010in}{6.280645in}}%
\pgfpathlineto{\pgfqpoint{5.909834in}{6.280424in}}%
\pgfpathlineto{\pgfqpoint{5.912265in}{6.280363in}}%
\pgfpathlineto{\pgfqpoint{5.914697in}{6.280665in}}%
\pgfpathlineto{\pgfqpoint{5.927461in}{6.280484in}}%
\pgfpathlineto{\pgfqpoint{5.929285in}{6.279981in}}%
\pgfpathlineto{\pgfqpoint{5.934148in}{6.280323in}}%
\pgfpathlineto{\pgfqpoint{5.939011in}{6.280001in}}%
\pgfpathlineto{\pgfqpoint{5.940834in}{6.279820in}}%
\pgfpathlineto{\pgfqpoint{5.978521in}{6.277708in}}%
\pgfpathlineto{\pgfqpoint{5.979736in}{6.277688in}}%
\pgfpathlineto{\pgfqpoint{5.981560in}{6.277708in}}%
\pgfpathlineto{\pgfqpoint{5.985207in}{6.277406in}}%
\pgfpathlineto{\pgfqpoint{5.986423in}{6.277386in}}%
\pgfpathlineto{\pgfqpoint{5.988854in}{6.277145in}}%
\pgfpathlineto{\pgfqpoint{5.990678in}{6.276521in}}%
\pgfpathlineto{\pgfqpoint{6.005874in}{6.275455in}}%
\pgfpathlineto{\pgfqpoint{6.008913in}{6.275254in}}%
\pgfpathlineto{\pgfqpoint{6.013168in}{6.274590in}}%
\pgfpathlineto{\pgfqpoint{6.014991in}{6.274892in}}%
\pgfpathlineto{\pgfqpoint{6.018639in}{6.274127in}}%
\pgfpathlineto{\pgfqpoint{6.023501in}{6.273705in}}%
\pgfpathlineto{\pgfqpoint{6.028364in}{6.272920in}}%
\pgfpathlineto{\pgfqpoint{6.030188in}{6.273162in}}%
\pgfpathlineto{\pgfqpoint{6.035050in}{6.272940in}}%
\pgfpathlineto{\pgfqpoint{6.037482in}{6.272518in}}%
\pgfpathlineto{\pgfqpoint{6.039913in}{6.272176in}}%
\pgfpathlineto{\pgfqpoint{6.043560in}{6.271734in}}%
\pgfpathlineto{\pgfqpoint{6.046600in}{6.271713in}}%
\pgfpathlineto{\pgfqpoint{6.048423in}{6.271070in}}%
\pgfpathlineto{\pgfqpoint{6.055109in}{6.270667in}}%
\pgfpathlineto{\pgfqpoint{6.057541in}{6.270506in}}%
\pgfpathlineto{\pgfqpoint{6.058756in}{6.270446in}}%
\pgfpathlineto{\pgfqpoint{6.059972in}{6.269883in}}%
\pgfpathlineto{\pgfqpoint{6.061188in}{6.270728in}}%
\pgfpathlineto{\pgfqpoint{6.062403in}{6.269923in}}%
\pgfpathlineto{\pgfqpoint{6.067266in}{6.269782in}}%
\pgfpathlineto{\pgfqpoint{6.069090in}{6.269179in}}%
\pgfpathlineto{\pgfqpoint{6.070306in}{6.270748in}}%
\pgfpathlineto{\pgfqpoint{6.072737in}{6.270024in}}%
\pgfpathlineto{\pgfqpoint{6.076992in}{6.269742in}}%
\pgfpathlineto{\pgfqpoint{6.078208in}{6.269742in}}%
\pgfpathlineto{\pgfqpoint{6.080031in}{6.270144in}}%
\pgfpathlineto{\pgfqpoint{6.081855in}{6.269501in}}%
\pgfpathlineto{\pgfqpoint{6.094619in}{6.268455in}}%
\pgfpathlineto{\pgfqpoint{6.099482in}{6.267811in}}%
\pgfpathlineto{\pgfqpoint{6.100698in}{6.267932in}}%
\pgfpathlineto{\pgfqpoint{6.105561in}{6.267187in}}%
\pgfpathlineto{\pgfqpoint{6.109816in}{6.266524in}}%
\pgfpathlineto{\pgfqpoint{6.111639in}{6.267811in}}%
\pgfpathlineto{\pgfqpoint{6.112855in}{6.266946in}}%
\pgfpathlineto{\pgfqpoint{6.114678in}{6.267248in}}%
\pgfpathlineto{\pgfqpoint{6.117110in}{6.266765in}}%
\pgfpathlineto{\pgfqpoint{6.121365in}{6.266745in}}%
\pgfpathlineto{\pgfqpoint{6.122580in}{6.266966in}}%
\pgfpathlineto{\pgfqpoint{6.125012in}{6.265900in}}%
\pgfpathlineto{\pgfqpoint{6.126228in}{6.266966in}}%
\pgfpathlineto{\pgfqpoint{6.127443in}{6.265578in}}%
\pgfpathlineto{\pgfqpoint{6.129267in}{6.266845in}}%
\pgfpathlineto{\pgfqpoint{6.130482in}{6.265538in}}%
\pgfpathlineto{\pgfqpoint{6.132306in}{6.266463in}}%
\pgfpathlineto{\pgfqpoint{6.133522in}{6.265598in}}%
\pgfpathlineto{\pgfqpoint{6.134130in}{6.266524in}}%
\pgfpathlineto{\pgfqpoint{6.134737in}{6.265960in}}%
\pgfpathlineto{\pgfqpoint{6.136561in}{6.265679in}}%
\pgfpathlineto{\pgfqpoint{6.138992in}{6.265860in}}%
\pgfpathlineto{\pgfqpoint{6.148110in}{6.265618in}}%
\pgfpathlineto{\pgfqpoint{6.149933in}{6.265397in}}%
\pgfpathlineto{\pgfqpoint{6.151757in}{6.265618in}}%
\pgfpathlineto{\pgfqpoint{6.152365in}{6.266564in}}%
\pgfpathlineto{\pgfqpoint{6.153581in}{6.265437in}}%
\pgfpathlineto{\pgfqpoint{6.154796in}{6.265377in}}%
\pgfpathlineto{\pgfqpoint{6.155404in}{6.266544in}}%
\pgfpathlineto{\pgfqpoint{6.156012in}{6.266001in}}%
\pgfpathlineto{\pgfqpoint{6.157228in}{6.265478in}}%
\pgfpathlineto{\pgfqpoint{6.190659in}{6.266242in}}%
\pgfpathlineto{\pgfqpoint{6.191875in}{6.266383in}}%
\pgfpathlineto{\pgfqpoint{6.194306in}{6.266001in}}%
\pgfpathlineto{\pgfqpoint{6.198561in}{6.266785in}}%
\pgfpathlineto{\pgfqpoint{6.200993in}{6.266343in}}%
\pgfpathlineto{\pgfqpoint{6.202816in}{6.266463in}}%
\pgfpathlineto{\pgfqpoint{6.206463in}{6.266302in}}%
\pgfpathlineto{\pgfqpoint{6.210110in}{6.266222in}}%
\pgfpathlineto{\pgfqpoint{6.211326in}{6.266141in}}%
\pgfpathlineto{\pgfqpoint{6.211934in}{6.267127in}}%
\pgfpathlineto{\pgfqpoint{6.212542in}{6.266604in}}%
\pgfpathlineto{\pgfqpoint{6.214973in}{6.266242in}}%
\pgfpathlineto{\pgfqpoint{6.217405in}{6.266584in}}%
\pgfpathlineto{\pgfqpoint{6.224699in}{6.266825in}}%
\pgfpathlineto{\pgfqpoint{6.227130in}{6.267006in}}%
\pgfpathlineto{\pgfqpoint{6.228346in}{6.267228in}}%
\pgfpathlineto{\pgfqpoint{6.233209in}{6.267308in}}%
\pgfpathlineto{\pgfqpoint{6.238071in}{6.267368in}}%
\pgfpathlineto{\pgfqpoint{6.239895in}{6.267348in}}%
\pgfpathlineto{\pgfqpoint{6.240503in}{6.267228in}}%
\pgfpathlineto{\pgfqpoint{6.241718in}{6.268435in}}%
\pgfpathlineto{\pgfqpoint{6.243542in}{6.267932in}}%
\pgfpathlineto{\pgfqpoint{6.247797in}{6.268797in}}%
\pgfpathlineto{\pgfqpoint{6.250836in}{6.268334in}}%
\pgfpathlineto{\pgfqpoint{6.252660in}{6.269139in}}%
\pgfpathlineto{\pgfqpoint{6.255699in}{6.269460in}}%
\pgfpathlineto{\pgfqpoint{6.257522in}{6.269199in}}%
\pgfpathlineto{\pgfqpoint{6.258738in}{6.269135in}}%
\pgfpathlineto{\pgfqpoint{6.284268in}{6.365328in}}%
\pgfpathlineto{\pgfqpoint{6.290346in}{6.387935in}}%
\pgfpathlineto{\pgfqpoint{6.297640in}{6.415626in}}%
\pgfpathlineto{\pgfqpoint{6.300680in}{6.427419in}}%
\pgfpathlineto{\pgfqpoint{6.314660in}{6.477074in}}%
\pgfpathlineto{\pgfqpoint{6.328641in}{6.528547in}}%
\pgfpathlineto{\pgfqpoint{6.366327in}{6.667288in}}%
\pgfpathlineto{\pgfqpoint{6.397327in}{6.777056in}}%
\pgfpathlineto{\pgfqpoint{6.411916in}{6.821495in}}%
\pgfpathlineto{\pgfqpoint{6.415563in}{6.831348in}}%
\pgfpathlineto{\pgfqpoint{6.422249in}{6.847699in}}%
\pgfpathlineto{\pgfqpoint{6.444131in}{6.890295in}}%
\pgfpathlineto{\pgfqpoint{6.446563in}{6.893022in}}%
\pgfpathlineto{\pgfqpoint{6.459936in}{6.907180in}}%
\pgfpathlineto{\pgfqpoint{6.464798in}{6.909740in}}%
\pgfpathlineto{\pgfqpoint{6.469053in}{6.909297in}}%
\pgfpathlineto{\pgfqpoint{6.470877in}{6.908050in}}%
\pgfpathlineto{\pgfqpoint{6.472092in}{6.908694in}}%
\pgfpathlineto{\pgfqpoint{6.482426in}{6.894121in}}%
\pgfpathlineto{\pgfqpoint{6.484857in}{6.889995in}}%
\pgfpathlineto{\pgfqpoint{6.487289in}{6.886124in}}%
\pgfpathlineto{\pgfqpoint{6.498838in}{6.866993in}}%
\pgfpathlineto{\pgfqpoint{6.501269in}{6.861761in}}%
\pgfpathlineto{\pgfqpoint{6.508563in}{6.848915in}}%
\pgfpathlineto{\pgfqpoint{6.517681in}{6.829107in}}%
\pgfpathlineto{\pgfqpoint{6.533485in}{6.801485in}}%
\pgfpathlineto{\pgfqpoint{6.535916in}{6.798388in}}%
\pgfpathlineto{\pgfqpoint{6.540779in}{6.792366in}}%
\pgfpathlineto{\pgfqpoint{6.546858in}{6.786294in}}%
\pgfpathlineto{\pgfqpoint{6.549289in}{6.784143in}}%
\pgfpathlineto{\pgfqpoint{6.554152in}{6.779907in}}%
\pgfpathlineto{\pgfqpoint{6.555975in}{6.778313in}}%
\pgfpathlineto{\pgfqpoint{6.557799in}{6.778590in}}%
\pgfpathlineto{\pgfqpoint{6.561446in}{6.775638in}}%
\pgfpathlineto{\pgfqpoint{6.563270in}{6.774897in}}%
\pgfpathlineto{\pgfqpoint{6.566917in}{6.774141in}}%
\pgfpathlineto{\pgfqpoint{6.569348in}{6.772728in}}%
\pgfpathlineto{\pgfqpoint{6.570564in}{6.772723in}}%
\pgfpathlineto{\pgfqpoint{6.573603in}{6.771510in}}%
\pgfpathlineto{\pgfqpoint{6.575426in}{6.771053in}}%
\pgfpathlineto{\pgfqpoint{6.577858in}{6.769869in}}%
\pgfpathlineto{\pgfqpoint{6.580897in}{6.768925in}}%
\pgfpathlineto{\pgfqpoint{6.583936in}{6.768161in}}%
\pgfpathlineto{\pgfqpoint{6.599133in}{6.766390in}}%
\pgfpathlineto{\pgfqpoint{6.601564in}{6.766627in}}%
\pgfpathlineto{\pgfqpoint{6.603995in}{6.766083in}}%
\pgfpathlineto{\pgfqpoint{6.607642in}{6.766275in}}%
\pgfpathlineto{\pgfqpoint{6.612505in}{6.765822in}}%
\pgfpathlineto{\pgfqpoint{6.614329in}{6.766250in}}%
\pgfpathlineto{\pgfqpoint{6.617368in}{6.765630in}}%
\pgfpathlineto{\pgfqpoint{6.619799in}{6.765968in}}%
\pgfpathlineto{\pgfqpoint{6.625270in}{6.766325in}}%
\pgfpathlineto{\pgfqpoint{6.627701in}{6.767032in}}%
\pgfpathlineto{\pgfqpoint{6.629525in}{6.766499in}}%
\pgfpathlineto{\pgfqpoint{6.631348in}{6.767172in}}%
\pgfpathlineto{\pgfqpoint{6.635603in}{6.767778in}}%
\pgfpathlineto{\pgfqpoint{6.637427in}{6.768298in}}%
\pgfpathlineto{\pgfqpoint{6.640466in}{6.768560in}}%
\pgfpathlineto{\pgfqpoint{6.642290in}{6.768633in}}%
\pgfpathlineto{\pgfqpoint{6.645937in}{6.770037in}}%
\pgfpathlineto{\pgfqpoint{6.648368in}{6.770259in}}%
\pgfpathlineto{\pgfqpoint{6.650192in}{6.771031in}}%
\pgfpathlineto{\pgfqpoint{6.669035in}{6.775261in}}%
\pgfpathlineto{\pgfqpoint{6.672074in}{6.776333in}}%
\pgfpathlineto{\pgfqpoint{6.673898in}{6.777291in}}%
\pgfpathlineto{\pgfqpoint{6.676329in}{6.776604in}}%
\pgfpathlineto{\pgfqpoint{6.686663in}{6.778680in}}%
\pgfpathlineto{\pgfqpoint{6.688486in}{6.778416in}}%
\pgfpathlineto{\pgfqpoint{6.691525in}{6.778583in}}%
\pgfpathlineto{\pgfqpoint{6.693957in}{6.778712in}}%
\pgfpathlineto{\pgfqpoint{6.696996in}{6.778494in}}%
\pgfpathlineto{\pgfqpoint{6.701251in}{6.778788in}}%
\pgfpathlineto{\pgfqpoint{6.703682in}{6.779107in}}%
\pgfpathlineto{\pgfqpoint{6.708545in}{6.778216in}}%
\pgfpathlineto{\pgfqpoint{6.710976in}{6.778271in}}%
\pgfpathlineto{\pgfqpoint{6.716447in}{6.778149in}}%
\pgfpathlineto{\pgfqpoint{6.720702in}{6.777353in}}%
\pgfpathlineto{\pgfqpoint{6.731035in}{6.775967in}}%
\pgfpathlineto{\pgfqpoint{6.734682in}{6.776017in}}%
\pgfpathlineto{\pgfqpoint{6.737722in}{6.775326in}}%
\pgfpathlineto{\pgfqpoint{6.740153in}{6.775331in}}%
\pgfpathlineto{\pgfqpoint{6.741369in}{6.775118in}}%
\pgfpathlineto{\pgfqpoint{6.745016in}{6.774739in}}%
\pgfpathlineto{\pgfqpoint{6.746839in}{6.774197in}}%
\pgfpathlineto{\pgfqpoint{6.750486in}{6.772986in}}%
\pgfpathlineto{\pgfqpoint{6.752310in}{6.772839in}}%
\pgfpathlineto{\pgfqpoint{6.754741in}{6.772353in}}%
\pgfpathlineto{\pgfqpoint{6.757173in}{6.771913in}}%
\pgfpathlineto{\pgfqpoint{6.761428in}{6.771794in}}%
\pgfpathlineto{\pgfqpoint{6.765075in}{6.770997in}}%
\pgfpathlineto{\pgfqpoint{6.767506in}{6.770625in}}%
\pgfpathlineto{\pgfqpoint{6.769330in}{6.770421in}}%
\pgfpathlineto{\pgfqpoint{6.801546in}{6.762545in}}%
\pgfpathlineto{\pgfqpoint{6.803369in}{6.761877in}}%
\pgfpathlineto{\pgfqpoint{6.805801in}{6.761812in}}%
\pgfpathlineto{\pgfqpoint{6.819173in}{6.758450in}}%
\pgfpathlineto{\pgfqpoint{6.821604in}{6.757551in}}%
\pgfpathlineto{\pgfqpoint{6.834977in}{6.755300in}}%
\pgfpathlineto{\pgfqpoint{6.837409in}{6.755065in}}%
\pgfpathlineto{\pgfqpoint{6.839840in}{6.753696in}}%
\pgfpathlineto{\pgfqpoint{6.841664in}{6.753504in}}%
\pgfpathlineto{\pgfqpoint{6.851389in}{6.750857in}}%
\pgfpathlineto{\pgfqpoint{6.853820in}{6.750507in}}%
\pgfpathlineto{\pgfqpoint{6.856252in}{6.749719in}}%
\pgfpathlineto{\pgfqpoint{6.859899in}{6.749363in}}%
\pgfpathlineto{\pgfqpoint{6.862938in}{6.749029in}}%
\pgfpathlineto{\pgfqpoint{6.896978in}{6.744253in}}%
\pgfpathlineto{\pgfqpoint{6.945605in}{6.740608in}}%
\pgfpathlineto{\pgfqpoint{6.955331in}{6.742196in}}%
\pgfpathlineto{\pgfqpoint{6.956547in}{6.742097in}}%
\pgfpathlineto{\pgfqpoint{6.960194in}{6.743086in}}%
\pgfpathlineto{\pgfqpoint{6.986939in}{6.741237in}}%
\pgfpathlineto{\pgfqpoint{6.988762in}{6.741328in}}%
\pgfpathlineto{\pgfqpoint{6.996664in}{6.741108in}}%
\pgfpathlineto{\pgfqpoint{7.000312in}{6.740666in}}%
\pgfpathlineto{\pgfqpoint{7.003351in}{6.740456in}}%
\pgfpathlineto{\pgfqpoint{7.006998in}{6.740317in}}%
\pgfpathlineto{\pgfqpoint{7.016116in}{6.739459in}}%
\pgfpathlineto{\pgfqpoint{7.017939in}{6.739204in}}%
\pgfpathlineto{\pgfqpoint{7.020978in}{6.739471in}}%
\pgfpathlineto{\pgfqpoint{7.025841in}{6.740126in}}%
\pgfpathlineto{\pgfqpoint{7.030704in}{6.740537in}}%
\pgfpathlineto{\pgfqpoint{7.032527in}{6.740471in}}%
\pgfpathlineto{\pgfqpoint{7.036782in}{6.740311in}}%
\pgfpathlineto{\pgfqpoint{7.039214in}{6.740287in}}%
\pgfpathlineto{\pgfqpoint{7.058057in}{6.739983in}}%
\pgfpathlineto{\pgfqpoint{7.076292in}{6.739703in}}%
\pgfpathlineto{\pgfqpoint{7.078116in}{6.739620in}}%
\pgfpathlineto{\pgfqpoint{7.079940in}{6.739624in}}%
\pgfpathlineto{\pgfqpoint{7.092704in}{6.739669in}}%
\pgfpathlineto{\pgfqpoint{7.096351in}{6.740044in}}%
\pgfpathlineto{\pgfqpoint{7.098175in}{6.740113in}}%
\pgfpathlineto{\pgfqpoint{7.101214in}{6.740067in}}%
\pgfpathlineto{\pgfqpoint{7.135254in}{6.742200in}}%
\pgfpathlineto{\pgfqpoint{7.137077in}{6.742781in}}%
\pgfpathlineto{\pgfqpoint{7.143764in}{6.742938in}}%
\pgfpathlineto{\pgfqpoint{7.144979in}{6.742928in}}%
\pgfpathlineto{\pgfqpoint{7.148018in}{6.743439in}}%
\pgfpathlineto{\pgfqpoint{7.149234in}{6.743616in}}%
\pgfpathlineto{\pgfqpoint{7.151666in}{6.742981in}}%
\pgfpathlineto{\pgfqpoint{7.153489in}{6.743503in}}%
\pgfpathlineto{\pgfqpoint{7.156528in}{6.743483in}}%
\pgfpathlineto{\pgfqpoint{7.159567in}{6.743638in}}%
\pgfpathlineto{\pgfqpoint{7.163215in}{6.743778in}}%
\pgfpathlineto{\pgfqpoint{7.175372in}{6.743464in}}%
\pgfpathlineto{\pgfqpoint{7.191783in}{6.743132in}}%
\pgfpathlineto{\pgfqpoint{7.194823in}{6.743228in}}%
\pgfpathlineto{\pgfqpoint{7.206980in}{6.741894in}}%
\pgfpathlineto{\pgfqpoint{7.207587in}{6.742556in}}%
\pgfpathlineto{\pgfqpoint{7.208195in}{6.741779in}}%
\pgfpathlineto{\pgfqpoint{7.223392in}{6.740142in}}%
\pgfpathlineto{\pgfqpoint{7.225823in}{6.740532in}}%
\pgfpathlineto{\pgfqpoint{7.234941in}{6.739449in}}%
\pgfpathlineto{\pgfqpoint{7.235548in}{6.740224in}}%
\pgfpathlineto{\pgfqpoint{7.236156in}{6.739065in}}%
\pgfpathlineto{\pgfqpoint{7.236764in}{6.739695in}}%
\pgfpathlineto{\pgfqpoint{7.254392in}{6.739126in}}%
\pgfpathlineto{\pgfqpoint{7.255607in}{6.739338in}}%
\pgfpathlineto{\pgfqpoint{7.258039in}{6.738881in}}%
\pgfpathlineto{\pgfqpoint{7.260470in}{6.738873in}}%
\pgfpathlineto{\pgfqpoint{7.264725in}{6.738704in}}%
\pgfpathlineto{\pgfqpoint{7.267157in}{6.738507in}}%
\pgfpathlineto{\pgfqpoint{7.268980in}{6.738513in}}%
\pgfpathlineto{\pgfqpoint{7.272627in}{6.737944in}}%
\pgfpathlineto{\pgfqpoint{7.274451in}{6.737949in}}%
\pgfpathlineto{\pgfqpoint{7.278098in}{6.738128in}}%
\pgfpathlineto{\pgfqpoint{7.288431in}{6.737245in}}%
\pgfpathlineto{\pgfqpoint{7.291470in}{6.736884in}}%
\pgfpathlineto{\pgfqpoint{7.293902in}{6.737321in}}%
\pgfpathlineto{\pgfqpoint{7.302412in}{6.736862in}}%
\pgfpathlineto{\pgfqpoint{7.308490in}{6.736785in}}%
\pgfpathlineto{\pgfqpoint{7.312137in}{6.737107in}}%
\pgfpathlineto{\pgfqpoint{7.343745in}{6.735113in}}%
\pgfpathlineto{\pgfqpoint{7.346177in}{6.735441in}}%
\pgfpathlineto{\pgfqpoint{7.348608in}{6.735803in}}%
\pgfpathlineto{\pgfqpoint{7.349216in}{6.736023in}}%
\pgfpathlineto{\pgfqpoint{7.351039in}{6.734798in}}%
\pgfpathlineto{\pgfqpoint{7.359549in}{6.734855in}}%
\pgfpathlineto{\pgfqpoint{7.363804in}{6.734500in}}%
\pgfpathlineto{\pgfqpoint{7.382040in}{6.734582in}}%
\pgfpathlineto{\pgfqpoint{7.383255in}{6.734542in}}%
\pgfpathlineto{\pgfqpoint{7.384471in}{6.734968in}}%
\pgfpathlineto{\pgfqpoint{7.386902in}{6.734429in}}%
\pgfpathlineto{\pgfqpoint{7.388118in}{6.734531in}}%
\pgfpathlineto{\pgfqpoint{7.389942in}{6.734085in}}%
\pgfpathlineto{\pgfqpoint{7.403922in}{6.734277in}}%
\pgfpathlineto{\pgfqpoint{7.405746in}{6.735031in}}%
\pgfpathlineto{\pgfqpoint{7.425197in}{6.735486in}}%
\pgfpathlineto{\pgfqpoint{7.427020in}{6.735570in}}%
\pgfpathlineto{\pgfqpoint{7.431275in}{6.735545in}}%
\pgfpathlineto{\pgfqpoint{7.434314in}{6.735387in}}%
\pgfpathlineto{\pgfqpoint{7.436138in}{6.735782in}}%
\pgfpathlineto{\pgfqpoint{7.437354in}{6.734870in}}%
\pgfpathlineto{\pgfqpoint{7.437962in}{6.735586in}}%
\pgfpathlineto{\pgfqpoint{7.445256in}{6.735384in}}%
\pgfpathlineto{\pgfqpoint{7.447687in}{6.736184in}}%
\pgfpathlineto{\pgfqpoint{7.450118in}{6.735950in}}%
\pgfpathlineto{\pgfqpoint{7.464099in}{6.736843in}}%
\pgfpathlineto{\pgfqpoint{7.467138in}{6.736676in}}%
\pgfpathlineto{\pgfqpoint{7.468962in}{6.736893in}}%
\pgfpathlineto{\pgfqpoint{7.475648in}{6.737415in}}%
\pgfpathlineto{\pgfqpoint{7.479903in}{6.737716in}}%
\pgfpathlineto{\pgfqpoint{7.485374in}{6.738160in}}%
\pgfpathlineto{\pgfqpoint{7.492668in}{6.738731in}}%
\pgfpathlineto{\pgfqpoint{7.501785in}{6.738627in}}%
\pgfpathlineto{\pgfqpoint{7.503609in}{6.738523in}}%
\pgfpathlineto{\pgfqpoint{7.509687in}{6.737618in}}%
\pgfpathlineto{\pgfqpoint{7.512727in}{6.739381in}}%
\pgfpathlineto{\pgfqpoint{7.521844in}{6.738747in}}%
\pgfpathlineto{\pgfqpoint{7.526099in}{6.738956in}}%
\pgfpathlineto{\pgfqpoint{7.529746in}{6.739160in}}%
\pgfpathlineto{\pgfqpoint{7.532178in}{6.739115in}}%
\pgfpathlineto{\pgfqpoint{7.535217in}{6.739461in}}%
\pgfpathlineto{\pgfqpoint{7.537041in}{6.739245in}}%
\pgfpathlineto{\pgfqpoint{7.578374in}{6.739304in}}%
\pgfpathlineto{\pgfqpoint{7.580806in}{6.739101in}}%
\pgfpathlineto{\pgfqpoint{7.582629in}{6.738914in}}%
\pgfpathlineto{\pgfqpoint{7.603296in}{6.738574in}}%
\pgfpathlineto{\pgfqpoint{7.605727in}{6.738386in}}%
\pgfpathlineto{\pgfqpoint{7.611198in}{6.738392in}}%
\pgfpathlineto{\pgfqpoint{7.614237in}{6.737556in}}%
\pgfpathlineto{\pgfqpoint{7.622139in}{6.737729in}}%
\pgfpathlineto{\pgfqpoint{7.623355in}{6.736834in}}%
\pgfpathlineto{\pgfqpoint{7.630041in}{6.737807in}}%
\pgfpathlineto{\pgfqpoint{7.634296in}{6.736805in}}%
\pgfpathlineto{\pgfqpoint{7.636120in}{6.736077in}}%
\pgfpathlineto{\pgfqpoint{7.642806in}{6.735891in}}%
\pgfpathlineto{\pgfqpoint{7.645237in}{6.735533in}}%
\pgfpathlineto{\pgfqpoint{7.648277in}{6.735874in}}%
\pgfpathlineto{\pgfqpoint{7.660433in}{6.734415in}}%
\pgfpathlineto{\pgfqpoint{7.662257in}{6.734617in}}%
\pgfpathlineto{\pgfqpoint{7.682316in}{6.733989in}}%
\pgfpathlineto{\pgfqpoint{7.683532in}{6.733681in}}%
\pgfpathlineto{\pgfqpoint{7.685355in}{6.734486in}}%
\pgfpathlineto{\pgfqpoint{7.686571in}{6.733260in}}%
\pgfpathlineto{\pgfqpoint{7.688395in}{6.733965in}}%
\pgfpathlineto{\pgfqpoint{7.690218in}{6.733620in}}%
\pgfpathlineto{\pgfqpoint{7.691434in}{6.734236in}}%
\pgfpathlineto{\pgfqpoint{7.694473in}{6.733506in}}%
\pgfpathlineto{\pgfqpoint{7.695689in}{6.733652in}}%
\pgfpathlineto{\pgfqpoint{7.702983in}{6.733371in}}%
\pgfpathlineto{\pgfqpoint{7.704806in}{6.733685in}}%
\pgfpathlineto{\pgfqpoint{7.708453in}{6.733316in}}%
\pgfpathlineto{\pgfqpoint{7.710885in}{6.733646in}}%
\pgfpathlineto{\pgfqpoint{7.714532in}{6.733306in}}%
\pgfpathlineto{\pgfqpoint{7.745532in}{6.732947in}}%
\pgfpathlineto{\pgfqpoint{7.757081in}{6.733437in}}%
\pgfpathlineto{\pgfqpoint{7.760728in}{6.733978in}}%
\pgfpathlineto{\pgfqpoint{7.763160in}{6.733837in}}%
\pgfpathlineto{\pgfqpoint{7.768630in}{6.734113in}}%
\pgfpathlineto{\pgfqpoint{7.837925in}{6.738108in}}%
\pgfpathlineto{\pgfqpoint{7.840356in}{6.738335in}}%
\pgfpathlineto{\pgfqpoint{7.846435in}{6.738603in}}%
\pgfpathlineto{\pgfqpoint{7.849474in}{6.739166in}}%
\pgfpathlineto{\pgfqpoint{7.851297in}{6.739660in}}%
\pgfpathlineto{\pgfqpoint{7.853121in}{6.739165in}}%
\pgfpathlineto{\pgfqpoint{7.855552in}{6.739931in}}%
\pgfpathlineto{\pgfqpoint{7.859200in}{6.739675in}}%
\pgfpathlineto{\pgfqpoint{7.873788in}{6.740226in}}%
\pgfpathlineto{\pgfqpoint{7.875611in}{6.740152in}}%
\pgfpathlineto{\pgfqpoint{7.884121in}{6.740749in}}%
\pgfpathlineto{\pgfqpoint{7.896278in}{6.740237in}}%
\pgfpathlineto{\pgfqpoint{7.922416in}{6.556295in}}%
\pgfpathlineto{\pgfqpoint{7.949769in}{6.394295in}}%
\pgfpathlineto{\pgfqpoint{7.962533in}{6.328756in}}%
\pgfpathlineto{\pgfqpoint{7.969220in}{6.291982in}}%
\pgfpathlineto{\pgfqpoint{7.983200in}{6.225125in}}%
\pgfpathlineto{\pgfqpoint{8.019063in}{6.072838in}}%
\pgfpathlineto{\pgfqpoint{8.082887in}{5.826962in}}%
\pgfpathlineto{\pgfqpoint{8.111456in}{5.737142in}}%
\pgfpathlineto{\pgfqpoint{8.130299in}{5.693966in}}%
\pgfpathlineto{\pgfqpoint{8.142456in}{5.675685in}}%
\pgfpathlineto{\pgfqpoint{8.149750in}{5.668478in}}%
\pgfpathlineto{\pgfqpoint{8.151574in}{5.667341in}}%
\pgfpathlineto{\pgfqpoint{8.158868in}{5.664628in}}%
\pgfpathlineto{\pgfqpoint{8.163123in}{5.664017in}}%
\pgfpathlineto{\pgfqpoint{8.165554in}{5.664101in}}%
\pgfpathlineto{\pgfqpoint{8.167986in}{5.664378in}}%
\pgfpathlineto{\pgfqpoint{8.173456in}{5.670727in}}%
\pgfpathlineto{\pgfqpoint{8.178319in}{5.675019in}}%
\pgfpathlineto{\pgfqpoint{8.180751in}{5.677413in}}%
\pgfpathlineto{\pgfqpoint{8.183790in}{5.681932in}}%
\pgfpathlineto{\pgfqpoint{8.208712in}{5.717794in}}%
\pgfpathlineto{\pgfqpoint{8.209927in}{5.718614in}}%
\pgfpathlineto{\pgfqpoint{8.223908in}{5.736592in}}%
\pgfpathlineto{\pgfqpoint{8.225731in}{5.740380in}}%
\pgfpathlineto{\pgfqpoint{8.228770in}{5.745114in}}%
\pgfpathlineto{\pgfqpoint{8.233025in}{5.750532in}}%
\pgfpathlineto{\pgfqpoint{8.254300in}{5.777804in}}%
\pgfpathlineto{\pgfqpoint{8.256124in}{5.779908in}}%
\pgfpathlineto{\pgfqpoint{8.258555in}{5.781728in}}%
\pgfpathlineto{\pgfqpoint{8.259771in}{5.782860in}}%
\pgfpathlineto{\pgfqpoint{8.261594in}{5.784763in}}%
\pgfpathlineto{\pgfqpoint{8.267065in}{5.788917in}}%
\pgfpathlineto{\pgfqpoint{8.273751in}{5.795049in}}%
\pgfpathlineto{\pgfqpoint{8.275575in}{5.796409in}}%
\pgfpathlineto{\pgfqpoint{8.276183in}{5.796315in}}%
\pgfpathlineto{\pgfqpoint{8.279830in}{5.800857in}}%
\pgfpathlineto{\pgfqpoint{8.281653in}{5.801779in}}%
\pgfpathlineto{\pgfqpoint{8.284692in}{5.804216in}}%
\pgfpathlineto{\pgfqpoint{8.291379in}{5.809986in}}%
\pgfpathlineto{\pgfqpoint{8.301712in}{5.817161in}}%
\pgfpathlineto{\pgfqpoint{8.302928in}{5.817442in}}%
\pgfpathlineto{\pgfqpoint{8.306575in}{5.818780in}}%
\pgfpathlineto{\pgfqpoint{8.309006in}{5.819552in}}%
\pgfpathlineto{\pgfqpoint{8.312653in}{5.821345in}}%
\pgfpathlineto{\pgfqpoint{8.316300in}{5.823218in}}%
\pgfpathlineto{\pgfqpoint{8.318732in}{5.825270in}}%
\pgfpathlineto{\pgfqpoint{8.321163in}{5.825909in}}%
\pgfpathlineto{\pgfqpoint{8.323595in}{5.827695in}}%
\pgfpathlineto{\pgfqpoint{8.324810in}{5.828096in}}%
\pgfpathlineto{\pgfqpoint{8.326634in}{5.828227in}}%
\pgfpathlineto{\pgfqpoint{8.330889in}{5.830000in}}%
\pgfpathlineto{\pgfqpoint{8.332712in}{5.831208in}}%
\pgfpathlineto{\pgfqpoint{8.335144in}{5.831063in}}%
\pgfpathlineto{\pgfqpoint{8.338183in}{5.833870in}}%
\pgfpathlineto{\pgfqpoint{8.340614in}{5.833183in}}%
\pgfpathlineto{\pgfqpoint{8.345477in}{5.835728in}}%
\pgfpathlineto{\pgfqpoint{8.346693in}{5.835144in}}%
\pgfpathlineto{\pgfqpoint{8.348516in}{5.836167in}}%
\pgfpathlineto{\pgfqpoint{8.350948in}{5.836143in}}%
\pgfpathlineto{\pgfqpoint{8.354595in}{5.838651in}}%
\pgfpathlineto{\pgfqpoint{8.356418in}{5.838003in}}%
\pgfpathlineto{\pgfqpoint{8.358850in}{5.838877in}}%
\pgfpathlineto{\pgfqpoint{8.364320in}{5.839747in}}%
\pgfpathlineto{\pgfqpoint{8.366144in}{5.839080in}}%
\pgfpathlineto{\pgfqpoint{8.368575in}{5.839579in}}%
\pgfpathlineto{\pgfqpoint{8.369791in}{5.838470in}}%
\pgfpathlineto{\pgfqpoint{8.374046in}{5.838375in}}%
\pgfpathlineto{\pgfqpoint{8.374654in}{5.837329in}}%
\pgfpathlineto{\pgfqpoint{8.375262in}{5.838169in}}%
\pgfpathlineto{\pgfqpoint{8.376477in}{5.838133in}}%
\pgfpathlineto{\pgfqpoint{8.378909in}{5.837723in}}%
\pgfpathlineto{\pgfqpoint{8.389850in}{5.839336in}}%
\pgfpathlineto{\pgfqpoint{8.392281in}{5.840219in}}%
\pgfpathlineto{\pgfqpoint{8.412340in}{5.845385in}}%
\pgfpathlineto{\pgfqpoint{8.414772in}{5.844522in}}%
\pgfpathlineto{\pgfqpoint{8.417811in}{5.845027in}}%
\pgfpathlineto{\pgfqpoint{8.421458in}{5.844085in}}%
\pgfpathlineto{\pgfqpoint{8.422674in}{5.844928in}}%
\pgfpathlineto{\pgfqpoint{8.426321in}{5.843275in}}%
\pgfpathlineto{\pgfqpoint{8.431184in}{5.842441in}}%
\pgfpathlineto{\pgfqpoint{8.432399in}{5.842496in}}%
\pgfpathlineto{\pgfqpoint{8.434831in}{5.841959in}}%
\pgfpathlineto{\pgfqpoint{8.436046in}{5.843001in}}%
\pgfpathlineto{\pgfqpoint{8.437262in}{5.842807in}}%
\pgfpathlineto{\pgfqpoint{8.439086in}{5.843665in}}%
\pgfpathlineto{\pgfqpoint{8.448811in}{5.847128in}}%
\pgfpathlineto{\pgfqpoint{8.450027in}{5.847499in}}%
\pgfpathlineto{\pgfqpoint{8.453066in}{5.848590in}}%
\pgfpathlineto{\pgfqpoint{8.460968in}{5.851993in}}%
\pgfpathlineto{\pgfqpoint{8.462792in}{5.852855in}}%
\pgfpathlineto{\pgfqpoint{8.464615in}{5.853391in}}%
\pgfpathlineto{\pgfqpoint{8.470694in}{5.854719in}}%
\pgfpathlineto{\pgfqpoint{8.498047in}{5.860411in}}%
\pgfpathlineto{\pgfqpoint{8.503517in}{5.861884in}}%
\pgfpathlineto{\pgfqpoint{8.597126in}{5.876559in}}%
\pgfpathlineto{\pgfqpoint{8.600773in}{5.877006in}}%
\pgfpathlineto{\pgfqpoint{8.618401in}{5.877431in}}%
\pgfpathlineto{\pgfqpoint{8.625087in}{5.877269in}}%
\pgfpathlineto{\pgfqpoint{8.627518in}{5.877752in}}%
\pgfpathlineto{\pgfqpoint{8.651224in}{5.879231in}}%
\pgfpathlineto{\pgfqpoint{8.654263in}{5.878913in}}%
\pgfpathlineto{\pgfqpoint{8.660342in}{5.878671in}}%
\pgfpathlineto{\pgfqpoint{8.660950in}{5.878534in}}%
\pgfpathlineto{\pgfqpoint{8.664597in}{5.879161in}}%
\pgfpathlineto{\pgfqpoint{8.695597in}{5.880584in}}%
\pgfpathlineto{\pgfqpoint{8.698028in}{5.880694in}}%
\pgfpathlineto{\pgfqpoint{8.707754in}{5.881499in}}%
\pgfpathlineto{\pgfqpoint{8.713225in}{5.881476in}}%
\pgfpathlineto{\pgfqpoint{8.716264in}{5.881442in}}%
\pgfpathlineto{\pgfqpoint{8.739970in}{5.881577in}}%
\pgfpathlineto{\pgfqpoint{8.741793in}{5.881662in}}%
\pgfpathlineto{\pgfqpoint{8.765499in}{5.880662in}}%
\pgfpathlineto{\pgfqpoint{8.772186in}{5.880978in}}%
\pgfpathlineto{\pgfqpoint{8.777656in}{5.880926in}}%
\pgfpathlineto{\pgfqpoint{8.780088in}{5.880809in}}%
\pgfpathlineto{\pgfqpoint{8.781911in}{5.880922in}}%
\pgfpathlineto{\pgfqpoint{8.791029in}{5.880994in}}%
\pgfpathlineto{\pgfqpoint{8.794068in}{5.880845in}}%
\pgfpathlineto{\pgfqpoint{8.820206in}{5.881486in}}%
\pgfpathlineto{\pgfqpoint{8.822029in}{5.881433in}}%
\pgfpathlineto{\pgfqpoint{8.825068in}{5.881776in}}%
\pgfpathlineto{\pgfqpoint{8.827500in}{5.881306in}}%
\pgfpathlineto{\pgfqpoint{8.836010in}{5.882432in}}%
\pgfpathlineto{\pgfqpoint{8.839049in}{5.881966in}}%
\pgfpathlineto{\pgfqpoint{8.843304in}{5.882102in}}%
\pgfpathlineto{\pgfqpoint{8.848775in}{5.882323in}}%
\pgfpathlineto{\pgfqpoint{8.851206in}{5.882570in}}%
\pgfpathlineto{\pgfqpoint{8.853637in}{5.882892in}}%
\pgfpathlineto{\pgfqpoint{8.873088in}{5.883458in}}%
\pgfpathlineto{\pgfqpoint{8.875520in}{5.883432in}}%
\pgfpathlineto{\pgfqpoint{8.877951in}{5.883111in}}%
\pgfpathlineto{\pgfqpoint{8.889500in}{5.883422in}}%
\pgfpathlineto{\pgfqpoint{8.891324in}{5.883890in}}%
\pgfpathlineto{\pgfqpoint{8.893147in}{5.883130in}}%
\pgfpathlineto{\pgfqpoint{8.895579in}{5.883812in}}%
\pgfpathlineto{\pgfqpoint{8.900441in}{5.883906in}}%
\pgfpathlineto{\pgfqpoint{8.906520in}{5.883791in}}%
\pgfpathlineto{\pgfqpoint{8.908344in}{5.883449in}}%
\pgfpathlineto{\pgfqpoint{8.910167in}{5.883345in}}%
\pgfpathlineto{\pgfqpoint{8.918677in}{5.883890in}}%
\pgfpathlineto{\pgfqpoint{8.921716in}{5.883779in}}%
\pgfpathlineto{\pgfqpoint{8.930834in}{5.883653in}}%
\pgfpathlineto{\pgfqpoint{8.932657in}{5.883484in}}%
\pgfpathlineto{\pgfqpoint{8.935697in}{5.883323in}}%
\pgfpathlineto{\pgfqpoint{8.966697in}{5.882870in}}%
\pgfpathlineto{\pgfqpoint{8.969736in}{5.883650in}}%
\pgfpathlineto{\pgfqpoint{8.973383in}{5.882752in}}%
\pgfpathlineto{\pgfqpoint{8.977638in}{5.883301in}}%
\pgfpathlineto{\pgfqpoint{8.980069in}{5.882874in}}%
\pgfpathlineto{\pgfqpoint{8.989187in}{5.883909in}}%
\pgfpathlineto{\pgfqpoint{8.991011in}{5.883915in}}%
\pgfpathlineto{\pgfqpoint{8.994050in}{5.884084in}}%
\pgfpathlineto{\pgfqpoint{9.004991in}{5.884102in}}%
\pgfpathlineto{\pgfqpoint{9.008638in}{5.883992in}}%
\pgfpathlineto{\pgfqpoint{9.010462in}{5.884068in}}%
\pgfpathlineto{\pgfqpoint{9.014717in}{5.884823in}}%
\pgfpathlineto{\pgfqpoint{9.020795in}{5.884284in}}%
\pgfpathlineto{\pgfqpoint{9.022619in}{5.884775in}}%
\pgfpathlineto{\pgfqpoint{9.027482in}{5.885109in}}%
\pgfpathlineto{\pgfqpoint{9.030521in}{5.885249in}}%
\pgfpathlineto{\pgfqpoint{9.031736in}{5.885179in}}%
\pgfpathlineto{\pgfqpoint{9.034776in}{5.884764in}}%
\pgfpathlineto{\pgfqpoint{9.049364in}{5.885729in}}%
\pgfpathlineto{\pgfqpoint{9.052403in}{5.885244in}}%
\pgfpathlineto{\pgfqpoint{9.056050in}{5.885343in}}%
\pgfpathlineto{\pgfqpoint{9.063952in}{5.885348in}}%
\pgfpathlineto{\pgfqpoint{9.067599in}{5.886158in}}%
\pgfpathlineto{\pgfqpoint{9.071246in}{5.885494in}}%
\pgfpathlineto{\pgfqpoint{9.074894in}{5.886425in}}%
\pgfpathlineto{\pgfqpoint{9.077325in}{5.886538in}}%
\pgfpathlineto{\pgfqpoint{9.083404in}{5.886538in}}%
\pgfpathlineto{\pgfqpoint{9.086443in}{5.887656in}}%
\pgfpathlineto{\pgfqpoint{9.091306in}{5.885615in}}%
\pgfpathlineto{\pgfqpoint{9.093129in}{5.885666in}}%
\pgfpathlineto{\pgfqpoint{9.106502in}{5.886251in}}%
\pgfpathlineto{\pgfqpoint{9.109541in}{5.885746in}}%
\pgfpathlineto{\pgfqpoint{9.113796in}{5.886160in}}%
\pgfpathlineto{\pgfqpoint{9.116227in}{5.885753in}}%
\pgfpathlineto{\pgfqpoint{9.136286in}{5.885245in}}%
\pgfpathlineto{\pgfqpoint{9.137502in}{5.885476in}}%
\pgfpathlineto{\pgfqpoint{9.139933in}{5.885078in}}%
\pgfpathlineto{\pgfqpoint{9.142365in}{5.885022in}}%
\pgfpathlineto{\pgfqpoint{9.152698in}{5.884805in}}%
\pgfpathlineto{\pgfqpoint{9.155129in}{5.884805in}}%
\pgfpathlineto{\pgfqpoint{9.157561in}{5.884520in}}%
\pgfpathlineto{\pgfqpoint{9.160600in}{5.885044in}}%
\pgfpathlineto{\pgfqpoint{9.169718in}{5.884265in}}%
\pgfpathlineto{\pgfqpoint{9.190992in}{5.883796in}}%
\pgfpathlineto{\pgfqpoint{9.192816in}{5.884108in}}%
\pgfpathlineto{\pgfqpoint{9.198287in}{5.884432in}}%
\pgfpathlineto{\pgfqpoint{9.203757in}{5.883840in}}%
\pgfpathlineto{\pgfqpoint{9.209228in}{5.885674in}}%
\pgfpathlineto{\pgfqpoint{9.211051in}{5.885695in}}%
\pgfpathlineto{\pgfqpoint{9.213483in}{5.886022in}}%
\pgfpathlineto{\pgfqpoint{9.215914in}{5.885991in}}%
\pgfpathlineto{\pgfqpoint{9.218346in}{5.885759in}}%
\pgfpathlineto{\pgfqpoint{9.220169in}{5.885464in}}%
\pgfpathlineto{\pgfqpoint{9.222600in}{5.886054in}}%
\pgfpathlineto{\pgfqpoint{9.225032in}{5.885944in}}%
\pgfpathlineto{\pgfqpoint{9.232934in}{5.884233in}}%
\pgfpathlineto{\pgfqpoint{9.233542in}{5.885303in}}%
\pgfpathlineto{\pgfqpoint{9.234150in}{5.884626in}}%
\pgfpathlineto{\pgfqpoint{9.236581in}{5.883751in}}%
\pgfpathlineto{\pgfqpoint{9.240228in}{5.883993in}}%
\pgfpathlineto{\pgfqpoint{9.245091in}{5.884234in}}%
\pgfpathlineto{\pgfqpoint{9.262718in}{5.884017in}}%
\pgfpathlineto{\pgfqpoint{9.265150in}{5.884198in}}%
\pgfpathlineto{\pgfqpoint{9.269405in}{5.884417in}}%
\pgfpathlineto{\pgfqpoint{9.271228in}{5.883589in}}%
\pgfpathlineto{\pgfqpoint{9.274267in}{5.884678in}}%
\pgfpathlineto{\pgfqpoint{9.274875in}{5.883586in}}%
\pgfpathlineto{\pgfqpoint{9.275483in}{5.884046in}}%
\pgfpathlineto{\pgfqpoint{9.277915in}{5.884623in}}%
\pgfpathlineto{\pgfqpoint{9.279130in}{5.883900in}}%
\pgfpathlineto{\pgfqpoint{9.280346in}{5.884403in}}%
\pgfpathlineto{\pgfqpoint{9.282169in}{5.884131in}}%
\pgfpathlineto{\pgfqpoint{9.283993in}{5.883960in}}%
\pgfpathlineto{\pgfqpoint{9.284601in}{5.884856in}}%
\pgfpathlineto{\pgfqpoint{9.285209in}{5.884118in}}%
\pgfpathlineto{\pgfqpoint{9.288248in}{5.884399in}}%
\pgfpathlineto{\pgfqpoint{9.289464in}{5.884352in}}%
\pgfpathlineto{\pgfqpoint{9.290679in}{5.884528in}}%
\pgfpathlineto{\pgfqpoint{9.292503in}{5.883871in}}%
\pgfpathlineto{\pgfqpoint{9.296758in}{5.885406in}}%
\pgfpathlineto{\pgfqpoint{9.297974in}{5.883934in}}%
\pgfpathlineto{\pgfqpoint{9.298581in}{5.884384in}}%
\pgfpathlineto{\pgfqpoint{9.301013in}{5.883724in}}%
\pgfpathlineto{\pgfqpoint{9.303444in}{5.884864in}}%
\pgfpathlineto{\pgfqpoint{9.308915in}{5.884246in}}%
\pgfpathlineto{\pgfqpoint{9.329582in}{5.885317in}}%
\pgfpathlineto{\pgfqpoint{9.330189in}{5.886076in}}%
\pgfpathlineto{\pgfqpoint{9.332013in}{5.884766in}}%
\pgfpathlineto{\pgfqpoint{9.334444in}{5.884029in}}%
\pgfpathlineto{\pgfqpoint{9.347817in}{5.884424in}}%
\pgfpathlineto{\pgfqpoint{9.349033in}{5.884582in}}%
\pgfpathlineto{\pgfqpoint{9.350248in}{5.885097in}}%
\pgfpathlineto{\pgfqpoint{9.350856in}{5.884142in}}%
\pgfpathlineto{\pgfqpoint{9.351464in}{5.884764in}}%
\pgfpathlineto{\pgfqpoint{9.353895in}{5.885054in}}%
\pgfpathlineto{\pgfqpoint{9.356327in}{5.885295in}}%
\pgfpathlineto{\pgfqpoint{9.360582in}{5.884982in}}%
\pgfpathlineto{\pgfqpoint{9.363621in}{5.886418in}}%
\pgfpathlineto{\pgfqpoint{9.366660in}{5.884963in}}%
\pgfpathlineto{\pgfqpoint{9.372739in}{5.884380in}}%
\pgfpathlineto{\pgfqpoint{9.375778in}{5.884623in}}%
\pgfpathlineto{\pgfqpoint{9.376994in}{5.885610in}}%
\pgfpathlineto{\pgfqpoint{9.378817in}{5.884655in}}%
\pgfpathlineto{\pgfqpoint{9.382464in}{5.884882in}}%
\pgfpathlineto{\pgfqpoint{9.383072in}{5.885825in}}%
\pgfpathlineto{\pgfqpoint{9.383680in}{5.884978in}}%
\pgfpathlineto{\pgfqpoint{9.387935in}{5.883926in}}%
\pgfpathlineto{\pgfqpoint{9.388543in}{5.885068in}}%
\pgfpathlineto{\pgfqpoint{9.389151in}{5.884641in}}%
\pgfpathlineto{\pgfqpoint{9.392798in}{5.884408in}}%
\pgfpathlineto{\pgfqpoint{9.394621in}{5.884207in}}%
\pgfpathlineto{\pgfqpoint{9.396445in}{5.885376in}}%
\pgfpathlineto{\pgfqpoint{9.398876in}{5.884246in}}%
\pgfpathlineto{\pgfqpoint{9.401915in}{5.883702in}}%
\pgfpathlineto{\pgfqpoint{9.403739in}{5.883645in}}%
\pgfpathlineto{\pgfqpoint{9.404955in}{5.882697in}}%
\pgfpathlineto{\pgfqpoint{9.406778in}{5.883886in}}%
\pgfpathlineto{\pgfqpoint{9.408602in}{5.883498in}}%
\pgfpathlineto{\pgfqpoint{9.409209in}{5.884549in}}%
\pgfpathlineto{\pgfqpoint{9.409817in}{5.883850in}}%
\pgfpathlineto{\pgfqpoint{9.412249in}{5.883816in}}%
\pgfpathlineto{\pgfqpoint{9.412857in}{5.882827in}}%
\pgfpathlineto{\pgfqpoint{9.413464in}{5.883709in}}%
\pgfpathlineto{\pgfqpoint{9.415896in}{5.882998in}}%
\pgfpathlineto{\pgfqpoint{9.417719in}{5.882162in}}%
\pgfpathlineto{\pgfqpoint{9.422582in}{5.883099in}}%
\pgfpathlineto{\pgfqpoint{9.423798in}{5.881880in}}%
\pgfpathlineto{\pgfqpoint{9.425014in}{5.883264in}}%
\pgfpathlineto{\pgfqpoint{9.429876in}{5.882053in}}%
\pgfpathlineto{\pgfqpoint{9.431092in}{5.882329in}}%
\pgfpathlineto{\pgfqpoint{9.433523in}{5.882024in}}%
\pgfpathlineto{\pgfqpoint{9.439602in}{5.882728in}}%
\pgfpathlineto{\pgfqpoint{9.441425in}{5.882167in}}%
\pgfpathlineto{\pgfqpoint{9.445680in}{5.882426in}}%
\pgfpathlineto{\pgfqpoint{9.447504in}{5.882036in}}%
\pgfpathlineto{\pgfqpoint{9.449327in}{5.882511in}}%
\pgfpathlineto{\pgfqpoint{9.451759in}{5.882152in}}%
\pgfpathlineto{\pgfqpoint{9.456014in}{5.882311in}}%
\pgfpathlineto{\pgfqpoint{9.462092in}{5.882717in}}%
\pgfpathlineto{\pgfqpoint{9.464524in}{5.883997in}}%
\pgfpathlineto{\pgfqpoint{9.471818in}{5.882796in}}%
\pgfpathlineto{\pgfqpoint{9.473641in}{5.882644in}}%
\pgfpathlineto{\pgfqpoint{9.513759in}{5.882626in}}%
\pgfpathlineto{\pgfqpoint{9.514975in}{5.883148in}}%
\pgfpathlineto{\pgfqpoint{9.517406in}{5.882857in}}%
\pgfpathlineto{\pgfqpoint{9.527132in}{5.883791in}}%
\pgfpathlineto{\pgfqpoint{9.528955in}{5.883838in}}%
\pgfpathlineto{\pgfqpoint{9.533818in}{5.883185in}}%
\pgfpathlineto{\pgfqpoint{9.564211in}{5.952951in}}%
\pgfpathlineto{\pgfqpoint{9.568465in}{5.960514in}}%
\pgfpathlineto{\pgfqpoint{9.572720in}{5.967671in}}%
\pgfpathlineto{\pgfqpoint{9.575152in}{5.971172in}}%
\pgfpathlineto{\pgfqpoint{9.583054in}{5.982999in}}%
\pgfpathlineto{\pgfqpoint{9.586701in}{5.988092in}}%
\pgfpathlineto{\pgfqpoint{9.589740in}{5.993865in}}%
\pgfpathlineto{\pgfqpoint{9.590348in}{5.993151in}}%
\pgfpathlineto{\pgfqpoint{9.592172in}{5.996965in}}%
\pgfpathlineto{\pgfqpoint{9.594603in}{5.999284in}}%
\pgfpathlineto{\pgfqpoint{9.597642in}{6.005831in}}%
\pgfpathlineto{\pgfqpoint{9.598250in}{6.006160in}}%
\pgfpathlineto{\pgfqpoint{9.601289in}{6.012229in}}%
\pgfpathlineto{\pgfqpoint{9.606152in}{6.018535in}}%
\pgfpathlineto{\pgfqpoint{9.611015in}{6.024685in}}%
\pgfpathlineto{\pgfqpoint{9.613446in}{6.027491in}}%
\pgfpathlineto{\pgfqpoint{9.620132in}{6.036355in}}%
\pgfpathlineto{\pgfqpoint{9.623172in}{6.040062in}}%
\pgfpathlineto{\pgfqpoint{9.633505in}{6.052540in}}%
\pgfpathlineto{\pgfqpoint{9.636544in}{6.057089in}}%
\pgfpathlineto{\pgfqpoint{9.638368in}{6.059129in}}%
\pgfpathlineto{\pgfqpoint{9.646270in}{6.069255in}}%
\pgfpathlineto{\pgfqpoint{9.651740in}{6.076809in}}%
\pgfpathlineto{\pgfqpoint{9.668152in}{6.097033in}}%
\pgfpathlineto{\pgfqpoint{9.668760in}{6.099104in}}%
\pgfpathlineto{\pgfqpoint{9.669368in}{6.098708in}}%
\pgfpathlineto{\pgfqpoint{9.670584in}{6.099684in}}%
\pgfpathlineto{\pgfqpoint{9.672407in}{6.102801in}}%
\pgfpathlineto{\pgfqpoint{9.680309in}{6.111208in}}%
\pgfpathlineto{\pgfqpoint{9.700368in}{6.144931in}}%
\pgfpathlineto{\pgfqpoint{9.718604in}{6.167885in}}%
\pgfpathlineto{\pgfqpoint{9.719819in}{6.168584in}}%
\pgfpathlineto{\pgfqpoint{9.723466in}{6.172036in}}%
\pgfpathlineto{\pgfqpoint{9.727721in}{6.174675in}}%
\pgfpathlineto{\pgfqpoint{9.730153in}{6.176893in}}%
\pgfpathlineto{\pgfqpoint{9.732584in}{6.178161in}}%
\pgfpathlineto{\pgfqpoint{9.741094in}{6.183103in}}%
\pgfpathlineto{\pgfqpoint{9.747172in}{6.185846in}}%
\pgfpathlineto{\pgfqpoint{9.770271in}{6.193336in}}%
\pgfpathlineto{\pgfqpoint{9.772094in}{6.192803in}}%
\pgfpathlineto{\pgfqpoint{9.773918in}{6.193236in}}%
\pgfpathlineto{\pgfqpoint{9.776957in}{6.193065in}}%
\pgfpathlineto{\pgfqpoint{9.779388in}{6.193704in}}%
\pgfpathlineto{\pgfqpoint{9.781212in}{6.192942in}}%
\pgfpathlineto{\pgfqpoint{9.783643in}{6.194171in}}%
\pgfpathlineto{\pgfqpoint{9.788506in}{6.193646in}}%
\pgfpathlineto{\pgfqpoint{9.789722in}{6.193794in}}%
\pgfpathlineto{\pgfqpoint{9.791545in}{6.194377in}}%
\pgfpathlineto{\pgfqpoint{9.793369in}{6.194419in}}%
\pgfpathlineto{\pgfqpoint{9.816467in}{6.190656in}}%
\pgfpathlineto{\pgfqpoint{9.821330in}{6.188357in}}%
\pgfpathlineto{\pgfqpoint{9.823153in}{6.188414in}}%
\pgfpathlineto{\pgfqpoint{9.844428in}{6.180297in}}%
\pgfpathlineto{\pgfqpoint{9.845644in}{6.179923in}}%
\pgfpathlineto{\pgfqpoint{9.846859in}{6.179852in}}%
\pgfpathlineto{\pgfqpoint{9.867526in}{6.170273in}}%
\pgfpathlineto{\pgfqpoint{9.876644in}{6.165567in}}%
\pgfpathlineto{\pgfqpoint{9.879683in}{6.164259in}}%
\pgfpathlineto{\pgfqpoint{9.881507in}{6.163610in}}%
\pgfpathlineto{\pgfqpoint{9.888193in}{6.160253in}}%
\pgfpathlineto{\pgfqpoint{9.891232in}{6.159107in}}%
\pgfpathlineto{\pgfqpoint{9.897919in}{6.155678in}}%
\pgfpathlineto{\pgfqpoint{9.902781in}{6.153587in}}%
\pgfpathlineto{\pgfqpoint{9.911899in}{6.149011in}}%
\pgfpathlineto{\pgfqpoint{9.914938in}{6.147159in}}%
\pgfpathlineto{\pgfqpoint{9.928311in}{6.141869in}}%
\pgfpathlineto{\pgfqpoint{9.930134in}{6.141236in}}%
\pgfpathlineto{\pgfqpoint{9.932566in}{6.139778in}}%
\pgfpathlineto{\pgfqpoint{9.940468in}{6.136483in}}%
\pgfpathlineto{\pgfqpoint{9.942899in}{6.135177in}}%
\pgfpathlineto{\pgfqpoint{9.944115in}{6.135283in}}%
\pgfpathlineto{\pgfqpoint{9.949586in}{6.132427in}}%
\pgfpathlineto{\pgfqpoint{9.950801in}{6.131880in}}%
\pgfpathlineto{\pgfqpoint{9.954448in}{6.130636in}}%
\pgfpathlineto{\pgfqpoint{9.959311in}{6.128217in}}%
\pgfpathlineto{\pgfqpoint{9.962350in}{6.125918in}}%
\pgfpathlineto{\pgfqpoint{9.965390in}{6.125391in}}%
\pgfpathlineto{\pgfqpoint{9.968429in}{6.123884in}}%
\pgfpathlineto{\pgfqpoint{9.975115in}{6.122075in}}%
\pgfpathlineto{\pgfqpoint{9.976939in}{6.121896in}}%
\pgfpathlineto{\pgfqpoint{9.981194in}{6.120343in}}%
\pgfpathlineto{\pgfqpoint{10.024959in}{6.117921in}}%
\pgfpathlineto{\pgfqpoint{10.026174in}{6.118712in}}%
\pgfpathlineto{\pgfqpoint{10.030429in}{6.118210in}}%
\pgfpathlineto{\pgfqpoint{10.032861in}{6.118592in}}%
\pgfpathlineto{\pgfqpoint{10.036508in}{6.118729in}}%
\pgfpathlineto{\pgfqpoint{10.038939in}{6.118438in}}%
\pgfpathlineto{\pgfqpoint{10.043194in}{6.119106in}}%
\pgfpathlineto{\pgfqpoint{10.052920in}{6.119768in}}%
\pgfpathlineto{\pgfqpoint{10.055351in}{6.119519in}}%
\pgfpathlineto{\pgfqpoint{10.057175in}{6.120221in}}%
\pgfpathlineto{\pgfqpoint{10.059606in}{6.120653in}}%
\pgfpathlineto{\pgfqpoint{10.062037in}{6.120710in}}%
\pgfpathlineto{\pgfqpoint{10.069939in}{6.121527in}}%
\pgfpathlineto{\pgfqpoint{10.077841in}{6.122618in}}%
\pgfpathlineto{\pgfqpoint{10.080881in}{6.122310in}}%
\pgfpathlineto{\pgfqpoint{10.082704in}{6.123258in}}%
\pgfpathlineto{\pgfqpoint{10.086351in}{6.123288in}}%
\pgfpathlineto{\pgfqpoint{10.138626in}{6.130135in}}%
\pgfpathlineto{\pgfqpoint{10.139842in}{6.130272in}}%
\pgfpathlineto{\pgfqpoint{10.141665in}{6.130931in}}%
\pgfpathlineto{\pgfqpoint{10.151999in}{6.133021in}}%
\pgfpathlineto{\pgfqpoint{10.153822in}{6.133419in}}%
\pgfpathlineto{\pgfqpoint{10.155646in}{6.132953in}}%
\pgfpathlineto{\pgfqpoint{10.161724in}{6.134224in}}%
\pgfpathlineto{\pgfqpoint{10.164156in}{6.134240in}}%
\pgfpathlineto{\pgfqpoint{10.165979in}{6.134426in}}%
\pgfpathlineto{\pgfqpoint{10.167195in}{6.134612in}}%
\pgfpathlineto{\pgfqpoint{10.169626in}{6.135553in}}%
\pgfpathlineto{\pgfqpoint{10.170842in}{6.134940in}}%
\pgfpathlineto{\pgfqpoint{10.173273in}{6.135950in}}%
\pgfpathlineto{\pgfqpoint{10.181783in}{6.137255in}}%
\pgfpathlineto{\pgfqpoint{10.185430in}{6.137950in}}%
\pgfpathlineto{\pgfqpoint{10.187862in}{6.137774in}}%
\pgfpathlineto{\pgfqpoint{10.191509in}{6.138566in}}%
\pgfpathlineto{\pgfqpoint{10.193332in}{6.138785in}}%
\pgfpathlineto{\pgfqpoint{10.195156in}{6.138566in}}%
\pgfpathlineto{\pgfqpoint{10.197587in}{6.139317in}}%
\pgfpathlineto{\pgfqpoint{10.206097in}{6.140284in}}%
\pgfpathlineto{\pgfqpoint{10.211568in}{6.140455in}}%
\pgfpathlineto{\pgfqpoint{10.212783in}{6.141380in}}%
\pgfpathlineto{\pgfqpoint{10.214607in}{6.140963in}}%
\pgfpathlineto{\pgfqpoint{10.223725in}{6.141896in}}%
\pgfpathlineto{\pgfqpoint{10.225548in}{6.141878in}}%
\pgfpathlineto{\pgfqpoint{10.227980in}{6.142310in}}%
\pgfpathlineto{\pgfqpoint{10.231019in}{6.142655in}}%
\pgfpathlineto{\pgfqpoint{10.232234in}{6.142476in}}%
\pgfpathlineto{\pgfqpoint{10.235274in}{6.143129in}}%
\pgfpathlineto{\pgfqpoint{10.240744in}{6.143469in}}%
\pgfpathlineto{\pgfqpoint{10.243784in}{6.143919in}}%
\pgfpathlineto{\pgfqpoint{10.246215in}{6.143735in}}%
\pgfpathlineto{\pgfqpoint{10.248646in}{6.144130in}}%
\pgfpathlineto{\pgfqpoint{10.251078in}{6.144500in}}%
\pgfpathlineto{\pgfqpoint{10.254117in}{6.144308in}}%
\pgfpathlineto{\pgfqpoint{10.257156in}{6.143811in}}%
\pgfpathlineto{\pgfqpoint{10.258980in}{6.144736in}}%
\pgfpathlineto{\pgfqpoint{10.263235in}{6.144731in}}%
\pgfpathlineto{\pgfqpoint{10.265666in}{6.145463in}}%
\pgfpathlineto{\pgfqpoint{10.268097in}{6.144898in}}%
\pgfpathlineto{\pgfqpoint{10.269921in}{6.145367in}}%
\pgfpathlineto{\pgfqpoint{10.280862in}{6.145715in}}%
\pgfpathlineto{\pgfqpoint{10.285725in}{6.145409in}}%
\pgfpathlineto{\pgfqpoint{10.288156in}{6.145685in}}%
\pgfpathlineto{\pgfqpoint{10.289980in}{6.146162in}}%
\pgfpathlineto{\pgfqpoint{10.292411in}{6.145631in}}%
\pgfpathlineto{\pgfqpoint{10.300313in}{6.146095in}}%
\pgfpathlineto{\pgfqpoint{10.302137in}{6.145937in}}%
\pgfpathlineto{\pgfqpoint{10.303960in}{6.146318in}}%
\pgfpathlineto{\pgfqpoint{10.307000in}{6.145941in}}%
\pgfpathlineto{\pgfqpoint{10.309431in}{6.146059in}}%
\pgfpathlineto{\pgfqpoint{10.311255in}{6.145340in}}%
\pgfpathlineto{\pgfqpoint{10.313686in}{6.144999in}}%
\pgfpathlineto{\pgfqpoint{10.316117in}{6.144599in}}%
\pgfpathlineto{\pgfqpoint{10.318549in}{6.144790in}}%
\pgfpathlineto{\pgfqpoint{10.324019in}{6.144320in}}%
\pgfpathlineto{\pgfqpoint{10.326451in}{6.144934in}}%
\pgfpathlineto{\pgfqpoint{10.328274in}{6.144530in}}%
\pgfpathlineto{\pgfqpoint{10.336784in}{6.144227in}}%
\pgfpathlineto{\pgfqpoint{10.338608in}{6.144468in}}%
\pgfpathlineto{\pgfqpoint{10.341039in}{6.143884in}}%
\pgfpathlineto{\pgfqpoint{10.344078in}{6.143777in}}%
\pgfpathlineto{\pgfqpoint{10.346510in}{6.143699in}}%
\pgfpathlineto{\pgfqpoint{10.348333in}{6.143930in}}%
\pgfpathlineto{\pgfqpoint{10.350765in}{6.143615in}}%
\pgfpathlineto{\pgfqpoint{10.353804in}{6.150156in}}%
\pgfpathlineto{\pgfqpoint{10.357451in}{6.159014in}}%
\pgfpathlineto{\pgfqpoint{10.362314in}{6.170675in}}%
\pgfpathlineto{\pgfqpoint{10.366569in}{6.180295in}}%
\pgfpathlineto{\pgfqpoint{10.370824in}{6.190554in}}%
\pgfpathlineto{\pgfqpoint{10.375079in}{6.200094in}}%
\pgfpathlineto{\pgfqpoint{10.378726in}{6.208520in}}%
\pgfpathlineto{\pgfqpoint{10.409118in}{6.273700in}}%
\pgfpathlineto{\pgfqpoint{10.416412in}{6.287472in}}%
\pgfpathlineto{\pgfqpoint{10.424922in}{6.304637in}}%
\pgfpathlineto{\pgfqpoint{10.435255in}{6.322106in}}%
\pgfpathlineto{\pgfqpoint{10.437079in}{6.324610in}}%
\pgfpathlineto{\pgfqpoint{10.444981in}{6.336645in}}%
\pgfpathlineto{\pgfqpoint{10.451059in}{6.343358in}}%
\pgfpathlineto{\pgfqpoint{10.453491in}{6.346446in}}%
\pgfpathlineto{\pgfqpoint{10.457746in}{6.351751in}}%
\pgfpathlineto{\pgfqpoint{10.462608in}{6.356325in}}%
\pgfpathlineto{\pgfqpoint{10.464432in}{6.357931in}}%
\pgfpathlineto{\pgfqpoint{10.473550in}{6.366258in}}%
\pgfpathlineto{\pgfqpoint{10.480236in}{6.371495in}}%
\pgfpathlineto{\pgfqpoint{10.483275in}{6.373514in}}%
\pgfpathlineto{\pgfqpoint{10.485099in}{6.375546in}}%
\pgfpathlineto{\pgfqpoint{10.487530in}{6.377000in}}%
\pgfpathlineto{\pgfqpoint{10.497256in}{6.383287in}}%
\pgfpathlineto{\pgfqpoint{10.502726in}{6.385439in}}%
\pgfpathlineto{\pgfqpoint{10.505158in}{6.387314in}}%
\pgfpathlineto{\pgfqpoint{10.514883in}{6.391980in}}%
\pgfpathlineto{\pgfqpoint{10.515491in}{6.391255in}}%
\pgfpathlineto{\pgfqpoint{10.517315in}{6.393307in}}%
\pgfpathlineto{\pgfqpoint{10.541021in}{6.401718in}}%
\pgfpathlineto{\pgfqpoint{10.542236in}{6.402282in}}%
\pgfpathlineto{\pgfqpoint{10.543452in}{6.402986in}}%
\pgfpathlineto{\pgfqpoint{10.545276in}{6.402770in}}%
\pgfpathlineto{\pgfqpoint{10.547099in}{6.403734in}}%
\pgfpathlineto{\pgfqpoint{10.548923in}{6.404636in}}%
\pgfpathlineto{\pgfqpoint{10.551354in}{6.405245in}}%
\pgfpathlineto{\pgfqpoint{10.552570in}{6.405825in}}%
\pgfpathlineto{\pgfqpoint{10.559256in}{6.407549in}}%
\pgfpathlineto{\pgfqpoint{10.561080in}{6.408115in}}%
\pgfpathlineto{\pgfqpoint{10.578707in}{6.412699in}}%
\pgfpathlineto{\pgfqpoint{10.579923in}{6.412160in}}%
\pgfpathlineto{\pgfqpoint{10.581139in}{6.412611in}}%
\pgfpathlineto{\pgfqpoint{10.581747in}{6.413764in}}%
\pgfpathlineto{\pgfqpoint{10.582354in}{6.412704in}}%
\pgfpathlineto{\pgfqpoint{10.584786in}{6.413873in}}%
\pgfpathlineto{\pgfqpoint{10.603021in}{6.415348in}}%
\pgfpathlineto{\pgfqpoint{10.604845in}{6.415896in}}%
\pgfpathlineto{\pgfqpoint{10.607884in}{6.416912in}}%
\pgfpathlineto{\pgfqpoint{10.609708in}{6.416928in}}%
\pgfpathlineto{\pgfqpoint{10.610923in}{6.416329in}}%
\pgfpathlineto{\pgfqpoint{10.612747in}{6.417191in}}%
\pgfpathlineto{\pgfqpoint{10.613962in}{6.416522in}}%
\pgfpathlineto{\pgfqpoint{10.615786in}{6.417586in}}%
\pgfpathlineto{\pgfqpoint{10.617002in}{6.417112in}}%
\pgfpathlineto{\pgfqpoint{10.623080in}{6.417861in}}%
\pgfpathlineto{\pgfqpoint{10.624904in}{6.417699in}}%
\pgfpathlineto{\pgfqpoint{10.628551in}{6.417874in}}%
\pgfpathlineto{\pgfqpoint{10.630374in}{6.418074in}}%
\pgfpathlineto{\pgfqpoint{10.631590in}{6.417872in}}%
\pgfpathlineto{\pgfqpoint{10.633413in}{6.418297in}}%
\pgfpathlineto{\pgfqpoint{10.634629in}{6.417487in}}%
\pgfpathlineto{\pgfqpoint{10.636453in}{6.418320in}}%
\pgfpathlineto{\pgfqpoint{10.638884in}{6.417642in}}%
\pgfpathlineto{\pgfqpoint{10.639492in}{6.418427in}}%
\pgfpathlineto{\pgfqpoint{10.640100in}{6.417663in}}%
\pgfpathlineto{\pgfqpoint{10.649218in}{6.418005in}}%
\pgfpathlineto{\pgfqpoint{10.651041in}{6.418124in}}%
\pgfpathlineto{\pgfqpoint{10.657727in}{6.417916in}}%
\pgfpathlineto{\pgfqpoint{10.660767in}{6.417520in}}%
\pgfpathlineto{\pgfqpoint{10.661982in}{6.417343in}}%
\pgfpathlineto{\pgfqpoint{10.663806in}{6.417854in}}%
\pgfpathlineto{\pgfqpoint{10.665022in}{6.417461in}}%
\pgfpathlineto{\pgfqpoint{10.665629in}{6.416684in}}%
\pgfpathlineto{\pgfqpoint{10.666237in}{6.417295in}}%
\pgfpathlineto{\pgfqpoint{10.668061in}{6.417317in}}%
\pgfpathlineto{\pgfqpoint{10.670492in}{6.417050in}}%
\pgfpathlineto{\pgfqpoint{10.672316in}{6.417445in}}%
\pgfpathlineto{\pgfqpoint{10.674747in}{6.417795in}}%
\pgfpathlineto{\pgfqpoint{10.676571in}{6.416460in}}%
\pgfpathlineto{\pgfqpoint{10.677786in}{6.417442in}}%
\pgfpathlineto{\pgfqpoint{10.679610in}{6.416717in}}%
\pgfpathlineto{\pgfqpoint{10.682041in}{6.416648in}}%
\pgfpathlineto{\pgfqpoint{10.686904in}{6.416170in}}%
\pgfpathlineto{\pgfqpoint{10.706963in}{6.415523in}}%
\pgfpathlineto{\pgfqpoint{10.708787in}{6.415123in}}%
\pgfpathlineto{\pgfqpoint{10.712434in}{6.415345in}}%
\pgfpathlineto{\pgfqpoint{10.714865in}{6.415168in}}%
\pgfpathlineto{\pgfqpoint{10.727022in}{6.413511in}}%
\pgfpathlineto{\pgfqpoint{10.728238in}{6.413394in}}%
\pgfpathlineto{\pgfqpoint{10.730061in}{6.412545in}}%
\pgfpathlineto{\pgfqpoint{10.743434in}{6.411348in}}%
\pgfpathlineto{\pgfqpoint{10.746473in}{6.410839in}}%
\pgfpathlineto{\pgfqpoint{10.751336in}{6.410738in}}%
\pgfpathlineto{\pgfqpoint{10.765924in}{6.409188in}}%
\pgfpathlineto{\pgfqpoint{10.768356in}{6.408843in}}%
\pgfpathlineto{\pgfqpoint{10.785375in}{6.407975in}}%
\pgfpathlineto{\pgfqpoint{10.787199in}{6.407827in}}%
\pgfpathlineto{\pgfqpoint{10.797532in}{6.406746in}}%
\pgfpathlineto{\pgfqpoint{10.799964in}{6.406697in}}%
\pgfpathlineto{\pgfqpoint{10.801179in}{6.407203in}}%
\pgfpathlineto{\pgfqpoint{10.803003in}{6.406351in}}%
\pgfpathlineto{\pgfqpoint{10.823062in}{6.405396in}}%
\pgfpathlineto{\pgfqpoint{10.829140in}{6.405318in}}%
\pgfpathlineto{\pgfqpoint{10.832787in}{6.404877in}}%
\pgfpathlineto{\pgfqpoint{10.835827in}{6.404899in}}%
\pgfpathlineto{\pgfqpoint{10.840082in}{6.404546in}}%
\pgfpathlineto{\pgfqpoint{10.854670in}{6.404177in}}%
\pgfpathlineto{\pgfqpoint{10.858317in}{6.404242in}}%
\pgfpathlineto{\pgfqpoint{10.859533in}{6.404075in}}%
\pgfpathlineto{\pgfqpoint{10.861356in}{6.404137in}}%
\pgfpathlineto{\pgfqpoint{10.865003in}{6.404103in}}%
\pgfpathlineto{\pgfqpoint{10.920925in}{6.403792in}}%
\pgfpathlineto{\pgfqpoint{10.931866in}{6.404275in}}%
\pgfpathlineto{\pgfqpoint{10.934298in}{6.404460in}}%
\pgfpathlineto{\pgfqpoint{10.940984in}{6.404909in}}%
\pgfpathlineto{\pgfqpoint{10.945847in}{6.404840in}}%
\pgfpathlineto{\pgfqpoint{10.953749in}{6.405672in}}%
\pgfpathlineto{\pgfqpoint{10.958004in}{6.406116in}}%
\pgfpathlineto{\pgfqpoint{11.052220in}{6.412085in}}%
\pgfpathlineto{\pgfqpoint{11.054044in}{6.411863in}}%
\pgfpathlineto{\pgfqpoint{11.139142in}{6.414466in}}%
\pgfpathlineto{\pgfqpoint{11.150083in}{6.414350in}}%
\pgfpathlineto{\pgfqpoint{11.167711in}{6.413908in}}%
\pgfpathlineto{\pgfqpoint{11.217555in}{6.538322in}}%
\pgfpathlineto{\pgfqpoint{11.248555in}{6.610551in}}%
\pgfpathlineto{\pgfqpoint{11.252810in}{6.614945in}}%
\pgfpathlineto{\pgfqpoint{11.257065in}{6.621279in}}%
\pgfpathlineto{\pgfqpoint{11.308732in}{6.713936in}}%
\pgfpathlineto{\pgfqpoint{11.311771in}{6.718102in}}%
\pgfpathlineto{\pgfqpoint{11.317241in}{6.724366in}}%
\pgfpathlineto{\pgfqpoint{11.319673in}{6.724164in}}%
\pgfpathlineto{\pgfqpoint{11.323320in}{6.722424in}}%
\pgfpathlineto{\pgfqpoint{11.325751in}{6.723272in}}%
\pgfpathlineto{\pgfqpoint{11.327575in}{6.723642in}}%
\pgfpathlineto{\pgfqpoint{11.331222in}{6.725226in}}%
\pgfpathlineto{\pgfqpoint{11.339732in}{6.722603in}}%
\pgfpathlineto{\pgfqpoint{11.345810in}{6.725734in}}%
\pgfpathlineto{\pgfqpoint{11.353104in}{6.729966in}}%
\pgfpathlineto{\pgfqpoint{11.363438in}{6.736302in}}%
\pgfpathlineto{\pgfqpoint{11.366477in}{6.737373in}}%
\pgfpathlineto{\pgfqpoint{11.369516in}{6.738801in}}%
\pgfpathlineto{\pgfqpoint{11.376811in}{6.741621in}}%
\pgfpathlineto{\pgfqpoint{11.399909in}{6.746744in}}%
\pgfpathlineto{\pgfqpoint{11.415105in}{6.745285in}}%
\pgfpathlineto{\pgfqpoint{11.418752in}{6.745017in}}%
\pgfpathlineto{\pgfqpoint{11.419360in}{6.745326in}}%
\pgfpathlineto{\pgfqpoint{11.419968in}{6.743930in}}%
\pgfpathlineto{\pgfqpoint{11.420575in}{6.744566in}}%
\pgfpathlineto{\pgfqpoint{11.422399in}{6.744703in}}%
\pgfpathlineto{\pgfqpoint{11.424830in}{6.744091in}}%
\pgfpathlineto{\pgfqpoint{11.428477in}{6.743425in}}%
\pgfpathlineto{\pgfqpoint{11.430301in}{6.742961in}}%
\pgfpathlineto{\pgfqpoint{11.442458in}{6.742414in}}%
\pgfpathlineto{\pgfqpoint{11.443066in}{6.741659in}}%
\pgfpathlineto{\pgfqpoint{11.443674in}{6.742171in}}%
\pgfpathlineto{\pgfqpoint{11.446105in}{6.741722in}}%
\pgfpathlineto{\pgfqpoint{11.460693in}{6.738441in}}%
\pgfpathlineto{\pgfqpoint{11.464340in}{6.737383in}}%
\pgfpathlineto{\pgfqpoint{11.471027in}{6.734480in}}%
\pgfpathlineto{\pgfqpoint{11.472242in}{6.734911in}}%
\pgfpathlineto{\pgfqpoint{11.474674in}{6.734098in}}%
\pgfpathlineto{\pgfqpoint{11.477105in}{6.734516in}}%
\pgfpathlineto{\pgfqpoint{11.478929in}{6.733952in}}%
\pgfpathlineto{\pgfqpoint{11.481360in}{6.733872in}}%
\pgfpathlineto{\pgfqpoint{11.483792in}{6.733799in}}%
\pgfpathlineto{\pgfqpoint{11.488654in}{6.732417in}}%
\pgfpathlineto{\pgfqpoint{11.490478in}{6.732734in}}%
\pgfpathlineto{\pgfqpoint{11.492301in}{6.732044in}}%
\pgfpathlineto{\pgfqpoint{11.495949in}{6.732232in}}%
\pgfpathlineto{\pgfqpoint{11.496556in}{6.731318in}}%
\pgfpathlineto{\pgfqpoint{11.497164in}{6.731933in}}%
\pgfpathlineto{\pgfqpoint{11.500811in}{6.731104in}}%
\pgfpathlineto{\pgfqpoint{11.509321in}{6.729791in}}%
\pgfpathlineto{\pgfqpoint{11.511145in}{6.729505in}}%
\pgfpathlineto{\pgfqpoint{11.514792in}{6.729414in}}%
\pgfpathlineto{\pgfqpoint{11.515400in}{6.728858in}}%
\pgfpathlineto{\pgfqpoint{11.516007in}{6.729570in}}%
\pgfpathlineto{\pgfqpoint{11.527556in}{6.727424in}}%
\pgfpathlineto{\pgfqpoint{11.529988in}{6.727117in}}%
\pgfpathlineto{\pgfqpoint{11.532419in}{6.726303in}}%
\pgfpathlineto{\pgfqpoint{11.543361in}{6.725027in}}%
\pgfpathlineto{\pgfqpoint{11.545184in}{6.724419in}}%
\pgfpathlineto{\pgfqpoint{11.548831in}{6.723476in}}%
\pgfpathlineto{\pgfqpoint{11.550655in}{6.724173in}}%
\pgfpathlineto{\pgfqpoint{11.557949in}{6.723737in}}%
\pgfpathlineto{\pgfqpoint{11.559165in}{6.723407in}}%
\pgfpathlineto{\pgfqpoint{11.566459in}{6.724480in}}%
\pgfpathlineto{\pgfqpoint{11.569498in}{6.724450in}}%
\pgfpathlineto{\pgfqpoint{11.570106in}{6.725327in}}%
\pgfpathlineto{\pgfqpoint{11.570714in}{6.724810in}}%
\pgfpathlineto{\pgfqpoint{11.572537in}{6.725633in}}%
\pgfpathlineto{\pgfqpoint{11.574361in}{6.725725in}}%
\pgfpathlineto{\pgfqpoint{11.574969in}{6.725010in}}%
\pgfpathlineto{\pgfqpoint{11.576792in}{6.726246in}}%
\pgfpathlineto{\pgfqpoint{11.579831in}{6.725829in}}%
\pgfpathlineto{\pgfqpoint{11.582263in}{6.725650in}}%
\pgfpathlineto{\pgfqpoint{11.584086in}{6.726156in}}%
\pgfpathlineto{\pgfqpoint{11.588341in}{6.725659in}}%
\pgfpathlineto{\pgfqpoint{11.592596in}{6.725555in}}%
\pgfpathlineto{\pgfqpoint{11.593812in}{6.725345in}}%
\pgfpathlineto{\pgfqpoint{11.596243in}{6.725464in}}%
\pgfpathlineto{\pgfqpoint{11.602322in}{6.724291in}}%
\pgfpathlineto{\pgfqpoint{11.602930in}{6.725011in}}%
\pgfpathlineto{\pgfqpoint{11.603537in}{6.724111in}}%
\pgfpathlineto{\pgfqpoint{11.606577in}{6.724200in}}%
\pgfpathlineto{\pgfqpoint{11.608400in}{6.723824in}}%
\pgfpathlineto{\pgfqpoint{11.612655in}{6.724177in}}%
\pgfpathlineto{\pgfqpoint{11.614479in}{6.723926in}}%
\pgfpathlineto{\pgfqpoint{11.618126in}{6.724176in}}%
\pgfpathlineto{\pgfqpoint{11.622381in}{6.723617in}}%
\pgfpathlineto{\pgfqpoint{11.624204in}{6.724812in}}%
\pgfpathlineto{\pgfqpoint{11.625420in}{6.724603in}}%
\pgfpathlineto{\pgfqpoint{11.627244in}{6.724762in}}%
\pgfpathlineto{\pgfqpoint{11.629067in}{6.723922in}}%
\pgfpathlineto{\pgfqpoint{11.631498in}{6.724500in}}%
\pgfpathlineto{\pgfqpoint{11.633930in}{6.723949in}}%
\pgfpathlineto{\pgfqpoint{11.636969in}{6.723399in}}%
\pgfpathlineto{\pgfqpoint{11.643655in}{6.722115in}}%
\pgfpathlineto{\pgfqpoint{11.648518in}{6.722005in}}%
\pgfpathlineto{\pgfqpoint{11.650342in}{6.722333in}}%
\pgfpathlineto{\pgfqpoint{11.651557in}{6.721724in}}%
\pgfpathlineto{\pgfqpoint{11.652773in}{6.722566in}}%
\pgfpathlineto{\pgfqpoint{11.655204in}{6.721611in}}%
\pgfpathlineto{\pgfqpoint{11.663714in}{6.720984in}}%
\pgfpathlineto{\pgfqpoint{11.665538in}{6.721368in}}%
\pgfpathlineto{\pgfqpoint{11.667969in}{6.720449in}}%
\pgfpathlineto{\pgfqpoint{11.669793in}{6.720697in}}%
\pgfpathlineto{\pgfqpoint{11.672224in}{6.720679in}}%
\pgfpathlineto{\pgfqpoint{11.673440in}{6.720651in}}%
\pgfpathlineto{\pgfqpoint{11.675263in}{6.720192in}}%
\pgfpathlineto{\pgfqpoint{11.680126in}{6.719826in}}%
\pgfpathlineto{\pgfqpoint{11.681950in}{6.719065in}}%
\pgfpathlineto{\pgfqpoint{11.684381in}{6.719331in}}%
\pgfpathlineto{\pgfqpoint{11.687420in}{6.718790in}}%
\pgfpathlineto{\pgfqpoint{11.717205in}{6.716866in}}%
\pgfpathlineto{\pgfqpoint{11.720244in}{6.716942in}}%
\pgfpathlineto{\pgfqpoint{11.723891in}{6.717106in}}%
\pgfpathlineto{\pgfqpoint{11.726323in}{6.716657in}}%
\pgfpathlineto{\pgfqpoint{11.733617in}{6.716235in}}%
\pgfpathlineto{\pgfqpoint{11.777990in}{6.713932in}}%
\pgfpathlineto{\pgfqpoint{11.780421in}{6.714400in}}%
\pgfpathlineto{\pgfqpoint{11.786499in}{6.713196in}}%
\pgfpathlineto{\pgfqpoint{11.790146in}{6.714016in}}%
\pgfpathlineto{\pgfqpoint{11.791362in}{6.713445in}}%
\pgfpathlineto{\pgfqpoint{11.793794in}{6.713983in}}%
\pgfpathlineto{\pgfqpoint{11.796833in}{6.713692in}}%
\pgfpathlineto{\pgfqpoint{11.799872in}{6.713364in}}%
\pgfpathlineto{\pgfqpoint{11.813853in}{6.713478in}}%
\pgfpathlineto{\pgfqpoint{11.818107in}{6.712586in}}%
\pgfpathlineto{\pgfqpoint{11.826009in}{6.713012in}}%
\pgfpathlineto{\pgfqpoint{11.858833in}{6.711312in}}%
\pgfpathlineto{\pgfqpoint{11.860657in}{6.711037in}}%
\pgfpathlineto{\pgfqpoint{11.864304in}{6.711251in}}%
\pgfpathlineto{\pgfqpoint{11.888618in}{6.710142in}}%
\pgfpathlineto{\pgfqpoint{11.892873in}{6.710021in}}%
\pgfpathlineto{\pgfqpoint{11.903814in}{6.710480in}}%
\pgfpathlineto{\pgfqpoint{11.905030in}{6.710238in}}%
\pgfpathlineto{\pgfqpoint{11.911108in}{6.710761in}}%
\pgfpathlineto{\pgfqpoint{11.918402in}{6.709978in}}%
\pgfpathlineto{\pgfqpoint{11.920834in}{6.710285in}}%
\pgfpathlineto{\pgfqpoint{11.929344in}{6.709581in}}%
\pgfpathlineto{\pgfqpoint{11.931775in}{6.710301in}}%
\pgfpathlineto{\pgfqpoint{11.933598in}{6.709942in}}%
\pgfpathlineto{\pgfqpoint{11.935422in}{6.710421in}}%
\pgfpathlineto{\pgfqpoint{11.937245in}{6.709995in}}%
\pgfpathlineto{\pgfqpoint{11.949402in}{6.709840in}}%
\pgfpathlineto{\pgfqpoint{11.951226in}{6.709845in}}%
\pgfpathlineto{\pgfqpoint{11.953049in}{6.709378in}}%
\pgfpathlineto{\pgfqpoint{11.954873in}{6.709469in}}%
\pgfpathlineto{\pgfqpoint{11.970069in}{6.709639in}}%
\pgfpathlineto{\pgfqpoint{11.971285in}{6.709341in}}%
\pgfpathlineto{\pgfqpoint{11.973716in}{6.709866in}}%
\pgfpathlineto{\pgfqpoint{11.988912in}{6.708548in}}%
\pgfpathlineto{\pgfqpoint{11.992560in}{6.709139in}}%
\pgfpathlineto{\pgfqpoint{12.001069in}{6.708625in}}%
\pgfpathlineto{\pgfqpoint{12.002893in}{6.707492in}}%
\pgfpathlineto{\pgfqpoint{12.011403in}{6.707201in}}%
\pgfpathlineto{\pgfqpoint{12.015050in}{6.706571in}}%
\pgfpathlineto{\pgfqpoint{12.017481in}{6.706926in}}%
\pgfpathlineto{\pgfqpoint{12.021128in}{6.707204in}}%
\pgfpathlineto{\pgfqpoint{12.024775in}{6.707510in}}%
\pgfpathlineto{\pgfqpoint{12.039972in}{6.706857in}}%
\pgfpathlineto{\pgfqpoint{12.041795in}{6.706997in}}%
\pgfpathlineto{\pgfqpoint{12.055776in}{6.706233in}}%
\pgfpathlineto{\pgfqpoint{12.057599in}{6.707197in}}%
\pgfpathlineto{\pgfqpoint{12.061246in}{6.706650in}}%
\pgfpathlineto{\pgfqpoint{12.064893in}{6.707285in}}%
\pgfpathlineto{\pgfqpoint{12.067325in}{6.707411in}}%
\pgfpathlineto{\pgfqpoint{12.069756in}{6.707117in}}%
\pgfpathlineto{\pgfqpoint{12.070972in}{6.707052in}}%
\pgfpathlineto{\pgfqpoint{12.074011in}{6.707753in}}%
\pgfpathlineto{\pgfqpoint{12.077658in}{6.707453in}}%
\pgfpathlineto{\pgfqpoint{12.079482in}{6.707382in}}%
\pgfpathlineto{\pgfqpoint{12.087992in}{6.705603in}}%
\pgfpathlineto{\pgfqpoint{12.092247in}{6.706213in}}%
\pgfpathlineto{\pgfqpoint{12.095286in}{6.706536in}}%
\pgfpathlineto{\pgfqpoint{12.117776in}{6.706601in}}%
\pgfpathlineto{\pgfqpoint{12.120207in}{6.706605in}}%
\pgfpathlineto{\pgfqpoint{12.123247in}{6.706535in}}%
\pgfpathlineto{\pgfqpoint{12.126286in}{6.706096in}}%
\pgfpathlineto{\pgfqpoint{12.139659in}{6.708104in}}%
\pgfpathlineto{\pgfqpoint{12.141482in}{6.708485in}}%
\pgfpathlineto{\pgfqpoint{12.143914in}{6.708322in}}%
\pgfpathlineto{\pgfqpoint{12.157286in}{6.709948in}}%
\pgfpathlineto{\pgfqpoint{12.157894in}{6.709085in}}%
\pgfpathlineto{\pgfqpoint{12.158502in}{6.709535in}}%
\pgfpathlineto{\pgfqpoint{12.160325in}{6.710887in}}%
\pgfpathlineto{\pgfqpoint{12.165796in}{6.710150in}}%
\pgfpathlineto{\pgfqpoint{12.168227in}{6.710129in}}%
\pgfpathlineto{\pgfqpoint{12.169443in}{6.709068in}}%
\pgfpathlineto{\pgfqpoint{12.171267in}{6.710106in}}%
\pgfpathlineto{\pgfqpoint{12.177345in}{6.710035in}}%
\pgfpathlineto{\pgfqpoint{12.179169in}{6.710053in}}%
\pgfpathlineto{\pgfqpoint{12.184031in}{6.709656in}}%
\pgfpathlineto{\pgfqpoint{12.188894in}{6.709844in}}%
\pgfpathlineto{\pgfqpoint{12.193757in}{6.709340in}}%
\pgfpathlineto{\pgfqpoint{12.195580in}{6.709694in}}%
\pgfpathlineto{\pgfqpoint{12.199835in}{6.709461in}}%
\pgfpathlineto{\pgfqpoint{12.241169in}{6.709811in}}%
\pgfpathlineto{\pgfqpoint{12.244208in}{6.710140in}}%
\pgfpathlineto{\pgfqpoint{12.254542in}{6.710089in}}%
\pgfpathlineto{\pgfqpoint{12.259404in}{6.709444in}}%
\pgfpathlineto{\pgfqpoint{12.266699in}{6.708886in}}%
\pgfpathlineto{\pgfqpoint{12.270346in}{6.709411in}}%
\pgfpathlineto{\pgfqpoint{12.275208in}{6.708438in}}%
\pgfpathlineto{\pgfqpoint{12.278248in}{6.708775in}}%
\pgfpathlineto{\pgfqpoint{12.280071in}{6.708250in}}%
\pgfpathlineto{\pgfqpoint{12.281895in}{6.708505in}}%
\pgfpathlineto{\pgfqpoint{12.283718in}{6.708282in}}%
\pgfpathlineto{\pgfqpoint{12.286758in}{6.709211in}}%
\pgfpathlineto{\pgfqpoint{12.288581in}{6.708771in}}%
\pgfpathlineto{\pgfqpoint{12.290405in}{6.708832in}}%
\pgfpathlineto{\pgfqpoint{12.295875in}{6.708104in}}%
\pgfpathlineto{\pgfqpoint{12.299522in}{6.708087in}}%
\pgfpathlineto{\pgfqpoint{12.306817in}{6.709084in}}%
\pgfpathlineto{\pgfqpoint{12.311071in}{6.707970in}}%
\pgfpathlineto{\pgfqpoint{12.318973in}{6.708791in}}%
\pgfpathlineto{\pgfqpoint{12.320797in}{6.708150in}}%
\pgfpathlineto{\pgfqpoint{12.322621in}{6.708819in}}%
\pgfpathlineto{\pgfqpoint{12.323836in}{6.708351in}}%
\pgfpathlineto{\pgfqpoint{12.328091in}{6.708323in}}%
\pgfpathlineto{\pgfqpoint{12.329915in}{6.708417in}}%
\pgfpathlineto{\pgfqpoint{12.337209in}{6.708604in}}%
\pgfpathlineto{\pgfqpoint{12.339032in}{6.708628in}}%
\pgfpathlineto{\pgfqpoint{12.342072in}{6.708780in}}%
\pgfpathlineto{\pgfqpoint{12.346327in}{6.708005in}}%
\pgfpathlineto{\pgfqpoint{12.349366in}{6.708422in}}%
\pgfpathlineto{\pgfqpoint{12.351189in}{6.708204in}}%
\pgfpathlineto{\pgfqpoint{12.353013in}{6.708733in}}%
\pgfpathlineto{\pgfqpoint{12.358483in}{6.708825in}}%
\pgfpathlineto{\pgfqpoint{12.366993in}{6.708521in}}%
\pgfpathlineto{\pgfqpoint{12.374287in}{6.709313in}}%
\pgfpathlineto{\pgfqpoint{12.389484in}{6.709341in}}%
\pgfpathlineto{\pgfqpoint{12.391915in}{6.709449in}}%
\pgfpathlineto{\pgfqpoint{12.416837in}{6.708695in}}%
\pgfpathlineto{\pgfqpoint{12.418052in}{6.708926in}}%
\pgfpathlineto{\pgfqpoint{12.421092in}{6.709325in}}%
\pgfpathlineto{\pgfqpoint{12.423523in}{6.708871in}}%
\pgfpathlineto{\pgfqpoint{12.427778in}{6.708990in}}%
\pgfpathlineto{\pgfqpoint{12.432033in}{6.708511in}}%
\pgfpathlineto{\pgfqpoint{12.435072in}{6.708661in}}%
\pgfpathlineto{\pgfqpoint{12.447229in}{6.708747in}}%
\pgfpathlineto{\pgfqpoint{12.450268in}{6.709180in}}%
\pgfpathlineto{\pgfqpoint{12.453308in}{6.709379in}}%
\pgfpathlineto{\pgfqpoint{12.469720in}{6.709300in}}%
\pgfpathlineto{\pgfqpoint{12.474582in}{6.709077in}}%
\pgfpathlineto{\pgfqpoint{12.477622in}{6.709340in}}%
\pgfpathlineto{\pgfqpoint{12.479445in}{6.708793in}}%
\pgfpathlineto{\pgfqpoint{12.481269in}{6.709044in}}%
\pgfpathlineto{\pgfqpoint{12.514092in}{6.708993in}}%
\pgfpathlineto{\pgfqpoint{12.520171in}{6.709175in}}%
\pgfpathlineto{\pgfqpoint{12.523210in}{6.709302in}}%
\pgfpathlineto{\pgfqpoint{12.525641in}{6.709362in}}%
\pgfpathlineto{\pgfqpoint{12.532936in}{6.709046in}}%
\pgfpathlineto{\pgfqpoint{12.534759in}{6.708773in}}%
\pgfpathlineto{\pgfqpoint{12.537190in}{6.708751in}}%
\pgfpathlineto{\pgfqpoint{12.539014in}{6.709131in}}%
\pgfpathlineto{\pgfqpoint{12.541445in}{6.709444in}}%
\pgfpathlineto{\pgfqpoint{12.562720in}{6.708765in}}%
\pgfpathlineto{\pgfqpoint{12.564544in}{6.708673in}}%
\pgfpathlineto{\pgfqpoint{12.570622in}{6.708671in}}%
\pgfpathlineto{\pgfqpoint{12.574269in}{6.709254in}}%
\pgfpathlineto{\pgfqpoint{12.576701in}{6.709149in}}%
\pgfpathlineto{\pgfqpoint{12.613779in}{6.709081in}}%
\pgfpathlineto{\pgfqpoint{12.615603in}{6.708509in}}%
\pgfpathlineto{\pgfqpoint{12.617426in}{6.708809in}}%
\pgfpathlineto{\pgfqpoint{12.622289in}{6.709518in}}%
\pgfpathlineto{\pgfqpoint{12.626544in}{6.708370in}}%
\pgfpathlineto{\pgfqpoint{12.629583in}{6.709208in}}%
\pgfpathlineto{\pgfqpoint{12.633838in}{6.709088in}}%
\pgfpathlineto{\pgfqpoint{12.640525in}{6.709302in}}%
\pgfpathlineto{\pgfqpoint{12.642348in}{6.708906in}}%
\pgfpathlineto{\pgfqpoint{12.645387in}{6.708905in}}%
\pgfpathlineto{\pgfqpoint{12.647211in}{6.708606in}}%
\pgfpathlineto{\pgfqpoint{12.650250in}{6.709020in}}%
\pgfpathlineto{\pgfqpoint{12.651466in}{6.709139in}}%
\pgfpathlineto{\pgfqpoint{12.653289in}{6.709757in}}%
\pgfpathlineto{\pgfqpoint{12.663623in}{6.709185in}}%
\pgfpathlineto{\pgfqpoint{12.675172in}{6.709382in}}%
\pgfpathlineto{\pgfqpoint{12.676995in}{6.709342in}}%
\pgfpathlineto{\pgfqpoint{12.691584in}{6.709450in}}%
\pgfpathlineto{\pgfqpoint{12.696446in}{6.709063in}}%
\pgfpathlineto{\pgfqpoint{12.702525in}{6.709804in}}%
\pgfpathlineto{\pgfqpoint{12.706172in}{6.709953in}}%
\pgfpathlineto{\pgfqpoint{12.707996in}{6.709486in}}%
\pgfpathlineto{\pgfqpoint{12.710427in}{6.709520in}}%
\pgfpathlineto{\pgfqpoint{12.720152in}{6.709543in}}%
\pgfpathlineto{\pgfqpoint{12.721976in}{6.709701in}}%
\pgfpathlineto{\pgfqpoint{12.734741in}{6.709945in}}%
\pgfpathlineto{\pgfqpoint{12.737172in}{6.709828in}}%
\pgfpathlineto{\pgfqpoint{12.740211in}{6.709021in}}%
\pgfpathlineto{\pgfqpoint{12.742643in}{6.710200in}}%
\pgfpathlineto{\pgfqpoint{12.746290in}{6.710325in}}%
\pgfpathlineto{\pgfqpoint{12.757231in}{6.709727in}}%
\pgfpathlineto{\pgfqpoint{12.763917in}{6.709746in}}%
\pgfpathlineto{\pgfqpoint{12.765133in}{6.710168in}}%
\pgfpathlineto{\pgfqpoint{12.785192in}{6.709988in}}%
\pgfpathlineto{\pgfqpoint{12.787624in}{6.709193in}}%
\pgfpathlineto{\pgfqpoint{12.791878in}{6.707462in}}%
\pgfpathlineto{\pgfqpoint{12.797349in}{6.708509in}}%
\pgfpathlineto{\pgfqpoint{12.799173in}{6.708803in}}%
\pgfpathlineto{\pgfqpoint{12.801604in}{6.708948in}}%
\pgfpathlineto{\pgfqpoint{12.833820in}{6.599640in}}%
\pgfpathlineto{\pgfqpoint{12.839898in}{6.583165in}}%
\pgfpathlineto{\pgfqpoint{12.847800in}{6.561963in}}%
\pgfpathlineto{\pgfqpoint{12.852663in}{6.551020in}}%
\pgfpathlineto{\pgfqpoint{12.859957in}{6.533841in}}%
\pgfpathlineto{\pgfqpoint{12.865428in}{6.522617in}}%
\pgfpathlineto{\pgfqpoint{12.876369in}{6.500027in}}%
\pgfpathlineto{\pgfqpoint{12.904938in}{6.451226in}}%
\pgfpathlineto{\pgfqpoint{12.920742in}{6.431130in}}%
\pgfpathlineto{\pgfqpoint{12.925605in}{6.424512in}}%
\pgfpathlineto{\pgfqpoint{12.929860in}{6.419705in}}%
\pgfpathlineto{\pgfqpoint{12.941409in}{6.407273in}}%
\pgfpathlineto{\pgfqpoint{12.969370in}{6.381042in}}%
\pgfpathlineto{\pgfqpoint{12.982135in}{6.372674in}}%
\pgfpathlineto{\pgfqpoint{12.989429in}{6.367203in}}%
\pgfpathlineto{\pgfqpoint{13.002194in}{6.360202in}}%
\pgfpathlineto{\pgfqpoint{13.011311in}{6.354570in}}%
\pgfpathlineto{\pgfqpoint{13.018605in}{6.351512in}}%
\pgfpathlineto{\pgfqpoint{13.021645in}{6.350084in}}%
\pgfpathlineto{\pgfqpoint{13.027723in}{6.347067in}}%
\pgfpathlineto{\pgfqpoint{13.042919in}{6.341696in}}%
\pgfpathlineto{\pgfqpoint{13.058723in}{6.337492in}}%
\pgfpathlineto{\pgfqpoint{13.064802in}{6.336043in}}%
\pgfpathlineto{\pgfqpoint{13.072704in}{6.333469in}}%
\pgfpathlineto{\pgfqpoint{13.090331in}{6.328339in}}%
\pgfpathlineto{\pgfqpoint{13.094586in}{6.325241in}}%
\pgfpathlineto{\pgfqpoint{13.100057in}{6.319005in}}%
\pgfpathlineto{\pgfqpoint{13.105528in}{6.313071in}}%
\pgfpathlineto{\pgfqpoint{13.112214in}{6.308545in}}%
\pgfpathlineto{\pgfqpoint{13.120116in}{6.307097in}}%
\pgfpathlineto{\pgfqpoint{13.127410in}{6.306654in}}%
\pgfpathlineto{\pgfqpoint{13.133488in}{6.305649in}}%
\pgfpathlineto{\pgfqpoint{13.142606in}{6.304945in}}%
\pgfpathlineto{\pgfqpoint{13.154763in}{6.303194in}}%
\pgfpathlineto{\pgfqpoint{13.269038in}{6.298326in}}%
\pgfpathlineto{\pgfqpoint{13.277548in}{6.298226in}}%
\pgfpathlineto{\pgfqpoint{13.318274in}{6.298809in}}%
\pgfpathlineto{\pgfqpoint{13.320705in}{6.298467in}}%
\pgfpathlineto{\pgfqpoint{13.390000in}{6.296778in}}%
\pgfpathlineto{\pgfqpoint{13.390000in}{6.296778in}}%
\pgfusepath{stroke}%
\end{pgfscope}%
\begin{pgfscope}%
\pgfpathrectangle{\pgfqpoint{1.021528in}{5.487778in}}{\pgfqpoint{12.368472in}{3.868889in}}%
\pgfusepath{clip}%
\pgfsetrectcap%
\pgfsetroundjoin%
\pgfsetlinewidth{1.505625pt}%
\definecolor{currentstroke}{rgb}{1.000000,0.498039,0.054902}%
\pgfsetstrokecolor{currentstroke}%
\pgfsetdash{}{0pt}%
\pgfpathmoveto{\pgfqpoint{1.011528in}{9.141289in}}%
\pgfpathlineto{\pgfqpoint{1.353845in}{9.140039in}}%
\pgfpathlineto{\pgfqpoint{1.365936in}{9.050617in}}%
\pgfpathlineto{\pgfqpoint{1.402879in}{8.609930in}}%
\pgfpathlineto{\pgfqpoint{1.427337in}{8.388311in}}%
\pgfpathlineto{\pgfqpoint{1.439751in}{8.351153in}}%
\pgfpathlineto{\pgfqpoint{1.452244in}{8.359701in}}%
\pgfpathlineto{\pgfqpoint{1.464581in}{8.403917in}}%
\pgfpathlineto{\pgfqpoint{1.477002in}{8.443219in}}%
\pgfpathlineto{\pgfqpoint{1.489340in}{8.501178in}}%
\pgfpathlineto{\pgfqpoint{1.501666in}{8.530638in}}%
\pgfpathlineto{\pgfqpoint{1.513989in}{8.588885in}}%
\pgfpathlineto{\pgfqpoint{1.526321in}{8.630213in}}%
\pgfpathlineto{\pgfqpoint{1.538635in}{8.641146in}}%
\pgfpathlineto{\pgfqpoint{1.563211in}{8.636674in}}%
\pgfpathlineto{\pgfqpoint{1.575484in}{8.613126in}}%
\pgfpathlineto{\pgfqpoint{1.587846in}{8.594950in}}%
\pgfpathlineto{\pgfqpoint{1.600133in}{8.539396in}}%
\pgfpathlineto{\pgfqpoint{1.612509in}{8.492038in}}%
\pgfpathlineto{\pgfqpoint{1.624855in}{8.407942in}}%
\pgfpathlineto{\pgfqpoint{1.649413in}{8.217591in}}%
\pgfpathlineto{\pgfqpoint{1.661750in}{8.108272in}}%
\pgfpathlineto{\pgfqpoint{1.673889in}{8.027914in}}%
\pgfpathlineto{\pgfqpoint{1.686043in}{7.908135in}}%
\pgfpathlineto{\pgfqpoint{1.698553in}{7.798295in}}%
\pgfpathlineto{\pgfqpoint{1.710971in}{7.705673in}}%
\pgfpathlineto{\pgfqpoint{1.723312in}{7.624427in}}%
\pgfpathlineto{\pgfqpoint{1.735567in}{7.509711in}}%
\pgfpathlineto{\pgfqpoint{1.772601in}{7.264793in}}%
\pgfpathlineto{\pgfqpoint{1.797303in}{7.128857in}}%
\pgfpathlineto{\pgfqpoint{1.809616in}{7.065553in}}%
\pgfpathlineto{\pgfqpoint{1.821919in}{7.019769in}}%
\pgfpathlineto{\pgfqpoint{1.834270in}{6.954554in}}%
\pgfpathlineto{\pgfqpoint{1.846536in}{6.900765in}}%
\pgfpathlineto{\pgfqpoint{1.871217in}{6.818210in}}%
\pgfpathlineto{\pgfqpoint{1.883501in}{6.792582in}}%
\pgfpathlineto{\pgfqpoint{1.895893in}{6.753095in}}%
\pgfpathlineto{\pgfqpoint{1.908056in}{6.706729in}}%
\pgfpathlineto{\pgfqpoint{1.920339in}{6.679251in}}%
\pgfpathlineto{\pgfqpoint{1.932629in}{6.643907in}}%
\pgfpathlineto{\pgfqpoint{1.944939in}{6.630812in}}%
\pgfpathlineto{\pgfqpoint{1.981864in}{6.568735in}}%
\pgfpathlineto{\pgfqpoint{1.994168in}{6.574387in}}%
\pgfpathlineto{\pgfqpoint{2.006472in}{6.567387in}}%
\pgfpathlineto{\pgfqpoint{2.018785in}{6.551174in}}%
\pgfpathlineto{\pgfqpoint{2.031105in}{6.537354in}}%
\pgfpathlineto{\pgfqpoint{2.043393in}{6.539909in}}%
\pgfpathlineto{\pgfqpoint{2.055733in}{6.540573in}}%
\pgfpathlineto{\pgfqpoint{2.068023in}{6.528343in}}%
\pgfpathlineto{\pgfqpoint{2.080340in}{6.524299in}}%
\pgfpathlineto{\pgfqpoint{2.092610in}{6.516313in}}%
\pgfpathlineto{\pgfqpoint{2.104937in}{6.529489in}}%
\pgfpathlineto{\pgfqpoint{2.117186in}{6.528524in}}%
\pgfpathlineto{\pgfqpoint{2.129480in}{6.532185in}}%
\pgfpathlineto{\pgfqpoint{2.141719in}{6.531259in}}%
\pgfpathlineto{\pgfqpoint{2.153988in}{6.549444in}}%
\pgfpathlineto{\pgfqpoint{2.166241in}{6.556645in}}%
\pgfpathlineto{\pgfqpoint{2.178459in}{6.553366in}}%
\pgfpathlineto{\pgfqpoint{2.190740in}{6.557390in}}%
\pgfpathlineto{\pgfqpoint{2.202950in}{6.552361in}}%
\pgfpathlineto{\pgfqpoint{2.215204in}{6.569298in}}%
\pgfpathlineto{\pgfqpoint{2.227432in}{6.571008in}}%
\pgfpathlineto{\pgfqpoint{2.251989in}{6.577908in}}%
\pgfpathlineto{\pgfqpoint{2.264290in}{6.577546in}}%
\pgfpathlineto{\pgfqpoint{2.276509in}{6.583942in}}%
\pgfpathlineto{\pgfqpoint{2.301066in}{6.582896in}}%
\pgfpathlineto{\pgfqpoint{2.313284in}{6.587020in}}%
\pgfpathlineto{\pgfqpoint{2.325582in}{6.605768in}}%
\pgfpathlineto{\pgfqpoint{2.337870in}{6.598245in}}%
\pgfpathlineto{\pgfqpoint{2.362419in}{6.598345in}}%
\pgfpathlineto{\pgfqpoint{2.374687in}{6.600276in}}%
\pgfpathlineto{\pgfqpoint{2.386956in}{6.595509in}}%
\pgfpathlineto{\pgfqpoint{2.399176in}{6.597118in}}%
\pgfpathlineto{\pgfqpoint{2.448266in}{6.597862in}}%
\pgfpathlineto{\pgfqpoint{2.472743in}{6.597541in}}%
\pgfpathlineto{\pgfqpoint{2.485015in}{6.609107in}}%
\pgfpathlineto{\pgfqpoint{2.497307in}{6.585089in}}%
\pgfpathlineto{\pgfqpoint{2.509536in}{6.595951in}}%
\pgfpathlineto{\pgfqpoint{2.521736in}{6.585934in}}%
\pgfpathlineto{\pgfqpoint{2.533989in}{6.584204in}}%
\pgfpathlineto{\pgfqpoint{2.546277in}{6.592371in}}%
\pgfpathlineto{\pgfqpoint{2.570780in}{6.596193in}}%
\pgfpathlineto{\pgfqpoint{2.583026in}{6.590882in}}%
\pgfpathlineto{\pgfqpoint{2.595251in}{6.594986in}}%
\pgfpathlineto{\pgfqpoint{2.607512in}{6.601725in}}%
\pgfpathlineto{\pgfqpoint{2.619805in}{6.602932in}}%
\pgfpathlineto{\pgfqpoint{2.632012in}{6.620654in}}%
\pgfpathlineto{\pgfqpoint{2.644272in}{6.619507in}}%
\pgfpathlineto{\pgfqpoint{2.656563in}{6.626789in}}%
\pgfpathlineto{\pgfqpoint{2.668884in}{6.621800in}}%
\pgfpathlineto{\pgfqpoint{2.681175in}{6.635157in}}%
\pgfpathlineto{\pgfqpoint{2.705771in}{6.641815in}}%
\pgfpathlineto{\pgfqpoint{2.718088in}{6.640628in}}%
\pgfpathlineto{\pgfqpoint{2.742693in}{6.646040in}}%
\pgfpathlineto{\pgfqpoint{2.754986in}{6.645054in}}%
\pgfpathlineto{\pgfqpoint{2.779604in}{6.645315in}}%
\pgfpathlineto{\pgfqpoint{2.791893in}{6.655796in}}%
\pgfpathlineto{\pgfqpoint{2.804199in}{6.656339in}}%
\pgfpathlineto{\pgfqpoint{2.816526in}{6.661911in}}%
\pgfpathlineto{\pgfqpoint{2.828806in}{6.659618in}}%
\pgfpathlineto{\pgfqpoint{2.841092in}{6.654347in}}%
\pgfpathlineto{\pgfqpoint{2.853341in}{6.652839in}}%
\pgfpathlineto{\pgfqpoint{2.877906in}{6.662514in}}%
\pgfpathlineto{\pgfqpoint{2.890194in}{6.678044in}}%
\pgfpathlineto{\pgfqpoint{2.902484in}{6.675268in}}%
\pgfpathlineto{\pgfqpoint{2.927039in}{6.661569in}}%
\pgfpathlineto{\pgfqpoint{2.939303in}{6.663741in}}%
\pgfpathlineto{\pgfqpoint{2.951591in}{6.664345in}}%
\pgfpathlineto{\pgfqpoint{2.963822in}{6.679150in}}%
\pgfpathlineto{\pgfqpoint{2.976166in}{6.659034in}}%
\pgfpathlineto{\pgfqpoint{2.988435in}{6.644180in}}%
\pgfpathlineto{\pgfqpoint{3.000715in}{6.602930in}}%
\pgfpathlineto{\pgfqpoint{3.025315in}{6.499647in}}%
\pgfpathlineto{\pgfqpoint{3.037564in}{6.441186in}}%
\pgfpathlineto{\pgfqpoint{3.049841in}{6.391089in}}%
\pgfpathlineto{\pgfqpoint{3.074240in}{6.275047in}}%
\pgfpathlineto{\pgfqpoint{3.098730in}{6.174103in}}%
\pgfpathlineto{\pgfqpoint{3.111013in}{6.141948in}}%
\pgfpathlineto{\pgfqpoint{3.123277in}{6.104641in}}%
\pgfpathlineto{\pgfqpoint{3.135509in}{6.076282in}}%
\pgfpathlineto{\pgfqpoint{3.147843in}{6.069722in}}%
\pgfpathlineto{\pgfqpoint{3.160058in}{6.068569in}}%
\pgfpathlineto{\pgfqpoint{3.172319in}{6.073731in}}%
\pgfpathlineto{\pgfqpoint{3.184558in}{6.071743in}}%
\pgfpathlineto{\pgfqpoint{3.196881in}{6.080815in}}%
\pgfpathlineto{\pgfqpoint{3.209177in}{6.107532in}}%
\pgfpathlineto{\pgfqpoint{3.221478in}{6.126024in}}%
\pgfpathlineto{\pgfqpoint{3.233733in}{6.137390in}}%
\pgfpathlineto{\pgfqpoint{3.246085in}{6.155191in}}%
\pgfpathlineto{\pgfqpoint{3.258465in}{6.176220in}}%
\pgfpathlineto{\pgfqpoint{3.270813in}{6.181885in}}%
\pgfpathlineto{\pgfqpoint{3.295503in}{6.185929in}}%
\pgfpathlineto{\pgfqpoint{3.320054in}{6.173206in}}%
\pgfpathlineto{\pgfqpoint{3.332408in}{6.161423in}}%
\pgfpathlineto{\pgfqpoint{3.344664in}{6.159641in}}%
\pgfpathlineto{\pgfqpoint{3.356955in}{6.146532in}}%
\pgfpathlineto{\pgfqpoint{3.369219in}{6.145470in}}%
\pgfpathlineto{\pgfqpoint{3.381521in}{6.142149in}}%
\pgfpathlineto{\pgfqpoint{3.393795in}{6.124112in}}%
\pgfpathlineto{\pgfqpoint{3.406081in}{6.117438in}}%
\pgfpathlineto{\pgfqpoint{3.418339in}{6.114595in}}%
\pgfpathlineto{\pgfqpoint{3.430647in}{6.102399in}}%
\pgfpathlineto{\pgfqpoint{3.442887in}{6.111857in}}%
\pgfpathlineto{\pgfqpoint{3.455170in}{6.099573in}}%
\pgfpathlineto{\pgfqpoint{3.467484in}{6.091657in}}%
\pgfpathlineto{\pgfqpoint{3.479971in}{6.094810in}}%
\pgfpathlineto{\pgfqpoint{3.492066in}{6.105786in}}%
\pgfpathlineto{\pgfqpoint{3.504341in}{6.097957in}}%
\pgfpathlineto{\pgfqpoint{3.516670in}{6.083072in}}%
\pgfpathlineto{\pgfqpoint{3.528976in}{6.094855in}}%
\pgfpathlineto{\pgfqpoint{3.541255in}{6.089780in}}%
\pgfpathlineto{\pgfqpoint{3.553586in}{6.093259in}}%
\pgfpathlineto{\pgfqpoint{3.565903in}{6.075823in}}%
\pgfpathlineto{\pgfqpoint{3.578312in}{6.080751in}}%
\pgfpathlineto{\pgfqpoint{3.590571in}{6.084113in}}%
\pgfpathlineto{\pgfqpoint{3.602910in}{6.084330in}}%
\pgfpathlineto{\pgfqpoint{3.615301in}{6.070230in}}%
\pgfpathlineto{\pgfqpoint{3.627589in}{6.081218in}}%
\pgfpathlineto{\pgfqpoint{3.639871in}{6.087444in}}%
\pgfpathlineto{\pgfqpoint{3.652129in}{6.077532in}}%
\pgfpathlineto{\pgfqpoint{3.664435in}{6.070817in}}%
\pgfpathlineto{\pgfqpoint{3.676728in}{6.071246in}}%
\pgfpathlineto{\pgfqpoint{3.689153in}{6.075852in}}%
\pgfpathlineto{\pgfqpoint{3.701287in}{6.068556in}}%
\pgfpathlineto{\pgfqpoint{3.713587in}{6.063996in}}%
\pgfpathlineto{\pgfqpoint{3.725921in}{6.050952in}}%
\pgfpathlineto{\pgfqpoint{3.738192in}{6.048378in}}%
\pgfpathlineto{\pgfqpoint{3.750507in}{6.044239in}}%
\pgfpathlineto{\pgfqpoint{3.762765in}{6.036910in}}%
\pgfpathlineto{\pgfqpoint{3.775096in}{6.035936in}}%
\pgfpathlineto{\pgfqpoint{3.787363in}{6.038475in}}%
\pgfpathlineto{\pgfqpoint{3.799673in}{6.023663in}}%
\pgfpathlineto{\pgfqpoint{3.811965in}{6.021256in}}%
\pgfpathlineto{\pgfqpoint{3.824248in}{6.024985in}}%
\pgfpathlineto{\pgfqpoint{3.836508in}{6.024415in}}%
\pgfpathlineto{\pgfqpoint{3.848773in}{6.014202in}}%
\pgfpathlineto{\pgfqpoint{3.861088in}{6.033857in}}%
\pgfpathlineto{\pgfqpoint{3.873321in}{6.019306in}}%
\pgfpathlineto{\pgfqpoint{3.885579in}{6.016601in}}%
\pgfpathlineto{\pgfqpoint{3.897871in}{6.006803in}}%
\pgfpathlineto{\pgfqpoint{3.910129in}{6.012880in}}%
\pgfpathlineto{\pgfqpoint{3.922412in}{6.027492in}}%
\pgfpathlineto{\pgfqpoint{3.934718in}{6.023891in}}%
\pgfpathlineto{\pgfqpoint{3.946872in}{6.008696in}}%
\pgfpathlineto{\pgfqpoint{3.959148in}{5.997947in}}%
\pgfpathlineto{\pgfqpoint{3.971376in}{6.011275in}}%
\pgfpathlineto{\pgfqpoint{3.983659in}{6.000527in}}%
\pgfpathlineto{\pgfqpoint{3.995880in}{6.006724in}}%
\pgfpathlineto{\pgfqpoint{4.008111in}{5.984947in}}%
\pgfpathlineto{\pgfqpoint{4.020388in}{6.000454in}}%
\pgfpathlineto{\pgfqpoint{4.032734in}{5.994599in}}%
\pgfpathlineto{\pgfqpoint{4.044913in}{5.983323in}}%
\pgfpathlineto{\pgfqpoint{4.057154in}{5.998305in}}%
\pgfpathlineto{\pgfqpoint{4.069435in}{5.988678in}}%
\pgfpathlineto{\pgfqpoint{4.081765in}{5.988177in}}%
\pgfpathlineto{\pgfqpoint{4.094084in}{5.984094in}}%
\pgfpathlineto{\pgfqpoint{4.106308in}{5.978531in}}%
\pgfpathlineto{\pgfqpoint{4.118613in}{5.985694in}}%
\pgfpathlineto{\pgfqpoint{4.130925in}{5.970691in}}%
\pgfpathlineto{\pgfqpoint{4.143204in}{5.968465in}}%
\pgfpathlineto{\pgfqpoint{4.155562in}{5.949906in}}%
\pgfpathlineto{\pgfqpoint{4.167884in}{5.947194in}}%
\pgfpathlineto{\pgfqpoint{4.180204in}{5.952734in}}%
\pgfpathlineto{\pgfqpoint{4.192515in}{5.950495in}}%
\pgfpathlineto{\pgfqpoint{4.204864in}{5.952874in}}%
\pgfpathlineto{\pgfqpoint{4.217226in}{5.947968in}}%
\pgfpathlineto{\pgfqpoint{4.229676in}{5.944244in}}%
\pgfpathlineto{\pgfqpoint{4.242008in}{5.949511in}}%
\pgfpathlineto{\pgfqpoint{4.278887in}{5.918868in}}%
\pgfpathlineto{\pgfqpoint{4.291223in}{5.932570in}}%
\pgfpathlineto{\pgfqpoint{4.303586in}{5.928081in}}%
\pgfpathlineto{\pgfqpoint{4.315894in}{5.920911in}}%
\pgfpathlineto{\pgfqpoint{4.328217in}{5.920364in}}%
\pgfpathlineto{\pgfqpoint{4.340589in}{5.917473in}}%
\pgfpathlineto{\pgfqpoint{4.365132in}{5.917177in}}%
\pgfpathlineto{\pgfqpoint{4.377409in}{5.920202in}}%
\pgfpathlineto{\pgfqpoint{4.389712in}{5.913446in}}%
\pgfpathlineto{\pgfqpoint{4.414266in}{5.902215in}}%
\pgfpathlineto{\pgfqpoint{4.426548in}{5.907662in}}%
\pgfpathlineto{\pgfqpoint{4.451054in}{5.897036in}}%
\pgfpathlineto{\pgfqpoint{4.463341in}{5.899808in}}%
\pgfpathlineto{\pgfqpoint{4.475634in}{5.878651in}}%
\pgfpathlineto{\pgfqpoint{4.487876in}{5.879425in}}%
\pgfpathlineto{\pgfqpoint{4.500111in}{5.876423in}}%
\pgfpathlineto{\pgfqpoint{4.512406in}{5.880099in}}%
\pgfpathlineto{\pgfqpoint{4.524605in}{5.874335in}}%
\pgfpathlineto{\pgfqpoint{4.536850in}{5.872550in}}%
\pgfpathlineto{\pgfqpoint{4.549155in}{5.877797in}}%
\pgfpathlineto{\pgfqpoint{4.561449in}{5.875127in}}%
\pgfpathlineto{\pgfqpoint{4.573692in}{5.869236in}}%
\pgfpathlineto{\pgfqpoint{4.586037in}{5.867253in}}%
\pgfpathlineto{\pgfqpoint{4.598301in}{5.868889in}}%
\pgfpathlineto{\pgfqpoint{4.610585in}{5.864340in}}%
\pgfpathlineto{\pgfqpoint{4.622902in}{5.872872in}}%
\pgfpathlineto{\pgfqpoint{4.635211in}{5.888603in}}%
\pgfpathlineto{\pgfqpoint{4.647533in}{5.911394in}}%
\pgfpathlineto{\pgfqpoint{4.659823in}{5.913486in}}%
\pgfpathlineto{\pgfqpoint{4.672159in}{5.935694in}}%
\pgfpathlineto{\pgfqpoint{4.684482in}{5.940079in}}%
\pgfpathlineto{\pgfqpoint{4.696826in}{5.953516in}}%
\pgfpathlineto{\pgfqpoint{4.709133in}{5.923926in}}%
\pgfpathlineto{\pgfqpoint{4.721485in}{5.930906in}}%
\pgfpathlineto{\pgfqpoint{4.746088in}{5.904554in}}%
\pgfpathlineto{\pgfqpoint{4.770726in}{5.895462in}}%
\pgfpathlineto{\pgfqpoint{4.782993in}{5.917549in}}%
\pgfpathlineto{\pgfqpoint{4.795316in}{5.932033in}}%
\pgfpathlineto{\pgfqpoint{4.819879in}{6.005897in}}%
\pgfpathlineto{\pgfqpoint{4.832212in}{6.046974in}}%
\pgfpathlineto{\pgfqpoint{4.844475in}{6.115568in}}%
\pgfpathlineto{\pgfqpoint{4.869045in}{6.216348in}}%
\pgfpathlineto{\pgfqpoint{4.881308in}{6.287980in}}%
\pgfpathlineto{\pgfqpoint{4.918129in}{6.411651in}}%
\pgfpathlineto{\pgfqpoint{4.930437in}{6.442750in}}%
\pgfpathlineto{\pgfqpoint{4.942663in}{6.465340in}}%
\pgfpathlineto{\pgfqpoint{4.955050in}{6.478254in}}%
\pgfpathlineto{\pgfqpoint{4.967280in}{6.487689in}}%
\pgfpathlineto{\pgfqpoint{4.979580in}{6.478214in}}%
\pgfpathlineto{\pgfqpoint{4.991907in}{6.486079in}}%
\pgfpathlineto{\pgfqpoint{5.004220in}{6.477953in}}%
\pgfpathlineto{\pgfqpoint{5.028632in}{6.455866in}}%
\pgfpathlineto{\pgfqpoint{5.040919in}{6.446250in}}%
\pgfpathlineto{\pgfqpoint{5.053133in}{6.445265in}}%
\pgfpathlineto{\pgfqpoint{5.065359in}{6.440497in}}%
\pgfpathlineto{\pgfqpoint{5.077556in}{6.423962in}}%
\pgfpathlineto{\pgfqpoint{5.089803in}{6.419496in}}%
\pgfpathlineto{\pgfqpoint{5.103142in}{6.403042in}}%
\pgfpathlineto{\pgfqpoint{5.114445in}{6.394593in}}%
\pgfpathlineto{\pgfqpoint{5.126788in}{6.390570in}}%
\pgfpathlineto{\pgfqpoint{5.138942in}{6.377475in}}%
\pgfpathlineto{\pgfqpoint{5.163511in}{6.365305in}}%
\pgfpathlineto{\pgfqpoint{5.175781in}{6.349112in}}%
\pgfpathlineto{\pgfqpoint{5.188068in}{6.356212in}}%
\pgfpathlineto{\pgfqpoint{5.200309in}{6.352632in}}%
\pgfpathlineto{\pgfqpoint{5.212567in}{6.346275in}}%
\pgfpathlineto{\pgfqpoint{5.224822in}{6.359290in}}%
\pgfpathlineto{\pgfqpoint{5.237095in}{6.358385in}}%
\pgfpathlineto{\pgfqpoint{5.249306in}{6.355207in}}%
\pgfpathlineto{\pgfqpoint{5.261596in}{6.360115in}}%
\pgfpathlineto{\pgfqpoint{5.273854in}{6.362851in}}%
\pgfpathlineto{\pgfqpoint{5.286123in}{6.368523in}}%
\pgfpathlineto{\pgfqpoint{5.298415in}{6.371360in}}%
\pgfpathlineto{\pgfqpoint{5.310645in}{6.369107in}}%
\pgfpathlineto{\pgfqpoint{5.322909in}{6.370173in}}%
\pgfpathlineto{\pgfqpoint{5.335205in}{6.388116in}}%
\pgfpathlineto{\pgfqpoint{5.347494in}{6.385903in}}%
\pgfpathlineto{\pgfqpoint{5.359737in}{6.386064in}}%
\pgfpathlineto{\pgfqpoint{5.372046in}{6.392903in}}%
\pgfpathlineto{\pgfqpoint{5.384315in}{6.395478in}}%
\pgfpathlineto{\pgfqpoint{5.396624in}{6.400467in}}%
\pgfpathlineto{\pgfqpoint{5.408968in}{6.407910in}}%
\pgfpathlineto{\pgfqpoint{5.421315in}{6.409559in}}%
\pgfpathlineto{\pgfqpoint{5.433613in}{6.406120in}}%
\pgfpathlineto{\pgfqpoint{5.445933in}{6.403867in}}%
\pgfpathlineto{\pgfqpoint{5.458209in}{6.424264in}}%
\pgfpathlineto{\pgfqpoint{5.482851in}{6.428186in}}%
\pgfpathlineto{\pgfqpoint{5.495157in}{6.425451in}}%
\pgfpathlineto{\pgfqpoint{5.507425in}{6.424405in}}%
\pgfpathlineto{\pgfqpoint{5.519727in}{6.425672in}}%
\pgfpathlineto{\pgfqpoint{5.531987in}{6.420100in}}%
\pgfpathlineto{\pgfqpoint{5.556515in}{6.423057in}}%
\pgfpathlineto{\pgfqpoint{5.568812in}{6.438325in}}%
\pgfpathlineto{\pgfqpoint{5.581095in}{6.431526in}}%
\pgfpathlineto{\pgfqpoint{5.593320in}{6.428347in}}%
\pgfpathlineto{\pgfqpoint{5.605559in}{6.419496in}}%
\pgfpathlineto{\pgfqpoint{5.617774in}{6.420080in}}%
\pgfpathlineto{\pgfqpoint{5.629985in}{6.430258in}}%
\pgfpathlineto{\pgfqpoint{5.642264in}{6.432914in}}%
\pgfpathlineto{\pgfqpoint{5.666912in}{6.427402in}}%
\pgfpathlineto{\pgfqpoint{5.679096in}{6.425350in}}%
\pgfpathlineto{\pgfqpoint{5.691319in}{6.427000in}}%
\pgfpathlineto{\pgfqpoint{5.703589in}{6.425531in}}%
\pgfpathlineto{\pgfqpoint{5.715924in}{6.442147in}}%
\pgfpathlineto{\pgfqpoint{5.728136in}{6.442811in}}%
\pgfpathlineto{\pgfqpoint{5.740402in}{6.442107in}}%
\pgfpathlineto{\pgfqpoint{5.752667in}{6.443535in}}%
\pgfpathlineto{\pgfqpoint{5.764890in}{6.440014in}}%
\pgfpathlineto{\pgfqpoint{5.777189in}{6.447055in}}%
\pgfpathlineto{\pgfqpoint{5.801686in}{6.451561in}}%
\pgfpathlineto{\pgfqpoint{5.813972in}{6.456429in}}%
\pgfpathlineto{\pgfqpoint{5.838481in}{6.458280in}}%
\pgfpathlineto{\pgfqpoint{5.850783in}{6.460452in}}%
\pgfpathlineto{\pgfqpoint{5.863053in}{6.474955in}}%
\pgfpathlineto{\pgfqpoint{5.875314in}{6.477490in}}%
\pgfpathlineto{\pgfqpoint{5.887608in}{6.476364in}}%
\pgfpathlineto{\pgfqpoint{5.900024in}{6.472401in}}%
\pgfpathlineto{\pgfqpoint{5.912179in}{6.471978in}}%
\pgfpathlineto{\pgfqpoint{5.924523in}{6.476384in}}%
\pgfpathlineto{\pgfqpoint{5.936793in}{6.494085in}}%
\pgfpathlineto{\pgfqpoint{5.949122in}{6.488332in}}%
\pgfpathlineto{\pgfqpoint{5.973715in}{6.488232in}}%
\pgfpathlineto{\pgfqpoint{5.985995in}{6.490002in}}%
\pgfpathlineto{\pgfqpoint{5.998313in}{6.488614in}}%
\pgfpathlineto{\pgfqpoint{6.010650in}{6.490163in}}%
\pgfpathlineto{\pgfqpoint{6.035253in}{6.484390in}}%
\pgfpathlineto{\pgfqpoint{6.047510in}{6.484490in}}%
\pgfpathlineto{\pgfqpoint{6.059849in}{6.486804in}}%
\pgfpathlineto{\pgfqpoint{6.072112in}{6.487125in}}%
\pgfpathlineto{\pgfqpoint{6.084396in}{6.492014in}}%
\pgfpathlineto{\pgfqpoint{6.096688in}{6.491269in}}%
\pgfpathlineto{\pgfqpoint{6.109005in}{6.492838in}}%
\pgfpathlineto{\pgfqpoint{6.121323in}{6.487186in}}%
\pgfpathlineto{\pgfqpoint{6.133582in}{6.491591in}}%
\pgfpathlineto{\pgfqpoint{6.145866in}{6.491169in}}%
\pgfpathlineto{\pgfqpoint{6.158122in}{6.480829in}}%
\pgfpathlineto{\pgfqpoint{6.170412in}{6.490907in}}%
\pgfpathlineto{\pgfqpoint{6.182665in}{6.496781in}}%
\pgfpathlineto{\pgfqpoint{6.194948in}{6.498652in}}%
\pgfpathlineto{\pgfqpoint{6.207218in}{6.506879in}}%
\pgfpathlineto{\pgfqpoint{6.219559in}{6.509031in}}%
\pgfpathlineto{\pgfqpoint{6.231859in}{6.494166in}}%
\pgfpathlineto{\pgfqpoint{6.244102in}{6.502494in}}%
\pgfpathlineto{\pgfqpoint{6.256411in}{6.502212in}}%
\pgfpathlineto{\pgfqpoint{6.268674in}{6.556969in}}%
\pgfpathlineto{\pgfqpoint{6.280964in}{6.591128in}}%
\pgfpathlineto{\pgfqpoint{6.305656in}{6.685752in}}%
\pgfpathlineto{\pgfqpoint{6.317936in}{6.745059in}}%
\pgfpathlineto{\pgfqpoint{6.392017in}{7.118404in}}%
\pgfpathlineto{\pgfqpoint{6.404361in}{7.193056in}}%
\pgfpathlineto{\pgfqpoint{6.416676in}{7.236800in}}%
\pgfpathlineto{\pgfqpoint{6.429121in}{7.285375in}}%
\pgfpathlineto{\pgfqpoint{6.441387in}{7.324018in}}%
\pgfpathlineto{\pgfqpoint{6.453653in}{7.327400in}}%
\pgfpathlineto{\pgfqpoint{6.466018in}{7.313282in}}%
\pgfpathlineto{\pgfqpoint{6.478315in}{7.303877in}}%
\pgfpathlineto{\pgfqpoint{6.490661in}{7.269811in}}%
\pgfpathlineto{\pgfqpoint{6.502991in}{7.219510in}}%
\pgfpathlineto{\pgfqpoint{6.515276in}{7.164106in}}%
\pgfpathlineto{\pgfqpoint{6.527575in}{7.130680in}}%
\pgfpathlineto{\pgfqpoint{6.539888in}{7.090485in}}%
\pgfpathlineto{\pgfqpoint{6.552131in}{7.071679in}}%
\pgfpathlineto{\pgfqpoint{6.564369in}{7.048116in}}%
\pgfpathlineto{\pgfqpoint{6.576664in}{7.028518in}}%
\pgfpathlineto{\pgfqpoint{6.588969in}{7.025608in}}%
\pgfpathlineto{\pgfqpoint{6.601282in}{7.024348in}}%
\pgfpathlineto{\pgfqpoint{6.613453in}{7.024790in}}%
\pgfpathlineto{\pgfqpoint{6.650164in}{7.035243in}}%
\pgfpathlineto{\pgfqpoint{6.662390in}{7.034897in}}%
\pgfpathlineto{\pgfqpoint{6.674603in}{7.028673in}}%
\pgfpathlineto{\pgfqpoint{6.686831in}{7.045885in}}%
\pgfpathlineto{\pgfqpoint{6.699020in}{7.042475in}}%
\pgfpathlineto{\pgfqpoint{6.711326in}{7.042858in}}%
\pgfpathlineto{\pgfqpoint{6.723539in}{7.051778in}}%
\pgfpathlineto{\pgfqpoint{6.735785in}{7.049341in}}%
\pgfpathlineto{\pgfqpoint{6.760278in}{7.049092in}}%
\pgfpathlineto{\pgfqpoint{6.772536in}{7.037773in}}%
\pgfpathlineto{\pgfqpoint{6.784771in}{7.035151in}}%
\pgfpathlineto{\pgfqpoint{6.797013in}{7.039697in}}%
\pgfpathlineto{\pgfqpoint{6.809306in}{7.033528in}}%
\pgfpathlineto{\pgfqpoint{6.821542in}{7.032543in}}%
\pgfpathlineto{\pgfqpoint{6.833802in}{7.037592in}}%
\pgfpathlineto{\pgfqpoint{6.846096in}{7.032440in}}%
\pgfpathlineto{\pgfqpoint{6.870650in}{7.025201in}}%
\pgfpathlineto{\pgfqpoint{6.882898in}{7.024765in}}%
\pgfpathlineto{\pgfqpoint{6.895320in}{7.026963in}}%
\pgfpathlineto{\pgfqpoint{6.907494in}{7.026557in}}%
\pgfpathlineto{\pgfqpoint{6.919739in}{7.022629in}}%
\pgfpathlineto{\pgfqpoint{6.931980in}{7.011322in}}%
\pgfpathlineto{\pgfqpoint{6.944286in}{7.027679in}}%
\pgfpathlineto{\pgfqpoint{6.956591in}{7.009001in}}%
\pgfpathlineto{\pgfqpoint{6.968797in}{7.014658in}}%
\pgfpathlineto{\pgfqpoint{6.981103in}{7.012514in}}%
\pgfpathlineto{\pgfqpoint{6.993378in}{7.030465in}}%
\pgfpathlineto{\pgfqpoint{7.005631in}{7.017941in}}%
\pgfpathlineto{\pgfqpoint{7.017872in}{7.019281in}}%
\pgfpathlineto{\pgfqpoint{7.030232in}{7.014065in}}%
\pgfpathlineto{\pgfqpoint{7.042473in}{7.013661in}}%
\pgfpathlineto{\pgfqpoint{7.067083in}{7.006026in}}%
\pgfpathlineto{\pgfqpoint{7.079417in}{7.012935in}}%
\pgfpathlineto{\pgfqpoint{7.091729in}{6.999561in}}%
\pgfpathlineto{\pgfqpoint{7.104025in}{7.014543in}}%
\pgfpathlineto{\pgfqpoint{7.116365in}{7.005560in}}%
\pgfpathlineto{\pgfqpoint{7.128574in}{7.025011in}}%
\pgfpathlineto{\pgfqpoint{7.140857in}{7.018878in}}%
\pgfpathlineto{\pgfqpoint{7.153115in}{7.007757in}}%
\pgfpathlineto{\pgfqpoint{7.165429in}{7.018789in}}%
\pgfpathlineto{\pgfqpoint{7.177722in}{7.004412in}}%
\pgfpathlineto{\pgfqpoint{7.190014in}{7.019652in}}%
\pgfpathlineto{\pgfqpoint{7.202323in}{7.003843in}}%
\pgfpathlineto{\pgfqpoint{7.214660in}{7.014795in}}%
\pgfpathlineto{\pgfqpoint{7.226940in}{7.010970in}}%
\pgfpathlineto{\pgfqpoint{7.239221in}{6.998801in}}%
\pgfpathlineto{\pgfqpoint{7.251515in}{7.002370in}}%
\pgfpathlineto{\pgfqpoint{7.263812in}{7.001634in}}%
\pgfpathlineto{\pgfqpoint{7.276092in}{6.999480in}}%
\pgfpathlineto{\pgfqpoint{7.300657in}{7.005454in}}%
\pgfpathlineto{\pgfqpoint{7.312950in}{6.991331in}}%
\pgfpathlineto{\pgfqpoint{7.325219in}{6.991137in}}%
\pgfpathlineto{\pgfqpoint{7.337550in}{6.995627in}}%
\pgfpathlineto{\pgfqpoint{7.349809in}{6.997477in}}%
\pgfpathlineto{\pgfqpoint{7.362136in}{6.996667in}}%
\pgfpathlineto{\pgfqpoint{7.374436in}{6.990318in}}%
\pgfpathlineto{\pgfqpoint{7.386756in}{6.988317in}}%
\pgfpathlineto{\pgfqpoint{7.399044in}{6.991476in}}%
\pgfpathlineto{\pgfqpoint{7.411406in}{6.995965in}}%
\pgfpathlineto{\pgfqpoint{7.423698in}{6.994751in}}%
\pgfpathlineto{\pgfqpoint{7.435934in}{6.995751in}}%
\pgfpathlineto{\pgfqpoint{7.448245in}{6.977601in}}%
\pgfpathlineto{\pgfqpoint{7.460511in}{6.979716in}}%
\pgfpathlineto{\pgfqpoint{7.472767in}{6.996462in}}%
\pgfpathlineto{\pgfqpoint{7.485009in}{6.992915in}}%
\pgfpathlineto{\pgfqpoint{7.497247in}{6.996684in}}%
\pgfpathlineto{\pgfqpoint{7.509592in}{6.996204in}}%
\pgfpathlineto{\pgfqpoint{7.521881in}{6.993853in}}%
\pgfpathlineto{\pgfqpoint{7.534193in}{6.996712in}}%
\pgfpathlineto{\pgfqpoint{7.558771in}{6.998656in}}%
\pgfpathlineto{\pgfqpoint{7.571089in}{6.997300in}}%
\pgfpathlineto{\pgfqpoint{7.583415in}{7.000066in}}%
\pgfpathlineto{\pgfqpoint{7.595699in}{6.999917in}}%
\pgfpathlineto{\pgfqpoint{7.608007in}{6.995155in}}%
\pgfpathlineto{\pgfqpoint{7.620286in}{6.995787in}}%
\pgfpathlineto{\pgfqpoint{7.632611in}{7.001345in}}%
\pgfpathlineto{\pgfqpoint{7.644902in}{6.988883in}}%
\pgfpathlineto{\pgfqpoint{7.681843in}{7.003892in}}%
\pgfpathlineto{\pgfqpoint{7.694160in}{6.988789in}}%
\pgfpathlineto{\pgfqpoint{7.706416in}{6.991826in}}%
\pgfpathlineto{\pgfqpoint{7.718690in}{6.997316in}}%
\pgfpathlineto{\pgfqpoint{7.730949in}{6.986698in}}%
\pgfpathlineto{\pgfqpoint{7.743226in}{6.987105in}}%
\pgfpathlineto{\pgfqpoint{7.755480in}{6.978598in}}%
\pgfpathlineto{\pgfqpoint{7.767779in}{6.983639in}}%
\pgfpathlineto{\pgfqpoint{7.780022in}{6.987032in}}%
\pgfpathlineto{\pgfqpoint{7.792354in}{6.987945in}}%
\pgfpathlineto{\pgfqpoint{7.804650in}{6.991415in}}%
\pgfpathlineto{\pgfqpoint{7.816896in}{6.993409in}}%
\pgfpathlineto{\pgfqpoint{7.829197in}{6.990595in}}%
\pgfpathlineto{\pgfqpoint{7.853785in}{6.997532in}}%
\pgfpathlineto{\pgfqpoint{7.865947in}{6.987197in}}%
\pgfpathlineto{\pgfqpoint{7.890385in}{6.996545in}}%
\pgfpathlineto{\pgfqpoint{7.902611in}{6.955342in}}%
\pgfpathlineto{\pgfqpoint{7.927134in}{6.826092in}}%
\pgfpathlineto{\pgfqpoint{7.939420in}{6.760245in}}%
\pgfpathlineto{\pgfqpoint{7.951705in}{6.687834in}}%
\pgfpathlineto{\pgfqpoint{7.976236in}{6.555548in}}%
\pgfpathlineto{\pgfqpoint{7.988525in}{6.468523in}}%
\pgfpathlineto{\pgfqpoint{8.013108in}{6.343771in}}%
\pgfpathlineto{\pgfqpoint{8.037625in}{6.191739in}}%
\pgfpathlineto{\pgfqpoint{8.062202in}{6.024580in}}%
\pgfpathlineto{\pgfqpoint{8.074561in}{5.951161in}}%
\pgfpathlineto{\pgfqpoint{8.111455in}{5.781434in}}%
\pgfpathlineto{\pgfqpoint{8.136094in}{5.701632in}}%
\pgfpathlineto{\pgfqpoint{8.148465in}{5.701477in}}%
\pgfpathlineto{\pgfqpoint{8.160747in}{5.685693in}}%
\pgfpathlineto{\pgfqpoint{8.173047in}{5.677141in}}%
\pgfpathlineto{\pgfqpoint{8.185384in}{5.706784in}}%
\pgfpathlineto{\pgfqpoint{8.197720in}{5.707665in}}%
\pgfpathlineto{\pgfqpoint{8.210018in}{5.736760in}}%
\pgfpathlineto{\pgfqpoint{8.222298in}{5.747753in}}%
\pgfpathlineto{\pgfqpoint{8.234578in}{5.762356in}}%
\pgfpathlineto{\pgfqpoint{8.246865in}{5.787126in}}%
\pgfpathlineto{\pgfqpoint{8.259028in}{5.787712in}}%
\pgfpathlineto{\pgfqpoint{8.271539in}{5.819734in}}%
\pgfpathlineto{\pgfqpoint{8.283817in}{5.817329in}}%
\pgfpathlineto{\pgfqpoint{8.296187in}{5.841880in}}%
\pgfpathlineto{\pgfqpoint{8.308399in}{5.848284in}}%
\pgfpathlineto{\pgfqpoint{8.320643in}{5.859635in}}%
\pgfpathlineto{\pgfqpoint{8.332937in}{5.865970in}}%
\pgfpathlineto{\pgfqpoint{8.345244in}{5.860795in}}%
\pgfpathlineto{\pgfqpoint{8.357498in}{5.859604in}}%
\pgfpathlineto{\pgfqpoint{8.369797in}{5.848179in}}%
\pgfpathlineto{\pgfqpoint{8.382081in}{5.865328in}}%
\pgfpathlineto{\pgfqpoint{8.406428in}{5.864593in}}%
\pgfpathlineto{\pgfqpoint{8.418709in}{5.862248in}}%
\pgfpathlineto{\pgfqpoint{8.431023in}{5.855178in}}%
\pgfpathlineto{\pgfqpoint{8.443305in}{5.853296in}}%
\pgfpathlineto{\pgfqpoint{8.455608in}{5.861394in}}%
\pgfpathlineto{\pgfqpoint{8.467915in}{5.857566in}}%
\pgfpathlineto{\pgfqpoint{8.480239in}{5.835478in}}%
\pgfpathlineto{\pgfqpoint{8.492533in}{5.850455in}}%
\pgfpathlineto{\pgfqpoint{8.504778in}{5.851155in}}%
\pgfpathlineto{\pgfqpoint{8.517110in}{5.847898in}}%
\pgfpathlineto{\pgfqpoint{8.529479in}{5.851686in}}%
\pgfpathlineto{\pgfqpoint{8.541864in}{5.846120in}}%
\pgfpathlineto{\pgfqpoint{8.554184in}{5.867344in}}%
\pgfpathlineto{\pgfqpoint{8.566463in}{5.851163in}}%
\pgfpathlineto{\pgfqpoint{8.578707in}{5.856026in}}%
\pgfpathlineto{\pgfqpoint{8.590975in}{5.855886in}}%
\pgfpathlineto{\pgfqpoint{8.603278in}{5.848385in}}%
\pgfpathlineto{\pgfqpoint{8.615537in}{5.852637in}}%
\pgfpathlineto{\pgfqpoint{8.627808in}{5.859935in}}%
\pgfpathlineto{\pgfqpoint{8.640024in}{5.854562in}}%
\pgfpathlineto{\pgfqpoint{8.652304in}{5.852517in}}%
\pgfpathlineto{\pgfqpoint{8.664534in}{5.853810in}}%
\pgfpathlineto{\pgfqpoint{8.676810in}{5.844229in}}%
\pgfpathlineto{\pgfqpoint{8.689078in}{5.841176in}}%
\pgfpathlineto{\pgfqpoint{8.713673in}{5.845081in}}%
\pgfpathlineto{\pgfqpoint{8.726061in}{5.839839in}}%
\pgfpathlineto{\pgfqpoint{8.738260in}{5.842994in}}%
\pgfpathlineto{\pgfqpoint{8.775078in}{5.838859in}}%
\pgfpathlineto{\pgfqpoint{8.787335in}{5.846285in}}%
\pgfpathlineto{\pgfqpoint{8.799612in}{5.845014in}}%
\pgfpathlineto{\pgfqpoint{8.811984in}{5.845875in}}%
\pgfpathlineto{\pgfqpoint{8.824305in}{5.844008in}}%
\pgfpathlineto{\pgfqpoint{8.836530in}{5.833530in}}%
\pgfpathlineto{\pgfqpoint{8.848842in}{5.841974in}}%
\pgfpathlineto{\pgfqpoint{8.873449in}{5.837783in}}%
\pgfpathlineto{\pgfqpoint{8.885737in}{5.829526in}}%
\pgfpathlineto{\pgfqpoint{8.897990in}{5.833037in}}%
\pgfpathlineto{\pgfqpoint{8.910275in}{5.829293in}}%
\pgfpathlineto{\pgfqpoint{8.922495in}{5.831933in}}%
\pgfpathlineto{\pgfqpoint{8.934849in}{5.828468in}}%
\pgfpathlineto{\pgfqpoint{8.947104in}{5.835622in}}%
\pgfpathlineto{\pgfqpoint{8.959428in}{5.821532in}}%
\pgfpathlineto{\pgfqpoint{8.971677in}{5.820373in}}%
\pgfpathlineto{\pgfqpoint{8.983944in}{5.834193in}}%
\pgfpathlineto{\pgfqpoint{8.996217in}{5.832048in}}%
\pgfpathlineto{\pgfqpoint{9.008511in}{5.832131in}}%
\pgfpathlineto{\pgfqpoint{9.020806in}{5.825254in}}%
\pgfpathlineto{\pgfqpoint{9.033063in}{5.828102in}}%
\pgfpathlineto{\pgfqpoint{9.045373in}{5.828718in}}%
\pgfpathlineto{\pgfqpoint{9.057635in}{5.828134in}}%
\pgfpathlineto{\pgfqpoint{9.069982in}{5.833023in}}%
\pgfpathlineto{\pgfqpoint{9.082186in}{5.832051in}}%
\pgfpathlineto{\pgfqpoint{9.094490in}{5.827062in}}%
\pgfpathlineto{\pgfqpoint{9.106778in}{5.825268in}}%
\pgfpathlineto{\pgfqpoint{9.131343in}{5.814272in}}%
\pgfpathlineto{\pgfqpoint{9.143598in}{5.806161in}}%
\pgfpathlineto{\pgfqpoint{9.168220in}{5.802505in}}%
\pgfpathlineto{\pgfqpoint{9.180483in}{5.796034in}}%
\pgfpathlineto{\pgfqpoint{9.192775in}{5.806885in}}%
\pgfpathlineto{\pgfqpoint{9.205147in}{5.806524in}}%
\pgfpathlineto{\pgfqpoint{9.217364in}{5.802573in}}%
\pgfpathlineto{\pgfqpoint{9.229650in}{5.791099in}}%
\pgfpathlineto{\pgfqpoint{9.241913in}{5.788809in}}%
\pgfpathlineto{\pgfqpoint{9.254153in}{5.793460in}}%
\pgfpathlineto{\pgfqpoint{9.266401in}{5.792876in}}%
\pgfpathlineto{\pgfqpoint{9.278692in}{5.805412in}}%
\pgfpathlineto{\pgfqpoint{9.290988in}{5.787221in}}%
\pgfpathlineto{\pgfqpoint{9.315536in}{5.796056in}}%
\pgfpathlineto{\pgfqpoint{9.327781in}{5.778041in}}%
\pgfpathlineto{\pgfqpoint{9.340096in}{5.794018in}}%
\pgfpathlineto{\pgfqpoint{9.352334in}{5.791097in}}%
\pgfpathlineto{\pgfqpoint{9.364627in}{5.785339in}}%
\pgfpathlineto{\pgfqpoint{9.376899in}{5.784389in}}%
\pgfpathlineto{\pgfqpoint{9.389206in}{5.776409in}}%
\pgfpathlineto{\pgfqpoint{9.401540in}{5.792272in}}%
\pgfpathlineto{\pgfqpoint{9.413786in}{5.784253in}}%
\pgfpathlineto{\pgfqpoint{9.426057in}{5.783008in}}%
\pgfpathlineto{\pgfqpoint{9.438333in}{5.779311in}}%
\pgfpathlineto{\pgfqpoint{9.450619in}{5.780733in}}%
\pgfpathlineto{\pgfqpoint{9.462959in}{5.776641in}}%
\pgfpathlineto{\pgfqpoint{9.475206in}{5.789753in}}%
\pgfpathlineto{\pgfqpoint{9.487548in}{5.773307in}}%
\pgfpathlineto{\pgfqpoint{9.499862in}{5.769378in}}%
\pgfpathlineto{\pgfqpoint{9.512168in}{5.769443in}}%
\pgfpathlineto{\pgfqpoint{9.524489in}{5.771264in}}%
\pgfpathlineto{\pgfqpoint{9.536812in}{5.776693in}}%
\pgfpathlineto{\pgfqpoint{9.549056in}{5.787792in}}%
\pgfpathlineto{\pgfqpoint{9.561364in}{5.782876in}}%
\pgfpathlineto{\pgfqpoint{9.573731in}{5.788275in}}%
\pgfpathlineto{\pgfqpoint{9.586031in}{5.811976in}}%
\pgfpathlineto{\pgfqpoint{9.598378in}{5.814286in}}%
\pgfpathlineto{\pgfqpoint{9.610774in}{5.837840in}}%
\pgfpathlineto{\pgfqpoint{9.623253in}{5.855855in}}%
\pgfpathlineto{\pgfqpoint{9.635662in}{5.882072in}}%
\pgfpathlineto{\pgfqpoint{9.684891in}{6.035569in}}%
\pgfpathlineto{\pgfqpoint{9.709509in}{6.083223in}}%
\pgfpathlineto{\pgfqpoint{9.721828in}{6.115057in}}%
\pgfpathlineto{\pgfqpoint{9.734189in}{6.150331in}}%
\pgfpathlineto{\pgfqpoint{9.746462in}{6.171920in}}%
\pgfpathlineto{\pgfqpoint{9.758804in}{6.188889in}}%
\pgfpathlineto{\pgfqpoint{9.771158in}{6.209507in}}%
\pgfpathlineto{\pgfqpoint{9.783428in}{6.205031in}}%
\pgfpathlineto{\pgfqpoint{9.795726in}{6.211369in}}%
\pgfpathlineto{\pgfqpoint{9.808012in}{6.228964in}}%
\pgfpathlineto{\pgfqpoint{9.820266in}{6.218894in}}%
\pgfpathlineto{\pgfqpoint{9.832583in}{6.215305in}}%
\pgfpathlineto{\pgfqpoint{9.844800in}{6.226861in}}%
\pgfpathlineto{\pgfqpoint{9.857043in}{6.211830in}}%
\pgfpathlineto{\pgfqpoint{9.869336in}{6.204630in}}%
\pgfpathlineto{\pgfqpoint{9.881634in}{6.199353in}}%
\pgfpathlineto{\pgfqpoint{9.893847in}{6.197751in}}%
\pgfpathlineto{\pgfqpoint{9.906155in}{6.189103in}}%
\pgfpathlineto{\pgfqpoint{9.918431in}{6.184645in}}%
\pgfpathlineto{\pgfqpoint{9.930722in}{6.178572in}}%
\pgfpathlineto{\pgfqpoint{9.942947in}{6.167115in}}%
\pgfpathlineto{\pgfqpoint{9.955220in}{6.165545in}}%
\pgfpathlineto{\pgfqpoint{9.967495in}{6.149795in}}%
\pgfpathlineto{\pgfqpoint{9.979761in}{6.156983in}}%
\pgfpathlineto{\pgfqpoint{9.992027in}{6.157228in}}%
\pgfpathlineto{\pgfqpoint{10.004357in}{6.153386in}}%
\pgfpathlineto{\pgfqpoint{10.016680in}{6.157350in}}%
\pgfpathlineto{\pgfqpoint{10.028949in}{6.147816in}}%
\pgfpathlineto{\pgfqpoint{10.041270in}{6.163372in}}%
\pgfpathlineto{\pgfqpoint{10.053550in}{6.162794in}}%
\pgfpathlineto{\pgfqpoint{10.065831in}{6.169565in}}%
\pgfpathlineto{\pgfqpoint{10.090342in}{6.163993in}}%
\pgfpathlineto{\pgfqpoint{10.102679in}{6.167394in}}%
\pgfpathlineto{\pgfqpoint{10.115008in}{6.169237in}}%
\pgfpathlineto{\pgfqpoint{10.127346in}{6.173018in}}%
\pgfpathlineto{\pgfqpoint{10.139648in}{6.170431in}}%
\pgfpathlineto{\pgfqpoint{10.152024in}{6.178525in}}%
\pgfpathlineto{\pgfqpoint{10.164317in}{6.184948in}}%
\pgfpathlineto{\pgfqpoint{10.176620in}{6.189062in}}%
\pgfpathlineto{\pgfqpoint{10.188950in}{6.178059in}}%
\pgfpathlineto{\pgfqpoint{10.201195in}{6.199260in}}%
\pgfpathlineto{\pgfqpoint{10.213507in}{6.197516in}}%
\pgfpathlineto{\pgfqpoint{10.225810in}{6.200527in}}%
\pgfpathlineto{\pgfqpoint{10.238172in}{6.199013in}}%
\pgfpathlineto{\pgfqpoint{10.250369in}{6.215623in}}%
\pgfpathlineto{\pgfqpoint{10.274908in}{6.213192in}}%
\pgfpathlineto{\pgfqpoint{10.287140in}{6.219278in}}%
\pgfpathlineto{\pgfqpoint{10.311760in}{6.218423in}}%
\pgfpathlineto{\pgfqpoint{10.323981in}{6.229590in}}%
\pgfpathlineto{\pgfqpoint{10.348537in}{6.227833in}}%
\pgfpathlineto{\pgfqpoint{10.360831in}{6.247193in}}%
\pgfpathlineto{\pgfqpoint{10.385397in}{6.311173in}}%
\pgfpathlineto{\pgfqpoint{10.410035in}{6.363813in}}%
\pgfpathlineto{\pgfqpoint{10.434627in}{6.439143in}}%
\pgfpathlineto{\pgfqpoint{10.446948in}{6.458028in}}%
\pgfpathlineto{\pgfqpoint{10.459245in}{6.471463in}}%
\pgfpathlineto{\pgfqpoint{10.471527in}{6.500399in}}%
\pgfpathlineto{\pgfqpoint{10.483883in}{6.509965in}}%
\pgfpathlineto{\pgfqpoint{10.496232in}{6.521009in}}%
\pgfpathlineto{\pgfqpoint{10.520797in}{6.562734in}}%
\pgfpathlineto{\pgfqpoint{10.533080in}{6.552352in}}%
\pgfpathlineto{\pgfqpoint{10.545384in}{6.560162in}}%
\pgfpathlineto{\pgfqpoint{10.557696in}{6.572295in}}%
\pgfpathlineto{\pgfqpoint{10.570313in}{6.578821in}}%
\pgfpathlineto{\pgfqpoint{10.582312in}{6.601129in}}%
\pgfpathlineto{\pgfqpoint{10.594609in}{6.601071in}}%
\pgfpathlineto{\pgfqpoint{10.606881in}{6.604300in}}%
\pgfpathlineto{\pgfqpoint{10.619179in}{6.610257in}}%
\pgfpathlineto{\pgfqpoint{10.631580in}{6.611828in}}%
\pgfpathlineto{\pgfqpoint{10.656075in}{6.618120in}}%
\pgfpathlineto{\pgfqpoint{10.668339in}{6.617041in}}%
\pgfpathlineto{\pgfqpoint{10.680650in}{6.614619in}}%
\pgfpathlineto{\pgfqpoint{10.692958in}{6.620320in}}%
\pgfpathlineto{\pgfqpoint{10.717543in}{6.620022in}}%
\pgfpathlineto{\pgfqpoint{10.729835in}{6.616841in}}%
\pgfpathlineto{\pgfqpoint{10.742107in}{6.618348in}}%
\pgfpathlineto{\pgfqpoint{10.766646in}{6.619813in}}%
\pgfpathlineto{\pgfqpoint{10.803478in}{6.621684in}}%
\pgfpathlineto{\pgfqpoint{10.815771in}{6.638345in}}%
\pgfpathlineto{\pgfqpoint{10.828073in}{6.631608in}}%
\pgfpathlineto{\pgfqpoint{10.852476in}{6.635856in}}%
\pgfpathlineto{\pgfqpoint{10.864752in}{6.634273in}}%
\pgfpathlineto{\pgfqpoint{10.877042in}{6.635663in}}%
\pgfpathlineto{\pgfqpoint{10.901568in}{6.662902in}}%
\pgfpathlineto{\pgfqpoint{10.913829in}{6.655463in}}%
\pgfpathlineto{\pgfqpoint{10.926101in}{6.653813in}}%
\pgfpathlineto{\pgfqpoint{10.963071in}{6.663463in}}%
\pgfpathlineto{\pgfqpoint{10.975356in}{6.660214in}}%
\pgfpathlineto{\pgfqpoint{10.987655in}{6.664386in}}%
\pgfpathlineto{\pgfqpoint{10.999935in}{6.662710in}}%
\pgfpathlineto{\pgfqpoint{11.012209in}{6.663978in}}%
\pgfpathlineto{\pgfqpoint{11.024488in}{6.667138in}}%
\pgfpathlineto{\pgfqpoint{11.036742in}{6.668737in}}%
\pgfpathlineto{\pgfqpoint{11.049045in}{6.673115in}}%
\pgfpathlineto{\pgfqpoint{11.073604in}{6.678589in}}%
\pgfpathlineto{\pgfqpoint{11.085856in}{6.675462in}}%
\pgfpathlineto{\pgfqpoint{11.110475in}{6.679875in}}%
\pgfpathlineto{\pgfqpoint{11.122729in}{6.686631in}}%
\pgfpathlineto{\pgfqpoint{11.159571in}{6.689076in}}%
\pgfpathlineto{\pgfqpoint{11.171813in}{6.703476in}}%
\pgfpathlineto{\pgfqpoint{11.184085in}{6.739914in}}%
\pgfpathlineto{\pgfqpoint{11.196431in}{6.765477in}}%
\pgfpathlineto{\pgfqpoint{11.208753in}{6.814119in}}%
\pgfpathlineto{\pgfqpoint{11.221093in}{6.846453in}}%
\pgfpathlineto{\pgfqpoint{11.233579in}{6.906539in}}%
\pgfpathlineto{\pgfqpoint{11.245701in}{6.924830in}}%
\pgfpathlineto{\pgfqpoint{11.270351in}{6.994239in}}%
\pgfpathlineto{\pgfqpoint{11.295078in}{7.061528in}}%
\pgfpathlineto{\pgfqpoint{11.307325in}{7.082180in}}%
\pgfpathlineto{\pgfqpoint{11.319662in}{7.106115in}}%
\pgfpathlineto{\pgfqpoint{11.331948in}{7.110186in}}%
\pgfpathlineto{\pgfqpoint{11.344294in}{7.137710in}}%
\pgfpathlineto{\pgfqpoint{11.368900in}{7.154739in}}%
\pgfpathlineto{\pgfqpoint{11.381178in}{7.157792in}}%
\pgfpathlineto{\pgfqpoint{11.393518in}{7.159107in}}%
\pgfpathlineto{\pgfqpoint{11.405847in}{7.149248in}}%
\pgfpathlineto{\pgfqpoint{11.430362in}{7.144405in}}%
\pgfpathlineto{\pgfqpoint{11.442815in}{7.129100in}}%
\pgfpathlineto{\pgfqpoint{11.454904in}{7.133378in}}%
\pgfpathlineto{\pgfqpoint{11.467200in}{7.119098in}}%
\pgfpathlineto{\pgfqpoint{11.479492in}{7.139838in}}%
\pgfpathlineto{\pgfqpoint{11.491800in}{7.117835in}}%
\pgfpathlineto{\pgfqpoint{11.504199in}{7.130900in}}%
\pgfpathlineto{\pgfqpoint{11.516470in}{7.141728in}}%
\pgfpathlineto{\pgfqpoint{11.528782in}{7.128731in}}%
\pgfpathlineto{\pgfqpoint{11.541117in}{7.122957in}}%
\pgfpathlineto{\pgfqpoint{11.553426in}{7.121847in}}%
\pgfpathlineto{\pgfqpoint{11.565722in}{7.128491in}}%
\pgfpathlineto{\pgfqpoint{11.578065in}{7.105057in}}%
\pgfpathlineto{\pgfqpoint{11.590315in}{7.112084in}}%
\pgfpathlineto{\pgfqpoint{11.602696in}{7.121525in}}%
\pgfpathlineto{\pgfqpoint{11.614948in}{7.123383in}}%
\pgfpathlineto{\pgfqpoint{11.627285in}{7.113359in}}%
\pgfpathlineto{\pgfqpoint{11.639574in}{7.130128in}}%
\pgfpathlineto{\pgfqpoint{11.651848in}{7.133683in}}%
\pgfpathlineto{\pgfqpoint{11.664151in}{7.133203in}}%
\pgfpathlineto{\pgfqpoint{11.676427in}{7.126995in}}%
\pgfpathlineto{\pgfqpoint{11.688694in}{7.124441in}}%
\pgfpathlineto{\pgfqpoint{11.700952in}{7.123198in}}%
\pgfpathlineto{\pgfqpoint{11.713255in}{7.134601in}}%
\pgfpathlineto{\pgfqpoint{11.725571in}{7.136343in}}%
\pgfpathlineto{\pgfqpoint{11.737926in}{7.136000in}}%
\pgfpathlineto{\pgfqpoint{11.750211in}{7.147875in}}%
\pgfpathlineto{\pgfqpoint{11.762524in}{7.144982in}}%
\pgfpathlineto{\pgfqpoint{11.774822in}{7.146900in}}%
\pgfpathlineto{\pgfqpoint{11.787151in}{7.147086in}}%
\pgfpathlineto{\pgfqpoint{11.799386in}{7.164660in}}%
\pgfpathlineto{\pgfqpoint{11.811667in}{7.166959in}}%
\pgfpathlineto{\pgfqpoint{11.824001in}{7.148992in}}%
\pgfpathlineto{\pgfqpoint{11.836243in}{7.157842in}}%
\pgfpathlineto{\pgfqpoint{11.848542in}{7.153233in}}%
\pgfpathlineto{\pgfqpoint{11.860825in}{7.162746in}}%
\pgfpathlineto{\pgfqpoint{11.873217in}{7.144686in}}%
\pgfpathlineto{\pgfqpoint{11.885438in}{7.154204in}}%
\pgfpathlineto{\pgfqpoint{11.897740in}{7.151952in}}%
\pgfpathlineto{\pgfqpoint{11.910033in}{7.147026in}}%
\pgfpathlineto{\pgfqpoint{11.922303in}{7.144289in}}%
\pgfpathlineto{\pgfqpoint{11.946883in}{7.148215in}}%
\pgfpathlineto{\pgfqpoint{11.959144in}{7.141757in}}%
\pgfpathlineto{\pgfqpoint{11.971437in}{7.142539in}}%
\pgfpathlineto{\pgfqpoint{11.983736in}{7.145040in}}%
\pgfpathlineto{\pgfqpoint{11.996012in}{7.140804in}}%
\pgfpathlineto{\pgfqpoint{12.008295in}{7.143572in}}%
\pgfpathlineto{\pgfqpoint{12.020592in}{7.150850in}}%
\pgfpathlineto{\pgfqpoint{12.057611in}{7.148270in}}%
\pgfpathlineto{\pgfqpoint{12.069895in}{7.157222in}}%
\pgfpathlineto{\pgfqpoint{12.082278in}{7.150502in}}%
\pgfpathlineto{\pgfqpoint{12.094595in}{7.152043in}}%
\pgfpathlineto{\pgfqpoint{12.106906in}{7.159262in}}%
\pgfpathlineto{\pgfqpoint{12.119235in}{7.154629in}}%
\pgfpathlineto{\pgfqpoint{12.131596in}{7.157448in}}%
\pgfpathlineto{\pgfqpoint{12.143920in}{7.171507in}}%
\pgfpathlineto{\pgfqpoint{12.156241in}{7.158086in}}%
\pgfpathlineto{\pgfqpoint{12.180852in}{7.181428in}}%
\pgfpathlineto{\pgfqpoint{12.193154in}{7.172848in}}%
\pgfpathlineto{\pgfqpoint{12.205461in}{7.182996in}}%
\pgfpathlineto{\pgfqpoint{12.217786in}{7.187906in}}%
\pgfpathlineto{\pgfqpoint{12.230093in}{7.189836in}}%
\pgfpathlineto{\pgfqpoint{12.242435in}{7.186169in}}%
\pgfpathlineto{\pgfqpoint{12.254793in}{7.187746in}}%
\pgfpathlineto{\pgfqpoint{12.267040in}{7.192691in}}%
\pgfpathlineto{\pgfqpoint{12.279417in}{7.191049in}}%
\pgfpathlineto{\pgfqpoint{12.291648in}{7.187384in}}%
\pgfpathlineto{\pgfqpoint{12.303888in}{7.186381in}}%
\pgfpathlineto{\pgfqpoint{12.316212in}{7.197503in}}%
\pgfpathlineto{\pgfqpoint{12.328453in}{7.191099in}}%
\pgfpathlineto{\pgfqpoint{12.365318in}{7.201792in}}%
\pgfpathlineto{\pgfqpoint{12.377559in}{7.189881in}}%
\pgfpathlineto{\pgfqpoint{12.389784in}{7.190602in}}%
\pgfpathlineto{\pgfqpoint{12.402101in}{7.196291in}}%
\pgfpathlineto{\pgfqpoint{12.414318in}{7.197352in}}%
\pgfpathlineto{\pgfqpoint{12.426552in}{7.209759in}}%
\pgfpathlineto{\pgfqpoint{12.438868in}{7.206329in}}%
\pgfpathlineto{\pgfqpoint{12.451045in}{7.204987in}}%
\pgfpathlineto{\pgfqpoint{12.463322in}{7.208923in}}%
\pgfpathlineto{\pgfqpoint{12.475551in}{7.192178in}}%
\pgfpathlineto{\pgfqpoint{12.487822in}{7.200371in}}%
\pgfpathlineto{\pgfqpoint{12.500090in}{7.215283in}}%
\pgfpathlineto{\pgfqpoint{12.512300in}{7.202310in}}%
\pgfpathlineto{\pgfqpoint{12.524557in}{7.211433in}}%
\pgfpathlineto{\pgfqpoint{12.536785in}{7.209493in}}%
\pgfpathlineto{\pgfqpoint{12.549021in}{7.216228in}}%
\pgfpathlineto{\pgfqpoint{12.561361in}{7.220938in}}%
\pgfpathlineto{\pgfqpoint{12.573599in}{7.217836in}}%
\pgfpathlineto{\pgfqpoint{12.585885in}{7.202787in}}%
\pgfpathlineto{\pgfqpoint{12.598181in}{7.217628in}}%
\pgfpathlineto{\pgfqpoint{12.610477in}{7.220481in}}%
\pgfpathlineto{\pgfqpoint{12.622780in}{7.206800in}}%
\pgfpathlineto{\pgfqpoint{12.635089in}{7.215019in}}%
\pgfpathlineto{\pgfqpoint{12.647435in}{7.217748in}}%
\pgfpathlineto{\pgfqpoint{12.659721in}{7.217008in}}%
\pgfpathlineto{\pgfqpoint{12.672038in}{7.218130in}}%
\pgfpathlineto{\pgfqpoint{12.684320in}{7.216657in}}%
\pgfpathlineto{\pgfqpoint{12.696630in}{7.227068in}}%
\pgfpathlineto{\pgfqpoint{12.708922in}{7.226150in}}%
\pgfpathlineto{\pgfqpoint{12.721184in}{7.222458in}}%
\pgfpathlineto{\pgfqpoint{12.733421in}{7.216161in}}%
\pgfpathlineto{\pgfqpoint{12.745725in}{7.225298in}}%
\pgfpathlineto{\pgfqpoint{12.757996in}{7.223060in}}%
\pgfpathlineto{\pgfqpoint{12.770271in}{7.222709in}}%
\pgfpathlineto{\pgfqpoint{12.782558in}{7.235366in}}%
\pgfpathlineto{\pgfqpoint{12.794883in}{7.221120in}}%
\pgfpathlineto{\pgfqpoint{12.807199in}{7.209658in}}%
\pgfpathlineto{\pgfqpoint{12.819481in}{7.173815in}}%
\pgfpathlineto{\pgfqpoint{12.831752in}{7.141992in}}%
\pgfpathlineto{\pgfqpoint{12.844082in}{7.115882in}}%
\pgfpathlineto{\pgfqpoint{12.856308in}{7.085688in}}%
\pgfpathlineto{\pgfqpoint{12.868545in}{7.074062in}}%
\pgfpathlineto{\pgfqpoint{12.905384in}{7.017898in}}%
\pgfpathlineto{\pgfqpoint{12.917658in}{7.015082in}}%
\pgfpathlineto{\pgfqpoint{12.942206in}{6.987443in}}%
\pgfpathlineto{\pgfqpoint{12.954519in}{6.971673in}}%
\pgfpathlineto{\pgfqpoint{12.991333in}{6.939306in}}%
\pgfpathlineto{\pgfqpoint{13.015912in}{6.936349in}}%
\pgfpathlineto{\pgfqpoint{13.077373in}{6.910380in}}%
\pgfpathlineto{\pgfqpoint{13.102017in}{6.892316in}}%
\pgfpathlineto{\pgfqpoint{13.102017in}{6.892316in}}%
\pgfusepath{stroke}%
\end{pgfscope}%
\begin{pgfscope}%
\pgfsetrectcap%
\pgfsetmiterjoin%
\pgfsetlinewidth{0.803000pt}%
\definecolor{currentstroke}{rgb}{0.000000,0.000000,0.000000}%
\pgfsetstrokecolor{currentstroke}%
\pgfsetdash{}{0pt}%
\pgfpathmoveto{\pgfqpoint{1.021528in}{5.487778in}}%
\pgfpathlineto{\pgfqpoint{1.021528in}{9.356667in}}%
\pgfusepath{stroke}%
\end{pgfscope}%
\begin{pgfscope}%
\pgfsetrectcap%
\pgfsetmiterjoin%
\pgfsetlinewidth{0.803000pt}%
\definecolor{currentstroke}{rgb}{0.000000,0.000000,0.000000}%
\pgfsetstrokecolor{currentstroke}%
\pgfsetdash{}{0pt}%
\pgfpathmoveto{\pgfqpoint{13.390000in}{5.487778in}}%
\pgfpathlineto{\pgfqpoint{13.390000in}{9.356667in}}%
\pgfusepath{stroke}%
\end{pgfscope}%
\begin{pgfscope}%
\pgfsetrectcap%
\pgfsetmiterjoin%
\pgfsetlinewidth{0.803000pt}%
\definecolor{currentstroke}{rgb}{0.000000,0.000000,0.000000}%
\pgfsetstrokecolor{currentstroke}%
\pgfsetdash{}{0pt}%
\pgfpathmoveto{\pgfqpoint{1.021528in}{5.487778in}}%
\pgfpathlineto{\pgfqpoint{13.390000in}{5.487778in}}%
\pgfusepath{stroke}%
\end{pgfscope}%
\begin{pgfscope}%
\pgfsetrectcap%
\pgfsetmiterjoin%
\pgfsetlinewidth{0.803000pt}%
\definecolor{currentstroke}{rgb}{0.000000,0.000000,0.000000}%
\pgfsetstrokecolor{currentstroke}%
\pgfsetdash{}{0pt}%
\pgfpathmoveto{\pgfqpoint{1.021528in}{9.356667in}}%
\pgfpathlineto{\pgfqpoint{13.390000in}{9.356667in}}%
\pgfusepath{stroke}%
\end{pgfscope}%
\begin{pgfscope}%
\definecolor{textcolor}{rgb}{0.000000,0.000000,0.000000}%
\pgfsetstrokecolor{textcolor}%
\pgfsetfillcolor{textcolor}%
\pgftext[x=7.205764in,y=9.440000in,,base]{\color{textcolor}\sffamily\fontsize{14.400000}{17.280000}\selectfont X Position Error vs Time}%
\end{pgfscope}%
\begin{pgfscope}%
\pgfsetbuttcap%
\pgfsetmiterjoin%
\definecolor{currentfill}{rgb}{1.000000,1.000000,1.000000}%
\pgfsetfillcolor{currentfill}%
\pgfsetfillopacity{0.800000}%
\pgfsetlinewidth{1.003750pt}%
\definecolor{currentstroke}{rgb}{0.800000,0.800000,0.800000}%
\pgfsetstrokecolor{currentstroke}%
\pgfsetstrokeopacity{0.800000}%
\pgfsetdash{}{0pt}%
\pgfpathmoveto{\pgfqpoint{11.715993in}{8.734076in}}%
\pgfpathlineto{\pgfqpoint{13.273333in}{8.734076in}}%
\pgfpathquadraticcurveto{\pgfqpoint{13.306667in}{8.734076in}}{\pgfqpoint{13.306667in}{8.767409in}}%
\pgfpathlineto{\pgfqpoint{13.306667in}{9.240000in}}%
\pgfpathquadraticcurveto{\pgfqpoint{13.306667in}{9.273333in}}{\pgfqpoint{13.273333in}{9.273333in}}%
\pgfpathlineto{\pgfqpoint{11.715993in}{9.273333in}}%
\pgfpathquadraticcurveto{\pgfqpoint{11.682659in}{9.273333in}}{\pgfqpoint{11.682659in}{9.240000in}}%
\pgfpathlineto{\pgfqpoint{11.682659in}{8.767409in}}%
\pgfpathquadraticcurveto{\pgfqpoint{11.682659in}{8.734076in}}{\pgfqpoint{11.715993in}{8.734076in}}%
\pgfpathlineto{\pgfqpoint{11.715993in}{8.734076in}}%
\pgfpathclose%
\pgfusepath{stroke,fill}%
\end{pgfscope}%
\begin{pgfscope}%
\pgfsetrectcap%
\pgfsetroundjoin%
\pgfsetlinewidth{1.505625pt}%
\definecolor{currentstroke}{rgb}{0.121569,0.466667,0.705882}%
\pgfsetstrokecolor{currentstroke}%
\pgfsetdash{}{0pt}%
\pgfpathmoveto{\pgfqpoint{11.749326in}{9.138372in}}%
\pgfpathlineto{\pgfqpoint{11.915993in}{9.138372in}}%
\pgfpathlineto{\pgfqpoint{12.082659in}{9.138372in}}%
\pgfusepath{stroke}%
\end{pgfscope}%
\begin{pgfscope}%
\definecolor{textcolor}{rgb}{0.000000,0.000000,0.000000}%
\pgfsetstrokecolor{textcolor}%
\pgfsetfillcolor{textcolor}%
\pgftext[x=12.215993in,y=9.080039in,left,base]{\color{textcolor}\sffamily\fontsize{12.000000}{14.400000}\selectfont vrpn X Error}%
\end{pgfscope}%
\begin{pgfscope}%
\pgfsetrectcap%
\pgfsetroundjoin%
\pgfsetlinewidth{1.505625pt}%
\definecolor{currentstroke}{rgb}{1.000000,0.498039,0.054902}%
\pgfsetstrokecolor{currentstroke}%
\pgfsetdash{}{0pt}%
\pgfpathmoveto{\pgfqpoint{11.749326in}{8.893744in}}%
\pgfpathlineto{\pgfqpoint{11.915993in}{8.893744in}}%
\pgfpathlineto{\pgfqpoint{12.082659in}{8.893744in}}%
\pgfusepath{stroke}%
\end{pgfscope}%
\begin{pgfscope}%
\definecolor{textcolor}{rgb}{0.000000,0.000000,0.000000}%
\pgfsetstrokecolor{textcolor}%
\pgfsetfillcolor{textcolor}%
\pgftext[x=12.215993in,y=8.835410in,left,base]{\color{textcolor}\sffamily\fontsize{12.000000}{14.400000}\selectfont slam X Error}%
\end{pgfscope}%
\begin{pgfscope}%
\pgfsetbuttcap%
\pgfsetmiterjoin%
\definecolor{currentfill}{rgb}{1.000000,1.000000,1.000000}%
\pgfsetfillcolor{currentfill}%
\pgfsetlinewidth{0.000000pt}%
\definecolor{currentstroke}{rgb}{0.000000,0.000000,0.000000}%
\pgfsetstrokecolor{currentstroke}%
\pgfsetstrokeopacity{0.000000}%
\pgfsetdash{}{0pt}%
\pgfpathmoveto{\pgfqpoint{1.021528in}{0.692778in}}%
\pgfpathlineto{\pgfqpoint{13.390000in}{0.692778in}}%
\pgfpathlineto{\pgfqpoint{13.390000in}{4.561667in}}%
\pgfpathlineto{\pgfqpoint{1.021528in}{4.561667in}}%
\pgfpathlineto{\pgfqpoint{1.021528in}{0.692778in}}%
\pgfpathclose%
\pgfusepath{fill}%
\end{pgfscope}%
\begin{pgfscope}%
\pgfpathrectangle{\pgfqpoint{1.021528in}{0.692778in}}{\pgfqpoint{12.368472in}{3.868889in}}%
\pgfusepath{clip}%
\pgfsetrectcap%
\pgfsetroundjoin%
\pgfsetlinewidth{0.803000pt}%
\definecolor{currentstroke}{rgb}{0.690196,0.690196,0.690196}%
\pgfsetstrokecolor{currentstroke}%
\pgfsetdash{}{0pt}%
\pgfpathmoveto{\pgfqpoint{1.021528in}{0.692778in}}%
\pgfpathlineto{\pgfqpoint{1.021528in}{4.561667in}}%
\pgfusepath{stroke}%
\end{pgfscope}%
\begin{pgfscope}%
\pgfsetbuttcap%
\pgfsetroundjoin%
\definecolor{currentfill}{rgb}{0.000000,0.000000,0.000000}%
\pgfsetfillcolor{currentfill}%
\pgfsetlinewidth{0.803000pt}%
\definecolor{currentstroke}{rgb}{0.000000,0.000000,0.000000}%
\pgfsetstrokecolor{currentstroke}%
\pgfsetdash{}{0pt}%
\pgfsys@defobject{currentmarker}{\pgfqpoint{0.000000in}{-0.048611in}}{\pgfqpoint{0.000000in}{0.000000in}}{%
\pgfpathmoveto{\pgfqpoint{0.000000in}{0.000000in}}%
\pgfpathlineto{\pgfqpoint{0.000000in}{-0.048611in}}%
\pgfusepath{stroke,fill}%
}%
\begin{pgfscope}%
\pgfsys@transformshift{1.021528in}{0.692778in}%
\pgfsys@useobject{currentmarker}{}%
\end{pgfscope}%
\end{pgfscope}%
\begin{pgfscope}%
\definecolor{textcolor}{rgb}{0.000000,0.000000,0.000000}%
\pgfsetstrokecolor{textcolor}%
\pgfsetfillcolor{textcolor}%
\pgftext[x=1.021528in,y=0.595556in,,top]{\color{textcolor}\sffamily\fontsize{12.000000}{14.400000}\selectfont 0}%
\end{pgfscope}%
\begin{pgfscope}%
\pgfpathrectangle{\pgfqpoint{1.021528in}{0.692778in}}{\pgfqpoint{12.368472in}{3.868889in}}%
\pgfusepath{clip}%
\pgfsetrectcap%
\pgfsetroundjoin%
\pgfsetlinewidth{0.803000pt}%
\definecolor{currentstroke}{rgb}{0.690196,0.690196,0.690196}%
\pgfsetstrokecolor{currentstroke}%
\pgfsetdash{}{0pt}%
\pgfpathmoveto{\pgfqpoint{3.209777in}{0.692778in}}%
\pgfpathlineto{\pgfqpoint{3.209777in}{4.561667in}}%
\pgfusepath{stroke}%
\end{pgfscope}%
\begin{pgfscope}%
\pgfsetbuttcap%
\pgfsetroundjoin%
\definecolor{currentfill}{rgb}{0.000000,0.000000,0.000000}%
\pgfsetfillcolor{currentfill}%
\pgfsetlinewidth{0.803000pt}%
\definecolor{currentstroke}{rgb}{0.000000,0.000000,0.000000}%
\pgfsetstrokecolor{currentstroke}%
\pgfsetdash{}{0pt}%
\pgfsys@defobject{currentmarker}{\pgfqpoint{0.000000in}{-0.048611in}}{\pgfqpoint{0.000000in}{0.000000in}}{%
\pgfpathmoveto{\pgfqpoint{0.000000in}{0.000000in}}%
\pgfpathlineto{\pgfqpoint{0.000000in}{-0.048611in}}%
\pgfusepath{stroke,fill}%
}%
\begin{pgfscope}%
\pgfsys@transformshift{3.209777in}{0.692778in}%
\pgfsys@useobject{currentmarker}{}%
\end{pgfscope}%
\end{pgfscope}%
\begin{pgfscope}%
\definecolor{textcolor}{rgb}{0.000000,0.000000,0.000000}%
\pgfsetstrokecolor{textcolor}%
\pgfsetfillcolor{textcolor}%
\pgftext[x=3.209777in,y=0.595556in,,top]{\color{textcolor}\sffamily\fontsize{12.000000}{14.400000}\selectfont 20}%
\end{pgfscope}%
\begin{pgfscope}%
\pgfpathrectangle{\pgfqpoint{1.021528in}{0.692778in}}{\pgfqpoint{12.368472in}{3.868889in}}%
\pgfusepath{clip}%
\pgfsetrectcap%
\pgfsetroundjoin%
\pgfsetlinewidth{0.803000pt}%
\definecolor{currentstroke}{rgb}{0.690196,0.690196,0.690196}%
\pgfsetstrokecolor{currentstroke}%
\pgfsetdash{}{0pt}%
\pgfpathmoveto{\pgfqpoint{5.398027in}{0.692778in}}%
\pgfpathlineto{\pgfqpoint{5.398027in}{4.561667in}}%
\pgfusepath{stroke}%
\end{pgfscope}%
\begin{pgfscope}%
\pgfsetbuttcap%
\pgfsetroundjoin%
\definecolor{currentfill}{rgb}{0.000000,0.000000,0.000000}%
\pgfsetfillcolor{currentfill}%
\pgfsetlinewidth{0.803000pt}%
\definecolor{currentstroke}{rgb}{0.000000,0.000000,0.000000}%
\pgfsetstrokecolor{currentstroke}%
\pgfsetdash{}{0pt}%
\pgfsys@defobject{currentmarker}{\pgfqpoint{0.000000in}{-0.048611in}}{\pgfqpoint{0.000000in}{0.000000in}}{%
\pgfpathmoveto{\pgfqpoint{0.000000in}{0.000000in}}%
\pgfpathlineto{\pgfqpoint{0.000000in}{-0.048611in}}%
\pgfusepath{stroke,fill}%
}%
\begin{pgfscope}%
\pgfsys@transformshift{5.398027in}{0.692778in}%
\pgfsys@useobject{currentmarker}{}%
\end{pgfscope}%
\end{pgfscope}%
\begin{pgfscope}%
\definecolor{textcolor}{rgb}{0.000000,0.000000,0.000000}%
\pgfsetstrokecolor{textcolor}%
\pgfsetfillcolor{textcolor}%
\pgftext[x=5.398027in,y=0.595556in,,top]{\color{textcolor}\sffamily\fontsize{12.000000}{14.400000}\selectfont 40}%
\end{pgfscope}%
\begin{pgfscope}%
\pgfpathrectangle{\pgfqpoint{1.021528in}{0.692778in}}{\pgfqpoint{12.368472in}{3.868889in}}%
\pgfusepath{clip}%
\pgfsetrectcap%
\pgfsetroundjoin%
\pgfsetlinewidth{0.803000pt}%
\definecolor{currentstroke}{rgb}{0.690196,0.690196,0.690196}%
\pgfsetstrokecolor{currentstroke}%
\pgfsetdash{}{0pt}%
\pgfpathmoveto{\pgfqpoint{7.586276in}{0.692778in}}%
\pgfpathlineto{\pgfqpoint{7.586276in}{4.561667in}}%
\pgfusepath{stroke}%
\end{pgfscope}%
\begin{pgfscope}%
\pgfsetbuttcap%
\pgfsetroundjoin%
\definecolor{currentfill}{rgb}{0.000000,0.000000,0.000000}%
\pgfsetfillcolor{currentfill}%
\pgfsetlinewidth{0.803000pt}%
\definecolor{currentstroke}{rgb}{0.000000,0.000000,0.000000}%
\pgfsetstrokecolor{currentstroke}%
\pgfsetdash{}{0pt}%
\pgfsys@defobject{currentmarker}{\pgfqpoint{0.000000in}{-0.048611in}}{\pgfqpoint{0.000000in}{0.000000in}}{%
\pgfpathmoveto{\pgfqpoint{0.000000in}{0.000000in}}%
\pgfpathlineto{\pgfqpoint{0.000000in}{-0.048611in}}%
\pgfusepath{stroke,fill}%
}%
\begin{pgfscope}%
\pgfsys@transformshift{7.586276in}{0.692778in}%
\pgfsys@useobject{currentmarker}{}%
\end{pgfscope}%
\end{pgfscope}%
\begin{pgfscope}%
\definecolor{textcolor}{rgb}{0.000000,0.000000,0.000000}%
\pgfsetstrokecolor{textcolor}%
\pgfsetfillcolor{textcolor}%
\pgftext[x=7.586276in,y=0.595556in,,top]{\color{textcolor}\sffamily\fontsize{12.000000}{14.400000}\selectfont 60}%
\end{pgfscope}%
\begin{pgfscope}%
\pgfpathrectangle{\pgfqpoint{1.021528in}{0.692778in}}{\pgfqpoint{12.368472in}{3.868889in}}%
\pgfusepath{clip}%
\pgfsetrectcap%
\pgfsetroundjoin%
\pgfsetlinewidth{0.803000pt}%
\definecolor{currentstroke}{rgb}{0.690196,0.690196,0.690196}%
\pgfsetstrokecolor{currentstroke}%
\pgfsetdash{}{0pt}%
\pgfpathmoveto{\pgfqpoint{9.774526in}{0.692778in}}%
\pgfpathlineto{\pgfqpoint{9.774526in}{4.561667in}}%
\pgfusepath{stroke}%
\end{pgfscope}%
\begin{pgfscope}%
\pgfsetbuttcap%
\pgfsetroundjoin%
\definecolor{currentfill}{rgb}{0.000000,0.000000,0.000000}%
\pgfsetfillcolor{currentfill}%
\pgfsetlinewidth{0.803000pt}%
\definecolor{currentstroke}{rgb}{0.000000,0.000000,0.000000}%
\pgfsetstrokecolor{currentstroke}%
\pgfsetdash{}{0pt}%
\pgfsys@defobject{currentmarker}{\pgfqpoint{0.000000in}{-0.048611in}}{\pgfqpoint{0.000000in}{0.000000in}}{%
\pgfpathmoveto{\pgfqpoint{0.000000in}{0.000000in}}%
\pgfpathlineto{\pgfqpoint{0.000000in}{-0.048611in}}%
\pgfusepath{stroke,fill}%
}%
\begin{pgfscope}%
\pgfsys@transformshift{9.774526in}{0.692778in}%
\pgfsys@useobject{currentmarker}{}%
\end{pgfscope}%
\end{pgfscope}%
\begin{pgfscope}%
\definecolor{textcolor}{rgb}{0.000000,0.000000,0.000000}%
\pgfsetstrokecolor{textcolor}%
\pgfsetfillcolor{textcolor}%
\pgftext[x=9.774526in,y=0.595556in,,top]{\color{textcolor}\sffamily\fontsize{12.000000}{14.400000}\selectfont 80}%
\end{pgfscope}%
\begin{pgfscope}%
\pgfpathrectangle{\pgfqpoint{1.021528in}{0.692778in}}{\pgfqpoint{12.368472in}{3.868889in}}%
\pgfusepath{clip}%
\pgfsetrectcap%
\pgfsetroundjoin%
\pgfsetlinewidth{0.803000pt}%
\definecolor{currentstroke}{rgb}{0.690196,0.690196,0.690196}%
\pgfsetstrokecolor{currentstroke}%
\pgfsetdash{}{0pt}%
\pgfpathmoveto{\pgfqpoint{11.962775in}{0.692778in}}%
\pgfpathlineto{\pgfqpoint{11.962775in}{4.561667in}}%
\pgfusepath{stroke}%
\end{pgfscope}%
\begin{pgfscope}%
\pgfsetbuttcap%
\pgfsetroundjoin%
\definecolor{currentfill}{rgb}{0.000000,0.000000,0.000000}%
\pgfsetfillcolor{currentfill}%
\pgfsetlinewidth{0.803000pt}%
\definecolor{currentstroke}{rgb}{0.000000,0.000000,0.000000}%
\pgfsetstrokecolor{currentstroke}%
\pgfsetdash{}{0pt}%
\pgfsys@defobject{currentmarker}{\pgfqpoint{0.000000in}{-0.048611in}}{\pgfqpoint{0.000000in}{0.000000in}}{%
\pgfpathmoveto{\pgfqpoint{0.000000in}{0.000000in}}%
\pgfpathlineto{\pgfqpoint{0.000000in}{-0.048611in}}%
\pgfusepath{stroke,fill}%
}%
\begin{pgfscope}%
\pgfsys@transformshift{11.962775in}{0.692778in}%
\pgfsys@useobject{currentmarker}{}%
\end{pgfscope}%
\end{pgfscope}%
\begin{pgfscope}%
\definecolor{textcolor}{rgb}{0.000000,0.000000,0.000000}%
\pgfsetstrokecolor{textcolor}%
\pgfsetfillcolor{textcolor}%
\pgftext[x=11.962775in,y=0.595556in,,top]{\color{textcolor}\sffamily\fontsize{12.000000}{14.400000}\selectfont 100}%
\end{pgfscope}%
\begin{pgfscope}%
\definecolor{textcolor}{rgb}{0.000000,0.000000,0.000000}%
\pgfsetstrokecolor{textcolor}%
\pgfsetfillcolor{textcolor}%
\pgftext[x=7.205764in,y=0.378705in,,top]{\color{textcolor}\sffamily\fontsize{12.000000}{14.400000}\selectfont Time (s)}%
\end{pgfscope}%
\begin{pgfscope}%
\pgfpathrectangle{\pgfqpoint{1.021528in}{0.692778in}}{\pgfqpoint{12.368472in}{3.868889in}}%
\pgfusepath{clip}%
\pgfsetrectcap%
\pgfsetroundjoin%
\pgfsetlinewidth{0.803000pt}%
\definecolor{currentstroke}{rgb}{0.690196,0.690196,0.690196}%
\pgfsetstrokecolor{currentstroke}%
\pgfsetdash{}{0pt}%
\pgfpathmoveto{\pgfqpoint{1.021528in}{0.834850in}}%
\pgfpathlineto{\pgfqpoint{13.390000in}{0.834850in}}%
\pgfusepath{stroke}%
\end{pgfscope}%
\begin{pgfscope}%
\pgfsetbuttcap%
\pgfsetroundjoin%
\definecolor{currentfill}{rgb}{0.000000,0.000000,0.000000}%
\pgfsetfillcolor{currentfill}%
\pgfsetlinewidth{0.803000pt}%
\definecolor{currentstroke}{rgb}{0.000000,0.000000,0.000000}%
\pgfsetstrokecolor{currentstroke}%
\pgfsetdash{}{0pt}%
\pgfsys@defobject{currentmarker}{\pgfqpoint{-0.048611in}{0.000000in}}{\pgfqpoint{-0.000000in}{0.000000in}}{%
\pgfpathmoveto{\pgfqpoint{-0.000000in}{0.000000in}}%
\pgfpathlineto{\pgfqpoint{-0.048611in}{0.000000in}}%
\pgfusepath{stroke,fill}%
}%
\begin{pgfscope}%
\pgfsys@transformshift{1.021528in}{0.834850in}%
\pgfsys@useobject{currentmarker}{}%
\end{pgfscope}%
\end{pgfscope}%
\begin{pgfscope}%
\definecolor{textcolor}{rgb}{0.000000,0.000000,0.000000}%
\pgfsetstrokecolor{textcolor}%
\pgfsetfillcolor{textcolor}%
\pgftext[x=0.423582in, y=0.771536in, left, base]{\color{textcolor}\sffamily\fontsize{12.000000}{14.400000}\selectfont \ensuremath{-}1.50}%
\end{pgfscope}%
\begin{pgfscope}%
\pgfpathrectangle{\pgfqpoint{1.021528in}{0.692778in}}{\pgfqpoint{12.368472in}{3.868889in}}%
\pgfusepath{clip}%
\pgfsetrectcap%
\pgfsetroundjoin%
\pgfsetlinewidth{0.803000pt}%
\definecolor{currentstroke}{rgb}{0.690196,0.690196,0.690196}%
\pgfsetstrokecolor{currentstroke}%
\pgfsetdash{}{0pt}%
\pgfpathmoveto{\pgfqpoint{1.021528in}{1.300447in}}%
\pgfpathlineto{\pgfqpoint{13.390000in}{1.300447in}}%
\pgfusepath{stroke}%
\end{pgfscope}%
\begin{pgfscope}%
\pgfsetbuttcap%
\pgfsetroundjoin%
\definecolor{currentfill}{rgb}{0.000000,0.000000,0.000000}%
\pgfsetfillcolor{currentfill}%
\pgfsetlinewidth{0.803000pt}%
\definecolor{currentstroke}{rgb}{0.000000,0.000000,0.000000}%
\pgfsetstrokecolor{currentstroke}%
\pgfsetdash{}{0pt}%
\pgfsys@defobject{currentmarker}{\pgfqpoint{-0.048611in}{0.000000in}}{\pgfqpoint{-0.000000in}{0.000000in}}{%
\pgfpathmoveto{\pgfqpoint{-0.000000in}{0.000000in}}%
\pgfpathlineto{\pgfqpoint{-0.048611in}{0.000000in}}%
\pgfusepath{stroke,fill}%
}%
\begin{pgfscope}%
\pgfsys@transformshift{1.021528in}{1.300447in}%
\pgfsys@useobject{currentmarker}{}%
\end{pgfscope}%
\end{pgfscope}%
\begin{pgfscope}%
\definecolor{textcolor}{rgb}{0.000000,0.000000,0.000000}%
\pgfsetstrokecolor{textcolor}%
\pgfsetfillcolor{textcolor}%
\pgftext[x=0.423582in, y=1.237133in, left, base]{\color{textcolor}\sffamily\fontsize{12.000000}{14.400000}\selectfont \ensuremath{-}1.25}%
\end{pgfscope}%
\begin{pgfscope}%
\pgfpathrectangle{\pgfqpoint{1.021528in}{0.692778in}}{\pgfqpoint{12.368472in}{3.868889in}}%
\pgfusepath{clip}%
\pgfsetrectcap%
\pgfsetroundjoin%
\pgfsetlinewidth{0.803000pt}%
\definecolor{currentstroke}{rgb}{0.690196,0.690196,0.690196}%
\pgfsetstrokecolor{currentstroke}%
\pgfsetdash{}{0pt}%
\pgfpathmoveto{\pgfqpoint{1.021528in}{1.766043in}}%
\pgfpathlineto{\pgfqpoint{13.390000in}{1.766043in}}%
\pgfusepath{stroke}%
\end{pgfscope}%
\begin{pgfscope}%
\pgfsetbuttcap%
\pgfsetroundjoin%
\definecolor{currentfill}{rgb}{0.000000,0.000000,0.000000}%
\pgfsetfillcolor{currentfill}%
\pgfsetlinewidth{0.803000pt}%
\definecolor{currentstroke}{rgb}{0.000000,0.000000,0.000000}%
\pgfsetstrokecolor{currentstroke}%
\pgfsetdash{}{0pt}%
\pgfsys@defobject{currentmarker}{\pgfqpoint{-0.048611in}{0.000000in}}{\pgfqpoint{-0.000000in}{0.000000in}}{%
\pgfpathmoveto{\pgfqpoint{-0.000000in}{0.000000in}}%
\pgfpathlineto{\pgfqpoint{-0.048611in}{0.000000in}}%
\pgfusepath{stroke,fill}%
}%
\begin{pgfscope}%
\pgfsys@transformshift{1.021528in}{1.766043in}%
\pgfsys@useobject{currentmarker}{}%
\end{pgfscope}%
\end{pgfscope}%
\begin{pgfscope}%
\definecolor{textcolor}{rgb}{0.000000,0.000000,0.000000}%
\pgfsetstrokecolor{textcolor}%
\pgfsetfillcolor{textcolor}%
\pgftext[x=0.423582in, y=1.702730in, left, base]{\color{textcolor}\sffamily\fontsize{12.000000}{14.400000}\selectfont \ensuremath{-}1.00}%
\end{pgfscope}%
\begin{pgfscope}%
\pgfpathrectangle{\pgfqpoint{1.021528in}{0.692778in}}{\pgfqpoint{12.368472in}{3.868889in}}%
\pgfusepath{clip}%
\pgfsetrectcap%
\pgfsetroundjoin%
\pgfsetlinewidth{0.803000pt}%
\definecolor{currentstroke}{rgb}{0.690196,0.690196,0.690196}%
\pgfsetstrokecolor{currentstroke}%
\pgfsetdash{}{0pt}%
\pgfpathmoveto{\pgfqpoint{1.021528in}{2.231640in}}%
\pgfpathlineto{\pgfqpoint{13.390000in}{2.231640in}}%
\pgfusepath{stroke}%
\end{pgfscope}%
\begin{pgfscope}%
\pgfsetbuttcap%
\pgfsetroundjoin%
\definecolor{currentfill}{rgb}{0.000000,0.000000,0.000000}%
\pgfsetfillcolor{currentfill}%
\pgfsetlinewidth{0.803000pt}%
\definecolor{currentstroke}{rgb}{0.000000,0.000000,0.000000}%
\pgfsetstrokecolor{currentstroke}%
\pgfsetdash{}{0pt}%
\pgfsys@defobject{currentmarker}{\pgfqpoint{-0.048611in}{0.000000in}}{\pgfqpoint{-0.000000in}{0.000000in}}{%
\pgfpathmoveto{\pgfqpoint{-0.000000in}{0.000000in}}%
\pgfpathlineto{\pgfqpoint{-0.048611in}{0.000000in}}%
\pgfusepath{stroke,fill}%
}%
\begin{pgfscope}%
\pgfsys@transformshift{1.021528in}{2.231640in}%
\pgfsys@useobject{currentmarker}{}%
\end{pgfscope}%
\end{pgfscope}%
\begin{pgfscope}%
\definecolor{textcolor}{rgb}{0.000000,0.000000,0.000000}%
\pgfsetstrokecolor{textcolor}%
\pgfsetfillcolor{textcolor}%
\pgftext[x=0.423582in, y=2.168326in, left, base]{\color{textcolor}\sffamily\fontsize{12.000000}{14.400000}\selectfont \ensuremath{-}0.75}%
\end{pgfscope}%
\begin{pgfscope}%
\pgfpathrectangle{\pgfqpoint{1.021528in}{0.692778in}}{\pgfqpoint{12.368472in}{3.868889in}}%
\pgfusepath{clip}%
\pgfsetrectcap%
\pgfsetroundjoin%
\pgfsetlinewidth{0.803000pt}%
\definecolor{currentstroke}{rgb}{0.690196,0.690196,0.690196}%
\pgfsetstrokecolor{currentstroke}%
\pgfsetdash{}{0pt}%
\pgfpathmoveto{\pgfqpoint{1.021528in}{2.697237in}}%
\pgfpathlineto{\pgfqpoint{13.390000in}{2.697237in}}%
\pgfusepath{stroke}%
\end{pgfscope}%
\begin{pgfscope}%
\pgfsetbuttcap%
\pgfsetroundjoin%
\definecolor{currentfill}{rgb}{0.000000,0.000000,0.000000}%
\pgfsetfillcolor{currentfill}%
\pgfsetlinewidth{0.803000pt}%
\definecolor{currentstroke}{rgb}{0.000000,0.000000,0.000000}%
\pgfsetstrokecolor{currentstroke}%
\pgfsetdash{}{0pt}%
\pgfsys@defobject{currentmarker}{\pgfqpoint{-0.048611in}{0.000000in}}{\pgfqpoint{-0.000000in}{0.000000in}}{%
\pgfpathmoveto{\pgfqpoint{-0.000000in}{0.000000in}}%
\pgfpathlineto{\pgfqpoint{-0.048611in}{0.000000in}}%
\pgfusepath{stroke,fill}%
}%
\begin{pgfscope}%
\pgfsys@transformshift{1.021528in}{2.697237in}%
\pgfsys@useobject{currentmarker}{}%
\end{pgfscope}%
\end{pgfscope}%
\begin{pgfscope}%
\definecolor{textcolor}{rgb}{0.000000,0.000000,0.000000}%
\pgfsetstrokecolor{textcolor}%
\pgfsetfillcolor{textcolor}%
\pgftext[x=0.423582in, y=2.633923in, left, base]{\color{textcolor}\sffamily\fontsize{12.000000}{14.400000}\selectfont \ensuremath{-}0.50}%
\end{pgfscope}%
\begin{pgfscope}%
\pgfpathrectangle{\pgfqpoint{1.021528in}{0.692778in}}{\pgfqpoint{12.368472in}{3.868889in}}%
\pgfusepath{clip}%
\pgfsetrectcap%
\pgfsetroundjoin%
\pgfsetlinewidth{0.803000pt}%
\definecolor{currentstroke}{rgb}{0.690196,0.690196,0.690196}%
\pgfsetstrokecolor{currentstroke}%
\pgfsetdash{}{0pt}%
\pgfpathmoveto{\pgfqpoint{1.021528in}{3.162834in}}%
\pgfpathlineto{\pgfqpoint{13.390000in}{3.162834in}}%
\pgfusepath{stroke}%
\end{pgfscope}%
\begin{pgfscope}%
\pgfsetbuttcap%
\pgfsetroundjoin%
\definecolor{currentfill}{rgb}{0.000000,0.000000,0.000000}%
\pgfsetfillcolor{currentfill}%
\pgfsetlinewidth{0.803000pt}%
\definecolor{currentstroke}{rgb}{0.000000,0.000000,0.000000}%
\pgfsetstrokecolor{currentstroke}%
\pgfsetdash{}{0pt}%
\pgfsys@defobject{currentmarker}{\pgfqpoint{-0.048611in}{0.000000in}}{\pgfqpoint{-0.000000in}{0.000000in}}{%
\pgfpathmoveto{\pgfqpoint{-0.000000in}{0.000000in}}%
\pgfpathlineto{\pgfqpoint{-0.048611in}{0.000000in}}%
\pgfusepath{stroke,fill}%
}%
\begin{pgfscope}%
\pgfsys@transformshift{1.021528in}{3.162834in}%
\pgfsys@useobject{currentmarker}{}%
\end{pgfscope}%
\end{pgfscope}%
\begin{pgfscope}%
\definecolor{textcolor}{rgb}{0.000000,0.000000,0.000000}%
\pgfsetstrokecolor{textcolor}%
\pgfsetfillcolor{textcolor}%
\pgftext[x=0.423582in, y=3.099520in, left, base]{\color{textcolor}\sffamily\fontsize{12.000000}{14.400000}\selectfont \ensuremath{-}0.25}%
\end{pgfscope}%
\begin{pgfscope}%
\pgfpathrectangle{\pgfqpoint{1.021528in}{0.692778in}}{\pgfqpoint{12.368472in}{3.868889in}}%
\pgfusepath{clip}%
\pgfsetrectcap%
\pgfsetroundjoin%
\pgfsetlinewidth{0.803000pt}%
\definecolor{currentstroke}{rgb}{0.690196,0.690196,0.690196}%
\pgfsetstrokecolor{currentstroke}%
\pgfsetdash{}{0pt}%
\pgfpathmoveto{\pgfqpoint{1.021528in}{3.628430in}}%
\pgfpathlineto{\pgfqpoint{13.390000in}{3.628430in}}%
\pgfusepath{stroke}%
\end{pgfscope}%
\begin{pgfscope}%
\pgfsetbuttcap%
\pgfsetroundjoin%
\definecolor{currentfill}{rgb}{0.000000,0.000000,0.000000}%
\pgfsetfillcolor{currentfill}%
\pgfsetlinewidth{0.803000pt}%
\definecolor{currentstroke}{rgb}{0.000000,0.000000,0.000000}%
\pgfsetstrokecolor{currentstroke}%
\pgfsetdash{}{0pt}%
\pgfsys@defobject{currentmarker}{\pgfqpoint{-0.048611in}{0.000000in}}{\pgfqpoint{-0.000000in}{0.000000in}}{%
\pgfpathmoveto{\pgfqpoint{-0.000000in}{0.000000in}}%
\pgfpathlineto{\pgfqpoint{-0.048611in}{0.000000in}}%
\pgfusepath{stroke,fill}%
}%
\begin{pgfscope}%
\pgfsys@transformshift{1.021528in}{3.628430in}%
\pgfsys@useobject{currentmarker}{}%
\end{pgfscope}%
\end{pgfscope}%
\begin{pgfscope}%
\definecolor{textcolor}{rgb}{0.000000,0.000000,0.000000}%
\pgfsetstrokecolor{textcolor}%
\pgfsetfillcolor{textcolor}%
\pgftext[x=0.553212in, y=3.565117in, left, base]{\color{textcolor}\sffamily\fontsize{12.000000}{14.400000}\selectfont 0.00}%
\end{pgfscope}%
\begin{pgfscope}%
\pgfpathrectangle{\pgfqpoint{1.021528in}{0.692778in}}{\pgfqpoint{12.368472in}{3.868889in}}%
\pgfusepath{clip}%
\pgfsetrectcap%
\pgfsetroundjoin%
\pgfsetlinewidth{0.803000pt}%
\definecolor{currentstroke}{rgb}{0.690196,0.690196,0.690196}%
\pgfsetstrokecolor{currentstroke}%
\pgfsetdash{}{0pt}%
\pgfpathmoveto{\pgfqpoint{1.021528in}{4.094027in}}%
\pgfpathlineto{\pgfqpoint{13.390000in}{4.094027in}}%
\pgfusepath{stroke}%
\end{pgfscope}%
\begin{pgfscope}%
\pgfsetbuttcap%
\pgfsetroundjoin%
\definecolor{currentfill}{rgb}{0.000000,0.000000,0.000000}%
\pgfsetfillcolor{currentfill}%
\pgfsetlinewidth{0.803000pt}%
\definecolor{currentstroke}{rgb}{0.000000,0.000000,0.000000}%
\pgfsetstrokecolor{currentstroke}%
\pgfsetdash{}{0pt}%
\pgfsys@defobject{currentmarker}{\pgfqpoint{-0.048611in}{0.000000in}}{\pgfqpoint{-0.000000in}{0.000000in}}{%
\pgfpathmoveto{\pgfqpoint{-0.000000in}{0.000000in}}%
\pgfpathlineto{\pgfqpoint{-0.048611in}{0.000000in}}%
\pgfusepath{stroke,fill}%
}%
\begin{pgfscope}%
\pgfsys@transformshift{1.021528in}{4.094027in}%
\pgfsys@useobject{currentmarker}{}%
\end{pgfscope}%
\end{pgfscope}%
\begin{pgfscope}%
\definecolor{textcolor}{rgb}{0.000000,0.000000,0.000000}%
\pgfsetstrokecolor{textcolor}%
\pgfsetfillcolor{textcolor}%
\pgftext[x=0.553212in, y=4.030713in, left, base]{\color{textcolor}\sffamily\fontsize{12.000000}{14.400000}\selectfont 0.25}%
\end{pgfscope}%
\begin{pgfscope}%
\pgfpathrectangle{\pgfqpoint{1.021528in}{0.692778in}}{\pgfqpoint{12.368472in}{3.868889in}}%
\pgfusepath{clip}%
\pgfsetrectcap%
\pgfsetroundjoin%
\pgfsetlinewidth{0.803000pt}%
\definecolor{currentstroke}{rgb}{0.690196,0.690196,0.690196}%
\pgfsetstrokecolor{currentstroke}%
\pgfsetdash{}{0pt}%
\pgfpathmoveto{\pgfqpoint{1.021528in}{4.559624in}}%
\pgfpathlineto{\pgfqpoint{13.390000in}{4.559624in}}%
\pgfusepath{stroke}%
\end{pgfscope}%
\begin{pgfscope}%
\pgfsetbuttcap%
\pgfsetroundjoin%
\definecolor{currentfill}{rgb}{0.000000,0.000000,0.000000}%
\pgfsetfillcolor{currentfill}%
\pgfsetlinewidth{0.803000pt}%
\definecolor{currentstroke}{rgb}{0.000000,0.000000,0.000000}%
\pgfsetstrokecolor{currentstroke}%
\pgfsetdash{}{0pt}%
\pgfsys@defobject{currentmarker}{\pgfqpoint{-0.048611in}{0.000000in}}{\pgfqpoint{-0.000000in}{0.000000in}}{%
\pgfpathmoveto{\pgfqpoint{-0.000000in}{0.000000in}}%
\pgfpathlineto{\pgfqpoint{-0.048611in}{0.000000in}}%
\pgfusepath{stroke,fill}%
}%
\begin{pgfscope}%
\pgfsys@transformshift{1.021528in}{4.559624in}%
\pgfsys@useobject{currentmarker}{}%
\end{pgfscope}%
\end{pgfscope}%
\begin{pgfscope}%
\definecolor{textcolor}{rgb}{0.000000,0.000000,0.000000}%
\pgfsetstrokecolor{textcolor}%
\pgfsetfillcolor{textcolor}%
\pgftext[x=0.553212in, y=4.496310in, left, base]{\color{textcolor}\sffamily\fontsize{12.000000}{14.400000}\selectfont 0.50}%
\end{pgfscope}%
\begin{pgfscope}%
\definecolor{textcolor}{rgb}{0.000000,0.000000,0.000000}%
\pgfsetstrokecolor{textcolor}%
\pgfsetfillcolor{textcolor}%
\pgftext[x=0.368026in,y=2.627222in,,bottom,rotate=90.000000]{\color{textcolor}\sffamily\fontsize{12.000000}{14.400000}\selectfont Y Error}%
\end{pgfscope}%
\begin{pgfscope}%
\pgfpathrectangle{\pgfqpoint{1.021528in}{0.692778in}}{\pgfqpoint{12.368472in}{3.868889in}}%
\pgfusepath{clip}%
\pgfsetrectcap%
\pgfsetroundjoin%
\pgfsetlinewidth{1.505625pt}%
\definecolor{currentstroke}{rgb}{0.121569,0.466667,0.705882}%
\pgfsetstrokecolor{currentstroke}%
\pgfsetdash{}{0pt}%
\pgfpathmoveto{\pgfqpoint{1.021528in}{0.868766in}}%
\pgfpathlineto{\pgfqpoint{1.028822in}{0.868841in}}%
\pgfpathlineto{\pgfqpoint{1.033685in}{0.868710in}}%
\pgfpathlineto{\pgfqpoint{1.036724in}{0.868857in}}%
\pgfpathlineto{\pgfqpoint{1.045234in}{0.868753in}}%
\pgfpathlineto{\pgfqpoint{1.054352in}{0.868929in}}%
\pgfpathlineto{\pgfqpoint{1.059214in}{0.868700in}}%
\pgfpathlineto{\pgfqpoint{1.063469in}{0.868995in}}%
\pgfpathlineto{\pgfqpoint{1.068332in}{0.868678in}}%
\pgfpathlineto{\pgfqpoint{1.082313in}{0.868762in}}%
\pgfpathlineto{\pgfqpoint{1.091430in}{0.868889in}}%
\pgfpathlineto{\pgfqpoint{1.105411in}{0.868665in}}%
\pgfpathlineto{\pgfqpoint{1.110881in}{0.868808in}}%
\pgfpathlineto{\pgfqpoint{1.141882in}{0.868709in}}%
\pgfpathlineto{\pgfqpoint{1.145529in}{0.868847in}}%
\pgfpathlineto{\pgfqpoint{1.152215in}{0.868733in}}%
\pgfpathlineto{\pgfqpoint{1.160117in}{0.868932in}}%
\pgfpathlineto{\pgfqpoint{1.164372in}{0.868782in}}%
\pgfpathlineto{\pgfqpoint{1.186862in}{0.868787in}}%
\pgfpathlineto{\pgfqpoint{1.192333in}{0.868762in}}%
\pgfpathlineto{\pgfqpoint{1.349157in}{0.868967in}}%
\pgfpathlineto{\pgfqpoint{1.350981in}{0.885607in}}%
\pgfpathlineto{\pgfqpoint{1.368608in}{0.943605in}}%
\pgfpathlineto{\pgfqpoint{1.376511in}{0.970046in}}%
\pgfpathlineto{\pgfqpoint{1.402040in}{1.040106in}}%
\pgfpathlineto{\pgfqpoint{1.405687in}{1.047525in}}%
\pgfpathlineto{\pgfqpoint{1.409942in}{1.056628in}}%
\pgfpathlineto{\pgfqpoint{1.419668in}{1.070195in}}%
\pgfpathlineto{\pgfqpoint{1.423315in}{1.072750in}}%
\pgfpathlineto{\pgfqpoint{1.425746in}{1.074233in}}%
\pgfpathlineto{\pgfqpoint{1.426962in}{1.074255in}}%
\pgfpathlineto{\pgfqpoint{1.428785in}{1.074582in}}%
\pgfpathlineto{\pgfqpoint{1.436687in}{1.075884in}}%
\pgfpathlineto{\pgfqpoint{1.447629in}{1.087623in}}%
\pgfpathlineto{\pgfqpoint{1.460393in}{1.124534in}}%
\pgfpathlineto{\pgfqpoint{1.468295in}{1.158700in}}%
\pgfpathlineto{\pgfqpoint{1.478629in}{1.213460in}}%
\pgfpathlineto{\pgfqpoint{1.488962in}{1.280349in}}%
\pgfpathlineto{\pgfqpoint{1.511453in}{1.475167in}}%
\pgfpathlineto{\pgfqpoint{1.526041in}{1.638018in}}%
\pgfpathlineto{\pgfqpoint{1.554610in}{2.002925in}}%
\pgfpathlineto{\pgfqpoint{1.572845in}{2.250072in}}%
\pgfpathlineto{\pgfqpoint{1.625728in}{2.968235in}}%
\pgfpathlineto{\pgfqpoint{1.651865in}{3.265602in}}%
\pgfpathlineto{\pgfqpoint{1.670101in}{3.438975in}}%
\pgfpathlineto{\pgfqpoint{1.684081in}{3.558549in}}%
\pgfpathlineto{\pgfqpoint{1.698062in}{3.660446in}}%
\pgfpathlineto{\pgfqpoint{1.713866in}{3.757022in}}%
\pgfpathlineto{\pgfqpoint{1.730278in}{3.838515in}}%
\pgfpathlineto{\pgfqpoint{1.748513in}{3.905503in}}%
\pgfpathlineto{\pgfqpoint{1.764317in}{3.945263in}}%
\pgfpathlineto{\pgfqpoint{1.798356in}{4.003360in}}%
\pgfpathlineto{\pgfqpoint{1.799572in}{4.004708in}}%
\pgfpathlineto{\pgfqpoint{1.803219in}{4.009374in}}%
\pgfpathlineto{\pgfqpoint{1.805651in}{4.012302in}}%
\pgfpathlineto{\pgfqpoint{1.811729in}{4.020697in}}%
\pgfpathlineto{\pgfqpoint{1.819631in}{4.029216in}}%
\pgfpathlineto{\pgfqpoint{1.822062in}{4.032145in}}%
\pgfpathlineto{\pgfqpoint{1.823278in}{4.033630in}}%
\pgfpathlineto{\pgfqpoint{1.826317in}{4.037017in}}%
\pgfpathlineto{\pgfqpoint{1.827533in}{4.037301in}}%
\pgfpathlineto{\pgfqpoint{1.831788in}{4.041652in}}%
\pgfpathlineto{\pgfqpoint{1.838474in}{4.048173in}}%
\pgfpathlineto{\pgfqpoint{1.845768in}{4.053343in}}%
\pgfpathlineto{\pgfqpoint{1.846984in}{4.053764in}}%
\pgfpathlineto{\pgfqpoint{1.859141in}{4.061990in}}%
\pgfpathlineto{\pgfqpoint{1.865220in}{4.063996in}}%
\pgfpathlineto{\pgfqpoint{1.867651in}{4.065371in}}%
\pgfpathlineto{\pgfqpoint{1.873122in}{4.067197in}}%
\pgfpathlineto{\pgfqpoint{1.874945in}{4.067468in}}%
\pgfpathlineto{\pgfqpoint{1.876769in}{4.069081in}}%
\pgfpathlineto{\pgfqpoint{1.877376in}{4.068167in}}%
\pgfpathlineto{\pgfqpoint{1.877984in}{4.068575in}}%
\pgfpathlineto{\pgfqpoint{1.884671in}{4.070610in}}%
\pgfpathlineto{\pgfqpoint{1.887102in}{4.071596in}}%
\pgfpathlineto{\pgfqpoint{1.890749in}{4.072198in}}%
\pgfpathlineto{\pgfqpoint{1.898651in}{4.074529in}}%
\pgfpathlineto{\pgfqpoint{1.899867in}{4.075128in}}%
\pgfpathlineto{\pgfqpoint{1.901083in}{4.074602in}}%
\pgfpathlineto{\pgfqpoint{1.904122in}{4.076126in}}%
\pgfpathlineto{\pgfqpoint{1.913847in}{4.078052in}}%
\pgfpathlineto{\pgfqpoint{1.922965in}{4.079487in}}%
\pgfpathlineto{\pgfqpoint{1.924789in}{4.079363in}}%
\pgfpathlineto{\pgfqpoint{1.927220in}{4.079525in}}%
\pgfpathlineto{\pgfqpoint{1.929044in}{4.079512in}}%
\pgfpathlineto{\pgfqpoint{1.935122in}{4.078956in}}%
\pgfpathlineto{\pgfqpoint{1.937553in}{4.079954in}}%
\pgfpathlineto{\pgfqpoint{1.947279in}{4.078817in}}%
\pgfpathlineto{\pgfqpoint{1.949102in}{4.079402in}}%
\pgfpathlineto{\pgfqpoint{1.950926in}{4.079203in}}%
\pgfpathlineto{\pgfqpoint{1.953357in}{4.078539in}}%
\pgfpathlineto{\pgfqpoint{1.957004in}{4.077685in}}%
\pgfpathlineto{\pgfqpoint{1.962475in}{4.076207in}}%
\pgfpathlineto{\pgfqpoint{1.964299in}{4.076279in}}%
\pgfpathlineto{\pgfqpoint{1.967338in}{4.075726in}}%
\pgfpathlineto{\pgfqpoint{1.970377in}{4.075303in}}%
\pgfpathlineto{\pgfqpoint{1.974024in}{4.074601in}}%
\pgfpathlineto{\pgfqpoint{1.975848in}{4.074238in}}%
\pgfpathlineto{\pgfqpoint{1.978279in}{4.073829in}}%
\pgfpathlineto{\pgfqpoint{1.979495in}{4.073126in}}%
\pgfpathlineto{\pgfqpoint{1.984358in}{4.072111in}}%
\pgfpathlineto{\pgfqpoint{1.987397in}{4.070669in}}%
\pgfpathlineto{\pgfqpoint{1.989220in}{4.070585in}}%
\pgfpathlineto{\pgfqpoint{1.992260in}{4.069800in}}%
\pgfpathlineto{\pgfqpoint{1.994691in}{4.069053in}}%
\pgfpathlineto{\pgfqpoint{2.003809in}{4.064932in}}%
\pgfpathlineto{\pgfqpoint{2.005024in}{4.064850in}}%
\pgfpathlineto{\pgfqpoint{2.008064in}{4.063375in}}%
\pgfpathlineto{\pgfqpoint{2.012319in}{4.061626in}}%
\pgfpathlineto{\pgfqpoint{2.014750in}{4.060882in}}%
\pgfpathlineto{\pgfqpoint{2.017789in}{4.059800in}}%
\pgfpathlineto{\pgfqpoint{2.039064in}{4.053921in}}%
\pgfpathlineto{\pgfqpoint{2.045142in}{4.052637in}}%
\pgfpathlineto{\pgfqpoint{2.047574in}{4.052345in}}%
\pgfpathlineto{\pgfqpoint{2.059123in}{4.048143in}}%
\pgfpathlineto{\pgfqpoint{2.070064in}{4.044856in}}%
\pgfpathlineto{\pgfqpoint{2.076750in}{4.042794in}}%
\pgfpathlineto{\pgfqpoint{2.079790in}{4.042496in}}%
\pgfpathlineto{\pgfqpoint{2.093162in}{4.037218in}}%
\pgfpathlineto{\pgfqpoint{2.104711in}{4.035324in}}%
\pgfpathlineto{\pgfqpoint{2.107751in}{4.034350in}}%
\pgfpathlineto{\pgfqpoint{2.112613in}{4.033350in}}%
\pgfpathlineto{\pgfqpoint{2.118084in}{4.031739in}}%
\pgfpathlineto{\pgfqpoint{2.120515in}{4.031418in}}%
\pgfpathlineto{\pgfqpoint{2.122947in}{4.031142in}}%
\pgfpathlineto{\pgfqpoint{2.133280in}{4.028339in}}%
\pgfpathlineto{\pgfqpoint{2.153947in}{4.024733in}}%
\pgfpathlineto{\pgfqpoint{2.156378in}{4.024982in}}%
\pgfpathlineto{\pgfqpoint{2.167927in}{4.022996in}}%
\pgfpathlineto{\pgfqpoint{2.170359in}{4.022992in}}%
\pgfpathlineto{\pgfqpoint{2.176437in}{4.022457in}}%
\pgfpathlineto{\pgfqpoint{2.178261in}{4.022855in}}%
\pgfpathlineto{\pgfqpoint{2.186163in}{4.022002in}}%
\pgfpathlineto{\pgfqpoint{2.241477in}{4.016323in}}%
\pgfpathlineto{\pgfqpoint{2.243300in}{4.016070in}}%
\pgfpathlineto{\pgfqpoint{2.249379in}{4.015919in}}%
\pgfpathlineto{\pgfqpoint{2.276732in}{4.013788in}}%
\pgfpathlineto{\pgfqpoint{2.279163in}{4.013918in}}%
\pgfpathlineto{\pgfqpoint{2.285242in}{4.013931in}}%
\pgfpathlineto{\pgfqpoint{2.288889in}{4.014442in}}%
\pgfpathlineto{\pgfqpoint{2.291320in}{4.013621in}}%
\pgfpathlineto{\pgfqpoint{2.294967in}{4.013751in}}%
\pgfpathlineto{\pgfqpoint{2.301654in}{4.012880in}}%
\pgfpathlineto{\pgfqpoint{2.312595in}{4.013235in}}%
\pgfpathlineto{\pgfqpoint{2.315026in}{4.012689in}}%
\pgfpathlineto{\pgfqpoint{2.317458in}{4.012393in}}%
\pgfpathlineto{\pgfqpoint{2.361831in}{4.011226in}}%
\pgfpathlineto{\pgfqpoint{2.373988in}{4.012088in}}%
\pgfpathlineto{\pgfqpoint{2.380674in}{4.012032in}}%
\pgfpathlineto{\pgfqpoint{2.383105in}{4.011679in}}%
\pgfpathlineto{\pgfqpoint{2.388576in}{4.011049in}}%
\pgfpathlineto{\pgfqpoint{2.391007in}{4.011104in}}%
\pgfpathlineto{\pgfqpoint{2.397086in}{4.010548in}}%
\pgfpathlineto{\pgfqpoint{2.400125in}{4.011292in}}%
\pgfpathlineto{\pgfqpoint{2.403772in}{4.011049in}}%
\pgfpathlineto{\pgfqpoint{2.406811in}{4.011263in}}%
\pgfpathlineto{\pgfqpoint{2.412890in}{4.010865in}}%
\pgfpathlineto{\pgfqpoint{2.415321in}{4.010587in}}%
\pgfpathlineto{\pgfqpoint{2.422007in}{4.010814in}}%
\pgfpathlineto{\pgfqpoint{2.426262in}{4.010737in}}%
\pgfpathlineto{\pgfqpoint{2.428694in}{4.010791in}}%
\pgfpathlineto{\pgfqpoint{2.431733in}{4.010636in}}%
\pgfpathlineto{\pgfqpoint{2.451184in}{4.009958in}}%
\pgfpathlineto{\pgfqpoint{2.453008in}{4.009632in}}%
\pgfpathlineto{\pgfqpoint{2.455439in}{4.009936in}}%
\pgfpathlineto{\pgfqpoint{2.465772in}{4.010102in}}%
\pgfpathlineto{\pgfqpoint{2.468204in}{4.009884in}}%
\pgfpathlineto{\pgfqpoint{2.470027in}{4.010172in}}%
\pgfpathlineto{\pgfqpoint{2.476714in}{4.010007in}}%
\pgfpathlineto{\pgfqpoint{2.488871in}{4.008776in}}%
\pgfpathlineto{\pgfqpoint{2.491302in}{4.008825in}}%
\pgfpathlineto{\pgfqpoint{2.496165in}{4.009482in}}%
\pgfpathlineto{\pgfqpoint{2.498596in}{4.009178in}}%
\pgfpathlineto{\pgfqpoint{2.505890in}{4.009587in}}%
\pgfpathlineto{\pgfqpoint{2.507714in}{4.009569in}}%
\pgfpathlineto{\pgfqpoint{2.509537in}{4.009691in}}%
\pgfpathlineto{\pgfqpoint{2.511361in}{4.009288in}}%
\pgfpathlineto{\pgfqpoint{2.516224in}{4.009727in}}%
\pgfpathlineto{\pgfqpoint{2.518047in}{4.009688in}}%
\pgfpathlineto{\pgfqpoint{2.529596in}{4.009405in}}%
\pgfpathlineto{\pgfqpoint{2.533243in}{4.009853in}}%
\pgfpathlineto{\pgfqpoint{2.536891in}{4.009279in}}%
\pgfpathlineto{\pgfqpoint{2.541753in}{4.009285in}}%
\pgfpathlineto{\pgfqpoint{2.544185in}{4.009357in}}%
\pgfpathlineto{\pgfqpoint{2.553910in}{4.008872in}}%
\pgfpathlineto{\pgfqpoint{2.559989in}{4.008933in}}%
\pgfpathlineto{\pgfqpoint{2.562420in}{4.008865in}}%
\pgfpathlineto{\pgfqpoint{2.568499in}{4.008999in}}%
\pgfpathlineto{\pgfqpoint{2.583087in}{4.007588in}}%
\pgfpathlineto{\pgfqpoint{2.587950in}{4.001898in}}%
\pgfpathlineto{\pgfqpoint{2.590989in}{4.008550in}}%
\pgfpathlineto{\pgfqpoint{2.597067in}{4.009525in}}%
\pgfpathlineto{\pgfqpoint{2.614087in}{4.008609in}}%
\pgfpathlineto{\pgfqpoint{2.617734in}{4.008335in}}%
\pgfpathlineto{\pgfqpoint{2.634754in}{4.009464in}}%
\pgfpathlineto{\pgfqpoint{2.638401in}{4.009248in}}%
\pgfpathlineto{\pgfqpoint{2.642656in}{4.009743in}}%
\pgfpathlineto{\pgfqpoint{2.645695in}{4.009405in}}%
\pgfpathlineto{\pgfqpoint{2.654205in}{4.009335in}}%
\pgfpathlineto{\pgfqpoint{2.656636in}{4.008835in}}%
\pgfpathlineto{\pgfqpoint{2.661499in}{4.010323in}}%
\pgfpathlineto{\pgfqpoint{2.666362in}{4.009831in}}%
\pgfpathlineto{\pgfqpoint{2.670617in}{4.010459in}}%
\pgfpathlineto{\pgfqpoint{2.673656in}{4.010272in}}%
\pgfpathlineto{\pgfqpoint{2.684597in}{4.010214in}}%
\pgfpathlineto{\pgfqpoint{2.699186in}{4.010700in}}%
\pgfpathlineto{\pgfqpoint{2.705872in}{4.010989in}}%
\pgfpathlineto{\pgfqpoint{2.708303in}{4.011215in}}%
\pgfpathlineto{\pgfqpoint{2.732010in}{4.011903in}}%
\pgfpathlineto{\pgfqpoint{2.734441in}{4.012007in}}%
\pgfpathlineto{\pgfqpoint{2.738696in}{4.012506in}}%
\pgfpathlineto{\pgfqpoint{2.789147in}{4.012511in}}%
\pgfpathlineto{\pgfqpoint{2.791578in}{4.012460in}}%
\pgfpathlineto{\pgfqpoint{2.807382in}{4.012897in}}%
\pgfpathlineto{\pgfqpoint{2.811030in}{4.013413in}}%
\pgfpathlineto{\pgfqpoint{2.884579in}{4.012596in}}%
\pgfpathlineto{\pgfqpoint{2.887010in}{4.012221in}}%
\pgfpathlineto{\pgfqpoint{2.890658in}{4.013151in}}%
\pgfpathlineto{\pgfqpoint{2.897344in}{4.013156in}}%
\pgfpathlineto{\pgfqpoint{2.900383in}{4.012787in}}%
\pgfpathlineto{\pgfqpoint{2.906462in}{4.012079in}}%
\pgfpathlineto{\pgfqpoint{2.949011in}{4.011669in}}%
\pgfpathlineto{\pgfqpoint{2.953266in}{4.011507in}}%
\pgfpathlineto{\pgfqpoint{2.959344in}{4.011267in}}%
\pgfpathlineto{\pgfqpoint{2.982442in}{4.011445in}}%
\pgfpathlineto{\pgfqpoint{2.983658in}{4.011114in}}%
\pgfpathlineto{\pgfqpoint{3.018913in}{3.791422in}}%
\pgfpathlineto{\pgfqpoint{3.040188in}{3.679468in}}%
\pgfpathlineto{\pgfqpoint{3.065718in}{3.568973in}}%
\pgfpathlineto{\pgfqpoint{3.079698in}{3.511233in}}%
\pgfpathlineto{\pgfqpoint{3.085776in}{3.485744in}}%
\pgfpathlineto{\pgfqpoint{3.119816in}{3.343952in}}%
\pgfpathlineto{\pgfqpoint{3.162365in}{3.153939in}}%
\pgfpathlineto{\pgfqpoint{3.165404in}{3.143650in}}%
\pgfpathlineto{\pgfqpoint{3.186679in}{3.057569in}}%
\pgfpathlineto{\pgfqpoint{3.198836in}{3.018958in}}%
\pgfpathlineto{\pgfqpoint{3.201267in}{3.012876in}}%
\pgfpathlineto{\pgfqpoint{3.207954in}{2.997096in}}%
\pgfpathlineto{\pgfqpoint{3.218895in}{2.979637in}}%
\pgfpathlineto{\pgfqpoint{3.224973in}{2.973706in}}%
\pgfpathlineto{\pgfqpoint{3.230444in}{2.970630in}}%
\pgfpathlineto{\pgfqpoint{3.232875in}{2.970864in}}%
\pgfpathlineto{\pgfqpoint{3.234699in}{2.970065in}}%
\pgfpathlineto{\pgfqpoint{3.237738in}{2.969911in}}%
\pgfpathlineto{\pgfqpoint{3.242601in}{2.969571in}}%
\pgfpathlineto{\pgfqpoint{3.246856in}{2.971790in}}%
\pgfpathlineto{\pgfqpoint{3.258405in}{2.982073in}}%
\pgfpathlineto{\pgfqpoint{3.265091in}{2.991128in}}%
\pgfpathlineto{\pgfqpoint{3.293660in}{3.036023in}}%
\pgfpathlineto{\pgfqpoint{3.298523in}{3.045002in}}%
\pgfpathlineto{\pgfqpoint{3.302170in}{3.049812in}}%
\pgfpathlineto{\pgfqpoint{3.311288in}{3.063434in}}%
\pgfpathlineto{\pgfqpoint{3.334386in}{3.095954in}}%
\pgfpathlineto{\pgfqpoint{3.338641in}{3.101048in}}%
\pgfpathlineto{\pgfqpoint{3.343504in}{3.107202in}}%
\pgfpathlineto{\pgfqpoint{3.347759in}{3.112376in}}%
\pgfpathlineto{\pgfqpoint{3.371465in}{3.138360in}}%
\pgfpathlineto{\pgfqpoint{3.384229in}{3.148751in}}%
\pgfpathlineto{\pgfqpoint{3.406720in}{3.164616in}}%
\pgfpathlineto{\pgfqpoint{3.411583in}{3.167793in}}%
\pgfpathlineto{\pgfqpoint{3.415837in}{3.170188in}}%
\pgfpathlineto{\pgfqpoint{3.418877in}{3.171727in}}%
\pgfpathlineto{\pgfqpoint{3.429818in}{3.178628in}}%
\pgfpathlineto{\pgfqpoint{3.433465in}{3.179313in}}%
\pgfpathlineto{\pgfqpoint{3.436504in}{3.180994in}}%
\pgfpathlineto{\pgfqpoint{3.446230in}{3.185341in}}%
\pgfpathlineto{\pgfqpoint{3.448661in}{3.186110in}}%
\pgfpathlineto{\pgfqpoint{3.452308in}{3.187131in}}%
\pgfpathlineto{\pgfqpoint{3.456563in}{3.188780in}}%
\pgfpathlineto{\pgfqpoint{3.469328in}{3.192783in}}%
\pgfpathlineto{\pgfqpoint{3.471759in}{3.193663in}}%
\pgfpathlineto{\pgfqpoint{3.476014in}{3.194279in}}%
\pgfpathlineto{\pgfqpoint{3.477838in}{3.195410in}}%
\pgfpathlineto{\pgfqpoint{3.481485in}{3.196271in}}%
\pgfpathlineto{\pgfqpoint{3.491818in}{3.197792in}}%
\pgfpathlineto{\pgfqpoint{3.496073in}{3.198462in}}%
\pgfpathlineto{\pgfqpoint{3.499113in}{3.198951in}}%
\pgfpathlineto{\pgfqpoint{3.522211in}{3.202482in}}%
\pgfpathlineto{\pgfqpoint{3.525250in}{3.202726in}}%
\pgfpathlineto{\pgfqpoint{3.529505in}{3.203498in}}%
\pgfpathlineto{\pgfqpoint{3.531328in}{3.203250in}}%
\pgfpathlineto{\pgfqpoint{3.533760in}{3.203859in}}%
\pgfpathlineto{\pgfqpoint{3.565976in}{3.205637in}}%
\pgfpathlineto{\pgfqpoint{3.567799in}{3.205764in}}%
\pgfpathlineto{\pgfqpoint{3.574486in}{3.206276in}}%
\pgfpathlineto{\pgfqpoint{3.575701in}{3.205941in}}%
\pgfpathlineto{\pgfqpoint{3.578133in}{3.206255in}}%
\pgfpathlineto{\pgfqpoint{3.580564in}{3.206308in}}%
\pgfpathlineto{\pgfqpoint{3.582995in}{3.206784in}}%
\pgfpathlineto{\pgfqpoint{3.641349in}{3.210025in}}%
\pgfpathlineto{\pgfqpoint{3.646819in}{3.210106in}}%
\pgfpathlineto{\pgfqpoint{3.668702in}{3.210844in}}%
\pgfpathlineto{\pgfqpoint{3.670525in}{3.211076in}}%
\pgfpathlineto{\pgfqpoint{3.684506in}{3.212058in}}%
\pgfpathlineto{\pgfqpoint{3.686937in}{3.212933in}}%
\pgfpathlineto{\pgfqpoint{3.717330in}{3.215654in}}%
\pgfpathlineto{\pgfqpoint{3.731310in}{3.216127in}}%
\pgfpathlineto{\pgfqpoint{3.736173in}{3.216790in}}%
\pgfpathlineto{\pgfqpoint{3.738604in}{3.216718in}}%
\pgfpathlineto{\pgfqpoint{3.741643in}{3.217165in}}%
\pgfpathlineto{\pgfqpoint{3.771428in}{3.218847in}}%
\pgfpathlineto{\pgfqpoint{3.775075in}{3.218558in}}%
\pgfpathlineto{\pgfqpoint{3.784801in}{3.219642in}}%
\pgfpathlineto{\pgfqpoint{3.787232in}{3.219935in}}%
\pgfpathlineto{\pgfqpoint{3.828566in}{3.218700in}}%
\pgfpathlineto{\pgfqpoint{3.830389in}{3.218425in}}%
\pgfpathlineto{\pgfqpoint{3.832213in}{3.218787in}}%
\pgfpathlineto{\pgfqpoint{3.833428in}{3.218121in}}%
\pgfpathlineto{\pgfqpoint{3.836468in}{3.219308in}}%
\pgfpathlineto{\pgfqpoint{3.838291in}{3.219020in}}%
\pgfpathlineto{\pgfqpoint{3.839507in}{3.219507in}}%
\pgfpathlineto{\pgfqpoint{3.843762in}{3.219616in}}%
\pgfpathlineto{\pgfqpoint{3.846193in}{3.219755in}}%
\pgfpathlineto{\pgfqpoint{3.849232in}{3.220080in}}%
\pgfpathlineto{\pgfqpoint{3.857742in}{3.220506in}}%
\pgfpathlineto{\pgfqpoint{3.884488in}{3.221142in}}%
\pgfpathlineto{\pgfqpoint{3.886919in}{3.221870in}}%
\pgfpathlineto{\pgfqpoint{3.889958in}{3.221713in}}%
\pgfpathlineto{\pgfqpoint{3.890566in}{3.221068in}}%
\pgfpathlineto{\pgfqpoint{3.891174in}{3.221890in}}%
\pgfpathlineto{\pgfqpoint{3.928253in}{3.223324in}}%
\pgfpathlineto{\pgfqpoint{3.930076in}{3.223071in}}%
\pgfpathlineto{\pgfqpoint{3.931292in}{3.223187in}}%
\pgfpathlineto{\pgfqpoint{3.933115in}{3.223222in}}%
\pgfpathlineto{\pgfqpoint{3.935547in}{3.223058in}}%
\pgfpathlineto{\pgfqpoint{3.937978in}{3.222815in}}%
\pgfpathlineto{\pgfqpoint{3.942233in}{3.222704in}}%
\pgfpathlineto{\pgfqpoint{3.944664in}{3.222556in}}%
\pgfpathlineto{\pgfqpoint{3.947096in}{3.223276in}}%
\pgfpathlineto{\pgfqpoint{3.948919in}{3.222142in}}%
\pgfpathlineto{\pgfqpoint{3.950743in}{3.222181in}}%
\pgfpathlineto{\pgfqpoint{3.951959in}{3.223672in}}%
\pgfpathlineto{\pgfqpoint{3.953782in}{3.222430in}}%
\pgfpathlineto{\pgfqpoint{3.956213in}{3.223542in}}%
\pgfpathlineto{\pgfqpoint{3.959861in}{3.223679in}}%
\pgfpathlineto{\pgfqpoint{3.962292in}{3.222293in}}%
\pgfpathlineto{\pgfqpoint{3.964116in}{3.223652in}}%
\pgfpathlineto{\pgfqpoint{3.964723in}{3.222373in}}%
\pgfpathlineto{\pgfqpoint{3.965331in}{3.222775in}}%
\pgfpathlineto{\pgfqpoint{3.965939in}{3.223553in}}%
\pgfpathlineto{\pgfqpoint{3.966547in}{3.222528in}}%
\pgfpathlineto{\pgfqpoint{3.968370in}{3.222783in}}%
\pgfpathlineto{\pgfqpoint{3.973233in}{3.222476in}}%
\pgfpathlineto{\pgfqpoint{3.974449in}{3.222635in}}%
\pgfpathlineto{\pgfqpoint{3.976272in}{3.222256in}}%
\pgfpathlineto{\pgfqpoint{3.979312in}{3.222538in}}%
\pgfpathlineto{\pgfqpoint{3.981135in}{3.222799in}}%
\pgfpathlineto{\pgfqpoint{3.983567in}{3.222825in}}%
\pgfpathlineto{\pgfqpoint{3.984782in}{3.222554in}}%
\pgfpathlineto{\pgfqpoint{3.987214in}{3.222237in}}%
\pgfpathlineto{\pgfqpoint{3.989037in}{3.222761in}}%
\pgfpathlineto{\pgfqpoint{3.990861in}{3.222140in}}%
\pgfpathlineto{\pgfqpoint{3.993900in}{3.222754in}}%
\pgfpathlineto{\pgfqpoint{3.996939in}{3.221976in}}%
\pgfpathlineto{\pgfqpoint{3.998763in}{3.221987in}}%
\pgfpathlineto{\pgfqpoint{3.999978in}{3.222245in}}%
\pgfpathlineto{\pgfqpoint{4.000586in}{3.222685in}}%
\pgfpathlineto{\pgfqpoint{4.001802in}{3.221767in}}%
\pgfpathlineto{\pgfqpoint{4.010920in}{3.221938in}}%
\pgfpathlineto{\pgfqpoint{4.013959in}{3.221794in}}%
\pgfpathlineto{\pgfqpoint{4.016390in}{3.221400in}}%
\pgfpathlineto{\pgfqpoint{4.017606in}{3.221790in}}%
\pgfpathlineto{\pgfqpoint{4.022469in}{3.221158in}}%
\pgfpathlineto{\pgfqpoint{4.032194in}{3.219623in}}%
\pgfpathlineto{\pgfqpoint{4.037057in}{3.218627in}}%
\pgfpathlineto{\pgfqpoint{4.042528in}{3.219298in}}%
\pgfpathlineto{\pgfqpoint{4.046175in}{3.219090in}}%
\pgfpathlineto{\pgfqpoint{4.051038in}{3.219690in}}%
\pgfpathlineto{\pgfqpoint{4.054077in}{3.219641in}}%
\pgfpathlineto{\pgfqpoint{4.056508in}{3.219413in}}%
\pgfpathlineto{\pgfqpoint{4.058940in}{3.218853in}}%
\pgfpathlineto{\pgfqpoint{4.063195in}{3.219608in}}%
\pgfpathlineto{\pgfqpoint{4.117293in}{3.220016in}}%
\pgfpathlineto{\pgfqpoint{4.120332in}{3.220237in}}%
\pgfpathlineto{\pgfqpoint{4.129450in}{3.220431in}}%
\pgfpathlineto{\pgfqpoint{4.134313in}{3.220478in}}%
\pgfpathlineto{\pgfqpoint{4.136136in}{3.220272in}}%
\pgfpathlineto{\pgfqpoint{4.142215in}{3.220390in}}%
\pgfpathlineto{\pgfqpoint{4.145862in}{3.220366in}}%
\pgfpathlineto{\pgfqpoint{4.147685in}{3.220361in}}%
\pgfpathlineto{\pgfqpoint{4.153156in}{3.221010in}}%
\pgfpathlineto{\pgfqpoint{4.155587in}{3.220246in}}%
\pgfpathlineto{\pgfqpoint{4.161666in}{3.220075in}}%
\pgfpathlineto{\pgfqpoint{4.196921in}{3.220558in}}%
\pgfpathlineto{\pgfqpoint{4.199960in}{3.220491in}}%
\pgfpathlineto{\pgfqpoint{4.203607in}{3.220483in}}%
\pgfpathlineto{\pgfqpoint{4.205431in}{3.220988in}}%
\pgfpathlineto{\pgfqpoint{4.208470in}{3.220268in}}%
\pgfpathlineto{\pgfqpoint{4.216980in}{3.219689in}}%
\pgfpathlineto{\pgfqpoint{4.219411in}{3.219780in}}%
\pgfpathlineto{\pgfqpoint{4.235215in}{3.219954in}}%
\pgfpathlineto{\pgfqpoint{4.235823in}{3.219333in}}%
\pgfpathlineto{\pgfqpoint{4.236431in}{3.219956in}}%
\pgfpathlineto{\pgfqpoint{4.238862in}{3.220118in}}%
\pgfpathlineto{\pgfqpoint{4.241902in}{3.219870in}}%
\pgfpathlineto{\pgfqpoint{4.246156in}{3.220567in}}%
\pgfpathlineto{\pgfqpoint{4.247372in}{3.220217in}}%
\pgfpathlineto{\pgfqpoint{4.248588in}{3.220535in}}%
\pgfpathlineto{\pgfqpoint{4.249804in}{3.219291in}}%
\pgfpathlineto{\pgfqpoint{4.251627in}{3.219981in}}%
\pgfpathlineto{\pgfqpoint{4.253451in}{3.219450in}}%
\pgfpathlineto{\pgfqpoint{4.259529in}{3.219302in}}%
\pgfpathlineto{\pgfqpoint{4.261353in}{3.219442in}}%
\pgfpathlineto{\pgfqpoint{4.263784in}{3.219689in}}%
\pgfpathlineto{\pgfqpoint{4.271078in}{3.219720in}}%
\pgfpathlineto{\pgfqpoint{4.272294in}{3.220190in}}%
\pgfpathlineto{\pgfqpoint{4.276549in}{3.220214in}}%
\pgfpathlineto{\pgfqpoint{4.280196in}{3.220472in}}%
\pgfpathlineto{\pgfqpoint{4.290529in}{3.219868in}}%
\pgfpathlineto{\pgfqpoint{4.292961in}{3.221210in}}%
\pgfpathlineto{\pgfqpoint{4.294176in}{3.221230in}}%
\pgfpathlineto{\pgfqpoint{4.299039in}{3.220754in}}%
\pgfpathlineto{\pgfqpoint{4.302078in}{3.220433in}}%
\pgfpathlineto{\pgfqpoint{4.305118in}{3.220300in}}%
\pgfpathlineto{\pgfqpoint{4.306941in}{3.220537in}}%
\pgfpathlineto{\pgfqpoint{4.311196in}{3.220235in}}%
\pgfpathlineto{\pgfqpoint{4.313628in}{3.220341in}}%
\pgfpathlineto{\pgfqpoint{4.317275in}{3.220151in}}%
\pgfpathlineto{\pgfqpoint{4.320314in}{3.220341in}}%
\pgfpathlineto{\pgfqpoint{4.322745in}{3.219648in}}%
\pgfpathlineto{\pgfqpoint{4.325784in}{3.220087in}}%
\pgfpathlineto{\pgfqpoint{4.328824in}{3.219717in}}%
\pgfpathlineto{\pgfqpoint{4.331863in}{3.219820in}}%
\pgfpathlineto{\pgfqpoint{4.333686in}{3.218952in}}%
\pgfpathlineto{\pgfqpoint{4.335510in}{3.219580in}}%
\pgfpathlineto{\pgfqpoint{4.337941in}{3.219558in}}%
\pgfpathlineto{\pgfqpoint{4.339765in}{3.218930in}}%
\pgfpathlineto{\pgfqpoint{4.344628in}{3.219132in}}%
\pgfpathlineto{\pgfqpoint{4.346451in}{3.219184in}}%
\pgfpathlineto{\pgfqpoint{4.348275in}{3.219214in}}%
\pgfpathlineto{\pgfqpoint{4.351922in}{3.218996in}}%
\pgfpathlineto{\pgfqpoint{4.353138in}{3.219685in}}%
\pgfpathlineto{\pgfqpoint{4.358000in}{3.219159in}}%
\pgfpathlineto{\pgfqpoint{4.361647in}{3.219619in}}%
\pgfpathlineto{\pgfqpoint{4.363471in}{3.219459in}}%
\pgfpathlineto{\pgfqpoint{4.366510in}{3.219883in}}%
\pgfpathlineto{\pgfqpoint{4.368334in}{3.219792in}}%
\pgfpathlineto{\pgfqpoint{4.371981in}{3.220131in}}%
\pgfpathlineto{\pgfqpoint{4.381099in}{3.219502in}}%
\pgfpathlineto{\pgfqpoint{4.383530in}{3.219969in}}%
\pgfpathlineto{\pgfqpoint{4.386569in}{3.219731in}}%
\pgfpathlineto{\pgfqpoint{4.391432in}{3.219286in}}%
\pgfpathlineto{\pgfqpoint{4.398726in}{3.219791in}}%
\pgfpathlineto{\pgfqpoint{4.404805in}{3.219334in}}%
\pgfpathlineto{\pgfqpoint{4.409059in}{3.219817in}}%
\pgfpathlineto{\pgfqpoint{4.412099in}{3.219340in}}%
\pgfpathlineto{\pgfqpoint{4.423648in}{3.219550in}}%
\pgfpathlineto{\pgfqpoint{4.426079in}{3.219591in}}%
\pgfpathlineto{\pgfqpoint{4.440060in}{3.219068in}}%
\pgfpathlineto{\pgfqpoint{4.443707in}{3.218307in}}%
\pgfpathlineto{\pgfqpoint{4.447354in}{3.218719in}}%
\pgfpathlineto{\pgfqpoint{4.453432in}{3.218495in}}%
\pgfpathlineto{\pgfqpoint{4.458295in}{3.219599in}}%
\pgfpathlineto{\pgfqpoint{4.461334in}{3.218270in}}%
\pgfpathlineto{\pgfqpoint{4.464374in}{3.218274in}}%
\pgfpathlineto{\pgfqpoint{4.465589in}{3.217875in}}%
\pgfpathlineto{\pgfqpoint{4.467413in}{3.218061in}}%
\pgfpathlineto{\pgfqpoint{4.488687in}{3.218962in}}%
\pgfpathlineto{\pgfqpoint{4.491727in}{3.219277in}}%
\pgfpathlineto{\pgfqpoint{4.495374in}{3.220233in}}%
\pgfpathlineto{\pgfqpoint{4.499021in}{3.220455in}}%
\pgfpathlineto{\pgfqpoint{4.501452in}{3.219503in}}%
\pgfpathlineto{\pgfqpoint{4.514217in}{3.219425in}}%
\pgfpathlineto{\pgfqpoint{4.516648in}{3.219572in}}%
\pgfpathlineto{\pgfqpoint{4.519688in}{3.218991in}}%
\pgfpathlineto{\pgfqpoint{4.523943in}{3.219567in}}%
\pgfpathlineto{\pgfqpoint{4.532452in}{3.219756in}}%
\pgfpathlineto{\pgfqpoint{4.535492in}{3.220021in}}%
\pgfpathlineto{\pgfqpoint{4.538531in}{3.219941in}}%
\pgfpathlineto{\pgfqpoint{4.540962in}{3.219352in}}%
\pgfpathlineto{\pgfqpoint{4.542786in}{3.219714in}}%
\pgfpathlineto{\pgfqpoint{4.602355in}{3.219915in}}%
\pgfpathlineto{\pgfqpoint{4.607218in}{3.220758in}}%
\pgfpathlineto{\pgfqpoint{4.609649in}{3.220270in}}%
\pgfpathlineto{\pgfqpoint{4.612688in}{3.220126in}}%
\pgfpathlineto{\pgfqpoint{4.615120in}{3.220458in}}%
\pgfpathlineto{\pgfqpoint{4.621198in}{3.220997in}}%
\pgfpathlineto{\pgfqpoint{4.653414in}{3.426140in}}%
\pgfpathlineto{\pgfqpoint{4.663747in}{3.481646in}}%
\pgfpathlineto{\pgfqpoint{4.697787in}{3.647274in}}%
\pgfpathlineto{\pgfqpoint{4.711159in}{3.703490in}}%
\pgfpathlineto{\pgfqpoint{4.714807in}{3.718313in}}%
\pgfpathlineto{\pgfqpoint{4.776199in}{3.940866in}}%
\pgfpathlineto{\pgfqpoint{4.777415in}{3.943697in}}%
\pgfpathlineto{\pgfqpoint{4.781670in}{3.959203in}}%
\pgfpathlineto{\pgfqpoint{4.788964in}{3.986009in}}%
\pgfpathlineto{\pgfqpoint{4.800513in}{4.023817in}}%
\pgfpathlineto{\pgfqpoint{4.830905in}{4.114102in}}%
\pgfpathlineto{\pgfqpoint{4.854004in}{4.162294in}}%
\pgfpathlineto{\pgfqpoint{4.854611in}{4.162361in}}%
\pgfpathlineto{\pgfqpoint{4.857651in}{4.168396in}}%
\pgfpathlineto{\pgfqpoint{4.872847in}{4.187962in}}%
\pgfpathlineto{\pgfqpoint{4.874670in}{4.188689in}}%
\pgfpathlineto{\pgfqpoint{4.879533in}{4.192860in}}%
\pgfpathlineto{\pgfqpoint{4.880749in}{4.193275in}}%
\pgfpathlineto{\pgfqpoint{4.881357in}{4.194477in}}%
\pgfpathlineto{\pgfqpoint{4.881965in}{4.194021in}}%
\pgfpathlineto{\pgfqpoint{4.884396in}{4.194705in}}%
\pgfpathlineto{\pgfqpoint{4.885004in}{4.195768in}}%
\pgfpathlineto{\pgfqpoint{4.885612in}{4.195342in}}%
\pgfpathlineto{\pgfqpoint{4.890474in}{4.196548in}}%
\pgfpathlineto{\pgfqpoint{4.892298in}{4.194446in}}%
\pgfpathlineto{\pgfqpoint{4.895337in}{4.192148in}}%
\pgfpathlineto{\pgfqpoint{4.900200in}{4.186791in}}%
\pgfpathlineto{\pgfqpoint{4.902024in}{4.186730in}}%
\pgfpathlineto{\pgfqpoint{4.906886in}{4.185679in}}%
\pgfpathlineto{\pgfqpoint{4.908710in}{4.184296in}}%
\pgfpathlineto{\pgfqpoint{4.916004in}{4.181287in}}%
\pgfpathlineto{\pgfqpoint{4.917827in}{4.180042in}}%
\pgfpathlineto{\pgfqpoint{4.921475in}{4.176910in}}%
\pgfpathlineto{\pgfqpoint{4.925122in}{4.173292in}}%
\pgfpathlineto{\pgfqpoint{4.936063in}{4.160994in}}%
\pgfpathlineto{\pgfqpoint{4.937887in}{4.159158in}}%
\pgfpathlineto{\pgfqpoint{4.940318in}{4.156796in}}%
\pgfpathlineto{\pgfqpoint{4.970102in}{4.120571in}}%
\pgfpathlineto{\pgfqpoint{4.973749in}{4.116910in}}%
\pgfpathlineto{\pgfqpoint{4.978612in}{4.112158in}}%
\pgfpathlineto{\pgfqpoint{4.986514in}{4.102829in}}%
\pgfpathlineto{\pgfqpoint{4.995024in}{4.095867in}}%
\pgfpathlineto{\pgfqpoint{4.997455in}{4.094492in}}%
\pgfpathlineto{\pgfqpoint{5.003534in}{4.090489in}}%
\pgfpathlineto{\pgfqpoint{5.005965in}{4.088831in}}%
\pgfpathlineto{\pgfqpoint{5.016907in}{4.081251in}}%
\pgfpathlineto{\pgfqpoint{5.018730in}{4.080605in}}%
\pgfpathlineto{\pgfqpoint{5.024809in}{4.076205in}}%
\pgfpathlineto{\pgfqpoint{5.026024in}{4.075804in}}%
\pgfpathlineto{\pgfqpoint{5.029671in}{4.073881in}}%
\pgfpathlineto{\pgfqpoint{5.036966in}{4.069579in}}%
\pgfpathlineto{\pgfqpoint{5.052162in}{4.062050in}}%
\pgfpathlineto{\pgfqpoint{5.055809in}{4.061197in}}%
\pgfpathlineto{\pgfqpoint{5.066750in}{4.056267in}}%
\pgfpathlineto{\pgfqpoint{5.067966in}{4.055824in}}%
\pgfpathlineto{\pgfqpoint{5.071613in}{4.054354in}}%
\pgfpathlineto{\pgfqpoint{5.077691in}{4.052749in}}%
\pgfpathlineto{\pgfqpoint{5.082554in}{4.051034in}}%
\pgfpathlineto{\pgfqpoint{5.088633in}{4.051169in}}%
\pgfpathlineto{\pgfqpoint{5.106260in}{4.048459in}}%
\pgfpathlineto{\pgfqpoint{5.116594in}{4.046669in}}%
\pgfpathlineto{\pgfqpoint{5.119025in}{4.046524in}}%
\pgfpathlineto{\pgfqpoint{5.133005in}{4.044013in}}%
\pgfpathlineto{\pgfqpoint{5.136652in}{4.043215in}}%
\pgfpathlineto{\pgfqpoint{5.141515in}{4.041940in}}%
\pgfpathlineto{\pgfqpoint{5.143947in}{4.041002in}}%
\pgfpathlineto{\pgfqpoint{5.145770in}{4.041601in}}%
\pgfpathlineto{\pgfqpoint{5.151241in}{4.040558in}}%
\pgfpathlineto{\pgfqpoint{5.154888in}{4.039892in}}%
\pgfpathlineto{\pgfqpoint{5.157927in}{4.039412in}}%
\pgfpathlineto{\pgfqpoint{5.162790in}{4.039015in}}%
\pgfpathlineto{\pgfqpoint{5.174947in}{4.037125in}}%
\pgfpathlineto{\pgfqpoint{5.177378in}{4.036942in}}%
\pgfpathlineto{\pgfqpoint{5.179810in}{4.036060in}}%
\pgfpathlineto{\pgfqpoint{5.182849in}{4.035543in}}%
\pgfpathlineto{\pgfqpoint{5.191967in}{4.033188in}}%
\pgfpathlineto{\pgfqpoint{5.198045in}{4.032346in}}%
\pgfpathlineto{\pgfqpoint{5.199869in}{4.032174in}}%
\pgfpathlineto{\pgfqpoint{5.203516in}{4.031724in}}%
\pgfpathlineto{\pgfqpoint{5.206555in}{4.030609in}}%
\pgfpathlineto{\pgfqpoint{5.213241in}{4.030034in}}%
\pgfpathlineto{\pgfqpoint{5.215673in}{4.029500in}}%
\pgfpathlineto{\pgfqpoint{5.218712in}{4.029093in}}%
\pgfpathlineto{\pgfqpoint{5.221143in}{4.028763in}}%
\pgfpathlineto{\pgfqpoint{5.225398in}{4.028092in}}%
\pgfpathlineto{\pgfqpoint{5.229045in}{4.026983in}}%
\pgfpathlineto{\pgfqpoint{5.232692in}{4.027141in}}%
\pgfpathlineto{\pgfqpoint{5.238163in}{4.025914in}}%
\pgfpathlineto{\pgfqpoint{5.243634in}{4.025060in}}%
\pgfpathlineto{\pgfqpoint{5.246065in}{4.024553in}}%
\pgfpathlineto{\pgfqpoint{5.250320in}{4.024294in}}%
\pgfpathlineto{\pgfqpoint{5.258222in}{4.023028in}}%
\pgfpathlineto{\pgfqpoint{5.260653in}{4.022902in}}%
\pgfpathlineto{\pgfqpoint{5.268555in}{4.022173in}}%
\pgfpathlineto{\pgfqpoint{5.274634in}{4.021103in}}%
\pgfpathlineto{\pgfqpoint{5.276457in}{4.021259in}}%
\pgfpathlineto{\pgfqpoint{5.282536in}{4.021112in}}%
\pgfpathlineto{\pgfqpoint{5.287399in}{4.019955in}}%
\pgfpathlineto{\pgfqpoint{5.291046in}{4.019795in}}%
\pgfpathlineto{\pgfqpoint{5.292869in}{4.019537in}}%
\pgfpathlineto{\pgfqpoint{5.295301in}{4.019969in}}%
\pgfpathlineto{\pgfqpoint{5.297732in}{4.019712in}}%
\pgfpathlineto{\pgfqpoint{5.300163in}{4.019574in}}%
\pgfpathlineto{\pgfqpoint{5.310497in}{4.018885in}}%
\pgfpathlineto{\pgfqpoint{5.312320in}{4.018766in}}%
\pgfpathlineto{\pgfqpoint{5.385870in}{4.013935in}}%
\pgfpathlineto{\pgfqpoint{5.387693in}{4.013155in}}%
\pgfpathlineto{\pgfqpoint{5.391340in}{4.014174in}}%
\pgfpathlineto{\pgfqpoint{5.393164in}{4.014185in}}%
\pgfpathlineto{\pgfqpoint{5.394987in}{4.014232in}}%
\pgfpathlineto{\pgfqpoint{5.396811in}{4.014178in}}%
\pgfpathlineto{\pgfqpoint{5.399850in}{4.014534in}}%
\pgfpathlineto{\pgfqpoint{5.407752in}{4.013643in}}%
\pgfpathlineto{\pgfqpoint{5.410792in}{4.014447in}}%
\pgfpathlineto{\pgfqpoint{5.413223in}{4.014105in}}%
\pgfpathlineto{\pgfqpoint{5.419909in}{4.015588in}}%
\pgfpathlineto{\pgfqpoint{5.421733in}{4.015116in}}%
\pgfpathlineto{\pgfqpoint{5.430243in}{4.015231in}}%
\pgfpathlineto{\pgfqpoint{5.433282in}{4.014616in}}%
\pgfpathlineto{\pgfqpoint{5.436929in}{4.014973in}}%
\pgfpathlineto{\pgfqpoint{5.438752in}{4.015061in}}%
\pgfpathlineto{\pgfqpoint{5.441792in}{4.015503in}}%
\pgfpathlineto{\pgfqpoint{5.444223in}{4.015590in}}%
\pgfpathlineto{\pgfqpoint{5.447870in}{4.015378in}}%
\pgfpathlineto{\pgfqpoint{5.450909in}{4.015165in}}%
\pgfpathlineto{\pgfqpoint{5.455164in}{4.014887in}}%
\pgfpathlineto{\pgfqpoint{5.514733in}{4.012797in}}%
\pgfpathlineto{\pgfqpoint{5.516557in}{4.012691in}}%
\pgfpathlineto{\pgfqpoint{5.520204in}{4.012308in}}%
\pgfpathlineto{\pgfqpoint{5.537832in}{4.010922in}}%
\pgfpathlineto{\pgfqpoint{5.540263in}{4.010161in}}%
\pgfpathlineto{\pgfqpoint{5.543910in}{4.009367in}}%
\pgfpathlineto{\pgfqpoint{5.545734in}{4.009874in}}%
\pgfpathlineto{\pgfqpoint{5.548165in}{4.011037in}}%
\pgfpathlineto{\pgfqpoint{5.553028in}{4.011105in}}%
\pgfpathlineto{\pgfqpoint{5.555459in}{4.010645in}}%
\pgfpathlineto{\pgfqpoint{5.557283in}{4.010642in}}%
\pgfpathlineto{\pgfqpoint{5.561538in}{4.010687in}}%
\pgfpathlineto{\pgfqpoint{5.567616in}{4.009766in}}%
\pgfpathlineto{\pgfqpoint{5.570047in}{4.010823in}}%
\pgfpathlineto{\pgfqpoint{5.574302in}{4.009457in}}%
\pgfpathlineto{\pgfqpoint{5.577949in}{4.010524in}}%
\pgfpathlineto{\pgfqpoint{5.580989in}{4.010665in}}%
\pgfpathlineto{\pgfqpoint{5.588891in}{4.010762in}}%
\pgfpathlineto{\pgfqpoint{5.594361in}{4.010685in}}%
\pgfpathlineto{\pgfqpoint{5.595577in}{4.010651in}}%
\pgfpathlineto{\pgfqpoint{5.600440in}{4.010476in}}%
\pgfpathlineto{\pgfqpoint{5.602263in}{4.010650in}}%
\pgfpathlineto{\pgfqpoint{5.606518in}{4.010587in}}%
\pgfpathlineto{\pgfqpoint{5.608342in}{4.010274in}}%
\pgfpathlineto{\pgfqpoint{5.610165in}{4.010563in}}%
\pgfpathlineto{\pgfqpoint{5.613205in}{4.010153in}}%
\pgfpathlineto{\pgfqpoint{5.616852in}{4.009848in}}%
\pgfpathlineto{\pgfqpoint{5.627793in}{4.010833in}}%
\pgfpathlineto{\pgfqpoint{5.630224in}{4.011595in}}%
\pgfpathlineto{\pgfqpoint{5.634479in}{4.011509in}}%
\pgfpathlineto{\pgfqpoint{5.665479in}{4.010871in}}%
\pgfpathlineto{\pgfqpoint{5.667303in}{4.011526in}}%
\pgfpathlineto{\pgfqpoint{5.669734in}{4.010687in}}%
\pgfpathlineto{\pgfqpoint{5.673381in}{4.011813in}}%
\pgfpathlineto{\pgfqpoint{5.675813in}{4.011045in}}%
\pgfpathlineto{\pgfqpoint{5.677636in}{4.011509in}}%
\pgfpathlineto{\pgfqpoint{5.684323in}{4.011528in}}%
\pgfpathlineto{\pgfqpoint{5.685538in}{4.011222in}}%
\pgfpathlineto{\pgfqpoint{5.687362in}{4.010932in}}%
\pgfpathlineto{\pgfqpoint{5.689793in}{4.011392in}}%
\pgfpathlineto{\pgfqpoint{5.703774in}{4.010296in}}%
\pgfpathlineto{\pgfqpoint{5.708637in}{4.010810in}}%
\pgfpathlineto{\pgfqpoint{5.715323in}{4.010618in}}%
\pgfpathlineto{\pgfqpoint{5.723225in}{4.009225in}}%
\pgfpathlineto{\pgfqpoint{5.727480in}{4.010948in}}%
\pgfpathlineto{\pgfqpoint{5.736598in}{4.010898in}}%
\pgfpathlineto{\pgfqpoint{5.742068in}{4.011167in}}%
\pgfpathlineto{\pgfqpoint{5.744500in}{4.010509in}}%
\pgfpathlineto{\pgfqpoint{5.746323in}{4.011019in}}%
\pgfpathlineto{\pgfqpoint{5.757872in}{4.010976in}}%
\pgfpathlineto{\pgfqpoint{5.760304in}{4.011053in}}%
\pgfpathlineto{\pgfqpoint{5.762127in}{4.010731in}}%
\pgfpathlineto{\pgfqpoint{5.779755in}{4.009819in}}%
\pgfpathlineto{\pgfqpoint{5.782794in}{4.009898in}}%
\pgfpathlineto{\pgfqpoint{5.786441in}{4.011461in}}%
\pgfpathlineto{\pgfqpoint{5.793127in}{4.010901in}}%
\pgfpathlineto{\pgfqpoint{5.799814in}{4.011338in}}%
\pgfpathlineto{\pgfqpoint{5.801029in}{4.011034in}}%
\pgfpathlineto{\pgfqpoint{5.807716in}{4.011543in}}%
\pgfpathlineto{\pgfqpoint{5.809539in}{4.011268in}}%
\pgfpathlineto{\pgfqpoint{5.811971in}{4.011062in}}%
\pgfpathlineto{\pgfqpoint{5.814402in}{4.010936in}}%
\pgfpathlineto{\pgfqpoint{5.817441in}{4.011241in}}%
\pgfpathlineto{\pgfqpoint{5.822304in}{4.010564in}}%
\pgfpathlineto{\pgfqpoint{5.824735in}{4.011249in}}%
\pgfpathlineto{\pgfqpoint{5.910442in}{4.010434in}}%
\pgfpathlineto{\pgfqpoint{5.924422in}{4.011118in}}%
\pgfpathlineto{\pgfqpoint{5.926246in}{4.010652in}}%
\pgfpathlineto{\pgfqpoint{5.929893in}{4.011497in}}%
\pgfpathlineto{\pgfqpoint{5.932932in}{4.011658in}}%
\pgfpathlineto{\pgfqpoint{5.940834in}{4.012049in}}%
\pgfpathlineto{\pgfqpoint{5.943266in}{4.012404in}}%
\pgfpathlineto{\pgfqpoint{5.946913in}{4.012051in}}%
\pgfpathlineto{\pgfqpoint{5.949344in}{4.012158in}}%
\pgfpathlineto{\pgfqpoint{5.962109in}{4.013422in}}%
\pgfpathlineto{\pgfqpoint{5.977305in}{4.013118in}}%
\pgfpathlineto{\pgfqpoint{5.980952in}{4.013353in}}%
\pgfpathlineto{\pgfqpoint{5.985815in}{4.012840in}}%
\pgfpathlineto{\pgfqpoint{5.989462in}{4.012972in}}%
\pgfpathlineto{\pgfqpoint{5.991285in}{4.012832in}}%
\pgfpathlineto{\pgfqpoint{5.993109in}{4.012603in}}%
\pgfpathlineto{\pgfqpoint{5.997364in}{4.011187in}}%
\pgfpathlineto{\pgfqpoint{6.000403in}{4.013388in}}%
\pgfpathlineto{\pgfqpoint{6.020462in}{4.013113in}}%
\pgfpathlineto{\pgfqpoint{6.025933in}{4.013640in}}%
\pgfpathlineto{\pgfqpoint{6.029580in}{4.013160in}}%
\pgfpathlineto{\pgfqpoint{6.048423in}{4.012331in}}%
\pgfpathlineto{\pgfqpoint{6.050247in}{4.012667in}}%
\pgfpathlineto{\pgfqpoint{6.057541in}{4.012539in}}%
\pgfpathlineto{\pgfqpoint{6.058149in}{4.013257in}}%
\pgfpathlineto{\pgfqpoint{6.058756in}{4.012566in}}%
\pgfpathlineto{\pgfqpoint{6.060580in}{4.012214in}}%
\pgfpathlineto{\pgfqpoint{6.063011in}{4.011534in}}%
\pgfpathlineto{\pgfqpoint{6.064835in}{4.012308in}}%
\pgfpathlineto{\pgfqpoint{6.068482in}{4.012761in}}%
\pgfpathlineto{\pgfqpoint{6.071521in}{4.013015in}}%
\pgfpathlineto{\pgfqpoint{6.073953in}{4.013591in}}%
\pgfpathlineto{\pgfqpoint{6.075168in}{4.013408in}}%
\pgfpathlineto{\pgfqpoint{6.079423in}{4.013956in}}%
\pgfpathlineto{\pgfqpoint{6.081855in}{4.013688in}}%
\pgfpathlineto{\pgfqpoint{6.084286in}{4.013355in}}%
\pgfpathlineto{\pgfqpoint{6.087325in}{4.013135in}}%
\pgfpathlineto{\pgfqpoint{6.090972in}{4.013219in}}%
\pgfpathlineto{\pgfqpoint{6.092188in}{4.013956in}}%
\pgfpathlineto{\pgfqpoint{6.094619in}{4.013019in}}%
\pgfpathlineto{\pgfqpoint{6.095835in}{4.013442in}}%
\pgfpathlineto{\pgfqpoint{6.097051in}{4.012993in}}%
\pgfpathlineto{\pgfqpoint{6.099482in}{4.013831in}}%
\pgfpathlineto{\pgfqpoint{6.101914in}{4.013285in}}%
\pgfpathlineto{\pgfqpoint{6.106776in}{4.013631in}}%
\pgfpathlineto{\pgfqpoint{6.109208in}{4.014274in}}%
\pgfpathlineto{\pgfqpoint{6.113463in}{4.013256in}}%
\pgfpathlineto{\pgfqpoint{6.117110in}{4.013012in}}%
\pgfpathlineto{\pgfqpoint{6.118326in}{4.012455in}}%
\pgfpathlineto{\pgfqpoint{6.120149in}{4.012586in}}%
\pgfpathlineto{\pgfqpoint{6.120757in}{4.011616in}}%
\pgfpathlineto{\pgfqpoint{6.121365in}{4.012480in}}%
\pgfpathlineto{\pgfqpoint{6.123188in}{4.013212in}}%
\pgfpathlineto{\pgfqpoint{6.124404in}{4.013480in}}%
\pgfpathlineto{\pgfqpoint{6.126228in}{4.013591in}}%
\pgfpathlineto{\pgfqpoint{6.131090in}{4.013571in}}%
\pgfpathlineto{\pgfqpoint{6.132914in}{4.013667in}}%
\pgfpathlineto{\pgfqpoint{6.135345in}{4.013076in}}%
\pgfpathlineto{\pgfqpoint{6.137169in}{4.012901in}}%
\pgfpathlineto{\pgfqpoint{6.137777in}{4.013705in}}%
\pgfpathlineto{\pgfqpoint{6.138384in}{4.012759in}}%
\pgfpathlineto{\pgfqpoint{6.148110in}{4.011638in}}%
\pgfpathlineto{\pgfqpoint{6.151757in}{4.011219in}}%
\pgfpathlineto{\pgfqpoint{6.153581in}{4.012587in}}%
\pgfpathlineto{\pgfqpoint{6.154796in}{4.011696in}}%
\pgfpathlineto{\pgfqpoint{6.155404in}{4.012423in}}%
\pgfpathlineto{\pgfqpoint{6.156012in}{4.011810in}}%
\pgfpathlineto{\pgfqpoint{6.160875in}{4.011819in}}%
\pgfpathlineto{\pgfqpoint{6.162698in}{4.012215in}}%
\pgfpathlineto{\pgfqpoint{6.163914in}{4.011399in}}%
\pgfpathlineto{\pgfqpoint{6.165130in}{4.011974in}}%
\pgfpathlineto{\pgfqpoint{6.166953in}{4.011142in}}%
\pgfpathlineto{\pgfqpoint{6.168777in}{4.011541in}}%
\pgfpathlineto{\pgfqpoint{6.176679in}{4.011947in}}%
\pgfpathlineto{\pgfqpoint{6.178502in}{4.010792in}}%
\pgfpathlineto{\pgfqpoint{6.180326in}{4.011520in}}%
\pgfpathlineto{\pgfqpoint{6.182149in}{4.010630in}}%
\pgfpathlineto{\pgfqpoint{6.184581in}{4.010645in}}%
\pgfpathlineto{\pgfqpoint{6.190051in}{4.011577in}}%
\pgfpathlineto{\pgfqpoint{6.192483in}{4.011467in}}%
\pgfpathlineto{\pgfqpoint{6.195522in}{4.011087in}}%
\pgfpathlineto{\pgfqpoint{6.196738in}{4.010557in}}%
\pgfpathlineto{\pgfqpoint{6.208895in}{4.011371in}}%
\pgfpathlineto{\pgfqpoint{6.215581in}{4.011034in}}%
\pgfpathlineto{\pgfqpoint{6.217405in}{4.010922in}}%
\pgfpathlineto{\pgfqpoint{6.222267in}{4.010688in}}%
\pgfpathlineto{\pgfqpoint{6.224091in}{4.010523in}}%
\pgfpathlineto{\pgfqpoint{6.230169in}{4.010672in}}%
\pgfpathlineto{\pgfqpoint{6.232601in}{4.010196in}}%
\pgfpathlineto{\pgfqpoint{6.242326in}{4.010126in}}%
\pgfpathlineto{\pgfqpoint{6.244150in}{4.010871in}}%
\pgfpathlineto{\pgfqpoint{6.245973in}{4.009663in}}%
\pgfpathlineto{\pgfqpoint{6.247797in}{4.010132in}}%
\pgfpathlineto{\pgfqpoint{6.252660in}{4.009952in}}%
\pgfpathlineto{\pgfqpoint{6.255091in}{4.009835in}}%
\pgfpathlineto{\pgfqpoint{6.258738in}{4.010223in}}%
\pgfpathlineto{\pgfqpoint{6.290954in}{3.809227in}}%
\pgfpathlineto{\pgfqpoint{6.307366in}{3.728279in}}%
\pgfpathlineto{\pgfqpoint{6.320739in}{3.668625in}}%
\pgfpathlineto{\pgfqpoint{6.328033in}{3.639472in}}%
\pgfpathlineto{\pgfqpoint{6.351739in}{3.552498in}}%
\pgfpathlineto{\pgfqpoint{6.395504in}{3.391641in}}%
\pgfpathlineto{\pgfqpoint{6.428935in}{3.259372in}}%
\pgfpathlineto{\pgfqpoint{6.434406in}{3.239372in}}%
\pgfpathlineto{\pgfqpoint{6.444131in}{3.205014in}}%
\pgfpathlineto{\pgfqpoint{6.465406in}{3.134429in}}%
\pgfpathlineto{\pgfqpoint{6.492759in}{3.067042in}}%
\pgfpathlineto{\pgfqpoint{6.497622in}{3.058611in}}%
\pgfpathlineto{\pgfqpoint{6.503093in}{3.049298in}}%
\pgfpathlineto{\pgfqpoint{6.504308in}{3.048189in}}%
\pgfpathlineto{\pgfqpoint{6.507956in}{3.043325in}}%
\pgfpathlineto{\pgfqpoint{6.520720in}{3.035510in}}%
\pgfpathlineto{\pgfqpoint{6.523759in}{3.035211in}}%
\pgfpathlineto{\pgfqpoint{6.527407in}{3.035492in}}%
\pgfpathlineto{\pgfqpoint{6.529230in}{3.036899in}}%
\pgfpathlineto{\pgfqpoint{6.530446in}{3.036533in}}%
\pgfpathlineto{\pgfqpoint{6.534093in}{3.038859in}}%
\pgfpathlineto{\pgfqpoint{6.534701in}{3.038533in}}%
\pgfpathlineto{\pgfqpoint{6.536524in}{3.040133in}}%
\pgfpathlineto{\pgfqpoint{6.539564in}{3.041727in}}%
\pgfpathlineto{\pgfqpoint{6.546858in}{3.047943in}}%
\pgfpathlineto{\pgfqpoint{6.548681in}{3.049068in}}%
\pgfpathlineto{\pgfqpoint{6.551720in}{3.053132in}}%
\pgfpathlineto{\pgfqpoint{6.552328in}{3.053081in}}%
\pgfpathlineto{\pgfqpoint{6.554760in}{3.055521in}}%
\pgfpathlineto{\pgfqpoint{6.565701in}{3.068237in}}%
\pgfpathlineto{\pgfqpoint{6.574211in}{3.079569in}}%
\pgfpathlineto{\pgfqpoint{6.578466in}{3.085707in}}%
\pgfpathlineto{\pgfqpoint{6.583936in}{3.093448in}}%
\pgfpathlineto{\pgfqpoint{6.588799in}{3.100606in}}%
\pgfpathlineto{\pgfqpoint{6.589407in}{3.100697in}}%
\pgfpathlineto{\pgfqpoint{6.593054in}{3.106366in}}%
\pgfpathlineto{\pgfqpoint{6.626486in}{3.151577in}}%
\pgfpathlineto{\pgfqpoint{6.627094in}{3.151678in}}%
\pgfpathlineto{\pgfqpoint{6.629525in}{3.154640in}}%
\pgfpathlineto{\pgfqpoint{6.633172in}{3.159035in}}%
\pgfpathlineto{\pgfqpoint{6.642898in}{3.169550in}}%
\pgfpathlineto{\pgfqpoint{6.643505in}{3.169469in}}%
\pgfpathlineto{\pgfqpoint{6.647152in}{3.173945in}}%
\pgfpathlineto{\pgfqpoint{6.693349in}{3.212168in}}%
\pgfpathlineto{\pgfqpoint{6.694564in}{3.212613in}}%
\pgfpathlineto{\pgfqpoint{6.696996in}{3.214424in}}%
\pgfpathlineto{\pgfqpoint{6.701251in}{3.216527in}}%
\pgfpathlineto{\pgfqpoint{6.710976in}{3.221908in}}%
\pgfpathlineto{\pgfqpoint{6.720702in}{3.226183in}}%
\pgfpathlineto{\pgfqpoint{6.724957in}{3.228955in}}%
\pgfpathlineto{\pgfqpoint{6.759604in}{3.241210in}}%
\pgfpathlineto{\pgfqpoint{6.760820in}{3.240632in}}%
\pgfpathlineto{\pgfqpoint{6.762643in}{3.241415in}}%
\pgfpathlineto{\pgfqpoint{6.766290in}{3.241743in}}%
\pgfpathlineto{\pgfqpoint{6.767506in}{3.241637in}}%
\pgfpathlineto{\pgfqpoint{6.768722in}{3.242199in}}%
\pgfpathlineto{\pgfqpoint{6.771153in}{3.242270in}}%
\pgfpathlineto{\pgfqpoint{6.775408in}{3.243633in}}%
\pgfpathlineto{\pgfqpoint{6.777840in}{3.244165in}}%
\pgfpathlineto{\pgfqpoint{6.779663in}{3.244729in}}%
\pgfpathlineto{\pgfqpoint{6.794859in}{3.247866in}}%
\pgfpathlineto{\pgfqpoint{6.796683in}{3.248033in}}%
\pgfpathlineto{\pgfqpoint{6.801546in}{3.248691in}}%
\pgfpathlineto{\pgfqpoint{6.807624in}{3.249678in}}%
\pgfpathlineto{\pgfqpoint{6.808840in}{3.249575in}}%
\pgfpathlineto{\pgfqpoint{6.810663in}{3.249969in}}%
\pgfpathlineto{\pgfqpoint{6.812487in}{3.249913in}}%
\pgfpathlineto{\pgfqpoint{6.817957in}{3.250950in}}%
\pgfpathlineto{\pgfqpoint{6.820997in}{3.251159in}}%
\pgfpathlineto{\pgfqpoint{6.825859in}{3.252127in}}%
\pgfpathlineto{\pgfqpoint{6.829507in}{3.251494in}}%
\pgfpathlineto{\pgfqpoint{6.830722in}{3.252237in}}%
\pgfpathlineto{\pgfqpoint{6.835585in}{3.252278in}}%
\pgfpathlineto{\pgfqpoint{6.837409in}{3.252622in}}%
\pgfpathlineto{\pgfqpoint{6.841664in}{3.252624in}}%
\pgfpathlineto{\pgfqpoint{6.844095in}{3.253356in}}%
\pgfpathlineto{\pgfqpoint{6.853820in}{3.254627in}}%
\pgfpathlineto{\pgfqpoint{6.856252in}{3.254394in}}%
\pgfpathlineto{\pgfqpoint{6.859899in}{3.254562in}}%
\pgfpathlineto{\pgfqpoint{6.862330in}{3.254455in}}%
\pgfpathlineto{\pgfqpoint{6.864762in}{3.255425in}}%
\pgfpathlineto{\pgfqpoint{6.870840in}{3.255008in}}%
\pgfpathlineto{\pgfqpoint{6.876919in}{3.255357in}}%
\pgfpathlineto{\pgfqpoint{6.878742in}{3.255179in}}%
\pgfpathlineto{\pgfqpoint{6.881781in}{3.255545in}}%
\pgfpathlineto{\pgfqpoint{6.884213in}{3.255360in}}%
\pgfpathlineto{\pgfqpoint{6.886644in}{3.255362in}}%
\pgfpathlineto{\pgfqpoint{6.892723in}{3.255486in}}%
\pgfpathlineto{\pgfqpoint{6.922507in}{3.255659in}}%
\pgfpathlineto{\pgfqpoint{6.923723in}{3.255898in}}%
\pgfpathlineto{\pgfqpoint{6.926154in}{3.255378in}}%
\pgfpathlineto{\pgfqpoint{6.931625in}{3.255806in}}%
\pgfpathlineto{\pgfqpoint{6.934664in}{3.255707in}}%
\pgfpathlineto{\pgfqpoint{6.941350in}{3.256427in}}%
\pgfpathlineto{\pgfqpoint{6.944997in}{3.256316in}}%
\pgfpathlineto{\pgfqpoint{6.946821in}{3.256743in}}%
\pgfpathlineto{\pgfqpoint{6.951684in}{3.255614in}}%
\pgfpathlineto{\pgfqpoint{6.958370in}{3.255007in}}%
\pgfpathlineto{\pgfqpoint{6.961409in}{3.255029in}}%
\pgfpathlineto{\pgfqpoint{6.965056in}{3.255310in}}%
\pgfpathlineto{\pgfqpoint{6.969311in}{3.255485in}}%
\pgfpathlineto{\pgfqpoint{7.000312in}{3.257559in}}%
\pgfpathlineto{\pgfqpoint{7.003351in}{3.257716in}}%
\pgfpathlineto{\pgfqpoint{7.014292in}{3.259132in}}%
\pgfpathlineto{\pgfqpoint{7.016723in}{3.259185in}}%
\pgfpathlineto{\pgfqpoint{7.024018in}{3.258947in}}%
\pgfpathlineto{\pgfqpoint{7.034351in}{3.259351in}}%
\pgfpathlineto{\pgfqpoint{7.036782in}{3.260222in}}%
\pgfpathlineto{\pgfqpoint{7.057449in}{3.261345in}}%
\pgfpathlineto{\pgfqpoint{7.068390in}{3.262855in}}%
\pgfpathlineto{\pgfqpoint{7.073253in}{3.263448in}}%
\pgfpathlineto{\pgfqpoint{7.076900in}{3.264188in}}%
\pgfpathlineto{\pgfqpoint{7.080547in}{3.264082in}}%
\pgfpathlineto{\pgfqpoint{7.082371in}{3.264451in}}%
\pgfpathlineto{\pgfqpoint{7.087842in}{3.265040in}}%
\pgfpathlineto{\pgfqpoint{7.089057in}{3.265131in}}%
\pgfpathlineto{\pgfqpoint{7.090881in}{3.265293in}}%
\pgfpathlineto{\pgfqpoint{7.114587in}{3.266238in}}%
\pgfpathlineto{\pgfqpoint{7.116410in}{3.266563in}}%
\pgfpathlineto{\pgfqpoint{7.129175in}{3.265700in}}%
\pgfpathlineto{\pgfqpoint{7.131607in}{3.265720in}}%
\pgfpathlineto{\pgfqpoint{7.136469in}{3.265778in}}%
\pgfpathlineto{\pgfqpoint{7.143156in}{3.265688in}}%
\pgfpathlineto{\pgfqpoint{7.154097in}{3.265498in}}%
\pgfpathlineto{\pgfqpoint{7.156528in}{3.265478in}}%
\pgfpathlineto{\pgfqpoint{7.159567in}{3.265652in}}%
\pgfpathlineto{\pgfqpoint{7.165646in}{3.265611in}}%
\pgfpathlineto{\pgfqpoint{7.180842in}{3.266600in}}%
\pgfpathlineto{\pgfqpoint{7.186921in}{3.266794in}}%
\pgfpathlineto{\pgfqpoint{7.189960in}{3.266603in}}%
\pgfpathlineto{\pgfqpoint{7.192391in}{3.266895in}}%
\pgfpathlineto{\pgfqpoint{7.194215in}{3.267133in}}%
\pgfpathlineto{\pgfqpoint{7.196038in}{3.266999in}}%
\pgfpathlineto{\pgfqpoint{7.198470in}{3.267046in}}%
\pgfpathlineto{\pgfqpoint{7.200293in}{3.266950in}}%
\pgfpathlineto{\pgfqpoint{7.203332in}{3.267341in}}%
\pgfpathlineto{\pgfqpoint{7.225823in}{3.268737in}}%
\pgfpathlineto{\pgfqpoint{7.227646in}{3.268482in}}%
\pgfpathlineto{\pgfqpoint{7.230686in}{3.268488in}}%
\pgfpathlineto{\pgfqpoint{7.235548in}{3.268912in}}%
\pgfpathlineto{\pgfqpoint{7.237372in}{3.269309in}}%
\pgfpathlineto{\pgfqpoint{7.239803in}{3.268608in}}%
\pgfpathlineto{\pgfqpoint{7.243450in}{3.268278in}}%
\pgfpathlineto{\pgfqpoint{7.249529in}{3.268266in}}%
\pgfpathlineto{\pgfqpoint{7.256823in}{3.267602in}}%
\pgfpathlineto{\pgfqpoint{7.259862in}{3.267795in}}%
\pgfpathlineto{\pgfqpoint{7.265333in}{3.267374in}}%
\pgfpathlineto{\pgfqpoint{7.268372in}{3.267524in}}%
\pgfpathlineto{\pgfqpoint{7.269588in}{3.266917in}}%
\pgfpathlineto{\pgfqpoint{7.271411in}{3.267115in}}%
\pgfpathlineto{\pgfqpoint{7.279313in}{3.266883in}}%
\pgfpathlineto{\pgfqpoint{7.281745in}{3.266625in}}%
\pgfpathlineto{\pgfqpoint{7.283568in}{3.266647in}}%
\pgfpathlineto{\pgfqpoint{7.293294in}{3.266142in}}%
\pgfpathlineto{\pgfqpoint{7.294510in}{3.266644in}}%
\pgfpathlineto{\pgfqpoint{7.297549in}{3.265844in}}%
\pgfpathlineto{\pgfqpoint{7.302412in}{3.265584in}}%
\pgfpathlineto{\pgfqpoint{7.305451in}{3.265565in}}%
\pgfpathlineto{\pgfqpoint{7.307882in}{3.265328in}}%
\pgfpathlineto{\pgfqpoint{7.323078in}{3.264952in}}%
\pgfpathlineto{\pgfqpoint{7.329157in}{3.264755in}}%
\pgfpathlineto{\pgfqpoint{7.332804in}{3.264845in}}%
\pgfpathlineto{\pgfqpoint{7.339490in}{3.264758in}}%
\pgfpathlineto{\pgfqpoint{7.344353in}{3.264061in}}%
\pgfpathlineto{\pgfqpoint{7.346784in}{3.264085in}}%
\pgfpathlineto{\pgfqpoint{7.355294in}{3.264251in}}%
\pgfpathlineto{\pgfqpoint{7.358941in}{3.263964in}}%
\pgfpathlineto{\pgfqpoint{7.361981in}{3.264147in}}%
\pgfpathlineto{\pgfqpoint{7.363804in}{3.263996in}}%
\pgfpathlineto{\pgfqpoint{7.369275in}{3.264784in}}%
\pgfpathlineto{\pgfqpoint{7.371098in}{3.264178in}}%
\pgfpathlineto{\pgfqpoint{7.373530in}{3.263633in}}%
\pgfpathlineto{\pgfqpoint{7.379608in}{3.264376in}}%
\pgfpathlineto{\pgfqpoint{7.383255in}{3.263955in}}%
\pgfpathlineto{\pgfqpoint{7.384471in}{3.264455in}}%
\pgfpathlineto{\pgfqpoint{7.388118in}{3.264051in}}%
\pgfpathlineto{\pgfqpoint{7.406353in}{3.264637in}}%
\pgfpathlineto{\pgfqpoint{7.408177in}{3.264355in}}%
\pgfpathlineto{\pgfqpoint{7.413648in}{3.264703in}}%
\pgfpathlineto{\pgfqpoint{7.416687in}{3.264424in}}%
\pgfpathlineto{\pgfqpoint{7.422765in}{3.264631in}}%
\pgfpathlineto{\pgfqpoint{7.428236in}{3.265394in}}%
\pgfpathlineto{\pgfqpoint{7.430667in}{3.265029in}}%
\pgfpathlineto{\pgfqpoint{7.431883in}{3.265516in}}%
\pgfpathlineto{\pgfqpoint{7.433707in}{3.265349in}}%
\pgfpathlineto{\pgfqpoint{7.436746in}{3.265212in}}%
\pgfpathlineto{\pgfqpoint{7.442824in}{3.266493in}}%
\pgfpathlineto{\pgfqpoint{7.447687in}{3.265970in}}%
\pgfpathlineto{\pgfqpoint{7.449511in}{3.266358in}}%
\pgfpathlineto{\pgfqpoint{7.454373in}{3.266470in}}%
\pgfpathlineto{\pgfqpoint{7.463491in}{3.266436in}}%
\pgfpathlineto{\pgfqpoint{7.465922in}{3.266817in}}%
\pgfpathlineto{\pgfqpoint{7.481726in}{3.266239in}}%
\pgfpathlineto{\pgfqpoint{7.483550in}{3.266291in}}%
\pgfpathlineto{\pgfqpoint{7.489021in}{3.266384in}}%
\pgfpathlineto{\pgfqpoint{7.490844in}{3.266149in}}%
\pgfpathlineto{\pgfqpoint{7.495707in}{3.266854in}}%
\pgfpathlineto{\pgfqpoint{7.498746in}{3.266657in}}%
\pgfpathlineto{\pgfqpoint{7.509080in}{3.267603in}}%
\pgfpathlineto{\pgfqpoint{7.516982in}{3.266932in}}%
\pgfpathlineto{\pgfqpoint{7.521237in}{3.267376in}}%
\pgfpathlineto{\pgfqpoint{7.532178in}{3.267305in}}%
\pgfpathlineto{\pgfqpoint{7.539472in}{3.267653in}}%
\pgfpathlineto{\pgfqpoint{7.545550in}{3.267090in}}%
\pgfpathlineto{\pgfqpoint{7.547982in}{3.267191in}}%
\pgfpathlineto{\pgfqpoint{7.552845in}{3.267465in}}%
\pgfpathlineto{\pgfqpoint{7.561354in}{3.266775in}}%
\pgfpathlineto{\pgfqpoint{7.563178in}{3.266775in}}%
\pgfpathlineto{\pgfqpoint{7.579590in}{3.267115in}}%
\pgfpathlineto{\pgfqpoint{7.582021in}{3.267281in}}%
\pgfpathlineto{\pgfqpoint{7.584453in}{3.267578in}}%
\pgfpathlineto{\pgfqpoint{7.587492in}{3.267803in}}%
\pgfpathlineto{\pgfqpoint{7.590531in}{3.267067in}}%
\pgfpathlineto{\pgfqpoint{7.599649in}{3.266607in}}%
\pgfpathlineto{\pgfqpoint{7.630649in}{3.264423in}}%
\pgfpathlineto{\pgfqpoint{7.632472in}{3.264856in}}%
\pgfpathlineto{\pgfqpoint{7.639159in}{3.264744in}}%
\pgfpathlineto{\pgfqpoint{7.641590in}{3.264320in}}%
\pgfpathlineto{\pgfqpoint{7.644630in}{3.264553in}}%
\pgfpathlineto{\pgfqpoint{7.647061in}{3.264314in}}%
\pgfpathlineto{\pgfqpoint{7.654355in}{3.264153in}}%
\pgfpathlineto{\pgfqpoint{7.659218in}{3.264279in}}%
\pgfpathlineto{\pgfqpoint{7.661649in}{3.264292in}}%
\pgfpathlineto{\pgfqpoint{7.664688in}{3.264457in}}%
\pgfpathlineto{\pgfqpoint{7.667120in}{3.264003in}}%
\pgfpathlineto{\pgfqpoint{7.669551in}{3.264010in}}%
\pgfpathlineto{\pgfqpoint{7.676237in}{3.263574in}}%
\pgfpathlineto{\pgfqpoint{7.678061in}{3.263259in}}%
\pgfpathlineto{\pgfqpoint{7.681100in}{3.263256in}}%
\pgfpathlineto{\pgfqpoint{7.687787in}{3.263221in}}%
\pgfpathlineto{\pgfqpoint{7.715748in}{3.261793in}}%
\pgfpathlineto{\pgfqpoint{7.718787in}{3.261959in}}%
\pgfpathlineto{\pgfqpoint{7.722434in}{3.261901in}}%
\pgfpathlineto{\pgfqpoint{7.725473in}{3.261695in}}%
\pgfpathlineto{\pgfqpoint{7.749787in}{3.261826in}}%
\pgfpathlineto{\pgfqpoint{7.751003in}{3.262080in}}%
\pgfpathlineto{\pgfqpoint{7.755258in}{3.262003in}}%
\pgfpathlineto{\pgfqpoint{7.814827in}{3.265114in}}%
\pgfpathlineto{\pgfqpoint{7.817866in}{3.264838in}}%
\pgfpathlineto{\pgfqpoint{7.825768in}{3.265558in}}%
\pgfpathlineto{\pgfqpoint{7.840356in}{3.266031in}}%
\pgfpathlineto{\pgfqpoint{7.844611in}{3.266029in}}%
\pgfpathlineto{\pgfqpoint{7.847650in}{3.265895in}}%
\pgfpathlineto{\pgfqpoint{7.851905in}{3.265731in}}%
\pgfpathlineto{\pgfqpoint{7.860415in}{3.266007in}}%
\pgfpathlineto{\pgfqpoint{7.863454in}{3.265941in}}%
\pgfpathlineto{\pgfqpoint{7.867102in}{3.266410in}}%
\pgfpathlineto{\pgfqpoint{7.871964in}{3.266672in}}%
\pgfpathlineto{\pgfqpoint{7.875003in}{3.266552in}}%
\pgfpathlineto{\pgfqpoint{7.883513in}{3.266580in}}%
\pgfpathlineto{\pgfqpoint{7.886553in}{3.266778in}}%
\pgfpathlineto{\pgfqpoint{7.896278in}{3.267779in}}%
\pgfpathlineto{\pgfqpoint{7.913298in}{3.324675in}}%
\pgfpathlineto{\pgfqpoint{7.916945in}{3.334378in}}%
\pgfpathlineto{\pgfqpoint{7.924847in}{3.355255in}}%
\pgfpathlineto{\pgfqpoint{7.947337in}{3.400325in}}%
\pgfpathlineto{\pgfqpoint{7.961926in}{3.424108in}}%
\pgfpathlineto{\pgfqpoint{7.968004in}{3.435785in}}%
\pgfpathlineto{\pgfqpoint{7.971651in}{3.440273in}}%
\pgfpathlineto{\pgfqpoint{7.979553in}{3.450349in}}%
\pgfpathlineto{\pgfqpoint{7.992318in}{3.462939in}}%
\pgfpathlineto{\pgfqpoint{8.030612in}{3.483946in}}%
\pgfpathlineto{\pgfqpoint{8.064652in}{3.503706in}}%
\pgfpathlineto{\pgfqpoint{8.068907in}{3.507487in}}%
\pgfpathlineto{\pgfqpoint{8.078632in}{3.515197in}}%
\pgfpathlineto{\pgfqpoint{8.082279in}{3.519015in}}%
\pgfpathlineto{\pgfqpoint{8.085926in}{3.523448in}}%
\pgfpathlineto{\pgfqpoint{8.094436in}{3.532983in}}%
\pgfpathlineto{\pgfqpoint{8.098691in}{3.539371in}}%
\pgfpathlineto{\pgfqpoint{8.109025in}{3.555779in}}%
\pgfpathlineto{\pgfqpoint{8.116927in}{3.570380in}}%
\pgfpathlineto{\pgfqpoint{8.122397in}{3.581703in}}%
\pgfpathlineto{\pgfqpoint{8.143064in}{3.625264in}}%
\pgfpathlineto{\pgfqpoint{8.163731in}{3.672532in}}%
\pgfpathlineto{\pgfqpoint{8.171025in}{3.689144in}}%
\pgfpathlineto{\pgfqpoint{8.183790in}{3.715888in}}%
\pgfpathlineto{\pgfqpoint{8.192908in}{3.730042in}}%
\pgfpathlineto{\pgfqpoint{8.212966in}{3.751925in}}%
\pgfpathlineto{\pgfqpoint{8.214790in}{3.752987in}}%
\pgfpathlineto{\pgfqpoint{8.217829in}{3.754663in}}%
\pgfpathlineto{\pgfqpoint{8.220868in}{3.755278in}}%
\pgfpathlineto{\pgfqpoint{8.223300in}{3.756171in}}%
\pgfpathlineto{\pgfqpoint{8.226947in}{3.758965in}}%
\pgfpathlineto{\pgfqpoint{8.231202in}{3.761554in}}%
\pgfpathlineto{\pgfqpoint{8.239104in}{3.762820in}}%
\pgfpathlineto{\pgfqpoint{8.241535in}{3.760865in}}%
\pgfpathlineto{\pgfqpoint{8.245790in}{3.760306in}}%
\pgfpathlineto{\pgfqpoint{8.248222in}{3.759859in}}%
\pgfpathlineto{\pgfqpoint{8.251261in}{3.759524in}}%
\pgfpathlineto{\pgfqpoint{8.262202in}{3.753732in}}%
\pgfpathlineto{\pgfqpoint{8.267673in}{3.750845in}}%
\pgfpathlineto{\pgfqpoint{8.269496in}{3.749877in}}%
\pgfpathlineto{\pgfqpoint{8.273143in}{3.748498in}}%
\pgfpathlineto{\pgfqpoint{8.278614in}{3.744420in}}%
\pgfpathlineto{\pgfqpoint{8.284085in}{3.739484in}}%
\pgfpathlineto{\pgfqpoint{8.291987in}{3.733525in}}%
\pgfpathlineto{\pgfqpoint{8.296849in}{3.729018in}}%
\pgfpathlineto{\pgfqpoint{8.302928in}{3.723654in}}%
\pgfpathlineto{\pgfqpoint{8.307183in}{3.719184in}}%
\pgfpathlineto{\pgfqpoint{8.316300in}{3.709668in}}%
\pgfpathlineto{\pgfqpoint{8.317516in}{3.708308in}}%
\pgfpathlineto{\pgfqpoint{8.319948in}{3.706557in}}%
\pgfpathlineto{\pgfqpoint{8.322379in}{3.703969in}}%
\pgfpathlineto{\pgfqpoint{8.325418in}{3.701101in}}%
\pgfpathlineto{\pgfqpoint{8.326634in}{3.700356in}}%
\pgfpathlineto{\pgfqpoint{8.329065in}{3.697581in}}%
\pgfpathlineto{\pgfqpoint{8.334536in}{3.692161in}}%
\pgfpathlineto{\pgfqpoint{8.363105in}{3.664412in}}%
\pgfpathlineto{\pgfqpoint{8.365536in}{3.662847in}}%
\pgfpathlineto{\pgfqpoint{8.368575in}{3.659420in}}%
\pgfpathlineto{\pgfqpoint{8.371615in}{3.657390in}}%
\pgfpathlineto{\pgfqpoint{8.373438in}{3.655956in}}%
\pgfpathlineto{\pgfqpoint{8.379517in}{3.650984in}}%
\pgfpathlineto{\pgfqpoint{8.383772in}{3.646570in}}%
\pgfpathlineto{\pgfqpoint{8.389850in}{3.640629in}}%
\pgfpathlineto{\pgfqpoint{8.391673in}{3.639623in}}%
\pgfpathlineto{\pgfqpoint{8.392889in}{3.638338in}}%
\pgfpathlineto{\pgfqpoint{8.395321in}{3.636085in}}%
\pgfpathlineto{\pgfqpoint{8.397144in}{3.634166in}}%
\pgfpathlineto{\pgfqpoint{8.405046in}{3.626903in}}%
\pgfpathlineto{\pgfqpoint{8.406870in}{3.625469in}}%
\pgfpathlineto{\pgfqpoint{8.414164in}{3.618634in}}%
\pgfpathlineto{\pgfqpoint{8.417203in}{3.615841in}}%
\pgfpathlineto{\pgfqpoint{8.424497in}{3.609434in}}%
\pgfpathlineto{\pgfqpoint{8.433007in}{3.602506in}}%
\pgfpathlineto{\pgfqpoint{8.456105in}{3.592933in}}%
\pgfpathlineto{\pgfqpoint{8.460360in}{3.592393in}}%
\pgfpathlineto{\pgfqpoint{8.468870in}{3.590847in}}%
\pgfpathlineto{\pgfqpoint{8.471909in}{3.590196in}}%
\pgfpathlineto{\pgfqpoint{8.473733in}{3.589674in}}%
\pgfpathlineto{\pgfqpoint{8.478596in}{3.588668in}}%
\pgfpathlineto{\pgfqpoint{8.481635in}{3.588482in}}%
\pgfpathlineto{\pgfqpoint{8.483458in}{3.588650in}}%
\pgfpathlineto{\pgfqpoint{8.484674in}{3.588724in}}%
\pgfpathlineto{\pgfqpoint{8.487713in}{3.588315in}}%
\pgfpathlineto{\pgfqpoint{8.493792in}{3.588017in}}%
\pgfpathlineto{\pgfqpoint{8.498047in}{3.586992in}}%
\pgfpathlineto{\pgfqpoint{8.503517in}{3.587085in}}%
\pgfpathlineto{\pgfqpoint{8.520537in}{3.587793in}}%
\pgfpathlineto{\pgfqpoint{8.524184in}{3.588464in}}%
\pgfpathlineto{\pgfqpoint{8.527223in}{3.588259in}}%
\pgfpathlineto{\pgfqpoint{8.597734in}{3.597887in}}%
\pgfpathlineto{\pgfqpoint{8.603812in}{3.598967in}}%
\pgfpathlineto{\pgfqpoint{8.607459in}{3.599638in}}%
\pgfpathlineto{\pgfqpoint{8.614146in}{3.600662in}}%
\pgfpathlineto{\pgfqpoint{8.615969in}{3.600848in}}%
\pgfpathlineto{\pgfqpoint{8.624479in}{3.602413in}}%
\pgfpathlineto{\pgfqpoint{8.629342in}{3.603698in}}%
\pgfpathlineto{\pgfqpoint{8.634205in}{3.604331in}}%
\pgfpathlineto{\pgfqpoint{8.646969in}{3.606212in}}%
\pgfpathlineto{\pgfqpoint{8.648793in}{3.606659in}}%
\pgfpathlineto{\pgfqpoint{8.660950in}{3.608708in}}%
\pgfpathlineto{\pgfqpoint{8.677970in}{3.611427in}}%
\pgfpathlineto{\pgfqpoint{8.681617in}{3.611930in}}%
\pgfpathlineto{\pgfqpoint{8.687695in}{3.612526in}}%
\pgfpathlineto{\pgfqpoint{8.691950in}{3.613010in}}%
\pgfpathlineto{\pgfqpoint{8.694989in}{3.613419in}}%
\pgfpathlineto{\pgfqpoint{8.697421in}{3.613550in}}%
\pgfpathlineto{\pgfqpoint{8.701676in}{3.614015in}}%
\pgfpathlineto{\pgfqpoint{8.732068in}{3.615859in}}%
\pgfpathlineto{\pgfqpoint{8.736931in}{3.615915in}}%
\pgfpathlineto{\pgfqpoint{8.742401in}{3.615952in}}%
\pgfpathlineto{\pgfqpoint{8.753950in}{3.615785in}}%
\pgfpathlineto{\pgfqpoint{8.755774in}{3.616139in}}%
\pgfpathlineto{\pgfqpoint{8.758813in}{3.616027in}}%
\pgfpathlineto{\pgfqpoint{8.760637in}{3.616083in}}%
\pgfpathlineto{\pgfqpoint{8.764284in}{3.616008in}}%
\pgfpathlineto{\pgfqpoint{8.772794in}{3.615692in}}%
\pgfpathlineto{\pgfqpoint{8.780088in}{3.615003in}}%
\pgfpathlineto{\pgfqpoint{8.783127in}{3.614742in}}%
\pgfpathlineto{\pgfqpoint{8.784951in}{3.614872in}}%
\pgfpathlineto{\pgfqpoint{8.789813in}{3.614481in}}%
\pgfpathlineto{\pgfqpoint{8.791637in}{3.614183in}}%
\pgfpathlineto{\pgfqpoint{8.793460in}{3.614369in}}%
\pgfpathlineto{\pgfqpoint{8.800755in}{3.613587in}}%
\pgfpathlineto{\pgfqpoint{8.803186in}{3.613717in}}%
\pgfpathlineto{\pgfqpoint{8.811088in}{3.612917in}}%
\pgfpathlineto{\pgfqpoint{8.817166in}{3.612600in}}%
\pgfpathlineto{\pgfqpoint{8.820814in}{3.612209in}}%
\pgfpathlineto{\pgfqpoint{8.826892in}{3.611967in}}%
\pgfpathlineto{\pgfqpoint{8.828716in}{3.611315in}}%
\pgfpathlineto{\pgfqpoint{8.834794in}{3.610570in}}%
\pgfpathlineto{\pgfqpoint{8.838441in}{3.610291in}}%
\pgfpathlineto{\pgfqpoint{8.847559in}{3.609639in}}%
\pgfpathlineto{\pgfqpoint{8.851814in}{3.608950in}}%
\pgfpathlineto{\pgfqpoint{8.859108in}{3.607888in}}%
\pgfpathlineto{\pgfqpoint{8.863363in}{3.607721in}}%
\pgfpathlineto{\pgfqpoint{8.870049in}{3.607050in}}%
\pgfpathlineto{\pgfqpoint{8.874912in}{3.606305in}}%
\pgfpathlineto{\pgfqpoint{8.880990in}{3.605393in}}%
\pgfpathlineto{\pgfqpoint{8.904696in}{3.602897in}}%
\pgfpathlineto{\pgfqpoint{8.907736in}{3.602748in}}%
\pgfpathlineto{\pgfqpoint{8.917461in}{3.601761in}}%
\pgfpathlineto{\pgfqpoint{8.919893in}{3.601295in}}%
\pgfpathlineto{\pgfqpoint{8.929618in}{3.600830in}}%
\pgfpathlineto{\pgfqpoint{8.932050in}{3.600718in}}%
\pgfpathlineto{\pgfqpoint{8.937520in}{3.600439in}}%
\pgfpathlineto{\pgfqpoint{8.958795in}{3.599414in}}%
\pgfpathlineto{\pgfqpoint{8.964266in}{3.599452in}}%
\pgfpathlineto{\pgfqpoint{9.020187in}{3.600476in}}%
\pgfpathlineto{\pgfqpoint{9.025050in}{3.600718in}}%
\pgfpathlineto{\pgfqpoint{9.029305in}{3.601314in}}%
\pgfpathlineto{\pgfqpoint{9.032952in}{3.601314in}}%
\pgfpathlineto{\pgfqpoint{9.039031in}{3.601482in}}%
\pgfpathlineto{\pgfqpoint{9.042678in}{3.601910in}}%
\pgfpathlineto{\pgfqpoint{9.045717in}{3.601929in}}%
\pgfpathlineto{\pgfqpoint{9.048148in}{3.602189in}}%
\pgfpathlineto{\pgfqpoint{9.055443in}{3.603065in}}%
\pgfpathlineto{\pgfqpoint{9.056050in}{3.603977in}}%
\pgfpathlineto{\pgfqpoint{9.056658in}{3.603083in}}%
\pgfpathlineto{\pgfqpoint{9.068207in}{3.604201in}}%
\pgfpathlineto{\pgfqpoint{9.072462in}{3.604406in}}%
\pgfpathlineto{\pgfqpoint{9.075501in}{3.604983in}}%
\pgfpathlineto{\pgfqpoint{9.080364in}{3.604629in}}%
\pgfpathlineto{\pgfqpoint{9.082188in}{3.605132in}}%
\pgfpathlineto{\pgfqpoint{9.090698in}{3.606342in}}%
\pgfpathlineto{\pgfqpoint{9.094345in}{3.606342in}}%
\pgfpathlineto{\pgfqpoint{9.097992in}{3.607032in}}%
\pgfpathlineto{\pgfqpoint{9.102247in}{3.606845in}}%
\pgfpathlineto{\pgfqpoint{9.104070in}{3.607478in}}%
\pgfpathlineto{\pgfqpoint{9.108325in}{3.607776in}}%
\pgfpathlineto{\pgfqpoint{9.110149in}{3.608391in}}%
\pgfpathlineto{\pgfqpoint{9.111364in}{3.607944in}}%
\pgfpathlineto{\pgfqpoint{9.113188in}{3.609117in}}%
\pgfpathlineto{\pgfqpoint{9.113796in}{3.607832in}}%
\pgfpathlineto{\pgfqpoint{9.114404in}{3.608317in}}%
\pgfpathlineto{\pgfqpoint{9.118659in}{3.608596in}}%
\pgfpathlineto{\pgfqpoint{9.119874in}{3.609210in}}%
\pgfpathlineto{\pgfqpoint{9.127776in}{3.609676in}}%
\pgfpathlineto{\pgfqpoint{9.132639in}{3.610179in}}%
\pgfpathlineto{\pgfqpoint{9.135071in}{3.610496in}}%
\pgfpathlineto{\pgfqpoint{9.142973in}{3.611110in}}%
\pgfpathlineto{\pgfqpoint{9.146012in}{3.611222in}}%
\pgfpathlineto{\pgfqpoint{9.173365in}{3.613066in}}%
\pgfpathlineto{\pgfqpoint{9.174581in}{3.613196in}}%
\pgfpathlineto{\pgfqpoint{9.177620in}{3.613848in}}%
\pgfpathlineto{\pgfqpoint{9.186130in}{3.614109in}}%
\pgfpathlineto{\pgfqpoint{9.190992in}{3.614220in}}%
\pgfpathlineto{\pgfqpoint{9.192208in}{3.614593in}}%
\pgfpathlineto{\pgfqpoint{9.193424in}{3.614816in}}%
\pgfpathlineto{\pgfqpoint{9.195247in}{3.614332in}}%
\pgfpathlineto{\pgfqpoint{9.209228in}{3.615394in}}%
\pgfpathlineto{\pgfqpoint{9.210444in}{3.615114in}}%
\pgfpathlineto{\pgfqpoint{9.212267in}{3.614742in}}%
\pgfpathlineto{\pgfqpoint{9.215306in}{3.614202in}}%
\pgfpathlineto{\pgfqpoint{9.216522in}{3.614928in}}%
\pgfpathlineto{\pgfqpoint{9.218953in}{3.614574in}}%
\pgfpathlineto{\pgfqpoint{9.262111in}{3.614015in}}%
\pgfpathlineto{\pgfqpoint{9.263934in}{3.613252in}}%
\pgfpathlineto{\pgfqpoint{9.266365in}{3.613885in}}%
\pgfpathlineto{\pgfqpoint{9.268797in}{3.613680in}}%
\pgfpathlineto{\pgfqpoint{9.271228in}{3.613326in}}%
\pgfpathlineto{\pgfqpoint{9.273660in}{3.613364in}}%
\pgfpathlineto{\pgfqpoint{9.280346in}{3.613606in}}%
\pgfpathlineto{\pgfqpoint{9.281562in}{3.612228in}}%
\pgfpathlineto{\pgfqpoint{9.282169in}{3.613122in}}%
\pgfpathlineto{\pgfqpoint{9.282777in}{3.612581in}}%
\pgfpathlineto{\pgfqpoint{9.287032in}{3.612544in}}%
\pgfpathlineto{\pgfqpoint{9.288248in}{3.613270in}}%
\pgfpathlineto{\pgfqpoint{9.289464in}{3.612041in}}%
\pgfpathlineto{\pgfqpoint{9.290679in}{3.612544in}}%
\pgfpathlineto{\pgfqpoint{9.292503in}{3.611725in}}%
\pgfpathlineto{\pgfqpoint{9.296758in}{3.613066in}}%
\pgfpathlineto{\pgfqpoint{9.297974in}{3.611557in}}%
\pgfpathlineto{\pgfqpoint{9.306483in}{3.610775in}}%
\pgfpathlineto{\pgfqpoint{9.308307in}{3.610756in}}%
\pgfpathlineto{\pgfqpoint{9.311346in}{3.610328in}}%
\pgfpathlineto{\pgfqpoint{9.312562in}{3.610365in}}%
\pgfpathlineto{\pgfqpoint{9.313170in}{3.610775in}}%
\pgfpathlineto{\pgfqpoint{9.313778in}{3.609639in}}%
\pgfpathlineto{\pgfqpoint{9.314385in}{3.610179in}}%
\pgfpathlineto{\pgfqpoint{9.315601in}{3.610719in}}%
\pgfpathlineto{\pgfqpoint{9.317425in}{3.609881in}}%
\pgfpathlineto{\pgfqpoint{9.319248in}{3.609471in}}%
\pgfpathlineto{\pgfqpoint{9.321072in}{3.609713in}}%
\pgfpathlineto{\pgfqpoint{9.322287in}{3.609583in}}%
\pgfpathlineto{\pgfqpoint{9.328974in}{3.608186in}}%
\pgfpathlineto{\pgfqpoint{9.330189in}{3.609676in}}%
\pgfpathlineto{\pgfqpoint{9.332013in}{3.608186in}}%
\pgfpathlineto{\pgfqpoint{9.335052in}{3.607478in}}%
\pgfpathlineto{\pgfqpoint{9.336876in}{3.609248in}}%
\pgfpathlineto{\pgfqpoint{9.338699in}{3.607572in}}%
\pgfpathlineto{\pgfqpoint{9.345386in}{3.606510in}}%
\pgfpathlineto{\pgfqpoint{9.345993in}{3.607292in}}%
\pgfpathlineto{\pgfqpoint{9.346601in}{3.606659in}}%
\pgfpathlineto{\pgfqpoint{9.347817in}{3.606398in}}%
\pgfpathlineto{\pgfqpoint{9.352072in}{3.606249in}}%
\pgfpathlineto{\pgfqpoint{9.353895in}{3.605020in}}%
\pgfpathlineto{\pgfqpoint{9.356935in}{3.605951in}}%
\pgfpathlineto{\pgfqpoint{9.358758in}{3.604983in}}%
\pgfpathlineto{\pgfqpoint{9.359366in}{3.605672in}}%
\pgfpathlineto{\pgfqpoint{9.359974in}{3.604983in}}%
\pgfpathlineto{\pgfqpoint{9.361797in}{3.604741in}}%
\pgfpathlineto{\pgfqpoint{9.364837in}{3.604927in}}%
\pgfpathlineto{\pgfqpoint{9.366660in}{3.604871in}}%
\pgfpathlineto{\pgfqpoint{9.367876in}{3.604219in}}%
\pgfpathlineto{\pgfqpoint{9.368484in}{3.606268in}}%
\pgfpathlineto{\pgfqpoint{9.369092in}{3.606026in}}%
\pgfpathlineto{\pgfqpoint{9.369699in}{3.604685in}}%
\pgfpathlineto{\pgfqpoint{9.370307in}{3.606175in}}%
\pgfpathlineto{\pgfqpoint{9.371523in}{3.603754in}}%
\pgfpathlineto{\pgfqpoint{9.375778in}{3.603884in}}%
\pgfpathlineto{\pgfqpoint{9.376994in}{3.605244in}}%
\pgfpathlineto{\pgfqpoint{9.378817in}{3.603865in}}%
\pgfpathlineto{\pgfqpoint{9.381249in}{3.603139in}}%
\pgfpathlineto{\pgfqpoint{9.383072in}{3.605150in}}%
\pgfpathlineto{\pgfqpoint{9.384288in}{3.602636in}}%
\pgfpathlineto{\pgfqpoint{9.384896in}{3.603139in}}%
\pgfpathlineto{\pgfqpoint{9.387935in}{3.602469in}}%
\pgfpathlineto{\pgfqpoint{9.388543in}{3.604406in}}%
\pgfpathlineto{\pgfqpoint{9.389151in}{3.603251in}}%
\pgfpathlineto{\pgfqpoint{9.394621in}{3.602003in}}%
\pgfpathlineto{\pgfqpoint{9.396445in}{3.604331in}}%
\pgfpathlineto{\pgfqpoint{9.398268in}{3.602506in}}%
\pgfpathlineto{\pgfqpoint{9.400092in}{3.601631in}}%
\pgfpathlineto{\pgfqpoint{9.403739in}{3.601910in}}%
\pgfpathlineto{\pgfqpoint{9.404347in}{3.600085in}}%
\pgfpathlineto{\pgfqpoint{9.404955in}{3.600439in}}%
\pgfpathlineto{\pgfqpoint{9.406778in}{3.602040in}}%
\pgfpathlineto{\pgfqpoint{9.408602in}{3.601351in}}%
\pgfpathlineto{\pgfqpoint{9.409209in}{3.603921in}}%
\pgfpathlineto{\pgfqpoint{9.409817in}{3.602189in}}%
\pgfpathlineto{\pgfqpoint{9.413464in}{3.600848in}}%
\pgfpathlineto{\pgfqpoint{9.414680in}{3.601090in}}%
\pgfpathlineto{\pgfqpoint{9.416504in}{3.599880in}}%
\pgfpathlineto{\pgfqpoint{9.418327in}{3.600644in}}%
\pgfpathlineto{\pgfqpoint{9.419543in}{3.599954in}}%
\pgfpathlineto{\pgfqpoint{9.421974in}{3.600588in}}%
\pgfpathlineto{\pgfqpoint{9.423798in}{3.599340in}}%
\pgfpathlineto{\pgfqpoint{9.425014in}{3.600662in}}%
\pgfpathlineto{\pgfqpoint{9.425621in}{3.600383in}}%
\pgfpathlineto{\pgfqpoint{9.426229in}{3.598874in}}%
\pgfpathlineto{\pgfqpoint{9.426837in}{3.599154in}}%
\pgfpathlineto{\pgfqpoint{9.428661in}{3.599489in}}%
\pgfpathlineto{\pgfqpoint{9.431700in}{3.598874in}}%
\pgfpathlineto{\pgfqpoint{9.433523in}{3.598725in}}%
\pgfpathlineto{\pgfqpoint{9.451151in}{3.598092in}}%
\pgfpathlineto{\pgfqpoint{9.452974in}{3.597589in}}%
\pgfpathlineto{\pgfqpoint{9.479112in}{3.597980in}}%
\pgfpathlineto{\pgfqpoint{9.480328in}{3.597124in}}%
\pgfpathlineto{\pgfqpoint{9.480935in}{3.598129in}}%
\pgfpathlineto{\pgfqpoint{9.481543in}{3.597608in}}%
\pgfpathlineto{\pgfqpoint{9.482151in}{3.597608in}}%
\pgfpathlineto{\pgfqpoint{9.482759in}{3.598725in}}%
\pgfpathlineto{\pgfqpoint{9.483367in}{3.597366in}}%
\pgfpathlineto{\pgfqpoint{9.483975in}{3.599079in}}%
\pgfpathlineto{\pgfqpoint{9.484583in}{3.598409in}}%
\pgfpathlineto{\pgfqpoint{9.487014in}{3.597422in}}%
\pgfpathlineto{\pgfqpoint{9.497347in}{3.598334in}}%
\pgfpathlineto{\pgfqpoint{9.499171in}{3.597626in}}%
\pgfpathlineto{\pgfqpoint{9.500387in}{3.597999in}}%
\pgfpathlineto{\pgfqpoint{9.500994in}{3.599284in}}%
\pgfpathlineto{\pgfqpoint{9.502210in}{3.597869in}}%
\pgfpathlineto{\pgfqpoint{9.502818in}{3.600066in}}%
\pgfpathlineto{\pgfqpoint{9.503426in}{3.598297in}}%
\pgfpathlineto{\pgfqpoint{9.505857in}{3.597869in}}%
\pgfpathlineto{\pgfqpoint{9.510112in}{3.598930in}}%
\pgfpathlineto{\pgfqpoint{9.511936in}{3.598390in}}%
\pgfpathlineto{\pgfqpoint{9.513759in}{3.598167in}}%
\pgfpathlineto{\pgfqpoint{9.514367in}{3.599880in}}%
\pgfpathlineto{\pgfqpoint{9.514975in}{3.598502in}}%
\pgfpathlineto{\pgfqpoint{9.519838in}{3.598818in}}%
\pgfpathlineto{\pgfqpoint{9.522877in}{3.599619in}}%
\pgfpathlineto{\pgfqpoint{9.524700in}{3.598669in}}%
\pgfpathlineto{\pgfqpoint{9.526524in}{3.598967in}}%
\pgfpathlineto{\pgfqpoint{9.527132in}{3.598856in}}%
\pgfpathlineto{\pgfqpoint{9.527740in}{3.600588in}}%
\pgfpathlineto{\pgfqpoint{9.528348in}{3.600364in}}%
\pgfpathlineto{\pgfqpoint{9.530171in}{3.599452in}}%
\pgfpathlineto{\pgfqpoint{9.531995in}{3.599526in}}%
\pgfpathlineto{\pgfqpoint{9.533818in}{3.599527in}}%
\pgfpathlineto{\pgfqpoint{9.595819in}{3.811139in}}%
\pgfpathlineto{\pgfqpoint{9.612230in}{3.859904in}}%
\pgfpathlineto{\pgfqpoint{9.620740in}{3.884820in}}%
\pgfpathlineto{\pgfqpoint{9.629858in}{3.910188in}}%
\pgfpathlineto{\pgfqpoint{9.646878in}{3.954530in}}%
\pgfpathlineto{\pgfqpoint{9.657211in}{3.976885in}}%
\pgfpathlineto{\pgfqpoint{9.661466in}{3.985779in}}%
\pgfpathlineto{\pgfqpoint{9.678486in}{4.017085in}}%
\pgfpathlineto{\pgfqpoint{9.684564in}{4.019305in}}%
\pgfpathlineto{\pgfqpoint{9.685780in}{4.019465in}}%
\pgfpathlineto{\pgfqpoint{9.688819in}{4.021639in}}%
\pgfpathlineto{\pgfqpoint{9.722251in}{4.051011in}}%
\pgfpathlineto{\pgfqpoint{9.726506in}{4.053862in}}%
\pgfpathlineto{\pgfqpoint{9.728329in}{4.054273in}}%
\pgfpathlineto{\pgfqpoint{9.733800in}{4.057336in}}%
\pgfpathlineto{\pgfqpoint{9.743525in}{4.060411in}}%
\pgfpathlineto{\pgfqpoint{9.745349in}{4.060570in}}%
\pgfpathlineto{\pgfqpoint{9.747172in}{4.060691in}}%
\pgfpathlineto{\pgfqpoint{9.774526in}{4.060045in}}%
\pgfpathlineto{\pgfqpoint{9.783643in}{4.057886in}}%
\pgfpathlineto{\pgfqpoint{9.786075in}{4.058007in}}%
\pgfpathlineto{\pgfqpoint{9.797016in}{4.057735in}}%
\pgfpathlineto{\pgfqpoint{9.803094in}{4.055078in}}%
\pgfpathlineto{\pgfqpoint{9.804918in}{4.054667in}}%
\pgfpathlineto{\pgfqpoint{9.808565in}{4.053190in}}%
\pgfpathlineto{\pgfqpoint{9.812212in}{4.051899in}}%
\pgfpathlineto{\pgfqpoint{9.818291in}{4.050583in}}%
\pgfpathlineto{\pgfqpoint{9.820114in}{4.050537in}}%
\pgfpathlineto{\pgfqpoint{9.821938in}{4.050101in}}%
\pgfpathlineto{\pgfqpoint{9.836526in}{4.043800in}}%
\pgfpathlineto{\pgfqpoint{9.838350in}{4.043561in}}%
\pgfpathlineto{\pgfqpoint{9.865095in}{4.034869in}}%
\pgfpathlineto{\pgfqpoint{9.871781in}{4.033397in}}%
\pgfpathlineto{\pgfqpoint{9.875428in}{4.032265in}}%
\pgfpathlineto{\pgfqpoint{9.878467in}{4.031380in}}%
\pgfpathlineto{\pgfqpoint{9.880899in}{4.030696in}}%
\pgfpathlineto{\pgfqpoint{9.891840in}{4.027978in}}%
\pgfpathlineto{\pgfqpoint{9.894879in}{4.026895in}}%
\pgfpathlineto{\pgfqpoint{9.902781in}{4.024393in}}%
\pgfpathlineto{\pgfqpoint{9.908860in}{4.023464in}}%
\pgfpathlineto{\pgfqpoint{9.915546in}{4.022339in}}%
\pgfpathlineto{\pgfqpoint{9.920409in}{4.020716in}}%
\pgfpathlineto{\pgfqpoint{9.922232in}{4.019943in}}%
\pgfpathlineto{\pgfqpoint{9.929527in}{4.017812in}}%
\pgfpathlineto{\pgfqpoint{9.931958in}{4.018256in}}%
\pgfpathlineto{\pgfqpoint{9.936821in}{4.018221in}}%
\pgfpathlineto{\pgfqpoint{9.950801in}{4.019505in}}%
\pgfpathlineto{\pgfqpoint{9.955664in}{4.019007in}}%
\pgfpathlineto{\pgfqpoint{9.958703in}{4.018598in}}%
\pgfpathlineto{\pgfqpoint{9.968429in}{4.016193in}}%
\pgfpathlineto{\pgfqpoint{9.983625in}{4.014489in}}%
\pgfpathlineto{\pgfqpoint{9.992743in}{4.012699in}}%
\pgfpathlineto{\pgfqpoint{10.007331in}{4.010823in}}%
\pgfpathlineto{\pgfqpoint{10.009762in}{4.010549in}}%
\pgfpathlineto{\pgfqpoint{10.012802in}{4.010618in}}%
\pgfpathlineto{\pgfqpoint{10.037116in}{4.008015in}}%
\pgfpathlineto{\pgfqpoint{10.038939in}{4.008526in}}%
\pgfpathlineto{\pgfqpoint{10.040763in}{4.008233in}}%
\pgfpathlineto{\pgfqpoint{10.052920in}{4.008126in}}%
\pgfpathlineto{\pgfqpoint{10.057175in}{4.006814in}}%
\pgfpathlineto{\pgfqpoint{10.060214in}{4.006779in}}%
\pgfpathlineto{\pgfqpoint{10.062037in}{4.006193in}}%
\pgfpathlineto{\pgfqpoint{10.067508in}{4.006602in}}%
\pgfpathlineto{\pgfqpoint{10.069939in}{4.006733in}}%
\pgfpathlineto{\pgfqpoint{10.074194in}{4.006120in}}%
\pgfpathlineto{\pgfqpoint{10.077841in}{4.005944in}}%
\pgfpathlineto{\pgfqpoint{10.084528in}{4.006122in}}%
\pgfpathlineto{\pgfqpoint{10.113704in}{4.005111in}}%
\pgfpathlineto{\pgfqpoint{10.121606in}{4.004403in}}%
\pgfpathlineto{\pgfqpoint{10.124038in}{4.005166in}}%
\pgfpathlineto{\pgfqpoint{10.129508in}{4.005570in}}%
\pgfpathlineto{\pgfqpoint{10.130724in}{4.005872in}}%
\pgfpathlineto{\pgfqpoint{10.134371in}{4.005693in}}%
\pgfpathlineto{\pgfqpoint{10.141665in}{4.006280in}}%
\pgfpathlineto{\pgfqpoint{10.142881in}{4.005790in}}%
\pgfpathlineto{\pgfqpoint{10.147136in}{4.006015in}}%
\pgfpathlineto{\pgfqpoint{10.164156in}{4.007118in}}%
\pgfpathlineto{\pgfqpoint{10.168410in}{4.006415in}}%
\pgfpathlineto{\pgfqpoint{10.170842in}{4.006267in}}%
\pgfpathlineto{\pgfqpoint{10.172665in}{4.006357in}}%
\pgfpathlineto{\pgfqpoint{10.181175in}{4.005798in}}%
\pgfpathlineto{\pgfqpoint{10.182391in}{4.005227in}}%
\pgfpathlineto{\pgfqpoint{10.189685in}{4.006582in}}%
\pgfpathlineto{\pgfqpoint{10.190901in}{4.006078in}}%
\pgfpathlineto{\pgfqpoint{10.193940in}{4.007026in}}%
\pgfpathlineto{\pgfqpoint{10.198195in}{4.006770in}}%
\pgfpathlineto{\pgfqpoint{10.209136in}{4.007526in}}%
\pgfpathlineto{\pgfqpoint{10.210960in}{4.007271in}}%
\pgfpathlineto{\pgfqpoint{10.214607in}{4.007641in}}%
\pgfpathlineto{\pgfqpoint{10.216430in}{4.007398in}}%
\pgfpathlineto{\pgfqpoint{10.219470in}{4.007910in}}%
\pgfpathlineto{\pgfqpoint{10.221293in}{4.007643in}}%
\pgfpathlineto{\pgfqpoint{10.223725in}{4.008376in}}%
\pgfpathlineto{\pgfqpoint{10.225548in}{4.008398in}}%
\pgfpathlineto{\pgfqpoint{10.227372in}{4.008955in}}%
\pgfpathlineto{\pgfqpoint{10.231019in}{4.009001in}}%
\pgfpathlineto{\pgfqpoint{10.232234in}{4.008265in}}%
\pgfpathlineto{\pgfqpoint{10.238921in}{4.009203in}}%
\pgfpathlineto{\pgfqpoint{10.240744in}{4.008819in}}%
\pgfpathlineto{\pgfqpoint{10.242568in}{4.008749in}}%
\pgfpathlineto{\pgfqpoint{10.244999in}{4.007899in}}%
\pgfpathlineto{\pgfqpoint{10.249254in}{4.008436in}}%
\pgfpathlineto{\pgfqpoint{10.250470in}{4.008060in}}%
\pgfpathlineto{\pgfqpoint{10.254117in}{4.008407in}}%
\pgfpathlineto{\pgfqpoint{10.257764in}{4.009120in}}%
\pgfpathlineto{\pgfqpoint{10.261411in}{4.009343in}}%
\pgfpathlineto{\pgfqpoint{10.263842in}{4.009639in}}%
\pgfpathlineto{\pgfqpoint{10.270529in}{4.008819in}}%
\pgfpathlineto{\pgfqpoint{10.272960in}{4.008447in}}%
\pgfpathlineto{\pgfqpoint{10.274176in}{4.008177in}}%
\pgfpathlineto{\pgfqpoint{10.275999in}{4.008213in}}%
\pgfpathlineto{\pgfqpoint{10.277823in}{4.007755in}}%
\pgfpathlineto{\pgfqpoint{10.282686in}{4.007097in}}%
\pgfpathlineto{\pgfqpoint{10.285725in}{4.006946in}}%
\pgfpathlineto{\pgfqpoint{10.289372in}{4.007181in}}%
\pgfpathlineto{\pgfqpoint{10.292411in}{4.007546in}}%
\pgfpathlineto{\pgfqpoint{10.305176in}{4.005468in}}%
\pgfpathlineto{\pgfqpoint{10.307000in}{4.006231in}}%
\pgfpathlineto{\pgfqpoint{10.311255in}{4.005783in}}%
\pgfpathlineto{\pgfqpoint{10.316117in}{4.006438in}}%
\pgfpathlineto{\pgfqpoint{10.317941in}{4.006344in}}%
\pgfpathlineto{\pgfqpoint{10.320980in}{4.006838in}}%
\pgfpathlineto{\pgfqpoint{10.322804in}{4.006067in}}%
\pgfpathlineto{\pgfqpoint{10.324627in}{4.006126in}}%
\pgfpathlineto{\pgfqpoint{10.334353in}{4.005811in}}%
\pgfpathlineto{\pgfqpoint{10.335568in}{4.005508in}}%
\pgfpathlineto{\pgfqpoint{10.336784in}{4.006276in}}%
\pgfpathlineto{\pgfqpoint{10.339215in}{4.004963in}}%
\pgfpathlineto{\pgfqpoint{10.351372in}{4.006164in}}%
\pgfpathlineto{\pgfqpoint{10.352588in}{4.005014in}}%
\pgfpathlineto{\pgfqpoint{10.372039in}{4.005874in}}%
\pgfpathlineto{\pgfqpoint{10.373255in}{4.006397in}}%
\pgfpathlineto{\pgfqpoint{10.375079in}{4.005951in}}%
\pgfpathlineto{\pgfqpoint{10.379941in}{4.006920in}}%
\pgfpathlineto{\pgfqpoint{10.384804in}{4.007485in}}%
\pgfpathlineto{\pgfqpoint{10.386628in}{4.007954in}}%
\pgfpathlineto{\pgfqpoint{10.390883in}{4.009458in}}%
\pgfpathlineto{\pgfqpoint{10.392706in}{4.009190in}}%
\pgfpathlineto{\pgfqpoint{10.395745in}{4.009886in}}%
\pgfpathlineto{\pgfqpoint{10.397569in}{4.010218in}}%
\pgfpathlineto{\pgfqpoint{10.399392in}{4.010986in}}%
\pgfpathlineto{\pgfqpoint{10.400000in}{4.010595in}}%
\pgfpathlineto{\pgfqpoint{10.400608in}{4.012063in}}%
\pgfpathlineto{\pgfqpoint{10.401216in}{4.011205in}}%
\pgfpathlineto{\pgfqpoint{10.403040in}{4.011532in}}%
\pgfpathlineto{\pgfqpoint{10.404863in}{4.012653in}}%
\pgfpathlineto{\pgfqpoint{10.406687in}{4.012417in}}%
\pgfpathlineto{\pgfqpoint{10.413981in}{4.014068in}}%
\pgfpathlineto{\pgfqpoint{10.415196in}{4.013071in}}%
\pgfpathlineto{\pgfqpoint{10.432216in}{4.015981in}}%
\pgfpathlineto{\pgfqpoint{10.440118in}{4.016766in}}%
\pgfpathlineto{\pgfqpoint{10.443765in}{4.019245in}}%
\pgfpathlineto{\pgfqpoint{10.445589in}{4.020038in}}%
\pgfpathlineto{\pgfqpoint{10.452275in}{4.022722in}}%
\pgfpathlineto{\pgfqpoint{10.455922in}{4.025539in}}%
\pgfpathlineto{\pgfqpoint{10.457138in}{4.026028in}}%
\pgfpathlineto{\pgfqpoint{10.460177in}{4.028993in}}%
\pgfpathlineto{\pgfqpoint{10.461393in}{4.028843in}}%
\pgfpathlineto{\pgfqpoint{10.463216in}{4.029697in}}%
\pgfpathlineto{\pgfqpoint{10.469903in}{4.032879in}}%
\pgfpathlineto{\pgfqpoint{10.475373in}{4.036105in}}%
\pgfpathlineto{\pgfqpoint{10.479628in}{4.037334in}}%
\pgfpathlineto{\pgfqpoint{10.482667in}{4.040329in}}%
\pgfpathlineto{\pgfqpoint{10.483883in}{4.040186in}}%
\pgfpathlineto{\pgfqpoint{10.485707in}{4.041950in}}%
\pgfpathlineto{\pgfqpoint{10.487530in}{4.042187in}}%
\pgfpathlineto{\pgfqpoint{10.492393in}{4.044862in}}%
\pgfpathlineto{\pgfqpoint{10.495432in}{4.046038in}}%
\pgfpathlineto{\pgfqpoint{10.497256in}{4.046657in}}%
\pgfpathlineto{\pgfqpoint{10.501511in}{4.047604in}}%
\pgfpathlineto{\pgfqpoint{10.504550in}{4.049633in}}%
\pgfpathlineto{\pgfqpoint{10.508197in}{4.050752in}}%
\pgfpathlineto{\pgfqpoint{10.512452in}{4.052855in}}%
\pgfpathlineto{\pgfqpoint{10.513060in}{4.052071in}}%
\pgfpathlineto{\pgfqpoint{10.513668in}{4.052576in}}%
\pgfpathlineto{\pgfqpoint{10.517923in}{4.053329in}}%
\pgfpathlineto{\pgfqpoint{10.519138in}{4.053651in}}%
\pgfpathlineto{\pgfqpoint{10.523393in}{4.053683in}}%
\pgfpathlineto{\pgfqpoint{10.528864in}{4.054213in}}%
\pgfpathlineto{\pgfqpoint{10.531295in}{4.053776in}}%
\pgfpathlineto{\pgfqpoint{10.533119in}{4.053671in}}%
\pgfpathlineto{\pgfqpoint{10.543452in}{4.052694in}}%
\pgfpathlineto{\pgfqpoint{10.544668in}{4.051972in}}%
\pgfpathlineto{\pgfqpoint{10.548923in}{4.053055in}}%
\pgfpathlineto{\pgfqpoint{10.551354in}{4.052922in}}%
\pgfpathlineto{\pgfqpoint{10.553786in}{4.051665in}}%
\pgfpathlineto{\pgfqpoint{10.555609in}{4.051837in}}%
\pgfpathlineto{\pgfqpoint{10.558648in}{4.051082in}}%
\pgfpathlineto{\pgfqpoint{10.615786in}{4.034126in}}%
\pgfpathlineto{\pgfqpoint{10.617610in}{4.034197in}}%
\pgfpathlineto{\pgfqpoint{10.620649in}{4.034125in}}%
\pgfpathlineto{\pgfqpoint{10.627943in}{4.032904in}}%
\pgfpathlineto{\pgfqpoint{10.630374in}{4.032051in}}%
\pgfpathlineto{\pgfqpoint{10.630982in}{4.032677in}}%
\pgfpathlineto{\pgfqpoint{10.631590in}{4.032073in}}%
\pgfpathlineto{\pgfqpoint{10.633413in}{4.031001in}}%
\pgfpathlineto{\pgfqpoint{10.634629in}{4.032027in}}%
\pgfpathlineto{\pgfqpoint{10.636453in}{4.030897in}}%
\pgfpathlineto{\pgfqpoint{10.638276in}{4.030519in}}%
\pgfpathlineto{\pgfqpoint{10.640100in}{4.030219in}}%
\pgfpathlineto{\pgfqpoint{10.641923in}{4.030043in}}%
\pgfpathlineto{\pgfqpoint{10.643747in}{4.030229in}}%
\pgfpathlineto{\pgfqpoint{10.708787in}{4.024830in}}%
\pgfpathlineto{\pgfqpoint{10.713041in}{4.024904in}}%
\pgfpathlineto{\pgfqpoint{10.719120in}{4.024779in}}%
\pgfpathlineto{\pgfqpoint{10.721551in}{4.023580in}}%
\pgfpathlineto{\pgfqpoint{10.727630in}{4.023779in}}%
\pgfpathlineto{\pgfqpoint{10.730061in}{4.023357in}}%
\pgfpathlineto{\pgfqpoint{10.750120in}{4.023330in}}%
\pgfpathlineto{\pgfqpoint{10.753767in}{4.023609in}}%
\pgfpathlineto{\pgfqpoint{10.754983in}{4.023343in}}%
\pgfpathlineto{\pgfqpoint{10.759846in}{4.023283in}}%
\pgfpathlineto{\pgfqpoint{10.764101in}{4.022392in}}%
\pgfpathlineto{\pgfqpoint{10.765316in}{4.021964in}}%
\pgfpathlineto{\pgfqpoint{10.767140in}{4.021971in}}%
\pgfpathlineto{\pgfqpoint{10.768356in}{4.022065in}}%
\pgfpathlineto{\pgfqpoint{10.771395in}{4.022397in}}%
\pgfpathlineto{\pgfqpoint{10.775042in}{4.022686in}}%
\pgfpathlineto{\pgfqpoint{10.776865in}{4.022673in}}%
\pgfpathlineto{\pgfqpoint{10.779905in}{4.022807in}}%
\pgfpathlineto{\pgfqpoint{10.784767in}{4.022745in}}%
\pgfpathlineto{\pgfqpoint{10.785983in}{4.022673in}}%
\pgfpathlineto{\pgfqpoint{10.787807in}{4.022652in}}%
\pgfpathlineto{\pgfqpoint{10.788415in}{4.023114in}}%
\pgfpathlineto{\pgfqpoint{10.789022in}{4.022226in}}%
\pgfpathlineto{\pgfqpoint{10.809081in}{4.022606in}}%
\pgfpathlineto{\pgfqpoint{10.815768in}{4.023030in}}%
\pgfpathlineto{\pgfqpoint{10.872297in}{4.019589in}}%
\pgfpathlineto{\pgfqpoint{10.878376in}{4.019225in}}%
\pgfpathlineto{\pgfqpoint{10.880807in}{4.019357in}}%
\pgfpathlineto{\pgfqpoint{10.884454in}{4.019954in}}%
\pgfpathlineto{\pgfqpoint{10.888709in}{4.019475in}}%
\pgfpathlineto{\pgfqpoint{10.892356in}{4.019550in}}%
\pgfpathlineto{\pgfqpoint{10.896611in}{4.019451in}}%
\pgfpathlineto{\pgfqpoint{10.900258in}{4.019203in}}%
\pgfpathlineto{\pgfqpoint{10.902690in}{4.018962in}}%
\pgfpathlineto{\pgfqpoint{10.916062in}{4.018814in}}%
\pgfpathlineto{\pgfqpoint{10.919710in}{4.019061in}}%
\pgfpathlineto{\pgfqpoint{10.922141in}{4.018843in}}%
\pgfpathlineto{\pgfqpoint{10.931259in}{4.018083in}}%
\pgfpathlineto{\pgfqpoint{10.933082in}{4.018067in}}%
\pgfpathlineto{\pgfqpoint{10.950710in}{4.019636in}}%
\pgfpathlineto{\pgfqpoint{10.952533in}{4.019116in}}%
\pgfpathlineto{\pgfqpoint{10.956788in}{4.019795in}}%
\pgfpathlineto{\pgfqpoint{10.958612in}{4.019390in}}%
\pgfpathlineto{\pgfqpoint{10.963474in}{4.019469in}}%
\pgfpathlineto{\pgfqpoint{10.965906in}{4.019355in}}%
\pgfpathlineto{\pgfqpoint{10.967729in}{4.018798in}}%
\pgfpathlineto{\pgfqpoint{10.968945in}{4.018757in}}%
\pgfpathlineto{\pgfqpoint{10.971984in}{4.018999in}}%
\pgfpathlineto{\pgfqpoint{10.975024in}{4.018525in}}%
\pgfpathlineto{\pgfqpoint{10.981102in}{4.019314in}}%
\pgfpathlineto{\pgfqpoint{10.983533in}{4.018836in}}%
\pgfpathlineto{\pgfqpoint{10.986573in}{4.019329in}}%
\pgfpathlineto{\pgfqpoint{10.991435in}{4.019434in}}%
\pgfpathlineto{\pgfqpoint{10.994475in}{4.019196in}}%
\pgfpathlineto{\pgfqpoint{11.000553in}{4.018866in}}%
\pgfpathlineto{\pgfqpoint{11.005416in}{4.018694in}}%
\pgfpathlineto{\pgfqpoint{11.009063in}{4.018980in}}%
\pgfpathlineto{\pgfqpoint{11.015749in}{4.018398in}}%
\pgfpathlineto{\pgfqpoint{11.020004in}{4.018530in}}%
\pgfpathlineto{\pgfqpoint{11.022436in}{4.018254in}}%
\pgfpathlineto{\pgfqpoint{11.025475in}{4.018361in}}%
\pgfpathlineto{\pgfqpoint{11.036416in}{4.017654in}}%
\pgfpathlineto{\pgfqpoint{11.054044in}{4.017172in}}%
\pgfpathlineto{\pgfqpoint{11.055867in}{4.016908in}}%
\pgfpathlineto{\pgfqpoint{11.080789in}{4.016011in}}%
\pgfpathlineto{\pgfqpoint{11.082613in}{4.016372in}}%
\pgfpathlineto{\pgfqpoint{11.084436in}{4.015842in}}%
\pgfpathlineto{\pgfqpoint{11.089299in}{4.015656in}}%
\pgfpathlineto{\pgfqpoint{11.100848in}{4.015560in}}%
\pgfpathlineto{\pgfqpoint{11.104495in}{4.015792in}}%
\pgfpathlineto{\pgfqpoint{11.108750in}{4.015762in}}%
\pgfpathlineto{\pgfqpoint{11.112397in}{4.015815in}}%
\pgfpathlineto{\pgfqpoint{11.114221in}{4.015461in}}%
\pgfpathlineto{\pgfqpoint{11.121515in}{4.016142in}}%
\pgfpathlineto{\pgfqpoint{11.123338in}{4.016381in}}%
\pgfpathlineto{\pgfqpoint{11.167711in}{4.015974in}}%
\pgfpathlineto{\pgfqpoint{11.179868in}{3.974760in}}%
\pgfpathlineto{\pgfqpoint{11.188378in}{3.948072in}}%
\pgfpathlineto{\pgfqpoint{11.210260in}{3.885756in}}%
\pgfpathlineto{\pgfqpoint{11.217555in}{3.866015in}}%
\pgfpathlineto{\pgfqpoint{11.228496in}{3.838321in}}%
\pgfpathlineto{\pgfqpoint{11.242476in}{3.807741in}}%
\pgfpathlineto{\pgfqpoint{11.249163in}{3.792786in}}%
\pgfpathlineto{\pgfqpoint{11.272869in}{3.732538in}}%
\pgfpathlineto{\pgfqpoint{11.280163in}{3.719129in}}%
\pgfpathlineto{\pgfqpoint{11.290496in}{3.702330in}}%
\pgfpathlineto{\pgfqpoint{11.297183in}{3.692478in}}%
\pgfpathlineto{\pgfqpoint{11.311163in}{3.674189in}}%
\pgfpathlineto{\pgfqpoint{11.323928in}{3.656999in}}%
\pgfpathlineto{\pgfqpoint{11.334261in}{3.639567in}}%
\pgfpathlineto{\pgfqpoint{11.346418in}{3.623830in}}%
\pgfpathlineto{\pgfqpoint{11.360399in}{3.609248in}}%
\pgfpathlineto{\pgfqpoint{11.362222in}{3.607478in}}%
\pgfpathlineto{\pgfqpoint{11.367085in}{3.602953in}}%
\pgfpathlineto{\pgfqpoint{11.371948in}{3.599098in}}%
\pgfpathlineto{\pgfqpoint{11.376203in}{3.595112in}}%
\pgfpathlineto{\pgfqpoint{11.382889in}{3.590642in}}%
\pgfpathlineto{\pgfqpoint{11.387144in}{3.588836in}}%
\pgfpathlineto{\pgfqpoint{11.405379in}{3.583454in}}%
\pgfpathlineto{\pgfqpoint{11.408419in}{3.583603in}}%
\pgfpathlineto{\pgfqpoint{11.410850in}{3.583472in}}%
\pgfpathlineto{\pgfqpoint{11.416321in}{3.581815in}}%
\pgfpathlineto{\pgfqpoint{11.419360in}{3.581982in}}%
\pgfpathlineto{\pgfqpoint{11.436379in}{3.582225in}}%
\pgfpathlineto{\pgfqpoint{11.438203in}{3.583267in}}%
\pgfpathlineto{\pgfqpoint{11.444281in}{3.585353in}}%
\pgfpathlineto{\pgfqpoint{11.447321in}{3.585763in}}%
\pgfpathlineto{\pgfqpoint{11.450968in}{3.586825in}}%
\pgfpathlineto{\pgfqpoint{11.452184in}{3.587272in}}%
\pgfpathlineto{\pgfqpoint{11.453399in}{3.586787in}}%
\pgfpathlineto{\pgfqpoint{11.456438in}{3.588035in}}%
\pgfpathlineto{\pgfqpoint{11.465556in}{3.590605in}}%
\pgfpathlineto{\pgfqpoint{11.468595in}{3.592077in}}%
\pgfpathlineto{\pgfqpoint{11.471027in}{3.593082in}}%
\pgfpathlineto{\pgfqpoint{11.472242in}{3.593231in}}%
\pgfpathlineto{\pgfqpoint{11.474674in}{3.594554in}}%
\pgfpathlineto{\pgfqpoint{11.483184in}{3.598409in}}%
\pgfpathlineto{\pgfqpoint{11.485007in}{3.598427in}}%
\pgfpathlineto{\pgfqpoint{11.488654in}{3.600960in}}%
\pgfpathlineto{\pgfqpoint{11.489870in}{3.600941in}}%
\pgfpathlineto{\pgfqpoint{11.494733in}{3.603232in}}%
\pgfpathlineto{\pgfqpoint{11.499596in}{3.605355in}}%
\pgfpathlineto{\pgfqpoint{11.509929in}{3.610179in}}%
\pgfpathlineto{\pgfqpoint{11.514184in}{3.611930in}}%
\pgfpathlineto{\pgfqpoint{11.520262in}{3.614649in}}%
\pgfpathlineto{\pgfqpoint{11.526341in}{3.617833in}}%
\pgfpathlineto{\pgfqpoint{11.527556in}{3.618280in}}%
\pgfpathlineto{\pgfqpoint{11.529380in}{3.619305in}}%
\pgfpathlineto{\pgfqpoint{11.536066in}{3.622676in}}%
\pgfpathlineto{\pgfqpoint{11.537890in}{3.623383in}}%
\pgfpathlineto{\pgfqpoint{11.540321in}{3.624463in}}%
\pgfpathlineto{\pgfqpoint{11.548831in}{3.629026in}}%
\pgfpathlineto{\pgfqpoint{11.550047in}{3.628710in}}%
\pgfpathlineto{\pgfqpoint{11.558557in}{3.633161in}}%
\pgfpathlineto{\pgfqpoint{11.560380in}{3.633720in}}%
\pgfpathlineto{\pgfqpoint{11.569498in}{3.637221in}}%
\pgfpathlineto{\pgfqpoint{11.572537in}{3.638040in}}%
\pgfpathlineto{\pgfqpoint{11.589557in}{3.646197in}}%
\pgfpathlineto{\pgfqpoint{11.591381in}{3.646142in}}%
\pgfpathlineto{\pgfqpoint{11.605361in}{3.651058in}}%
\pgfpathlineto{\pgfqpoint{11.608400in}{3.652232in}}%
\pgfpathlineto{\pgfqpoint{11.614479in}{3.654858in}}%
\pgfpathlineto{\pgfqpoint{11.616910in}{3.654876in}}%
\pgfpathlineto{\pgfqpoint{11.621165in}{3.656180in}}%
\pgfpathlineto{\pgfqpoint{11.629067in}{3.658694in}}%
\pgfpathlineto{\pgfqpoint{11.631498in}{3.658638in}}%
\pgfpathlineto{\pgfqpoint{11.637577in}{3.659439in}}%
\pgfpathlineto{\pgfqpoint{11.639400in}{3.659495in}}%
\pgfpathlineto{\pgfqpoint{11.642440in}{3.660296in}}%
\pgfpathlineto{\pgfqpoint{11.671009in}{3.663741in}}%
\pgfpathlineto{\pgfqpoint{11.672224in}{3.663927in}}%
\pgfpathlineto{\pgfqpoint{11.674048in}{3.664039in}}%
\pgfpathlineto{\pgfqpoint{11.680126in}{3.663667in}}%
\pgfpathlineto{\pgfqpoint{11.682558in}{3.664598in}}%
\pgfpathlineto{\pgfqpoint{11.686812in}{3.664225in}}%
\pgfpathlineto{\pgfqpoint{11.688636in}{3.664468in}}%
\pgfpathlineto{\pgfqpoint{11.694107in}{3.664337in}}%
\pgfpathlineto{\pgfqpoint{11.695322in}{3.663927in}}%
\pgfpathlineto{\pgfqpoint{11.697146in}{3.663629in}}%
\pgfpathlineto{\pgfqpoint{11.698969in}{3.663723in}}%
\pgfpathlineto{\pgfqpoint{11.703832in}{3.664244in}}%
\pgfpathlineto{\pgfqpoint{11.706264in}{3.663611in}}%
\pgfpathlineto{\pgfqpoint{11.767048in}{3.656962in}}%
\pgfpathlineto{\pgfqpoint{11.768264in}{3.657167in}}%
\pgfpathlineto{\pgfqpoint{11.770088in}{3.657186in}}%
\pgfpathlineto{\pgfqpoint{11.781637in}{3.655193in}}%
\pgfpathlineto{\pgfqpoint{11.783460in}{3.655007in}}%
\pgfpathlineto{\pgfqpoint{11.791970in}{3.654038in}}%
\pgfpathlineto{\pgfqpoint{11.793794in}{3.653945in}}%
\pgfpathlineto{\pgfqpoint{11.798049in}{3.653200in}}%
\pgfpathlineto{\pgfqpoint{11.804735in}{3.652399in}}%
\pgfpathlineto{\pgfqpoint{11.813245in}{3.650704in}}%
\pgfpathlineto{\pgfqpoint{11.815676in}{3.650853in}}%
\pgfpathlineto{\pgfqpoint{11.821754in}{3.649904in}}%
\pgfpathlineto{\pgfqpoint{11.855186in}{3.645769in}}%
\pgfpathlineto{\pgfqpoint{11.860049in}{3.645266in}}%
\pgfpathlineto{\pgfqpoint{11.862480in}{3.645061in}}%
\pgfpathlineto{\pgfqpoint{11.867343in}{3.644670in}}%
\pgfpathlineto{\pgfqpoint{11.870990in}{3.644242in}}%
\pgfpathlineto{\pgfqpoint{11.874029in}{3.643963in}}%
\pgfpathlineto{\pgfqpoint{11.877069in}{3.643478in}}%
\pgfpathlineto{\pgfqpoint{11.880108in}{3.643497in}}%
\pgfpathlineto{\pgfqpoint{11.937245in}{3.642547in}}%
\pgfpathlineto{\pgfqpoint{11.939677in}{3.643013in}}%
\pgfpathlineto{\pgfqpoint{11.942108in}{3.642715in}}%
\pgfpathlineto{\pgfqpoint{11.959128in}{3.643236in}}%
\pgfpathlineto{\pgfqpoint{11.967638in}{3.643758in}}%
\pgfpathlineto{\pgfqpoint{11.970677in}{3.644372in}}%
\pgfpathlineto{\pgfqpoint{11.973716in}{3.644130in}}%
\pgfpathlineto{\pgfqpoint{11.980403in}{3.645006in}}%
\pgfpathlineto{\pgfqpoint{11.985873in}{3.645527in}}%
\pgfpathlineto{\pgfqpoint{11.988912in}{3.645378in}}%
\pgfpathlineto{\pgfqpoint{11.993167in}{3.646104in}}%
\pgfpathlineto{\pgfqpoint{11.994991in}{3.645881in}}%
\pgfpathlineto{\pgfqpoint{11.996814in}{3.646458in}}%
\pgfpathlineto{\pgfqpoint{12.007148in}{3.647296in}}%
\pgfpathlineto{\pgfqpoint{12.008971in}{3.647594in}}%
\pgfpathlineto{\pgfqpoint{12.010795in}{3.648265in}}%
\pgfpathlineto{\pgfqpoint{12.013226in}{3.648432in}}%
\pgfpathlineto{\pgfqpoint{12.018697in}{3.648619in}}%
\pgfpathlineto{\pgfqpoint{12.023560in}{3.649457in}}%
\pgfpathlineto{\pgfqpoint{12.060638in}{3.653368in}}%
\pgfpathlineto{\pgfqpoint{12.061246in}{3.652809in}}%
\pgfpathlineto{\pgfqpoint{12.064893in}{3.653591in}}%
\pgfpathlineto{\pgfqpoint{12.066717in}{3.654187in}}%
\pgfpathlineto{\pgfqpoint{12.071580in}{3.654355in}}%
\pgfpathlineto{\pgfqpoint{12.074619in}{3.654802in}}%
\pgfpathlineto{\pgfqpoint{12.148168in}{3.658210in}}%
\pgfpathlineto{\pgfqpoint{12.151816in}{3.658527in}}%
\pgfpathlineto{\pgfqpoint{12.155463in}{3.658191in}}%
\pgfpathlineto{\pgfqpoint{12.158502in}{3.658247in}}%
\pgfpathlineto{\pgfqpoint{12.161541in}{3.658806in}}%
\pgfpathlineto{\pgfqpoint{12.163365in}{3.658154in}}%
\pgfpathlineto{\pgfqpoint{12.188894in}{3.657372in}}%
\pgfpathlineto{\pgfqpoint{12.190110in}{3.657260in}}%
\pgfpathlineto{\pgfqpoint{12.203482in}{3.656571in}}%
\pgfpathlineto{\pgfqpoint{12.205914in}{3.656627in}}%
\pgfpathlineto{\pgfqpoint{12.207130in}{3.656478in}}%
\pgfpathlineto{\pgfqpoint{12.210169in}{3.655863in}}%
\pgfpathlineto{\pgfqpoint{12.212600in}{3.655807in}}%
\pgfpathlineto{\pgfqpoint{12.215032in}{3.655696in}}%
\pgfpathlineto{\pgfqpoint{12.219894in}{3.655323in}}%
\pgfpathlineto{\pgfqpoint{12.221718in}{3.654690in}}%
\pgfpathlineto{\pgfqpoint{12.222934in}{3.655249in}}%
\pgfpathlineto{\pgfqpoint{12.225973in}{3.654634in}}%
\pgfpathlineto{\pgfqpoint{12.234483in}{3.654131in}}%
\pgfpathlineto{\pgfqpoint{12.237522in}{3.653275in}}%
\pgfpathlineto{\pgfqpoint{12.239953in}{3.653330in}}%
\pgfpathlineto{\pgfqpoint{12.242385in}{3.653312in}}%
\pgfpathlineto{\pgfqpoint{12.263659in}{3.651170in}}%
\pgfpathlineto{\pgfqpoint{12.264875in}{3.650891in}}%
\pgfpathlineto{\pgfqpoint{12.266699in}{3.651431in}}%
\pgfpathlineto{\pgfqpoint{12.272777in}{3.650015in}}%
\pgfpathlineto{\pgfqpoint{12.275208in}{3.649941in}}%
\pgfpathlineto{\pgfqpoint{12.277032in}{3.649438in}}%
\pgfpathlineto{\pgfqpoint{12.278248in}{3.650332in}}%
\pgfpathlineto{\pgfqpoint{12.281287in}{3.649438in}}%
\pgfpathlineto{\pgfqpoint{12.281895in}{3.650239in}}%
\pgfpathlineto{\pgfqpoint{12.282503in}{3.649289in}}%
\pgfpathlineto{\pgfqpoint{12.285542in}{3.648954in}}%
\pgfpathlineto{\pgfqpoint{12.286758in}{3.649513in}}%
\pgfpathlineto{\pgfqpoint{12.288581in}{3.648172in}}%
\pgfpathlineto{\pgfqpoint{12.289797in}{3.649196in}}%
\pgfpathlineto{\pgfqpoint{12.290405in}{3.649568in}}%
\pgfpathlineto{\pgfqpoint{12.292836in}{3.648079in}}%
\pgfpathlineto{\pgfqpoint{12.298914in}{3.648153in}}%
\pgfpathlineto{\pgfqpoint{12.300130in}{3.647240in}}%
\pgfpathlineto{\pgfqpoint{12.302562in}{3.648004in}}%
\pgfpathlineto{\pgfqpoint{12.304385in}{3.646998in}}%
\pgfpathlineto{\pgfqpoint{12.306209in}{3.647036in}}%
\pgfpathlineto{\pgfqpoint{12.307424in}{3.646905in}}%
\pgfpathlineto{\pgfqpoint{12.309248in}{3.646384in}}%
\pgfpathlineto{\pgfqpoint{12.312287in}{3.646644in}}%
\pgfpathlineto{\pgfqpoint{12.314111in}{3.646272in}}%
\pgfpathlineto{\pgfqpoint{12.315326in}{3.645900in}}%
\pgfpathlineto{\pgfqpoint{12.345719in}{3.644074in}}%
\pgfpathlineto{\pgfqpoint{12.348150in}{3.643460in}}%
\pgfpathlineto{\pgfqpoint{12.363346in}{3.643087in}}%
\pgfpathlineto{\pgfqpoint{12.368817in}{3.642827in}}%
\pgfpathlineto{\pgfqpoint{12.370640in}{3.642566in}}%
\pgfpathlineto{\pgfqpoint{12.394347in}{3.642864in}}%
\pgfpathlineto{\pgfqpoint{12.396170in}{3.642752in}}%
\pgfpathlineto{\pgfqpoint{12.399209in}{3.642380in}}%
\pgfpathlineto{\pgfqpoint{12.401641in}{3.642827in}}%
\pgfpathlineto{\pgfqpoint{12.408327in}{3.642789in}}%
\pgfpathlineto{\pgfqpoint{12.411366in}{3.642994in}}%
\pgfpathlineto{\pgfqpoint{12.412582in}{3.642771in}}%
\pgfpathlineto{\pgfqpoint{12.415013in}{3.643292in}}%
\pgfpathlineto{\pgfqpoint{12.421700in}{3.643162in}}%
\pgfpathlineto{\pgfqpoint{12.428386in}{3.644093in}}%
\pgfpathlineto{\pgfqpoint{12.430817in}{3.643460in}}%
\pgfpathlineto{\pgfqpoint{12.432641in}{3.644261in}}%
\pgfpathlineto{\pgfqpoint{12.434464in}{3.643888in}}%
\pgfpathlineto{\pgfqpoint{12.436896in}{3.644540in}}%
\pgfpathlineto{\pgfqpoint{12.438719in}{3.644093in}}%
\pgfpathlineto{\pgfqpoint{12.442974in}{3.644186in}}%
\pgfpathlineto{\pgfqpoint{12.446013in}{3.644987in}}%
\pgfpathlineto{\pgfqpoint{12.449661in}{3.645229in}}%
\pgfpathlineto{\pgfqpoint{12.451484in}{3.645117in}}%
\pgfpathlineto{\pgfqpoint{12.456955in}{3.646067in}}%
\pgfpathlineto{\pgfqpoint{12.459386in}{3.645304in}}%
\pgfpathlineto{\pgfqpoint{12.461210in}{3.646756in}}%
\pgfpathlineto{\pgfqpoint{12.461817in}{3.646030in}}%
\pgfpathlineto{\pgfqpoint{12.466072in}{3.646514in}}%
\pgfpathlineto{\pgfqpoint{12.467288in}{3.646216in}}%
\pgfpathlineto{\pgfqpoint{12.472759in}{3.647445in}}%
\pgfpathlineto{\pgfqpoint{12.474582in}{3.646905in}}%
\pgfpathlineto{\pgfqpoint{12.476406in}{3.647278in}}%
\pgfpathlineto{\pgfqpoint{12.478229in}{3.648004in}}%
\pgfpathlineto{\pgfqpoint{12.480661in}{3.647389in}}%
\pgfpathlineto{\pgfqpoint{12.489778in}{3.648432in}}%
\pgfpathlineto{\pgfqpoint{12.492210in}{3.648432in}}%
\pgfpathlineto{\pgfqpoint{12.494641in}{3.649345in}}%
\pgfpathlineto{\pgfqpoint{12.497073in}{3.649121in}}%
\pgfpathlineto{\pgfqpoint{12.505582in}{3.650742in}}%
\pgfpathlineto{\pgfqpoint{12.506798in}{3.649941in}}%
\pgfpathlineto{\pgfqpoint{12.509837in}{3.651040in}}%
\pgfpathlineto{\pgfqpoint{12.512877in}{3.650406in}}%
\pgfpathlineto{\pgfqpoint{12.515916in}{3.650909in}}%
\pgfpathlineto{\pgfqpoint{12.517132in}{3.651263in}}%
\pgfpathlineto{\pgfqpoint{12.518955in}{3.651803in}}%
\pgfpathlineto{\pgfqpoint{12.520779in}{3.651226in}}%
\pgfpathlineto{\pgfqpoint{12.521387in}{3.651971in}}%
\pgfpathlineto{\pgfqpoint{12.521994in}{3.650798in}}%
\pgfpathlineto{\pgfqpoint{12.522602in}{3.651282in}}%
\pgfpathlineto{\pgfqpoint{12.523818in}{3.651915in}}%
\pgfpathlineto{\pgfqpoint{12.524426in}{3.652008in}}%
\pgfpathlineto{\pgfqpoint{12.525034in}{3.653684in}}%
\pgfpathlineto{\pgfqpoint{12.525641in}{3.652008in}}%
\pgfpathlineto{\pgfqpoint{12.526249in}{3.652232in}}%
\pgfpathlineto{\pgfqpoint{12.531720in}{3.652343in}}%
\pgfpathlineto{\pgfqpoint{12.535367in}{3.653479in}}%
\pgfpathlineto{\pgfqpoint{12.566367in}{3.656180in}}%
\pgfpathlineto{\pgfqpoint{12.567583in}{3.656199in}}%
\pgfpathlineto{\pgfqpoint{12.573661in}{3.656683in}}%
\pgfpathlineto{\pgfqpoint{12.576701in}{3.656720in}}%
\pgfpathlineto{\pgfqpoint{12.591897in}{3.657539in}}%
\pgfpathlineto{\pgfqpoint{12.594936in}{3.658080in}}%
\pgfpathlineto{\pgfqpoint{12.599799in}{3.657856in}}%
\pgfpathlineto{\pgfqpoint{12.605269in}{3.658098in}}%
\pgfpathlineto{\pgfqpoint{12.608917in}{3.658061in}}%
\pgfpathlineto{\pgfqpoint{12.614387in}{3.658340in}}%
\pgfpathlineto{\pgfqpoint{12.634446in}{3.658042in}}%
\pgfpathlineto{\pgfqpoint{12.638093in}{3.658061in}}%
\pgfpathlineto{\pgfqpoint{12.642348in}{3.657763in}}%
\pgfpathlineto{\pgfqpoint{12.645995in}{3.657893in}}%
\pgfpathlineto{\pgfqpoint{12.650250in}{3.657782in}}%
\pgfpathlineto{\pgfqpoint{12.656329in}{3.657614in}}%
\pgfpathlineto{\pgfqpoint{12.659976in}{3.657223in}}%
\pgfpathlineto{\pgfqpoint{12.664231in}{3.657037in}}%
\pgfpathlineto{\pgfqpoint{12.670917in}{3.656664in}}%
\pgfpathlineto{\pgfqpoint{12.677603in}{3.656310in}}%
\pgfpathlineto{\pgfqpoint{12.679427in}{3.655807in}}%
\pgfpathlineto{\pgfqpoint{12.686113in}{3.655509in}}%
\pgfpathlineto{\pgfqpoint{12.697054in}{3.654727in}}%
\pgfpathlineto{\pgfqpoint{12.700094in}{3.654224in}}%
\pgfpathlineto{\pgfqpoint{12.704956in}{3.654187in}}%
\pgfpathlineto{\pgfqpoint{12.713466in}{3.653032in}}%
\pgfpathlineto{\pgfqpoint{12.741427in}{3.650667in}}%
\pgfpathlineto{\pgfqpoint{12.743251in}{3.650053in}}%
\pgfpathlineto{\pgfqpoint{12.746290in}{3.649904in}}%
\pgfpathlineto{\pgfqpoint{12.759663in}{3.648339in}}%
\pgfpathlineto{\pgfqpoint{12.762094in}{3.647985in}}%
\pgfpathlineto{\pgfqpoint{12.806467in}{3.644931in}}%
\pgfpathlineto{\pgfqpoint{12.810722in}{3.644056in}}%
\pgfpathlineto{\pgfqpoint{12.813153in}{3.644354in}}%
\pgfpathlineto{\pgfqpoint{12.828957in}{3.643423in}}%
\pgfpathlineto{\pgfqpoint{12.830781in}{3.643646in}}%
\pgfpathlineto{\pgfqpoint{12.832604in}{3.643218in}}%
\pgfpathlineto{\pgfqpoint{12.835036in}{3.643236in}}%
\pgfpathlineto{\pgfqpoint{12.839290in}{3.642957in}}%
\pgfpathlineto{\pgfqpoint{12.845977in}{3.643087in}}%
\pgfpathlineto{\pgfqpoint{12.849624in}{3.642622in}}%
\pgfpathlineto{\pgfqpoint{12.853271in}{3.642603in}}%
\pgfpathlineto{\pgfqpoint{12.855702in}{3.642715in}}%
\pgfpathlineto{\pgfqpoint{12.859350in}{3.642901in}}%
\pgfpathlineto{\pgfqpoint{12.873938in}{3.642398in}}%
\pgfpathlineto{\pgfqpoint{12.876369in}{3.642398in}}%
\pgfpathlineto{\pgfqpoint{12.884879in}{3.642380in}}%
\pgfpathlineto{\pgfqpoint{12.890350in}{3.642473in}}%
\pgfpathlineto{\pgfqpoint{12.893997in}{3.642454in}}%
\pgfpathlineto{\pgfqpoint{12.902507in}{3.642249in}}%
\pgfpathlineto{\pgfqpoint{12.904938in}{3.642454in}}%
\pgfpathlineto{\pgfqpoint{12.914056in}{3.642137in}}%
\pgfpathlineto{\pgfqpoint{12.920134in}{3.642380in}}%
\pgfpathlineto{\pgfqpoint{12.923781in}{3.642286in}}%
\pgfpathlineto{\pgfqpoint{12.927428in}{3.642305in}}%
\pgfpathlineto{\pgfqpoint{12.929252in}{3.642212in}}%
\pgfpathlineto{\pgfqpoint{12.937762in}{3.642584in}}%
\pgfpathlineto{\pgfqpoint{12.941409in}{3.642678in}}%
\pgfpathlineto{\pgfqpoint{12.948703in}{3.642249in}}%
\pgfpathlineto{\pgfqpoint{12.968154in}{3.642510in}}%
\pgfpathlineto{\pgfqpoint{12.970585in}{3.642361in}}%
\pgfpathlineto{\pgfqpoint{13.011311in}{3.642212in}}%
\pgfpathlineto{\pgfqpoint{13.024076in}{3.642417in}}%
\pgfpathlineto{\pgfqpoint{13.030155in}{3.642175in}}%
\pgfpathlineto{\pgfqpoint{13.037449in}{3.642398in}}%
\pgfpathlineto{\pgfqpoint{13.048390in}{3.641672in}}%
\pgfpathlineto{\pgfqpoint{13.052037in}{3.641802in}}%
\pgfpathlineto{\pgfqpoint{13.057508in}{3.641542in}}%
\pgfpathlineto{\pgfqpoint{13.063586in}{3.641933in}}%
\pgfpathlineto{\pgfqpoint{13.095802in}{3.641840in}}%
\pgfpathlineto{\pgfqpoint{13.104312in}{3.641337in}}%
\pgfpathlineto{\pgfqpoint{13.118292in}{3.641542in}}%
\pgfpathlineto{\pgfqpoint{13.132881in}{3.641728in}}%
\pgfpathlineto{\pgfqpoint{13.140783in}{3.641728in}}%
\pgfpathlineto{\pgfqpoint{13.147469in}{3.641560in}}%
\pgfpathlineto{\pgfqpoint{13.163273in}{3.641616in}}%
\pgfpathlineto{\pgfqpoint{13.166312in}{3.641616in}}%
\pgfpathlineto{\pgfqpoint{13.180293in}{3.641393in}}%
\pgfpathlineto{\pgfqpoint{13.272078in}{3.642361in}}%
\pgfpathlineto{\pgfqpoint{13.273901in}{3.642435in}}%
\pgfpathlineto{\pgfqpoint{13.312196in}{3.642584in}}%
\pgfpathlineto{\pgfqpoint{13.319490in}{3.642584in}}%
\pgfpathlineto{\pgfqpoint{13.321921in}{3.642994in}}%
\pgfpathlineto{\pgfqpoint{13.324960in}{3.642808in}}%
\pgfpathlineto{\pgfqpoint{13.328000in}{3.643162in}}%
\pgfpathlineto{\pgfqpoint{13.330431in}{3.642920in}}%
\pgfpathlineto{\pgfqpoint{13.332862in}{3.643199in}}%
\pgfpathlineto{\pgfqpoint{13.335902in}{3.643125in}}%
\pgfpathlineto{\pgfqpoint{13.339549in}{3.643162in}}%
\pgfpathlineto{\pgfqpoint{13.341980in}{3.642938in}}%
\pgfpathlineto{\pgfqpoint{13.368118in}{3.643050in}}%
\pgfpathlineto{\pgfqpoint{13.370549in}{3.642976in}}%
\pgfpathlineto{\pgfqpoint{13.390000in}{3.642640in}}%
\pgfpathlineto{\pgfqpoint{13.390000in}{3.642640in}}%
\pgfusepath{stroke}%
\end{pgfscope}%
\begin{pgfscope}%
\pgfpathrectangle{\pgfqpoint{1.021528in}{0.692778in}}{\pgfqpoint{12.368472in}{3.868889in}}%
\pgfusepath{clip}%
\pgfsetrectcap%
\pgfsetroundjoin%
\pgfsetlinewidth{1.505625pt}%
\definecolor{currentstroke}{rgb}{1.000000,0.498039,0.054902}%
\pgfsetstrokecolor{currentstroke}%
\pgfsetdash{}{0pt}%
\pgfpathmoveto{\pgfqpoint{1.011528in}{0.886011in}}%
\pgfpathlineto{\pgfqpoint{1.353845in}{0.886568in}}%
\pgfpathlineto{\pgfqpoint{1.365936in}{0.909656in}}%
\pgfpathlineto{\pgfqpoint{1.378348in}{0.944461in}}%
\pgfpathlineto{\pgfqpoint{1.390595in}{0.950838in}}%
\pgfpathlineto{\pgfqpoint{1.402879in}{0.966205in}}%
\pgfpathlineto{\pgfqpoint{1.415172in}{0.937765in}}%
\pgfpathlineto{\pgfqpoint{1.427337in}{0.927334in}}%
\pgfpathlineto{\pgfqpoint{1.439751in}{0.930864in}}%
\pgfpathlineto{\pgfqpoint{1.452244in}{0.981384in}}%
\pgfpathlineto{\pgfqpoint{1.464581in}{1.061585in}}%
\pgfpathlineto{\pgfqpoint{1.477002in}{1.151813in}}%
\pgfpathlineto{\pgfqpoint{1.513989in}{1.493459in}}%
\pgfpathlineto{\pgfqpoint{1.526321in}{1.636617in}}%
\pgfpathlineto{\pgfqpoint{1.587846in}{2.431513in}}%
\pgfpathlineto{\pgfqpoint{1.600133in}{2.612775in}}%
\pgfpathlineto{\pgfqpoint{1.612509in}{2.750134in}}%
\pgfpathlineto{\pgfqpoint{1.624855in}{2.901760in}}%
\pgfpathlineto{\pgfqpoint{1.637148in}{3.031018in}}%
\pgfpathlineto{\pgfqpoint{1.649413in}{3.145121in}}%
\pgfpathlineto{\pgfqpoint{1.661750in}{3.279092in}}%
\pgfpathlineto{\pgfqpoint{1.686043in}{3.488824in}}%
\pgfpathlineto{\pgfqpoint{1.710971in}{3.643614in}}%
\pgfpathlineto{\pgfqpoint{1.723312in}{3.721207in}}%
\pgfpathlineto{\pgfqpoint{1.735567in}{3.773495in}}%
\pgfpathlineto{\pgfqpoint{1.772601in}{3.870241in}}%
\pgfpathlineto{\pgfqpoint{1.784986in}{3.880674in}}%
\pgfpathlineto{\pgfqpoint{1.797303in}{3.905533in}}%
\pgfpathlineto{\pgfqpoint{1.809616in}{3.927270in}}%
\pgfpathlineto{\pgfqpoint{1.821919in}{3.946684in}}%
\pgfpathlineto{\pgfqpoint{1.834270in}{3.958453in}}%
\pgfpathlineto{\pgfqpoint{1.858871in}{3.989913in}}%
\pgfpathlineto{\pgfqpoint{1.871217in}{4.001053in}}%
\pgfpathlineto{\pgfqpoint{1.883501in}{4.014017in}}%
\pgfpathlineto{\pgfqpoint{1.895893in}{4.014295in}}%
\pgfpathlineto{\pgfqpoint{1.908056in}{4.028495in}}%
\pgfpathlineto{\pgfqpoint{1.920339in}{4.031336in}}%
\pgfpathlineto{\pgfqpoint{1.932629in}{4.047808in}}%
\pgfpathlineto{\pgfqpoint{1.944939in}{4.046873in}}%
\pgfpathlineto{\pgfqpoint{1.957235in}{4.047917in}}%
\pgfpathlineto{\pgfqpoint{1.969557in}{4.062256in}}%
\pgfpathlineto{\pgfqpoint{1.981864in}{4.069566in}}%
\pgfpathlineto{\pgfqpoint{1.994168in}{4.056334in}}%
\pgfpathlineto{\pgfqpoint{2.006472in}{4.056893in}}%
\pgfpathlineto{\pgfqpoint{2.018785in}{4.061233in}}%
\pgfpathlineto{\pgfqpoint{2.031105in}{4.055571in}}%
\pgfpathlineto{\pgfqpoint{2.043393in}{4.051741in}}%
\pgfpathlineto{\pgfqpoint{2.055733in}{4.052148in}}%
\pgfpathlineto{\pgfqpoint{2.068023in}{4.045337in}}%
\pgfpathlineto{\pgfqpoint{2.080340in}{4.052004in}}%
\pgfpathlineto{\pgfqpoint{2.104937in}{4.048772in}}%
\pgfpathlineto{\pgfqpoint{2.117186in}{4.056089in}}%
\pgfpathlineto{\pgfqpoint{2.129480in}{4.052195in}}%
\pgfpathlineto{\pgfqpoint{2.141719in}{4.053669in}}%
\pgfpathlineto{\pgfqpoint{2.153988in}{4.057380in}}%
\pgfpathlineto{\pgfqpoint{2.166241in}{4.046759in}}%
\pgfpathlineto{\pgfqpoint{2.178459in}{4.056912in}}%
\pgfpathlineto{\pgfqpoint{2.190740in}{4.057738in}}%
\pgfpathlineto{\pgfqpoint{2.202950in}{4.063235in}}%
\pgfpathlineto{\pgfqpoint{2.215204in}{4.052890in}}%
\pgfpathlineto{\pgfqpoint{2.227432in}{4.053884in}}%
\pgfpathlineto{\pgfqpoint{2.239734in}{4.062170in}}%
\pgfpathlineto{\pgfqpoint{2.251989in}{4.058937in}}%
\pgfpathlineto{\pgfqpoint{2.264290in}{4.059451in}}%
\pgfpathlineto{\pgfqpoint{2.276509in}{4.055719in}}%
\pgfpathlineto{\pgfqpoint{2.288776in}{4.049215in}}%
\pgfpathlineto{\pgfqpoint{2.301066in}{4.054028in}}%
\pgfpathlineto{\pgfqpoint{2.313284in}{4.054298in}}%
\pgfpathlineto{\pgfqpoint{2.325582in}{4.059944in}}%
\pgfpathlineto{\pgfqpoint{2.337870in}{4.047260in}}%
\pgfpathlineto{\pgfqpoint{2.350124in}{4.053961in}}%
\pgfpathlineto{\pgfqpoint{2.362419in}{4.069368in}}%
\pgfpathlineto{\pgfqpoint{2.374687in}{4.059136in}}%
\pgfpathlineto{\pgfqpoint{2.386956in}{4.066226in}}%
\pgfpathlineto{\pgfqpoint{2.399176in}{4.057470in}}%
\pgfpathlineto{\pgfqpoint{2.411444in}{4.066604in}}%
\pgfpathlineto{\pgfqpoint{2.423753in}{4.050604in}}%
\pgfpathlineto{\pgfqpoint{2.435981in}{4.073767in}}%
\pgfpathlineto{\pgfqpoint{2.448266in}{4.062344in}}%
\pgfpathlineto{\pgfqpoint{2.460518in}{4.070801in}}%
\pgfpathlineto{\pgfqpoint{2.472743in}{4.066410in}}%
\pgfpathlineto{\pgfqpoint{2.485015in}{4.055606in}}%
\pgfpathlineto{\pgfqpoint{2.497307in}{4.070439in}}%
\pgfpathlineto{\pgfqpoint{2.509536in}{4.070925in}}%
\pgfpathlineto{\pgfqpoint{2.521736in}{4.072971in}}%
\pgfpathlineto{\pgfqpoint{2.533989in}{4.060605in}}%
\pgfpathlineto{\pgfqpoint{2.546277in}{4.067380in}}%
\pgfpathlineto{\pgfqpoint{2.558531in}{4.066099in}}%
\pgfpathlineto{\pgfqpoint{2.570780in}{4.079155in}}%
\pgfpathlineto{\pgfqpoint{2.583026in}{4.078377in}}%
\pgfpathlineto{\pgfqpoint{2.595251in}{4.079517in}}%
\pgfpathlineto{\pgfqpoint{2.607512in}{4.077059in}}%
\pgfpathlineto{\pgfqpoint{2.619805in}{4.087718in}}%
\pgfpathlineto{\pgfqpoint{2.644272in}{4.078673in}}%
\pgfpathlineto{\pgfqpoint{2.656563in}{4.085461in}}%
\pgfpathlineto{\pgfqpoint{2.668884in}{4.087684in}}%
\pgfpathlineto{\pgfqpoint{2.681175in}{4.098222in}}%
\pgfpathlineto{\pgfqpoint{2.705771in}{4.097820in}}%
\pgfpathlineto{\pgfqpoint{2.718088in}{4.095525in}}%
\pgfpathlineto{\pgfqpoint{2.730415in}{4.098834in}}%
\pgfpathlineto{\pgfqpoint{2.742693in}{4.105991in}}%
\pgfpathlineto{\pgfqpoint{2.754986in}{4.101436in}}%
\pgfpathlineto{\pgfqpoint{2.767312in}{4.122233in}}%
\pgfpathlineto{\pgfqpoint{2.779604in}{4.125142in}}%
\pgfpathlineto{\pgfqpoint{2.791893in}{4.104551in}}%
\pgfpathlineto{\pgfqpoint{2.804199in}{4.118949in}}%
\pgfpathlineto{\pgfqpoint{2.816526in}{4.097899in}}%
\pgfpathlineto{\pgfqpoint{2.828806in}{4.111639in}}%
\pgfpathlineto{\pgfqpoint{2.841092in}{4.115007in}}%
\pgfpathlineto{\pgfqpoint{2.853341in}{4.113229in}}%
\pgfpathlineto{\pgfqpoint{2.877906in}{4.118306in}}%
\pgfpathlineto{\pgfqpoint{2.890194in}{4.099452in}}%
\pgfpathlineto{\pgfqpoint{2.902484in}{4.112599in}}%
\pgfpathlineto{\pgfqpoint{2.914774in}{4.110078in}}%
\pgfpathlineto{\pgfqpoint{2.927039in}{4.109904in}}%
\pgfpathlineto{\pgfqpoint{2.951591in}{4.102377in}}%
\pgfpathlineto{\pgfqpoint{2.963822in}{4.105565in}}%
\pgfpathlineto{\pgfqpoint{2.976166in}{4.116199in}}%
\pgfpathlineto{\pgfqpoint{2.988435in}{4.082215in}}%
\pgfpathlineto{\pgfqpoint{3.000715in}{4.014879in}}%
\pgfpathlineto{\pgfqpoint{3.013036in}{3.936126in}}%
\pgfpathlineto{\pgfqpoint{3.025315in}{3.878450in}}%
\pgfpathlineto{\pgfqpoint{3.037564in}{3.813504in}}%
\pgfpathlineto{\pgfqpoint{3.049841in}{3.765782in}}%
\pgfpathlineto{\pgfqpoint{3.111013in}{3.437995in}}%
\pgfpathlineto{\pgfqpoint{3.123277in}{3.347658in}}%
\pgfpathlineto{\pgfqpoint{3.147843in}{3.205634in}}%
\pgfpathlineto{\pgfqpoint{3.160058in}{3.129978in}}%
\pgfpathlineto{\pgfqpoint{3.172319in}{3.084733in}}%
\pgfpathlineto{\pgfqpoint{3.184558in}{3.023966in}}%
\pgfpathlineto{\pgfqpoint{3.196881in}{2.984222in}}%
\pgfpathlineto{\pgfqpoint{3.209177in}{2.961859in}}%
\pgfpathlineto{\pgfqpoint{3.221478in}{2.944208in}}%
\pgfpathlineto{\pgfqpoint{3.233733in}{2.943091in}}%
\pgfpathlineto{\pgfqpoint{3.246085in}{2.930445in}}%
\pgfpathlineto{\pgfqpoint{3.258465in}{2.944103in}}%
\pgfpathlineto{\pgfqpoint{3.270813in}{2.950280in}}%
\pgfpathlineto{\pgfqpoint{3.283141in}{2.969597in}}%
\pgfpathlineto{\pgfqpoint{3.295503in}{2.977516in}}%
\pgfpathlineto{\pgfqpoint{3.320054in}{3.016840in}}%
\pgfpathlineto{\pgfqpoint{3.344664in}{3.039379in}}%
\pgfpathlineto{\pgfqpoint{3.356955in}{3.042232in}}%
\pgfpathlineto{\pgfqpoint{3.369219in}{3.049318in}}%
\pgfpathlineto{\pgfqpoint{3.381521in}{3.061501in}}%
\pgfpathlineto{\pgfqpoint{3.393795in}{3.077908in}}%
\pgfpathlineto{\pgfqpoint{3.406081in}{3.079728in}}%
\pgfpathlineto{\pgfqpoint{3.418339in}{3.097489in}}%
\pgfpathlineto{\pgfqpoint{3.442887in}{3.104221in}}%
\pgfpathlineto{\pgfqpoint{3.455170in}{3.103867in}}%
\pgfpathlineto{\pgfqpoint{3.467484in}{3.120201in}}%
\pgfpathlineto{\pgfqpoint{3.479971in}{3.126601in}}%
\pgfpathlineto{\pgfqpoint{3.492066in}{3.128142in}}%
\pgfpathlineto{\pgfqpoint{3.504341in}{3.127291in}}%
\pgfpathlineto{\pgfqpoint{3.516670in}{3.130483in}}%
\pgfpathlineto{\pgfqpoint{3.528976in}{3.122087in}}%
\pgfpathlineto{\pgfqpoint{3.541255in}{3.122182in}}%
\pgfpathlineto{\pgfqpoint{3.553586in}{3.128843in}}%
\pgfpathlineto{\pgfqpoint{3.565903in}{3.131525in}}%
\pgfpathlineto{\pgfqpoint{3.590571in}{3.130136in}}%
\pgfpathlineto{\pgfqpoint{3.602910in}{3.133825in}}%
\pgfpathlineto{\pgfqpoint{3.615301in}{3.123081in}}%
\pgfpathlineto{\pgfqpoint{3.627589in}{3.122526in}}%
\pgfpathlineto{\pgfqpoint{3.639871in}{3.138727in}}%
\pgfpathlineto{\pgfqpoint{3.652129in}{3.133659in}}%
\pgfpathlineto{\pgfqpoint{3.664435in}{3.137089in}}%
\pgfpathlineto{\pgfqpoint{3.676728in}{3.129741in}}%
\pgfpathlineto{\pgfqpoint{3.701287in}{3.135415in}}%
\pgfpathlineto{\pgfqpoint{3.713587in}{3.143010in}}%
\pgfpathlineto{\pgfqpoint{3.725921in}{3.134881in}}%
\pgfpathlineto{\pgfqpoint{3.738192in}{3.147353in}}%
\pgfpathlineto{\pgfqpoint{3.750507in}{3.155232in}}%
\pgfpathlineto{\pgfqpoint{3.762765in}{3.146805in}}%
\pgfpathlineto{\pgfqpoint{3.775096in}{3.146334in}}%
\pgfpathlineto{\pgfqpoint{3.799673in}{3.138350in}}%
\pgfpathlineto{\pgfqpoint{3.811965in}{3.147811in}}%
\pgfpathlineto{\pgfqpoint{3.824248in}{3.134395in}}%
\pgfpathlineto{\pgfqpoint{3.836508in}{3.138795in}}%
\pgfpathlineto{\pgfqpoint{3.848773in}{3.158024in}}%
\pgfpathlineto{\pgfqpoint{3.861088in}{3.142679in}}%
\pgfpathlineto{\pgfqpoint{3.873321in}{3.154515in}}%
\pgfpathlineto{\pgfqpoint{3.885579in}{3.147375in}}%
\pgfpathlineto{\pgfqpoint{3.910129in}{3.166009in}}%
\pgfpathlineto{\pgfqpoint{3.922412in}{3.149366in}}%
\pgfpathlineto{\pgfqpoint{3.934718in}{3.151832in}}%
\pgfpathlineto{\pgfqpoint{3.946872in}{3.157458in}}%
\pgfpathlineto{\pgfqpoint{3.959148in}{3.160467in}}%
\pgfpathlineto{\pgfqpoint{3.971376in}{3.155129in}}%
\pgfpathlineto{\pgfqpoint{3.983659in}{3.158842in}}%
\pgfpathlineto{\pgfqpoint{4.008111in}{3.157467in}}%
\pgfpathlineto{\pgfqpoint{4.020388in}{3.158877in}}%
\pgfpathlineto{\pgfqpoint{4.032734in}{3.156233in}}%
\pgfpathlineto{\pgfqpoint{4.044913in}{3.157534in}}%
\pgfpathlineto{\pgfqpoint{4.057154in}{3.167246in}}%
\pgfpathlineto{\pgfqpoint{4.069435in}{3.154088in}}%
\pgfpathlineto{\pgfqpoint{4.081765in}{3.159709in}}%
\pgfpathlineto{\pgfqpoint{4.094084in}{3.147037in}}%
\pgfpathlineto{\pgfqpoint{4.106308in}{3.151784in}}%
\pgfpathlineto{\pgfqpoint{4.130925in}{3.155510in}}%
\pgfpathlineto{\pgfqpoint{4.143204in}{3.162388in}}%
\pgfpathlineto{\pgfqpoint{4.155562in}{3.163358in}}%
\pgfpathlineto{\pgfqpoint{4.167884in}{3.154178in}}%
\pgfpathlineto{\pgfqpoint{4.180204in}{3.153159in}}%
\pgfpathlineto{\pgfqpoint{4.192515in}{3.159108in}}%
\pgfpathlineto{\pgfqpoint{4.204864in}{3.162076in}}%
\pgfpathlineto{\pgfqpoint{4.217226in}{3.160968in}}%
\pgfpathlineto{\pgfqpoint{4.229676in}{3.158599in}}%
\pgfpathlineto{\pgfqpoint{4.242008in}{3.159277in}}%
\pgfpathlineto{\pgfqpoint{4.254248in}{3.155404in}}%
\pgfpathlineto{\pgfqpoint{4.266551in}{3.165097in}}%
\pgfpathlineto{\pgfqpoint{4.278887in}{3.163893in}}%
\pgfpathlineto{\pgfqpoint{4.291223in}{3.160814in}}%
\pgfpathlineto{\pgfqpoint{4.303586in}{3.151470in}}%
\pgfpathlineto{\pgfqpoint{4.315894in}{3.155212in}}%
\pgfpathlineto{\pgfqpoint{4.328217in}{3.164935in}}%
\pgfpathlineto{\pgfqpoint{4.340589in}{3.156408in}}%
\pgfpathlineto{\pgfqpoint{4.352838in}{3.162774in}}%
\pgfpathlineto{\pgfqpoint{4.365132in}{3.162261in}}%
\pgfpathlineto{\pgfqpoint{4.377409in}{3.170743in}}%
\pgfpathlineto{\pgfqpoint{4.402034in}{3.174137in}}%
\pgfpathlineto{\pgfqpoint{4.414266in}{3.167695in}}%
\pgfpathlineto{\pgfqpoint{4.426548in}{3.168390in}}%
\pgfpathlineto{\pgfqpoint{4.438788in}{3.170287in}}%
\pgfpathlineto{\pgfqpoint{4.463341in}{3.172365in}}%
\pgfpathlineto{\pgfqpoint{4.475634in}{3.160662in}}%
\pgfpathlineto{\pgfqpoint{4.487876in}{3.177900in}}%
\pgfpathlineto{\pgfqpoint{4.500111in}{3.170334in}}%
\pgfpathlineto{\pgfqpoint{4.524605in}{3.168700in}}%
\pgfpathlineto{\pgfqpoint{4.536850in}{3.179655in}}%
\pgfpathlineto{\pgfqpoint{4.549155in}{3.177306in}}%
\pgfpathlineto{\pgfqpoint{4.573692in}{3.178896in}}%
\pgfpathlineto{\pgfqpoint{4.586037in}{3.185906in}}%
\pgfpathlineto{\pgfqpoint{4.598301in}{3.172392in}}%
\pgfpathlineto{\pgfqpoint{4.610585in}{3.176840in}}%
\pgfpathlineto{\pgfqpoint{4.622902in}{3.194092in}}%
\pgfpathlineto{\pgfqpoint{4.659823in}{3.390556in}}%
\pgfpathlineto{\pgfqpoint{4.672159in}{3.448486in}}%
\pgfpathlineto{\pgfqpoint{4.770726in}{3.969406in}}%
\pgfpathlineto{\pgfqpoint{4.782993in}{4.041130in}}%
\pgfpathlineto{\pgfqpoint{4.795316in}{4.106284in}}%
\pgfpathlineto{\pgfqpoint{4.807600in}{4.156166in}}%
\pgfpathlineto{\pgfqpoint{4.832212in}{4.284546in}}%
\pgfpathlineto{\pgfqpoint{4.844475in}{4.301901in}}%
\pgfpathlineto{\pgfqpoint{4.856752in}{4.335556in}}%
\pgfpathlineto{\pgfqpoint{4.869045in}{4.365830in}}%
\pgfpathlineto{\pgfqpoint{4.881308in}{4.367689in}}%
\pgfpathlineto{\pgfqpoint{4.893594in}{4.373956in}}%
\pgfpathlineto{\pgfqpoint{4.905830in}{4.385808in}}%
\pgfpathlineto{\pgfqpoint{4.918129in}{4.371666in}}%
\pgfpathlineto{\pgfqpoint{4.930437in}{4.355251in}}%
\pgfpathlineto{\pgfqpoint{4.942663in}{4.330748in}}%
\pgfpathlineto{\pgfqpoint{4.955050in}{4.317595in}}%
\pgfpathlineto{\pgfqpoint{4.967280in}{4.315796in}}%
\pgfpathlineto{\pgfqpoint{4.979580in}{4.284295in}}%
\pgfpathlineto{\pgfqpoint{5.016463in}{4.258474in}}%
\pgfpathlineto{\pgfqpoint{5.028632in}{4.257245in}}%
\pgfpathlineto{\pgfqpoint{5.040919in}{4.249606in}}%
\pgfpathlineto{\pgfqpoint{5.053133in}{4.240070in}}%
\pgfpathlineto{\pgfqpoint{5.065359in}{4.236614in}}%
\pgfpathlineto{\pgfqpoint{5.077556in}{4.228002in}}%
\pgfpathlineto{\pgfqpoint{5.089803in}{4.234830in}}%
\pgfpathlineto{\pgfqpoint{5.103142in}{4.225737in}}%
\pgfpathlineto{\pgfqpoint{5.114445in}{4.223822in}}%
\pgfpathlineto{\pgfqpoint{5.126788in}{4.232395in}}%
\pgfpathlineto{\pgfqpoint{5.138942in}{4.223094in}}%
\pgfpathlineto{\pgfqpoint{5.151264in}{4.234347in}}%
\pgfpathlineto{\pgfqpoint{5.163511in}{4.233732in}}%
\pgfpathlineto{\pgfqpoint{5.175781in}{4.225442in}}%
\pgfpathlineto{\pgfqpoint{5.188068in}{4.224234in}}%
\pgfpathlineto{\pgfqpoint{5.200309in}{4.228695in}}%
\pgfpathlineto{\pgfqpoint{5.212567in}{4.218926in}}%
\pgfpathlineto{\pgfqpoint{5.224822in}{4.225946in}}%
\pgfpathlineto{\pgfqpoint{5.237095in}{4.218915in}}%
\pgfpathlineto{\pgfqpoint{5.249306in}{4.226099in}}%
\pgfpathlineto{\pgfqpoint{5.261596in}{4.244094in}}%
\pgfpathlineto{\pgfqpoint{5.273854in}{4.232337in}}%
\pgfpathlineto{\pgfqpoint{5.298415in}{4.231178in}}%
\pgfpathlineto{\pgfqpoint{5.310645in}{4.235955in}}%
\pgfpathlineto{\pgfqpoint{5.322909in}{4.235858in}}%
\pgfpathlineto{\pgfqpoint{5.335205in}{4.242432in}}%
\pgfpathlineto{\pgfqpoint{5.347494in}{4.245523in}}%
\pgfpathlineto{\pgfqpoint{5.359737in}{4.251663in}}%
\pgfpathlineto{\pgfqpoint{5.372046in}{4.262927in}}%
\pgfpathlineto{\pgfqpoint{5.384315in}{4.243889in}}%
\pgfpathlineto{\pgfqpoint{5.396624in}{4.241031in}}%
\pgfpathlineto{\pgfqpoint{5.408968in}{4.252662in}}%
\pgfpathlineto{\pgfqpoint{5.421315in}{4.241432in}}%
\pgfpathlineto{\pgfqpoint{5.433613in}{4.246136in}}%
\pgfpathlineto{\pgfqpoint{5.445933in}{4.238639in}}%
\pgfpathlineto{\pgfqpoint{5.458209in}{4.243627in}}%
\pgfpathlineto{\pgfqpoint{5.470548in}{4.245696in}}%
\pgfpathlineto{\pgfqpoint{5.482851in}{4.246271in}}%
\pgfpathlineto{\pgfqpoint{5.495157in}{4.232047in}}%
\pgfpathlineto{\pgfqpoint{5.507425in}{4.248088in}}%
\pgfpathlineto{\pgfqpoint{5.519727in}{4.238986in}}%
\pgfpathlineto{\pgfqpoint{5.531987in}{4.249395in}}%
\pgfpathlineto{\pgfqpoint{5.544274in}{4.253701in}}%
\pgfpathlineto{\pgfqpoint{5.556515in}{4.241928in}}%
\pgfpathlineto{\pgfqpoint{5.568812in}{4.241442in}}%
\pgfpathlineto{\pgfqpoint{5.581095in}{4.252761in}}%
\pgfpathlineto{\pgfqpoint{5.593320in}{4.258492in}}%
\pgfpathlineto{\pgfqpoint{5.605559in}{4.256378in}}%
\pgfpathlineto{\pgfqpoint{5.617774in}{4.255649in}}%
\pgfpathlineto{\pgfqpoint{5.629985in}{4.248960in}}%
\pgfpathlineto{\pgfqpoint{5.642264in}{4.254590in}}%
\pgfpathlineto{\pgfqpoint{5.654504in}{4.255276in}}%
\pgfpathlineto{\pgfqpoint{5.666912in}{4.253046in}}%
\pgfpathlineto{\pgfqpoint{5.679096in}{4.248522in}}%
\pgfpathlineto{\pgfqpoint{5.703589in}{4.257650in}}%
\pgfpathlineto{\pgfqpoint{5.715924in}{4.245607in}}%
\pgfpathlineto{\pgfqpoint{5.728136in}{4.253690in}}%
\pgfpathlineto{\pgfqpoint{5.740402in}{4.266307in}}%
\pgfpathlineto{\pgfqpoint{5.752667in}{4.257130in}}%
\pgfpathlineto{\pgfqpoint{5.764890in}{4.273160in}}%
\pgfpathlineto{\pgfqpoint{5.777189in}{4.268549in}}%
\pgfpathlineto{\pgfqpoint{5.789417in}{4.259790in}}%
\pgfpathlineto{\pgfqpoint{5.801686in}{4.260729in}}%
\pgfpathlineto{\pgfqpoint{5.813972in}{4.254951in}}%
\pgfpathlineto{\pgfqpoint{5.826230in}{4.264159in}}%
\pgfpathlineto{\pgfqpoint{5.838481in}{4.276203in}}%
\pgfpathlineto{\pgfqpoint{5.850783in}{4.268575in}}%
\pgfpathlineto{\pgfqpoint{5.863053in}{4.283097in}}%
\pgfpathlineto{\pgfqpoint{5.875314in}{4.272050in}}%
\pgfpathlineto{\pgfqpoint{5.900024in}{4.277775in}}%
\pgfpathlineto{\pgfqpoint{5.912179in}{4.295178in}}%
\pgfpathlineto{\pgfqpoint{5.924523in}{4.283792in}}%
\pgfpathlineto{\pgfqpoint{5.936793in}{4.290238in}}%
\pgfpathlineto{\pgfqpoint{5.949122in}{4.302182in}}%
\pgfpathlineto{\pgfqpoint{5.961418in}{4.289959in}}%
\pgfpathlineto{\pgfqpoint{5.973715in}{4.299976in}}%
\pgfpathlineto{\pgfqpoint{5.985995in}{4.285918in}}%
\pgfpathlineto{\pgfqpoint{6.010650in}{4.284588in}}%
\pgfpathlineto{\pgfqpoint{6.022974in}{4.301397in}}%
\pgfpathlineto{\pgfqpoint{6.035253in}{4.280532in}}%
\pgfpathlineto{\pgfqpoint{6.047510in}{4.290012in}}%
\pgfpathlineto{\pgfqpoint{6.059849in}{4.303430in}}%
\pgfpathlineto{\pgfqpoint{6.072112in}{4.295464in}}%
\pgfpathlineto{\pgfqpoint{6.084396in}{4.290821in}}%
\pgfpathlineto{\pgfqpoint{6.096688in}{4.294815in}}%
\pgfpathlineto{\pgfqpoint{6.109005in}{4.301407in}}%
\pgfpathlineto{\pgfqpoint{6.121323in}{4.289308in}}%
\pgfpathlineto{\pgfqpoint{6.145866in}{4.299027in}}%
\pgfpathlineto{\pgfqpoint{6.158122in}{4.296839in}}%
\pgfpathlineto{\pgfqpoint{6.170412in}{4.299731in}}%
\pgfpathlineto{\pgfqpoint{6.182665in}{4.312558in}}%
\pgfpathlineto{\pgfqpoint{6.194948in}{4.294091in}}%
\pgfpathlineto{\pgfqpoint{6.207218in}{4.303238in}}%
\pgfpathlineto{\pgfqpoint{6.219559in}{4.300768in}}%
\pgfpathlineto{\pgfqpoint{6.231859in}{4.307570in}}%
\pgfpathlineto{\pgfqpoint{6.244102in}{4.303200in}}%
\pgfpathlineto{\pgfqpoint{6.256411in}{4.317258in}}%
\pgfpathlineto{\pgfqpoint{6.280964in}{4.195435in}}%
\pgfpathlineto{\pgfqpoint{6.317936in}{4.058916in}}%
\pgfpathlineto{\pgfqpoint{6.367330in}{3.837103in}}%
\pgfpathlineto{\pgfqpoint{6.392017in}{3.692084in}}%
\pgfpathlineto{\pgfqpoint{6.416676in}{3.518925in}}%
\pgfpathlineto{\pgfqpoint{6.429121in}{3.430278in}}%
\pgfpathlineto{\pgfqpoint{6.441387in}{3.364070in}}%
\pgfpathlineto{\pgfqpoint{6.453653in}{3.310207in}}%
\pgfpathlineto{\pgfqpoint{6.466018in}{3.244648in}}%
\pgfpathlineto{\pgfqpoint{6.490661in}{3.150785in}}%
\pgfpathlineto{\pgfqpoint{6.502991in}{3.117329in}}%
\pgfpathlineto{\pgfqpoint{6.515276in}{3.091410in}}%
\pgfpathlineto{\pgfqpoint{6.527575in}{3.095910in}}%
\pgfpathlineto{\pgfqpoint{6.539888in}{3.092934in}}%
\pgfpathlineto{\pgfqpoint{6.552131in}{3.087588in}}%
\pgfpathlineto{\pgfqpoint{6.564369in}{3.108658in}}%
\pgfpathlineto{\pgfqpoint{6.576664in}{3.133824in}}%
\pgfpathlineto{\pgfqpoint{6.588969in}{3.152329in}}%
\pgfpathlineto{\pgfqpoint{6.601282in}{3.159419in}}%
\pgfpathlineto{\pgfqpoint{6.613453in}{3.169997in}}%
\pgfpathlineto{\pgfqpoint{6.625711in}{3.178460in}}%
\pgfpathlineto{\pgfqpoint{6.637895in}{3.176392in}}%
\pgfpathlineto{\pgfqpoint{6.650164in}{3.196431in}}%
\pgfpathlineto{\pgfqpoint{6.662390in}{3.203078in}}%
\pgfpathlineto{\pgfqpoint{6.686831in}{3.220779in}}%
\pgfpathlineto{\pgfqpoint{6.699020in}{3.234593in}}%
\pgfpathlineto{\pgfqpoint{6.711326in}{3.222629in}}%
\pgfpathlineto{\pgfqpoint{6.723539in}{3.231243in}}%
\pgfpathlineto{\pgfqpoint{6.735785in}{3.242686in}}%
\pgfpathlineto{\pgfqpoint{6.760278in}{3.241762in}}%
\pgfpathlineto{\pgfqpoint{6.784771in}{3.250255in}}%
\pgfpathlineto{\pgfqpoint{6.797013in}{3.246620in}}%
\pgfpathlineto{\pgfqpoint{6.809306in}{3.250600in}}%
\pgfpathlineto{\pgfqpoint{6.821542in}{3.252141in}}%
\pgfpathlineto{\pgfqpoint{6.833802in}{3.239710in}}%
\pgfpathlineto{\pgfqpoint{6.846096in}{3.241092in}}%
\pgfpathlineto{\pgfqpoint{6.858395in}{3.239688in}}%
\pgfpathlineto{\pgfqpoint{6.870650in}{3.234744in}}%
\pgfpathlineto{\pgfqpoint{6.882898in}{3.238783in}}%
\pgfpathlineto{\pgfqpoint{6.895320in}{3.235320in}}%
\pgfpathlineto{\pgfqpoint{6.907494in}{3.220607in}}%
\pgfpathlineto{\pgfqpoint{6.919739in}{3.217805in}}%
\pgfpathlineto{\pgfqpoint{6.931980in}{3.222744in}}%
\pgfpathlineto{\pgfqpoint{6.944286in}{3.207623in}}%
\pgfpathlineto{\pgfqpoint{6.956591in}{3.221910in}}%
\pgfpathlineto{\pgfqpoint{6.968797in}{3.213867in}}%
\pgfpathlineto{\pgfqpoint{6.981103in}{3.208735in}}%
\pgfpathlineto{\pgfqpoint{6.993378in}{3.209548in}}%
\pgfpathlineto{\pgfqpoint{7.005631in}{3.206038in}}%
\pgfpathlineto{\pgfqpoint{7.017872in}{3.204987in}}%
\pgfpathlineto{\pgfqpoint{7.030232in}{3.199100in}}%
\pgfpathlineto{\pgfqpoint{7.042473in}{3.196786in}}%
\pgfpathlineto{\pgfqpoint{7.054766in}{3.211662in}}%
\pgfpathlineto{\pgfqpoint{7.067083in}{3.195162in}}%
\pgfpathlineto{\pgfqpoint{7.079417in}{3.199325in}}%
\pgfpathlineto{\pgfqpoint{7.091729in}{3.189945in}}%
\pgfpathlineto{\pgfqpoint{7.104025in}{3.190908in}}%
\pgfpathlineto{\pgfqpoint{7.116365in}{3.199726in}}%
\pgfpathlineto{\pgfqpoint{7.128574in}{3.189391in}}%
\pgfpathlineto{\pgfqpoint{7.140857in}{3.186474in}}%
\pgfpathlineto{\pgfqpoint{7.153115in}{3.185703in}}%
\pgfpathlineto{\pgfqpoint{7.165429in}{3.189907in}}%
\pgfpathlineto{\pgfqpoint{7.177722in}{3.184812in}}%
\pgfpathlineto{\pgfqpoint{7.190014in}{3.183695in}}%
\pgfpathlineto{\pgfqpoint{7.214660in}{3.170936in}}%
\pgfpathlineto{\pgfqpoint{7.226940in}{3.181959in}}%
\pgfpathlineto{\pgfqpoint{7.239221in}{3.180772in}}%
\pgfpathlineto{\pgfqpoint{7.251515in}{3.173351in}}%
\pgfpathlineto{\pgfqpoint{7.263812in}{3.169901in}}%
\pgfpathlineto{\pgfqpoint{7.276092in}{3.155799in}}%
\pgfpathlineto{\pgfqpoint{7.288351in}{3.166591in}}%
\pgfpathlineto{\pgfqpoint{7.300657in}{3.152597in}}%
\pgfpathlineto{\pgfqpoint{7.325219in}{3.150388in}}%
\pgfpathlineto{\pgfqpoint{7.337550in}{3.148799in}}%
\pgfpathlineto{\pgfqpoint{7.349809in}{3.141279in}}%
\pgfpathlineto{\pgfqpoint{7.411406in}{3.134297in}}%
\pgfpathlineto{\pgfqpoint{7.448245in}{3.141257in}}%
\pgfpathlineto{\pgfqpoint{7.460511in}{3.130727in}}%
\pgfpathlineto{\pgfqpoint{7.472767in}{3.135094in}}%
\pgfpathlineto{\pgfqpoint{7.485009in}{3.135893in}}%
\pgfpathlineto{\pgfqpoint{7.497247in}{3.121876in}}%
\pgfpathlineto{\pgfqpoint{7.509592in}{3.119973in}}%
\pgfpathlineto{\pgfqpoint{7.521881in}{3.113692in}}%
\pgfpathlineto{\pgfqpoint{7.534193in}{3.114180in}}%
\pgfpathlineto{\pgfqpoint{7.546462in}{3.103420in}}%
\pgfpathlineto{\pgfqpoint{7.558771in}{3.102845in}}%
\pgfpathlineto{\pgfqpoint{7.571089in}{3.103677in}}%
\pgfpathlineto{\pgfqpoint{7.583415in}{3.106467in}}%
\pgfpathlineto{\pgfqpoint{7.595699in}{3.091462in}}%
\pgfpathlineto{\pgfqpoint{7.608007in}{3.090813in}}%
\pgfpathlineto{\pgfqpoint{7.620286in}{3.088583in}}%
\pgfpathlineto{\pgfqpoint{7.632611in}{3.092016in}}%
\pgfpathlineto{\pgfqpoint{7.644902in}{3.082824in}}%
\pgfpathlineto{\pgfqpoint{7.657182in}{3.096128in}}%
\pgfpathlineto{\pgfqpoint{7.669522in}{3.083271in}}%
\pgfpathlineto{\pgfqpoint{7.681843in}{3.077142in}}%
\pgfpathlineto{\pgfqpoint{7.694160in}{3.083249in}}%
\pgfpathlineto{\pgfqpoint{7.706416in}{3.082050in}}%
\pgfpathlineto{\pgfqpoint{7.718690in}{3.074749in}}%
\pgfpathlineto{\pgfqpoint{7.730949in}{3.071668in}}%
\pgfpathlineto{\pgfqpoint{7.743226in}{3.066182in}}%
\pgfpathlineto{\pgfqpoint{7.755480in}{3.066369in}}%
\pgfpathlineto{\pgfqpoint{7.767779in}{3.069055in}}%
\pgfpathlineto{\pgfqpoint{7.780022in}{3.056714in}}%
\pgfpathlineto{\pgfqpoint{7.792354in}{3.062239in}}%
\pgfpathlineto{\pgfqpoint{7.804650in}{3.063760in}}%
\pgfpathlineto{\pgfqpoint{7.816896in}{3.057062in}}%
\pgfpathlineto{\pgfqpoint{7.829197in}{3.053768in}}%
\pgfpathlineto{\pgfqpoint{7.841491in}{3.062758in}}%
\pgfpathlineto{\pgfqpoint{7.853785in}{3.060639in}}%
\pgfpathlineto{\pgfqpoint{7.865947in}{3.055960in}}%
\pgfpathlineto{\pgfqpoint{7.878207in}{3.058239in}}%
\pgfpathlineto{\pgfqpoint{7.890385in}{3.064235in}}%
\pgfpathlineto{\pgfqpoint{7.902611in}{3.067428in}}%
\pgfpathlineto{\pgfqpoint{7.914842in}{3.095829in}}%
\pgfpathlineto{\pgfqpoint{7.927134in}{3.118122in}}%
\pgfpathlineto{\pgfqpoint{7.939420in}{3.126279in}}%
\pgfpathlineto{\pgfqpoint{7.951705in}{3.143227in}}%
\pgfpathlineto{\pgfqpoint{7.963942in}{3.150788in}}%
\pgfpathlineto{\pgfqpoint{7.976236in}{3.160380in}}%
\pgfpathlineto{\pgfqpoint{7.988525in}{3.163360in}}%
\pgfpathlineto{\pgfqpoint{8.013108in}{3.166581in}}%
\pgfpathlineto{\pgfqpoint{8.037625in}{3.147846in}}%
\pgfpathlineto{\pgfqpoint{8.062202in}{3.133617in}}%
\pgfpathlineto{\pgfqpoint{8.074561in}{3.133617in}}%
\pgfpathlineto{\pgfqpoint{8.086824in}{3.169785in}}%
\pgfpathlineto{\pgfqpoint{8.099107in}{3.209826in}}%
\pgfpathlineto{\pgfqpoint{8.111455in}{3.283037in}}%
\pgfpathlineto{\pgfqpoint{8.123801in}{3.344272in}}%
\pgfpathlineto{\pgfqpoint{8.136094in}{3.395748in}}%
\pgfpathlineto{\pgfqpoint{8.148465in}{3.464656in}}%
\pgfpathlineto{\pgfqpoint{8.160747in}{3.516245in}}%
\pgfpathlineto{\pgfqpoint{8.185384in}{3.595564in}}%
\pgfpathlineto{\pgfqpoint{8.197720in}{3.619216in}}%
\pgfpathlineto{\pgfqpoint{8.210018in}{3.650262in}}%
\pgfpathlineto{\pgfqpoint{8.222298in}{3.662405in}}%
\pgfpathlineto{\pgfqpoint{8.234578in}{3.670953in}}%
\pgfpathlineto{\pgfqpoint{8.246865in}{3.669687in}}%
\pgfpathlineto{\pgfqpoint{8.259028in}{3.664174in}}%
\pgfpathlineto{\pgfqpoint{8.271539in}{3.670450in}}%
\pgfpathlineto{\pgfqpoint{8.283817in}{3.663671in}}%
\pgfpathlineto{\pgfqpoint{8.296187in}{3.645271in}}%
\pgfpathlineto{\pgfqpoint{8.308399in}{3.631452in}}%
\pgfpathlineto{\pgfqpoint{8.320643in}{3.627355in}}%
\pgfpathlineto{\pgfqpoint{8.332937in}{3.607353in}}%
\pgfpathlineto{\pgfqpoint{8.345244in}{3.600890in}}%
\pgfpathlineto{\pgfqpoint{8.357498in}{3.586866in}}%
\pgfpathlineto{\pgfqpoint{8.369797in}{3.581596in}}%
\pgfpathlineto{\pgfqpoint{8.394176in}{3.548017in}}%
\pgfpathlineto{\pgfqpoint{8.406428in}{3.525054in}}%
\pgfpathlineto{\pgfqpoint{8.418709in}{3.508572in}}%
\pgfpathlineto{\pgfqpoint{8.431023in}{3.497714in}}%
\pgfpathlineto{\pgfqpoint{8.443305in}{3.496131in}}%
\pgfpathlineto{\pgfqpoint{8.455608in}{3.482647in}}%
\pgfpathlineto{\pgfqpoint{8.467915in}{3.478084in}}%
\pgfpathlineto{\pgfqpoint{8.480239in}{3.476073in}}%
\pgfpathlineto{\pgfqpoint{8.492533in}{3.476650in}}%
\pgfpathlineto{\pgfqpoint{8.504778in}{3.488085in}}%
\pgfpathlineto{\pgfqpoint{8.517110in}{3.467227in}}%
\pgfpathlineto{\pgfqpoint{8.566463in}{3.492127in}}%
\pgfpathlineto{\pgfqpoint{8.578707in}{3.489072in}}%
\pgfpathlineto{\pgfqpoint{8.590975in}{3.492462in}}%
\pgfpathlineto{\pgfqpoint{8.603278in}{3.500247in}}%
\pgfpathlineto{\pgfqpoint{8.615537in}{3.505182in}}%
\pgfpathlineto{\pgfqpoint{8.627808in}{3.502835in}}%
\pgfpathlineto{\pgfqpoint{8.640024in}{3.508553in}}%
\pgfpathlineto{\pgfqpoint{8.664534in}{3.523247in}}%
\pgfpathlineto{\pgfqpoint{8.676810in}{3.524365in}}%
\pgfpathlineto{\pgfqpoint{8.689078in}{3.527810in}}%
\pgfpathlineto{\pgfqpoint{8.701416in}{3.539077in}}%
\pgfpathlineto{\pgfqpoint{8.726061in}{3.536079in}}%
\pgfpathlineto{\pgfqpoint{8.738260in}{3.538947in}}%
\pgfpathlineto{\pgfqpoint{8.750541in}{3.543789in}}%
\pgfpathlineto{\pgfqpoint{8.762810in}{3.543082in}}%
\pgfpathlineto{\pgfqpoint{8.775078in}{3.549637in}}%
\pgfpathlineto{\pgfqpoint{8.787335in}{3.553120in}}%
\pgfpathlineto{\pgfqpoint{8.799612in}{3.554535in}}%
\pgfpathlineto{\pgfqpoint{8.811984in}{3.553232in}}%
\pgfpathlineto{\pgfqpoint{8.836530in}{3.557962in}}%
\pgfpathlineto{\pgfqpoint{8.848842in}{3.550513in}}%
\pgfpathlineto{\pgfqpoint{8.861160in}{3.558539in}}%
\pgfpathlineto{\pgfqpoint{8.873449in}{3.555466in}}%
\pgfpathlineto{\pgfqpoint{8.885737in}{3.554722in}}%
\pgfpathlineto{\pgfqpoint{8.897990in}{3.551947in}}%
\pgfpathlineto{\pgfqpoint{8.910275in}{3.552971in}}%
\pgfpathlineto{\pgfqpoint{8.922495in}{3.558186in}}%
\pgfpathlineto{\pgfqpoint{8.934849in}{3.552803in}}%
\pgfpathlineto{\pgfqpoint{8.947104in}{3.562152in}}%
\pgfpathlineto{\pgfqpoint{8.959428in}{3.552636in}}%
\pgfpathlineto{\pgfqpoint{8.983944in}{3.554498in}}%
\pgfpathlineto{\pgfqpoint{8.996217in}{3.558074in}}%
\pgfpathlineto{\pgfqpoint{9.008511in}{3.558875in}}%
\pgfpathlineto{\pgfqpoint{9.020806in}{3.562525in}}%
\pgfpathlineto{\pgfqpoint{9.033063in}{3.562134in}}%
\pgfpathlineto{\pgfqpoint{9.045373in}{3.576791in}}%
\pgfpathlineto{\pgfqpoint{9.057635in}{3.569602in}}%
\pgfpathlineto{\pgfqpoint{9.069982in}{3.583123in}}%
\pgfpathlineto{\pgfqpoint{9.082186in}{3.578392in}}%
\pgfpathlineto{\pgfqpoint{9.106778in}{3.584594in}}%
\pgfpathlineto{\pgfqpoint{9.119078in}{3.585861in}}%
\pgfpathlineto{\pgfqpoint{9.131343in}{3.588524in}}%
\pgfpathlineto{\pgfqpoint{9.143598in}{3.595359in}}%
\pgfpathlineto{\pgfqpoint{9.168220in}{3.594279in}}%
\pgfpathlineto{\pgfqpoint{9.180483in}{3.594092in}}%
\pgfpathlineto{\pgfqpoint{9.192775in}{3.597556in}}%
\pgfpathlineto{\pgfqpoint{9.205147in}{3.604987in}}%
\pgfpathlineto{\pgfqpoint{9.217364in}{3.610835in}}%
\pgfpathlineto{\pgfqpoint{9.229650in}{3.609494in}}%
\pgfpathlineto{\pgfqpoint{9.241913in}{3.610072in}}%
\pgfpathlineto{\pgfqpoint{9.254153in}{3.605974in}}%
\pgfpathlineto{\pgfqpoint{9.266401in}{3.611319in}}%
\pgfpathlineto{\pgfqpoint{9.278692in}{3.606831in}}%
\pgfpathlineto{\pgfqpoint{9.290988in}{3.605546in}}%
\pgfpathlineto{\pgfqpoint{9.303292in}{3.608396in}}%
\pgfpathlineto{\pgfqpoint{9.315536in}{3.602678in}}%
\pgfpathlineto{\pgfqpoint{9.327781in}{3.610034in}}%
\pgfpathlineto{\pgfqpoint{9.340096in}{3.624431in}}%
\pgfpathlineto{\pgfqpoint{9.352334in}{3.609066in}}%
\pgfpathlineto{\pgfqpoint{9.364627in}{3.598450in}}%
\pgfpathlineto{\pgfqpoint{9.376899in}{3.608954in}}%
\pgfpathlineto{\pgfqpoint{9.401540in}{3.616832in}}%
\pgfpathlineto{\pgfqpoint{9.413786in}{3.608060in}}%
\pgfpathlineto{\pgfqpoint{9.426057in}{3.607930in}}%
\pgfpathlineto{\pgfqpoint{9.438333in}{3.617894in}}%
\pgfpathlineto{\pgfqpoint{9.450619in}{3.611562in}}%
\pgfpathlineto{\pgfqpoint{9.462959in}{3.606645in}}%
\pgfpathlineto{\pgfqpoint{9.475206in}{3.609848in}}%
\pgfpathlineto{\pgfqpoint{9.487548in}{3.617409in}}%
\pgfpathlineto{\pgfqpoint{9.499862in}{3.617801in}}%
\pgfpathlineto{\pgfqpoint{9.512168in}{3.614299in}}%
\pgfpathlineto{\pgfqpoint{9.524489in}{3.619384in}}%
\pgfpathlineto{\pgfqpoint{9.536812in}{3.633831in}}%
\pgfpathlineto{\pgfqpoint{9.549056in}{3.666942in}}%
\pgfpathlineto{\pgfqpoint{9.561364in}{3.720346in}}%
\pgfpathlineto{\pgfqpoint{9.598378in}{3.901331in}}%
\pgfpathlineto{\pgfqpoint{9.610774in}{3.938311in}}%
\pgfpathlineto{\pgfqpoint{9.623253in}{4.000845in}}%
\pgfpathlineto{\pgfqpoint{9.635662in}{4.043596in}}%
\pgfpathlineto{\pgfqpoint{9.647992in}{4.090988in}}%
\pgfpathlineto{\pgfqpoint{9.660340in}{4.125072in}}%
\pgfpathlineto{\pgfqpoint{9.672597in}{4.138983in}}%
\pgfpathlineto{\pgfqpoint{9.684891in}{4.166389in}}%
\pgfpathlineto{\pgfqpoint{9.697161in}{4.180745in}}%
\pgfpathlineto{\pgfqpoint{9.709509in}{4.207382in}}%
\pgfpathlineto{\pgfqpoint{9.734189in}{4.217724in}}%
\pgfpathlineto{\pgfqpoint{9.746462in}{4.237850in}}%
\pgfpathlineto{\pgfqpoint{9.758804in}{4.237713in}}%
\pgfpathlineto{\pgfqpoint{9.771158in}{4.246752in}}%
\pgfpathlineto{\pgfqpoint{9.783428in}{4.240838in}}%
\pgfpathlineto{\pgfqpoint{9.795726in}{4.237259in}}%
\pgfpathlineto{\pgfqpoint{9.808012in}{4.231119in}}%
\pgfpathlineto{\pgfqpoint{9.820266in}{4.232365in}}%
\pgfpathlineto{\pgfqpoint{9.832583in}{4.225785in}}%
\pgfpathlineto{\pgfqpoint{9.844800in}{4.228904in}}%
\pgfpathlineto{\pgfqpoint{9.857043in}{4.227580in}}%
\pgfpathlineto{\pgfqpoint{9.869336in}{4.222678in}}%
\pgfpathlineto{\pgfqpoint{9.881634in}{4.220390in}}%
\pgfpathlineto{\pgfqpoint{9.893847in}{4.219891in}}%
\pgfpathlineto{\pgfqpoint{9.906155in}{4.217637in}}%
\pgfpathlineto{\pgfqpoint{9.918431in}{4.219163in}}%
\pgfpathlineto{\pgfqpoint{9.930722in}{4.214672in}}%
\pgfpathlineto{\pgfqpoint{9.942947in}{4.216718in}}%
\pgfpathlineto{\pgfqpoint{9.955220in}{4.203800in}}%
\pgfpathlineto{\pgfqpoint{9.967495in}{4.217749in}}%
\pgfpathlineto{\pgfqpoint{9.979761in}{4.209307in}}%
\pgfpathlineto{\pgfqpoint{9.992027in}{4.214068in}}%
\pgfpathlineto{\pgfqpoint{10.028949in}{4.218480in}}%
\pgfpathlineto{\pgfqpoint{10.041270in}{4.217866in}}%
\pgfpathlineto{\pgfqpoint{10.053550in}{4.223689in}}%
\pgfpathlineto{\pgfqpoint{10.065831in}{4.218728in}}%
\pgfpathlineto{\pgfqpoint{10.078066in}{4.216802in}}%
\pgfpathlineto{\pgfqpoint{10.090342in}{4.219052in}}%
\pgfpathlineto{\pgfqpoint{10.102679in}{4.223341in}}%
\pgfpathlineto{\pgfqpoint{10.115008in}{4.223263in}}%
\pgfpathlineto{\pgfqpoint{10.127346in}{4.226828in}}%
\pgfpathlineto{\pgfqpoint{10.139648in}{4.228153in}}%
\pgfpathlineto{\pgfqpoint{10.164317in}{4.242267in}}%
\pgfpathlineto{\pgfqpoint{10.176620in}{4.223981in}}%
\pgfpathlineto{\pgfqpoint{10.188950in}{4.232005in}}%
\pgfpathlineto{\pgfqpoint{10.201195in}{4.235760in}}%
\pgfpathlineto{\pgfqpoint{10.213507in}{4.243015in}}%
\pgfpathlineto{\pgfqpoint{10.225810in}{4.244137in}}%
\pgfpathlineto{\pgfqpoint{10.238172in}{4.239840in}}%
\pgfpathlineto{\pgfqpoint{10.250369in}{4.242768in}}%
\pgfpathlineto{\pgfqpoint{10.262627in}{4.236057in}}%
\pgfpathlineto{\pgfqpoint{10.274908in}{4.239062in}}%
\pgfpathlineto{\pgfqpoint{10.287140in}{4.240534in}}%
\pgfpathlineto{\pgfqpoint{10.299399in}{4.247930in}}%
\pgfpathlineto{\pgfqpoint{10.311760in}{4.244828in}}%
\pgfpathlineto{\pgfqpoint{10.323981in}{4.247113in}}%
\pgfpathlineto{\pgfqpoint{10.336235in}{4.256939in}}%
\pgfpathlineto{\pgfqpoint{10.348537in}{4.261930in}}%
\pgfpathlineto{\pgfqpoint{10.360831in}{4.264779in}}%
\pgfpathlineto{\pgfqpoint{10.373112in}{4.273793in}}%
\pgfpathlineto{\pgfqpoint{10.385397in}{4.277408in}}%
\pgfpathlineto{\pgfqpoint{10.397716in}{4.299813in}}%
\pgfpathlineto{\pgfqpoint{10.410035in}{4.301819in}}%
\pgfpathlineto{\pgfqpoint{10.422350in}{4.290052in}}%
\pgfpathlineto{\pgfqpoint{10.434627in}{4.316968in}}%
\pgfpathlineto{\pgfqpoint{10.446948in}{4.322217in}}%
\pgfpathlineto{\pgfqpoint{10.459245in}{4.330747in}}%
\pgfpathlineto{\pgfqpoint{10.471527in}{4.329649in}}%
\pgfpathlineto{\pgfqpoint{10.496232in}{4.329434in}}%
\pgfpathlineto{\pgfqpoint{10.508466in}{4.325299in}}%
\pgfpathlineto{\pgfqpoint{10.520797in}{4.331617in}}%
\pgfpathlineto{\pgfqpoint{10.545384in}{4.320920in}}%
\pgfpathlineto{\pgfqpoint{10.557696in}{4.322581in}}%
\pgfpathlineto{\pgfqpoint{10.570313in}{4.325842in}}%
\pgfpathlineto{\pgfqpoint{10.582312in}{4.303402in}}%
\pgfpathlineto{\pgfqpoint{10.606881in}{4.293707in}}%
\pgfpathlineto{\pgfqpoint{10.619179in}{4.292483in}}%
\pgfpathlineto{\pgfqpoint{10.631580in}{4.288553in}}%
\pgfpathlineto{\pgfqpoint{10.668339in}{4.284325in}}%
\pgfpathlineto{\pgfqpoint{10.680650in}{4.285477in}}%
\pgfpathlineto{\pgfqpoint{10.705258in}{4.293119in}}%
\pgfpathlineto{\pgfqpoint{10.717543in}{4.280985in}}%
\pgfpathlineto{\pgfqpoint{10.729835in}{4.280421in}}%
\pgfpathlineto{\pgfqpoint{10.742107in}{4.295136in}}%
\pgfpathlineto{\pgfqpoint{10.754414in}{4.285615in}}%
\pgfpathlineto{\pgfqpoint{10.766646in}{4.283788in}}%
\pgfpathlineto{\pgfqpoint{10.791376in}{4.284886in}}%
\pgfpathlineto{\pgfqpoint{10.803478in}{4.284925in}}%
\pgfpathlineto{\pgfqpoint{10.828073in}{4.280704in}}%
\pgfpathlineto{\pgfqpoint{10.840246in}{4.285645in}}%
\pgfpathlineto{\pgfqpoint{10.852476in}{4.293147in}}%
\pgfpathlineto{\pgfqpoint{10.877042in}{4.287399in}}%
\pgfpathlineto{\pgfqpoint{10.913829in}{4.296281in}}%
\pgfpathlineto{\pgfqpoint{10.926101in}{4.304762in}}%
\pgfpathlineto{\pgfqpoint{10.938412in}{4.303392in}}%
\pgfpathlineto{\pgfqpoint{10.950740in}{4.322093in}}%
\pgfpathlineto{\pgfqpoint{10.963071in}{4.317333in}}%
\pgfpathlineto{\pgfqpoint{10.975356in}{4.299795in}}%
\pgfpathlineto{\pgfqpoint{10.987655in}{4.318676in}}%
\pgfpathlineto{\pgfqpoint{10.999935in}{4.300031in}}%
\pgfpathlineto{\pgfqpoint{11.024488in}{4.308639in}}%
\pgfpathlineto{\pgfqpoint{11.036742in}{4.300135in}}%
\pgfpathlineto{\pgfqpoint{11.049045in}{4.301851in}}%
\pgfpathlineto{\pgfqpoint{11.061326in}{4.297390in}}%
\pgfpathlineto{\pgfqpoint{11.073604in}{4.311519in}}%
\pgfpathlineto{\pgfqpoint{11.085856in}{4.304128in}}%
\pgfpathlineto{\pgfqpoint{11.098187in}{4.299748in}}%
\pgfpathlineto{\pgfqpoint{11.110475in}{4.307220in}}%
\pgfpathlineto{\pgfqpoint{11.122729in}{4.296206in}}%
\pgfpathlineto{\pgfqpoint{11.134994in}{4.303504in}}%
\pgfpathlineto{\pgfqpoint{11.147290in}{4.296468in}}%
\pgfpathlineto{\pgfqpoint{11.159571in}{4.296358in}}%
\pgfpathlineto{\pgfqpoint{11.171813in}{4.298726in}}%
\pgfpathlineto{\pgfqpoint{11.184085in}{4.254607in}}%
\pgfpathlineto{\pgfqpoint{11.196431in}{4.235871in}}%
\pgfpathlineto{\pgfqpoint{11.208753in}{4.199890in}}%
\pgfpathlineto{\pgfqpoint{11.221093in}{4.169570in}}%
\pgfpathlineto{\pgfqpoint{11.233579in}{4.114909in}}%
\pgfpathlineto{\pgfqpoint{11.245701in}{4.088370in}}%
\pgfpathlineto{\pgfqpoint{11.257952in}{4.040376in}}%
\pgfpathlineto{\pgfqpoint{11.282684in}{3.968563in}}%
\pgfpathlineto{\pgfqpoint{11.295078in}{3.936213in}}%
\pgfpathlineto{\pgfqpoint{11.307325in}{3.896823in}}%
\pgfpathlineto{\pgfqpoint{11.319662in}{3.865964in}}%
\pgfpathlineto{\pgfqpoint{11.344294in}{3.793349in}}%
\pgfpathlineto{\pgfqpoint{11.356640in}{3.775992in}}%
\pgfpathlineto{\pgfqpoint{11.368900in}{3.745691in}}%
\pgfpathlineto{\pgfqpoint{11.393518in}{3.722187in}}%
\pgfpathlineto{\pgfqpoint{11.405847in}{3.715259in}}%
\pgfpathlineto{\pgfqpoint{11.418118in}{3.699746in}}%
\pgfpathlineto{\pgfqpoint{11.430362in}{3.686634in}}%
\pgfpathlineto{\pgfqpoint{11.442815in}{3.702651in}}%
\pgfpathlineto{\pgfqpoint{11.454904in}{3.699746in}}%
\pgfpathlineto{\pgfqpoint{11.479492in}{3.707642in}}%
\pgfpathlineto{\pgfqpoint{11.491800in}{3.706283in}}%
\pgfpathlineto{\pgfqpoint{11.504199in}{3.708201in}}%
\pgfpathlineto{\pgfqpoint{11.516470in}{3.715017in}}%
\pgfpathlineto{\pgfqpoint{11.528782in}{3.708275in}}%
\pgfpathlineto{\pgfqpoint{11.541117in}{3.732971in}}%
\pgfpathlineto{\pgfqpoint{11.553426in}{3.729283in}}%
\pgfpathlineto{\pgfqpoint{11.565722in}{3.730158in}}%
\pgfpathlineto{\pgfqpoint{11.590315in}{3.749583in}}%
\pgfpathlineto{\pgfqpoint{11.602696in}{3.751315in}}%
\pgfpathlineto{\pgfqpoint{11.614948in}{3.760608in}}%
\pgfpathlineto{\pgfqpoint{11.627285in}{3.759938in}}%
\pgfpathlineto{\pgfqpoint{11.651848in}{3.771447in}}%
\pgfpathlineto{\pgfqpoint{11.664151in}{3.774483in}}%
\pgfpathlineto{\pgfqpoint{11.676427in}{3.784242in}}%
\pgfpathlineto{\pgfqpoint{11.688694in}{3.784670in}}%
\pgfpathlineto{\pgfqpoint{11.700952in}{3.783460in}}%
\pgfpathlineto{\pgfqpoint{11.725571in}{3.775265in}}%
\pgfpathlineto{\pgfqpoint{11.737926in}{3.778096in}}%
\pgfpathlineto{\pgfqpoint{11.750211in}{3.775656in}}%
\pgfpathlineto{\pgfqpoint{11.762524in}{3.770200in}}%
\pgfpathlineto{\pgfqpoint{11.774822in}{3.762825in}}%
\pgfpathlineto{\pgfqpoint{11.787151in}{3.759174in}}%
\pgfpathlineto{\pgfqpoint{11.799386in}{3.759454in}}%
\pgfpathlineto{\pgfqpoint{11.811667in}{3.756306in}}%
\pgfpathlineto{\pgfqpoint{11.824001in}{3.758653in}}%
\pgfpathlineto{\pgfqpoint{11.836243in}{3.749527in}}%
\pgfpathlineto{\pgfqpoint{11.873217in}{3.736509in}}%
\pgfpathlineto{\pgfqpoint{11.885438in}{3.736118in}}%
\pgfpathlineto{\pgfqpoint{11.897740in}{3.728817in}}%
\pgfpathlineto{\pgfqpoint{11.910033in}{3.718779in}}%
\pgfpathlineto{\pgfqpoint{11.922303in}{3.731592in}}%
\pgfpathlineto{\pgfqpoint{11.934629in}{3.717178in}}%
\pgfpathlineto{\pgfqpoint{11.946883in}{3.719450in}}%
\pgfpathlineto{\pgfqpoint{11.959144in}{3.726489in}}%
\pgfpathlineto{\pgfqpoint{11.971437in}{3.717867in}}%
\pgfpathlineto{\pgfqpoint{11.983736in}{3.719543in}}%
\pgfpathlineto{\pgfqpoint{11.996012in}{3.712503in}}%
\pgfpathlineto{\pgfqpoint{12.008295in}{3.708704in}}%
\pgfpathlineto{\pgfqpoint{12.020592in}{3.712596in}}%
\pgfpathlineto{\pgfqpoint{12.032931in}{3.718854in}}%
\pgfpathlineto{\pgfqpoint{12.045260in}{3.719059in}}%
\pgfpathlineto{\pgfqpoint{12.057611in}{3.703452in}}%
\pgfpathlineto{\pgfqpoint{12.069895in}{3.727979in}}%
\pgfpathlineto{\pgfqpoint{12.082278in}{3.711553in}}%
\pgfpathlineto{\pgfqpoint{12.094595in}{3.726266in}}%
\pgfpathlineto{\pgfqpoint{12.106906in}{3.721927in}}%
\pgfpathlineto{\pgfqpoint{12.119235in}{3.719878in}}%
\pgfpathlineto{\pgfqpoint{12.131596in}{3.713043in}}%
\pgfpathlineto{\pgfqpoint{12.143920in}{3.721480in}}%
\pgfpathlineto{\pgfqpoint{12.156241in}{3.716935in}}%
\pgfpathlineto{\pgfqpoint{12.168551in}{3.715501in}}%
\pgfpathlineto{\pgfqpoint{12.180852in}{3.711572in}}%
\pgfpathlineto{\pgfqpoint{12.193154in}{3.697529in}}%
\pgfpathlineto{\pgfqpoint{12.205461in}{3.707009in}}%
\pgfpathlineto{\pgfqpoint{12.217786in}{3.701440in}}%
\pgfpathlineto{\pgfqpoint{12.230093in}{3.704029in}}%
\pgfpathlineto{\pgfqpoint{12.242435in}{3.700342in}}%
\pgfpathlineto{\pgfqpoint{12.254793in}{3.703899in}}%
\pgfpathlineto{\pgfqpoint{12.267040in}{3.692873in}}%
\pgfpathlineto{\pgfqpoint{12.279417in}{3.700081in}}%
\pgfpathlineto{\pgfqpoint{12.291648in}{3.691570in}}%
\pgfpathlineto{\pgfqpoint{12.303888in}{3.687398in}}%
\pgfpathlineto{\pgfqpoint{12.316212in}{3.678496in}}%
\pgfpathlineto{\pgfqpoint{12.328453in}{3.685014in}}%
\pgfpathlineto{\pgfqpoint{12.340754in}{3.680172in}}%
\pgfpathlineto{\pgfqpoint{12.353027in}{3.668457in}}%
\pgfpathlineto{\pgfqpoint{12.365318in}{3.672611in}}%
\pgfpathlineto{\pgfqpoint{12.377559in}{3.672387in}}%
\pgfpathlineto{\pgfqpoint{12.389784in}{3.675032in}}%
\pgfpathlineto{\pgfqpoint{12.402101in}{3.667135in}}%
\pgfpathlineto{\pgfqpoint{12.414318in}{3.667582in}}%
\pgfpathlineto{\pgfqpoint{12.426552in}{3.671624in}}%
\pgfpathlineto{\pgfqpoint{12.438868in}{3.665608in}}%
\pgfpathlineto{\pgfqpoint{12.451045in}{3.662200in}}%
\pgfpathlineto{\pgfqpoint{12.463322in}{3.661958in}}%
\pgfpathlineto{\pgfqpoint{12.487822in}{3.669351in}}%
\pgfpathlineto{\pgfqpoint{12.500090in}{3.662647in}}%
\pgfpathlineto{\pgfqpoint{12.512300in}{3.667210in}}%
\pgfpathlineto{\pgfqpoint{12.536785in}{3.658978in}}%
\pgfpathlineto{\pgfqpoint{12.549021in}{3.659909in}}%
\pgfpathlineto{\pgfqpoint{12.561361in}{3.652106in}}%
\pgfpathlineto{\pgfqpoint{12.573599in}{3.654303in}}%
\pgfpathlineto{\pgfqpoint{12.598181in}{3.668960in}}%
\pgfpathlineto{\pgfqpoint{12.610477in}{3.661027in}}%
\pgfpathlineto{\pgfqpoint{12.622780in}{3.650392in}}%
\pgfpathlineto{\pgfqpoint{12.635089in}{3.651584in}}%
\pgfpathlineto{\pgfqpoint{12.647435in}{3.657451in}}%
\pgfpathlineto{\pgfqpoint{12.659721in}{3.649592in}}%
\pgfpathlineto{\pgfqpoint{12.672038in}{3.652702in}}%
\pgfpathlineto{\pgfqpoint{12.684320in}{3.654247in}}%
\pgfpathlineto{\pgfqpoint{12.708922in}{3.651137in}}%
\pgfpathlineto{\pgfqpoint{12.721184in}{3.636164in}}%
\pgfpathlineto{\pgfqpoint{12.733421in}{3.636983in}}%
\pgfpathlineto{\pgfqpoint{12.745725in}{3.642626in}}%
\pgfpathlineto{\pgfqpoint{12.757996in}{3.643781in}}%
\pgfpathlineto{\pgfqpoint{12.770271in}{3.633463in}}%
\pgfpathlineto{\pgfqpoint{12.782558in}{3.631731in}}%
\pgfpathlineto{\pgfqpoint{12.794883in}{3.626591in}}%
\pgfpathlineto{\pgfqpoint{12.807199in}{3.638548in}}%
\pgfpathlineto{\pgfqpoint{12.819481in}{3.626889in}}%
\pgfpathlineto{\pgfqpoint{12.831752in}{3.620278in}}%
\pgfpathlineto{\pgfqpoint{12.844082in}{3.632662in}}%
\pgfpathlineto{\pgfqpoint{12.856308in}{3.616851in}}%
\pgfpathlineto{\pgfqpoint{12.868545in}{3.626405in}}%
\pgfpathlineto{\pgfqpoint{12.893124in}{3.619421in}}%
\pgfpathlineto{\pgfqpoint{12.905384in}{3.609848in}}%
\pgfpathlineto{\pgfqpoint{12.917658in}{3.618397in}}%
\pgfpathlineto{\pgfqpoint{12.942206in}{3.611282in}}%
\pgfpathlineto{\pgfqpoint{12.954519in}{3.600853in}}%
\pgfpathlineto{\pgfqpoint{13.015912in}{3.621153in}}%
\pgfpathlineto{\pgfqpoint{13.077373in}{3.618918in}}%
\pgfpathlineto{\pgfqpoint{13.102017in}{3.612791in}}%
\pgfpathlineto{\pgfqpoint{13.102017in}{3.612791in}}%
\pgfusepath{stroke}%
\end{pgfscope}%
\begin{pgfscope}%
\pgfsetrectcap%
\pgfsetmiterjoin%
\pgfsetlinewidth{0.803000pt}%
\definecolor{currentstroke}{rgb}{0.000000,0.000000,0.000000}%
\pgfsetstrokecolor{currentstroke}%
\pgfsetdash{}{0pt}%
\pgfpathmoveto{\pgfqpoint{1.021528in}{0.692778in}}%
\pgfpathlineto{\pgfqpoint{1.021528in}{4.561667in}}%
\pgfusepath{stroke}%
\end{pgfscope}%
\begin{pgfscope}%
\pgfsetrectcap%
\pgfsetmiterjoin%
\pgfsetlinewidth{0.803000pt}%
\definecolor{currentstroke}{rgb}{0.000000,0.000000,0.000000}%
\pgfsetstrokecolor{currentstroke}%
\pgfsetdash{}{0pt}%
\pgfpathmoveto{\pgfqpoint{13.390000in}{0.692778in}}%
\pgfpathlineto{\pgfqpoint{13.390000in}{4.561667in}}%
\pgfusepath{stroke}%
\end{pgfscope}%
\begin{pgfscope}%
\pgfsetrectcap%
\pgfsetmiterjoin%
\pgfsetlinewidth{0.803000pt}%
\definecolor{currentstroke}{rgb}{0.000000,0.000000,0.000000}%
\pgfsetstrokecolor{currentstroke}%
\pgfsetdash{}{0pt}%
\pgfpathmoveto{\pgfqpoint{1.021528in}{0.692778in}}%
\pgfpathlineto{\pgfqpoint{13.390000in}{0.692778in}}%
\pgfusepath{stroke}%
\end{pgfscope}%
\begin{pgfscope}%
\pgfsetrectcap%
\pgfsetmiterjoin%
\pgfsetlinewidth{0.803000pt}%
\definecolor{currentstroke}{rgb}{0.000000,0.000000,0.000000}%
\pgfsetstrokecolor{currentstroke}%
\pgfsetdash{}{0pt}%
\pgfpathmoveto{\pgfqpoint{1.021528in}{4.561667in}}%
\pgfpathlineto{\pgfqpoint{13.390000in}{4.561667in}}%
\pgfusepath{stroke}%
\end{pgfscope}%
\begin{pgfscope}%
\definecolor{textcolor}{rgb}{0.000000,0.000000,0.000000}%
\pgfsetstrokecolor{textcolor}%
\pgfsetfillcolor{textcolor}%
\pgftext[x=7.205764in,y=4.645000in,,base]{\color{textcolor}\sffamily\fontsize{14.400000}{17.280000}\selectfont Y Position Error vs Time}%
\end{pgfscope}%
\begin{pgfscope}%
\pgfsetbuttcap%
\pgfsetmiterjoin%
\definecolor{currentfill}{rgb}{1.000000,1.000000,1.000000}%
\pgfsetfillcolor{currentfill}%
\pgfsetfillopacity{0.800000}%
\pgfsetlinewidth{1.003750pt}%
\definecolor{currentstroke}{rgb}{0.800000,0.800000,0.800000}%
\pgfsetstrokecolor{currentstroke}%
\pgfsetstrokeopacity{0.800000}%
\pgfsetdash{}{0pt}%
\pgfpathmoveto{\pgfqpoint{11.728363in}{3.939076in}}%
\pgfpathlineto{\pgfqpoint{13.273333in}{3.939076in}}%
\pgfpathquadraticcurveto{\pgfqpoint{13.306667in}{3.939076in}}{\pgfqpoint{13.306667in}{3.972409in}}%
\pgfpathlineto{\pgfqpoint{13.306667in}{4.445000in}}%
\pgfpathquadraticcurveto{\pgfqpoint{13.306667in}{4.478333in}}{\pgfqpoint{13.273333in}{4.478333in}}%
\pgfpathlineto{\pgfqpoint{11.728363in}{4.478333in}}%
\pgfpathquadraticcurveto{\pgfqpoint{11.695029in}{4.478333in}}{\pgfqpoint{11.695029in}{4.445000in}}%
\pgfpathlineto{\pgfqpoint{11.695029in}{3.972409in}}%
\pgfpathquadraticcurveto{\pgfqpoint{11.695029in}{3.939076in}}{\pgfqpoint{11.728363in}{3.939076in}}%
\pgfpathlineto{\pgfqpoint{11.728363in}{3.939076in}}%
\pgfpathclose%
\pgfusepath{stroke,fill}%
\end{pgfscope}%
\begin{pgfscope}%
\pgfsetrectcap%
\pgfsetroundjoin%
\pgfsetlinewidth{1.505625pt}%
\definecolor{currentstroke}{rgb}{0.121569,0.466667,0.705882}%
\pgfsetstrokecolor{currentstroke}%
\pgfsetdash{}{0pt}%
\pgfpathmoveto{\pgfqpoint{11.761696in}{4.343372in}}%
\pgfpathlineto{\pgfqpoint{11.928363in}{4.343372in}}%
\pgfpathlineto{\pgfqpoint{12.095029in}{4.343372in}}%
\pgfusepath{stroke}%
\end{pgfscope}%
\begin{pgfscope}%
\definecolor{textcolor}{rgb}{0.000000,0.000000,0.000000}%
\pgfsetstrokecolor{textcolor}%
\pgfsetfillcolor{textcolor}%
\pgftext[x=12.228363in,y=4.285039in,left,base]{\color{textcolor}\sffamily\fontsize{12.000000}{14.400000}\selectfont vrpn Y Error}%
\end{pgfscope}%
\begin{pgfscope}%
\pgfsetrectcap%
\pgfsetroundjoin%
\pgfsetlinewidth{1.505625pt}%
\definecolor{currentstroke}{rgb}{1.000000,0.498039,0.054902}%
\pgfsetstrokecolor{currentstroke}%
\pgfsetdash{}{0pt}%
\pgfpathmoveto{\pgfqpoint{11.761696in}{4.098744in}}%
\pgfpathlineto{\pgfqpoint{11.928363in}{4.098744in}}%
\pgfpathlineto{\pgfqpoint{12.095029in}{4.098744in}}%
\pgfusepath{stroke}%
\end{pgfscope}%
\begin{pgfscope}%
\definecolor{textcolor}{rgb}{0.000000,0.000000,0.000000}%
\pgfsetstrokecolor{textcolor}%
\pgfsetfillcolor{textcolor}%
\pgftext[x=12.228363in,y=4.040410in,left,base]{\color{textcolor}\sffamily\fontsize{12.000000}{14.400000}\selectfont slam Y Error}%
\end{pgfscope}%
\end{pgfpicture}%
\makeatother%
\endgroup%

%     }
%     \label{fig:Exp1_Erros}
%     \source
% \end{figure}

A segunda trajetória realizada no segundo e quarto experimentos foi uma trajetória em formato de lemniscata de Bernoulli, caracterizada pelas equações:

\begin{equation}
\label{eq:lemniscata}
    \bs{x}_{des}=\begin{bmatrix} x_{des}=r_x\cos{\omega t} \\ y_{des}=r_y\sin{2 \omega t} \end{bmatrix}
\end{equation}

com $r_x = 1.5m$, $r_y = 1.0m$ e $\omega = 0.20\ Rad/s$. A velocidade desejada da trajetória é obtida pela derivada da equação anterior:

\begin{equation}
    \dot{\bs{x}}_{des}=\begin{bmatrix} \dot{x}_{des} = -r_x \omega \sin{\omega t} \\
    \dot{y}_{des}=r_y 2\omega\cos{2 \omega t} \end{bmatrix}
\end{equation}

A Figura \ref{fig:Exp2_TrajetoriaVRPN} mostra o trajeto realizado pelo robô em comparação com a trajetória desejada no Experimento 2 e a Figura \ref{fig:Exp2_Comparacao_Posicao_v_tempo} mostra a evolução das posições $x$ e $y$ do robô durante o experimento. Além das posições obtidas pelo OptiTrack (denominadas no gráfico de \ASPASIMPLES{vrpn}, também é mostrada a posição obtida pela SLAM Toolbox (denominada de \ASPASIMPLES{slam} nos gráficos).

\begin{figure}[htb]
    \centering
    \caption{Experimento 2 - Trajetória desejada e realizada pelo robô}
    \includegraphics[width=0.8\linewidth]{img/Resultados/Exp2_VRPN_Control_LEMNISCATA/Trajetoria_VRPN.pdf}
    \source
    \label{fig:Exp2_TrajetoriaVRPN}
\end{figure}

\begin{figure}[htb]
    \centering
    \caption{Experimento 2 - Comparação entre as posições registradas pelo OptiTrack e pela SLAM Toolbox durante o experimento}
    \includegraphics[width=0.8\linewidth]{img/Resultados/Exp2_VRPN_Control_LEMNISCATA/Position_Comparison.pdf}
    \source
    \label{fig:Exp2_Comparacao_Posicao_v_tempo}
\end{figure}


As Figuras \ref{fig:Exp3_Trajetoria} e \ref{fig:Exp3_Posicao_Tempo} mostram os resultados do terceiro experimento realizado, utilizando a pose obtida da SLAM Toolbox para controle do robô na tarefa de seguimento da trajetória \ASPASDUPLAS{CASA}. Observa-se na \ref{fig:Exp3_Trajetoria} que o robô conseguiu seguir a trajetória desejada de modo satisfatório, porém com maiores erros de posição que no Experimento 1 (Figura \ref{fig:Exp1_Trajetoria_VRPN_LINEAR}) e um comportamento oscilatório em alguns trechos da trajetória.

Na Figura \ref{fig:Exp3_Posicao_Tempo} mostra-se a evolução das posições no tempo, novamente com a posição equivalente ao sistema de câmeras OptiTrack aparecendo com o nome \ASPASIMPLES{vrpn} no gráfico e as posições obtidas da SLAM Toolbox com o nome \ASPASIMPLES{slam}. Nesta Figura também é possível notar o comportamento oscilatório durante o experimento, embora a diferença entre a posição do OptiTrack e da SLAM Toolbox não seja muito grande.


\begin{figure}[htb]
    \centering
    \caption{Experimento 3: Trajetória desejada e realizada pelo robô}
    \includegraphics[width=0.7\linewidth]{img/Resultados/Exp3_SLAM_Control_LINEAR/Trajetoria_Controller.pdf}
    \source
    \label{fig:Exp3_Trajetoria}
\end{figure}

\begin{figure}[htb]
    \centering
    \caption{Experimento 3: Comparação entre as posições registradas pelo OptiTrack e pela SLAM Toolbox durante o experimento}
    \includegraphics[width=0.8\linewidth]{img/Resultados/Exp3_SLAM_Control_LINEAR/XY_Position_v_Time_ALL.png}
    \source
    \label{fig:Exp3_Posicao_Tempo}
\end{figure}


Os resultados do experimento 4 podem ser conferidos nas Figuras \ref{fig:Exp4_Trajetoria} e \ref{fig:Exp4_Posicao_Tempo}, em que a trajetória desejada foi a Lemniscata de Bernoulli da Equação \ref{eq:lemniscata}. Podemos notar um comportamento bastante oscilatório durante algumas etapas do experimento. A Figura \ref{fig:Exp4_Posicao_Tempo} mostra a evolução das posições durante o experimento.

\begin{figure}[htb]
    \centering
    \caption{Experimento 4: Trajetória desejada e realizada pelo robô}
    \includegraphics[width=0.8\linewidth]{img/Resultados/Exp4_SLAM_Control_LEMNISCATA/Trajetoria_SLAM.pdf}
    \source
    \label{fig:Exp4_Trajetoria}
\end{figure}


\begin{figure}[htb]
    \centering
    \caption{Experimento 4: Comparação entre as posições registradas pelo OptiTrack e pela SLAM Toolbox durante o experimento}
    \includegraphics[width=\linewidth]{img/Resultados/Exp4_SLAM_Control_LEMNISCATA/Position_v_time_Comparison.pdf}
    \label{fig:Exp4_Posicao_Tempo}
    \source
\end{figure}