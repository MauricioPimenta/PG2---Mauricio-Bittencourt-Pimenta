% CHAPTER 01 - INTRODUÇÃO
%
%
% Ajustar esse \vspace de acordo com o necessário
\vspace{-42pt}

A crescente demanda por sistemas de transporte autônomos e eficientes impulsionou o desenvolvimento de tecnologias inovadoras no campo da robótica e da inteligência artificial \cite{Chen2023MilestonesSurveys} \cite{Teng2023MotionPerspectives}. Em específico, muito se discute do uso de veículos autônomos para a entrega de pacotes e transporte autônomo de cargas. \cite{Bacheti2022Leader-FollowerDelivery} \cite{Vale2021VeiculoIndustrial} \cite{Bacheti2021ALast-Mile-Delivery}
Em geral, assume-se na literatura que o transporte de cargas ocorre em ambiente estruturado, previamente conhecido. No entanto, o alevar esta tecnologia da academia para o mundo real, não necessariamente isso será uma realidade. Por vezes, o transporte terá um destino em que apenas uma breve descrição da estrutura local será apresentada para o sistema. Sendo assim, sistemas que visam a realização do transporte autônomo de cargas por robôs móveis se beneficiam de subsistemas de localização e mapeamento.\cite{Cadena2016PastAge}

Neste contexto, a cooperação entre robôs móveis aéreos e terrestres tem se destacado como uma abordagem promissora para a realização de tarefas complexas de navegação autônoma e mapeamento. A principal vantagem de mapear uma área de forma cooperativa está na capacidade de cobrir uma área maior em menos tempo. Ao utilizar um sistema multirrobô para mapeamento, pode-se trabalhar simultaneamente em áreas diferentes, aumentando a eficiência do mapeamento\cite{Feng2020AnSystems}. Além disso, a combinação das habilidades complementares dos veículos aéreos e terrestres oferece vantagens significativas no cumprimento desta tarefa, permitindo a realização de operações com maior eficiência e flexibilidade \cite{Bacheti2022Leader-FollowerDelivery}. Outra vantagem dessa cooperação é o mapeamento em terrenos complexos, como áreas urbanas ou locais de difícil acesso. A utilização de veículos aéreos para capturar imagens e dados tridimensionais do ambiente, combinada com a capacidade dos veículos terrestres de navegar de forma precisa e interagir com o terreno, pode fornecer informações detalhadas para a construção de mapas atualizados e de alta resolução.

Apesar das vantagens, a realização de tarefas de mapeamento de forma cooperativa apresenta diversos desafios. Primeiramente, se faz necessário que os robôs sigam uma rota eficiente para o mapeamento, fazendo com que estes precisem realizar menos movimentos para cobrir toda a área a ser mapeada.  Mais ainda, os mapas gerados individualmente pelos robôs devem ser combinados em um único mapa, com o cuidade para que o algoritmo não gere distorções geométricas, artefatos, ou sofra com variações de resolução, escala e iluminação \cite{Bavle2023FromSurvey}\cite{AbaspurKazerouni2022ASLAM}. Por fim, é importante que os robôs sejam capazes de desviar de obstáculos em ambientes estáticos, e de outros agentes de forma ágil em ambientes dinâmicos \cite{Li2022VisualDetection}. Esses requisitos exigem um planejamento ótimo para as rotas de navegação cooperativa\cite{Vasconcelos2020Real-timeMissions}.


\section{Objetivos}

\subsection{Objetivo Geral}

O objetivo geral deste projeto consiste em explorar algoritmos para o controle da navegação autônoma de robôs móveis e sistemas para o mapeamento do entorno destes robôs a fim de realizar tarefas de navegação de forma autônoma em ambientes desconhecidos. Para isso, buscou-se implementar e validar algoritmos de controle cinemático de robôs, de planejamento de rotas e seguimento de trajetória, além de técnicas de análise de informações obtidas de sensores para a detecção de características do ambiente e realização das tarefas de localização e mapeamento.

\subsection{Objetivos Específicos}

\begin{itemize}
\item
Realizar pesquisa bibliográfica para aprofundamento no estado da arte a respeito das técnicas de Localização e Mapeamento Simultâneos (do inglês, \textit{Simultaneous Localization and Mapping} - SLAM), planejamento de rotas e controle de robôs móveis.
\item
Implementar e validar experimentalmente algoritmos de controle de robôs móveis.
\item
Implementar e validar experimentalmente algoritmos de SLAM para localizalição no ambiente do laboratório.
\item
Analisar os resultados dos experimentos obtidos, comparando os dados de posição obtidos utilizando os algoritmos de SLAM com os dados de posição obtidos através do Sistema de Câmeras OptiTrack.
\end{itemize}

\section{Estrutura do Texto}

No Capítulo \ref{Cap02:Referencial_Teorico} é apresentado o referencial teórico e conceitos necessário para a compreensão da pesquisa relizada. Os recursos utilizados, a metodologia adotada e as etapas de desenvolvimento do projeto estão descritas no Capítulo \ref{Cap03:Metodologia}. Os resultados obtidos nos experimentos realizados são apresentados no Capítulo \ref{Cap04} e, por fim, no Capítulo \ref{Cap05} são discutidas as conclusões do trabalho, abordando os resultados obtidos e as perspectivas de trabalhos futuros.