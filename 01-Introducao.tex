% CHAPTER 01 - INTRODUÇÃO
%
%
% Ajustar esse \vspace de acordo com o necessário
\vspace{-42pt}

A crescente demanda por sistemas de transporte autônomos e eficientes impulsionou o desenvolvimento de tecnologias inovadoras no campo da robótica e da inteligência artificial \cite{Chen2023MilestonesSurveys} \cite{Teng2023MotionPerspectives}. Em específico, muito se discute do uso de veículos autônomos para a entrega de pacotes e transporte autônomo de cargas. \cite{Bacheti2022Leader-FollowerDelivery} \cite{Vale2021VeiculoIndustrial} \cite{Bacheti2021ALast-Mile-Delivery}
Em geral, assume-se na literatura que o transporte de cargas ocorre em ambiente estruturado, previamente conhecido. No entanto, o alevar esta tecnologia da academia para o mundo real, não necessariamente isso será uma realidade. Por vezes, o transporte terá um destino em que apenas uma breve descrição da estrutura local será apresentada para o sistema. Sendo assim, sistemas que visam a realização do transporte autônomo de cargas por robôs móveis se beneficiam de subsistemas de localização e mapeamento.\cite{Cadena2016PastAge}

Neste contexto, a cooperação entre robôs móveis aéreos e terrestres tem se destacado como uma abordagem promissora para a realização de tarefas complexas de navegação autônoma e mapeamento. A principal vantagem de mapear uma área de forma cooperativa está na capacidade de cobrir uma área maior em menos tempo. Ao utilizar um sistema multirrobô para mapeamento, pode-se trabalhar simultaneamente em áreas diferentes, aumentando a eficiência do mapeamento\cite{Feng2020AnSystems}. Além disso, a combinação das habilidades complementares dos veículos aéreos e terrestres oferece vantagens significativas no cumprimento desta tarefa, permitindo a realização de operações com maior eficiência e flexibilidade \cite{Bacheti2022Leader-FollowerDelivery}. Outra vantagem dessa cooperação é o mapeamento em terrenos complexos, como áreas urbanas ou locais de difícil acesso. A utilização de veículos aéreos para capturar imagens e dados tridimensionais do ambiente, combinada com a capacidade dos veículos terrestres de navegar de forma precisa e interagir com o terreno, pode fornecer informações detalhadas para a construção de mapas atualizados e de alta resolução.

Apesar das vantagens, a realização de tarefas de mapeamento de forma cooperativa apresenta diversos desafios. Primeiramente, se faz necessário que os robôs sigam uma rota eficiente para o mapeamento, fazendo com que estes precisem realizar menos movimentos para cobrir toda a área a ser mapeada.  Mais ainda, os mapas gerados individualmente pelos robôs devem ser combinados em um único mapa, com o cuidade para que o algoritmo não gere distorções geométricas, artefatos, ou sofra com variações de resolução, escala e iluminação \cite{Bavle2023FromSurvey}\cite{AbaspurKazerouni2022ASLAM}. Por fim, é importante que os robôs sejam capazes de desviar de obstáculos em ambientes estáticos, e de outros agentes de forma ágil em ambientes dinâmicos \cite{Li2022VisualDetection}. Esses requisitos exigem um planejamento ótimo para as rotas de navegação cooperativa\cite{Vasconcelos2020Real-timeMissions}.

Sendo assim, a realização deste projeto visa contribuir para o avanço do conhecimento na área de navegação autônoma cooperativa com ênfase em tarefas de mapeamento, explorando as possibilidades e os desafios da cooperação entre veículos aéreos e terrestres, com foco na geração de algoritmos ótimos para as rotas de navegação cooperativa. Espera-se que os resultados obtidos forneçam contribuições importantes para o desenvolvimento de sistemas multirrobôs para taregas de mapeamento, permitindo que o transporte autônomo de cargas possa ser realizado em ambiente desconhecido a priori.

\section{Apresentação}

\section{Trabalhos Relacionados}


\section{Objetivos}

\paragraph*{Objetivo Geral}

\begin{itemize}
\item 
O desenvolvimento de xxxxxx. Para isso, pretende-se aplicar xxxx.
\end{itemize}

\paragraph*{Objetivos Específicos}

\begin{itemize}
\item
Estudar o xxxxx;
\item
Obter um xxxxx;
\item
Implementar um xxxx;
\item
Validar e testar o xxxxx;
\end{itemize}

\section{Estrutura do Texto}
O presente trabalho está estruturado da seguinte maneira:
\begin{itemize}
\item
\textbf{Introdução}: este capítulo inicial tem como objetivo xxx;
\item
\textbf{Proposta}: neste capítulo é apresentado xxx; 
\item
\textbf{Resultados}: neste capítulo xxx; 
\item
\textbf{Conclusão}: no capítulo final deste trabalho são xxx.
\end{itemize}